\documentclass[uplatex]{jsarticle}

\usepackage{amssymb}
\usepackage{amsmath}
\usepackage{mathrsfs}
\usepackage{amsfonts}
\usepackage{mathtools}

\usepackage{xcolor}
\usepackage[dvipdfmx]{graphicx}



\usepackage{ulem}

\usepackage{braket}

%%%%%ハイパーリンク
%\usepackage[colorlinks=true,urlcolor=blue!70!black,citecolor=blue!60!black,linkcolor=blue!60!black]{hyperref}
%\usepackage{aliascnt} %for creating different biblatex references for different theoremstyles
\usepackage[setpagesize=false,dvipdfmx]{hyperref}
\usepackage{aliascnt}
\hypersetup{
    colorlinks=true,
    citecolor=blue,
    linkcolor=blue,
    urlcolor=blue,
}

\renewcommand{\eqref}[1]{\textcolor{blue}{(\ref{#1})}}

%%%%%%ハイパーリンク


%%%%%図式
%\usepackage{tikz}%%%図
\usepackage{amscd}%%%簡単な図式

\usepackage{tikz}
\usepackage{tikz-cd} %commutative diagrams in TikZ
\usetikzlibrary{calc}
\usetikzlibrary{matrix,arrows}
\usetikzlibrary{decorations.markings}

%%%%%図式



%%%%%%%%%%%%定理環境%%%%%%%%%%%%
%%%%%%%%%%%%定理環境%%%%%%%%%%%%
%%%%%%%%%%%%定理環境%%%%%%%%%%%%

\usepackage{amsthm}

%%%%%%%%%%%%Plain型%%%%%%%%%%%%


%%%%%%%%%%%%definition型%%%%%%%%%%%%

\theoremstyle{definition}

\renewcommand{\sectionautorefname}{Section}

\newtheorem{thm}{Theorem}[section]
\newcommand{\thmautorefname}{Theorem}


\newaliascnt{prop}{thm}%%%カウンター「prop」の定義(thmと同じ)
\newtheorem{prop}[prop]{Proposition}
\aliascntresetthe{prop}
\newcommand{\propautorefname}{Proposition}%%%カウンター名propは「命題」で参照する

\newaliascnt{cor}{thm}
\newtheorem{cor}[cor]{Corollary}
\aliascntresetthe{cor}
\newcommand{\corautorefname}{Corollary}

\newaliascnt{lem}{thm}
\newtheorem{lem}[lem]{Lemma}
\aliascntresetthe{lem}
\newcommand{\lemautorefname}{Lemma}

%%%%%%%アルファベットで番号づける定理環境
\newtheorem{thmA}{Theorem}[section]
\newcommand{\thmAautorefname}{Theorem}
\renewcommand\thethmA{\Alph{thmA}}

\newtheorem{corA}{Theorem}[section]
\newcommand{\corAautorefname}{Corollary}
\renewcommand\thecorA{\Alph{corA}}

\newaliascnt{defi}{thm}
\newtheorem{defi}[defi]{Definition}
\aliascntresetthe{defi}
\newcommand{\defiautorefname}{Definition}

\newaliascnt{rem}{thm}
\newtheorem{rem}[rem]{Remark}
\aliascntresetthe{rem}
\newcommand{\remautorefname}{Remark}

\newaliascnt{reconstruction}{thm}
\newtheorem{reconstruction}[reconstruction]{Reconstruction}
\aliascntresetthe{reconstruction}
\newcommand{\reconstructionautorefname}{Reconstruction}

%%%%%%%番号づけない定理環境
\newtheorem*{exam*}{Example}
\newtheorem*{rrem*}{Remark}
\newtheorem*{defi*}{Definition}

%%%%%%%%%%%%定理環境%%%%%%%%%%%%
%%%%%%%%%%%%定理環境%%%%%%%%%%%%
%%%%%%%%%%%%定理環境%%%%%%%%%%%%





%%%%%箇条書き環境
\usepackage[]{enumitem}

\makeatletter
\AddEnumerateCounter{\fnsymbol}{\c@fnsymbol}{9}%%%%fnsymbolという文字をenumerate環境のパラメーターで使えるようにする。
\makeatother

\makeatletter
\renewcommand{\p@enumii}{}
\makeatother

\renewcommand{\theenumi}{(\roman{enumi})}%%%%%itemは(1),(2),(3)で番号付ける。
\renewcommand{\labelenumi}{\theenumi}

\renewcommand{\theenumii}{(\alph{enumii})}%%%%%itemは(1),(2),(3)で番号付ける。
\renewcommand{\labelenumii}{\theenumii}

\usepackage{moreenum}
%%%%%箇条書き環境



\usepackage{mandorasymb}
\usepackage{applekeys}
\renewcommand{\qedsymbol}{\pencilkey}
%\renewcommand{\qedsymbol}{\kinoposymbniko}




\usepackage{latexsym}
\DeclareMathOperator{\Hom}{Hom}
\DeclareMathOperator{\Isom}{Isom}
\DeclareMathOperator{\ISOM}{\mathbf{Isom}}
\DeclareMathOperator{\id}{\mathrm{id}}
\DeclareMathOperator{\im}{\mathrm{Im}}
\DeclareMathOperator{\Spec}{\mathrm{Spec}}
\newcommand{\Supp}{\mathrm{Supp}}
\DeclareMathOperator{\Aut}{\mathrm{Aut}}

\newcommand{\coker}{\mathrm{coker}}

\DeclareMathOperator{\Tor}{\mathrm{Tor}}
\DeclareMathOperator{\Ext}{\mathrm{Ext}}

\DeclareMathOperator{\colim}{\mathrm{colim}}
\DeclareMathOperator{\plim}{\mathrm{lim}}
\newcommand{\Lotimes}[1]{\mathop{\otimes^{\mathbf{L}}_{#1}}}
\newcommand{\bLotimes}[1]{\mathop{\bar{\otimes}^{\mathbf{L}}_{#1}}}
\DeclareMathOperator{\RHom}{\mathbf{R}Hom}
\DeclareMathOperator{\bRHom}{\underline{\mathbf{R}Hom}}
\DeclareMathOperator{\inHom}{\mathcal{H}om}
\newcommand{\Ob}{\mathrm{Ob}}
\newcommand{\FP}[1]{\mathsf{FP}_{/#1}}

\newcommand{\rsa}{\rightsquigarrow}
\renewcommand{\coprod}{\amalg}
\renewcommand{\emptyset}{\varnothing}
\newcommand{\ep}{\varepsilon}
\newcommand{\dg}{\mathrm{dg}}
\newcommand{\op}{\mathrm{op}}

\newcommand{\dfn}{:\overset{\mbox{{\scriptsize def}}}{=}}
\newcommand{\deff}{:\hspace{-3pt}\overset{\text{def}}{\iff}}

\newcommand{\univ}[1]{\mathbb{#1}}
\newcommand{\usm}{\(\univ{U}\)-small}
\newcommand{\vsm}{\(\univ{V}\)-small}
\newcommand{\unit}[1]{\mathbf{1}_{\mathcal{#1}}}

\newcommand{\Qcoh}{\mathsf{Qcoh}}
\newcommand{\Coh}{\mathsf{Coh}}
\newcommand{\Pic}{\mathrm{Pic}}
\newcommand{\Sym}{\mathrm{Sym}}
\newcommand{\Mod}{\mathsf{Mod}}


\newcommand{\A}{\mathbb{A}}
\newcommand{\C}{\mathbb{C}}
\renewcommand{\P}{\mathbb{P}}
\newcommand{\R}{\mathbb{R}}
\newcommand{\Q}{\mathbb{Q}}
\newcommand{\Z}{\mathbb{Z}}
\newcommand{\N}{\mathbb{N}}



\newcommand{\mcA}{\mathcal{A}}
\newcommand{\mcB}{\mathcal{B}}
\newcommand{\mcC}{\mathcal{C}}
\newcommand{\mcD}{\mathcal{D}}
\newcommand{\mcE}{\mathcal{E}}
\newcommand{\mcF}{\mathcal{F}}
\newcommand{\mcG}{\mathcal{G}}
\newcommand{\mcH}{\mathcal{H}}
\newcommand{\mcI}{\mathcal{I}}
\newcommand{\mcJ}{\mathcal{J}}
\newcommand{\mcK}{\mathcal{K}}
\newcommand{\mcL}{\mathcal{L}}
\newcommand{\mcM}{\mathcal{M}}
\newcommand{\mcN}{\mathcal{N}}
\newcommand{\mcO}{\mathcal{O}}
\newcommand{\mcP}{\mathcal{P}}
\newcommand{\mcQ}{\mathcal{Q}}
\newcommand{\mcR}{\mathcal{R}}
\newcommand{\mcS}{\mathcal{S}}
\newcommand{\mcT}{\mathcal{T}}
\newcommand{\mcU}{\mathcal{U}}
\newcommand{\mcV}{\mathcal{V}}
\newcommand{\mcW}{\mathcal{W}}
\newcommand{\mcX}{\mathcal{X}}
\newcommand{\mcY}{\mathcal{Y}}
\newcommand{\mcZ}{\mathcal{Z}}

\DeclareMathOperator{\OOO}{\mcO}

\newcommand{\OC}{{\OOO_C}}
\newcommand{\OD}{{\OOO_D}}
\renewcommand{\OE}{{\OOO_E}}
\newcommand{\OF}{{\OOO_F}}
\newcommand{\OH}{{\OOO_H}}
\newcommand{\OM}{{\OOO_N}}
\newcommand{\ON}{{\OOO_N}}
\newcommand{\OS}{{\OOO_S}}
\newcommand{\OT}{{\OOO_T}}
\newcommand{\OU}{{\OOO_U}}
\newcommand{\OV}{{\OOO_V}}
\newcommand{\OW}{{\OOO_W}}
\newcommand{\OX}{{\OOO_X}}
\newcommand{\OY}{{\OOO_Y}}
\newcommand{\OZ}{{\OOO_Z}}

\newcommand{\OO}[1]{\OOO_{#1}}



\title{接続に関する基礎事項をjetバンドルの用語で書き直す}

\author{ゆじ}

\begin{document}

\maketitle

これは接続に関する基礎事項をjetバンドル (のような何か) を用いて線形に書き直したノートである。




\section{スキーム論的な視点で見たjet}


\(X\)が\(S\)-スキームであるときには、
\(X\times_SX\)の対角の\(r\)-次無限小近傍を\(X^{(r)}\)と書き、
第一、第二射影を\(p,q:X^{(r)}\to X\)で表す。

\begin{defi}
  \(J^r(E) \dfn q_*p^*E\)
\end{defi}


\begin{rem}
  \(J^r(-)\)という操作は函手的である。
\end{rem}


\subsection{基本的な完全系列}

\(r\)-次無限小部分を\(0\)にすることによって
自然な完全列
\[
\begin{CD}
  0 @>>> \Sym^r(\Omega_{X/S}) \otimes E @>>> J^r(E) @>>> J^{r-1}(E) @>>> 0
\end{CD}
\]
を得る。
とくに\(r=1\)とすることで完全列
\[
\begin{CD}
  0 @>>> \Omega_{X/S} \otimes E @>>> J^1(E) @>>> E @>>> 0
\end{CD}
\]
を得る。

後で見るように、\(E\)上の接続とは、
この (\(r=1\)の場合の) 完全系列の分裂\(E\to J^1(E)\)のことである
(cf. \autoref{sub: conn 1-jet})。



\subsection{テンソル}

二つのベクトル束\(E_1,E_2\)に対して、自然な射
\[q^*q_*p^*E_i \to p^*E_i\]
をテンソルすることで、射
\[
q^*(J^r(E_1) \otimes_{\mcO_X} J^r(E_2))
= q^*((q_*p^*E_1) \otimes_{\mcO_X} (q_*p^*E^2)) \xrightarrow{\sim}
(q^*q_*p^*E_1)\otimes_{\mcO_{X^{(1)}}} (q^*q_*p^*E_2)
\to p^*E_1\otimes_{\mcO_{X^{(1)}}} p^*E_2
\to p^*(E_1\otimes_{\mcO_{X^{(1)}}} E_2)
\]
を得る。
\(q_*\)は\(q^*\)の右随伴であるから、射
\[
J^r(E_1) \otimes_{\mcO_X} J^r(E_2) \to J^r(E_1\otimes E_2)
\]
を得る。
この射は\(r\)を小さくすることによって得られる全射と可換であり、
さらに\(E_1,E_2\)について函手的である。





\subsection{アーベル群の層としての直和分解}


\(p,q:X^{(r)}\to X\)は、下部位相空間の間の射はどちらも\(\id_{|X|}\)であるため、
アーベル群の層としては\(q_*(-) = p_*(-)\)となる。
従って、アーベル群の層としては
\(q_*p^*E = p_*p^*E\)であり、
自然な全射\(J^r(E) \to E\)のアーベル群の層としての自然な分裂
\(E\to J^r(E)\)を得る。
これを\(d^r\)で表す。
\(d^r\)は\(\OX\)-線形ではない。

\(r=1\)とする。
このとき、\(J^1(E)\)はアーベル群として
\(E\oplus (\Omega_X\otimes E)\)
と直和分解する。









\section{接続}


\begin{defi}[接続]
  \(M\)を多様体、
  \(E\)をベクトル束とする。
  \(E\)上の\textbf{接続}とは、
  \(\R\)-線形写像\(\nabla:E\to \Omega\otimes E\)であって、
  任意の\(s\in E\)と\(f\in \OM\)に対して
  \(\nabla(fs) = df\otimes s + f\nabla(s)\)が成り立つもののことを言う。
\end{defi}


\subsection{Jetによる解釈}
\label{sub: conn 1-jet}






\subsection{双対接続}
\label{sub: dual conn}


\begin{defi}[双対接続]
  \(\nabla:E\to J^1(E)\)を接続とする。
  \(\nabla\)は\(X^{(1)}\)上の層の射
  \(\tilde{\nabla}: q^*E \xrightarrow{\sim} p^*E\)と対応する。
  \(\tilde{\nabla}\)の双対の逆射
  \((\tilde{\nabla}^*)^{-1}: q^*E^* \xrightarrow{\sim} p^*E^*\)
  と対応する射
  \(E^* \to J^1(E^*)\)を
  \(\nabla\)の\textbf{双対接続}という。
\end{defi}


\begin{rem}
  
\end{rem}





\subsection{テンソル積}




\subsection{計量}


ベクトル束\(E\)上に計量を与えることは、同型射
\(h:E\xrightarrow{\sim}E^*\)
を与えることと等しい。

\begin{lem}\label{lem: met comm}
  \(h\)を\(E\)上の計量、
  \(\nabla\)をベクトル束\(E\)上の接続、
  \(\nabla^*\)を双対接続とする。
  \(\nabla\)が\(h\)と可換であるための必要十分条件は、
  以下の図式が可換であることである:
  \[
  \begin{CD}
    E @>{\nabla}>> J^1(E) \\
    @V{h}VV @VV{J^1(h)}V \\
    E^* @>{\nabla^*}>> J^1(E^*).
  \end{CD}
  \]
\end{lem}

\begin{proof}
  \(\{e_a\}\)を局所frame,
  \(\{e_*^a\}\)をその双対frame,
  \(h(e_a) = h_{ab}e_*^b\),
  \(\underline{\nabla}(e_a) = A_a^b\otimes e_b\)
  (ここで\(A_a^b\)はこのframeをとっている開集合上の1-form)
  とおいて計算すると、
  \begin{align*}
    (J^1(h)\circ \nabla) (e_a) &= J^1(h)(e_a,\underline{\nabla}(e_a)) \\
    &= (h_{ab}e_*^b, h_{bc}A_a^b\otimes e_*^c), \\
    (\nabla^*\circ h) (e_a) &= \nabla^*(h_{ab}e_*^b) \\
    &= (h_{ab}e_*^b, \underline{\nabla}^*(h_{ab}e_*^b)) \\
    &= (h_{ab}e_*^b, dh_{ab}\otimes e_*^b - h_{ab}A_b^c \otimes e_*^c)
  \end{align*}
  となる。
  従って、上記の図式の可換性は
  \(dh_{ab} = h_{ab}A_b^c + h_{bc}A_a^b\)
  が成り立つことと同値であり、これは\(h\)が\(\nabla\)と可換であることに他ならない。
\end{proof}





\section{\(J^1J^1\)}



\subsection{共変外微分}



\subsection{曲率}











\section{Gaussの方程式}

このsectionでは、
\[
\begin{CD}
  0 @>>> E_1 @>{i}>> E @>{p}>> E_2 @>>> 0
\end{CD}
\]
をベクトル束の完全列とし、
\(\nabla:E\to J^1(E)\)
を接続とする。


\subsection{第二基本形式とシェイプ作用素}


\begin{defi}[第二基本形式]
  \(S^{\nabla}\dfn J^1(p)\circ \nabla \circ i : E_1 \to J^1(E_2)\)
  を\textbf{第二基本形式}という
  (たんに\(S\)と書くこともある)。
\end{defi}

\begin{rem}
  自然な全射を\(p_E:J^1(E)\to E, p_{E_2}:J^1(E_2)\to E_2\)と書くと、
  \[
  p_{E_2}\circ S = p_{E_2}\circ J^1(p)\circ \nabla \circ i
  = p\circ p_E \circ \nabla \circ i
  = p\circ i = 0
  \]
  となるので、\(S:E_1\to J^1(E_2)\)は実際には
  \(\Omega\otimes E_2 \subset J^1(E_2)\)
  を一意的に経由する。
\end{rem}


\begin{rem}
  第二基本形式は、\(E\)の計量によらない。
  実際、この節ではまだ\(E\)に計量が入っていることを仮定していない。
\end{rem}



\(h:E\xrightarrow{\sim} E^*\)を\(E\)の計量とする。
さらに、
\begin{itemize}
  \item
  \(h_1\dfn i^* \circ h \circ i : E_1 \xrightarrow{\sim} E_1^*\)を
  \(E_1\)上の誘導計量、
  \item
  \(h_2 \dfn (q\circ h^{-1}\circ q^*)^{-1}: E_2 \xrightarrow{\sim} E_2^*\)を
  \(E_2\)上の誘導計量、
  \item
  \(p\dfn h_1^{-1}\circ i^*\circ h : E\to E_1\)を
  \(h\)が定める\(i:E_1 \to E\)のretract、
  \item
  \(j\dfn h^{-1}\circ q^*\circ h_2 : E_2 \to E\)を
  \(h\)が定める\(q:E\to E_2\)のsplit
\end{itemize}
とする。
このとき、定義より、
\(h_1\circ p = i^*\circ h, h\circ j = q^*\circ h_2\)
が成り立つ。


\begin{defi}[シェイプ作用素]
  \(A^{\nabla}\dfn J^1(p) \circ \nabla \circ j : E_2 \to J^1(E_1)\)
  を\textbf{シェイプ作用素} (shape operator) という。
  しばしば\(\nabla\)を省略してたんに\(A\)と書く。
\end{defi}


\begin{rem}
  第二基本形式の場合と同様に、
  実際には、\(A:E_2\to J^1(E_1)\)は
  \(\Omega\otimes E_1\subset J^1(E_1)\)を一意的に経由する。
\end{rem}


\begin{prop}\label{prop: second and shape}
  \(S\)を\(\nabla\)の第二基本形式、
  \(A\)を\(\nabla\)のシェイプ作用素、
  \(\nabla^*:E^*\to J^1(E^*)\)を双対接続、
  \(S^*\)を\(\nabla^*\)の第二基本形式、
  \(A^*\)を\(\nabla^*\)のシェイプ作用素とする。
  \begin{enumerate}
    \item \label{enumi: 1 prop: second and shape}
    \(S^* = J^1(i^*) \circ \nabla^* \circ q^* : E_2^* \to J^1(E_1^*)\)
    である。
    \item \label{enumi: 2 prop: second and shape}
    \(A^* = J^1(j^*) \circ \nabla^* \circ p^* : E_1^* \to J^1(E_2^*)\)
    である。
    \item \label{enumi: 3 prop: second and shape}
    以下の図式は可換である:
    \[
    \begin{CD}
      E_1 @>{S}>> J^1(E_2) \\
      @V{h_1}VV @VV{J^1(h_2)}V \\
      E_1^* @>{A^*}>> J^1(E_2^*).
    \end{CD}
    \]
    \item \label{enumi: 4 prop: second and shape}
    以下の図式は可換である:
    \[
    \begin{CD}
      E_2 @>{A}>> J^1(E_1) \\
      @V{h_2}VV @VV{J^1(h_1)}V \\
      E_2^* @>{S^*}>> J^1(E_1^*).
    \end{CD}
    \]
  \end{enumerate}
\end{prop}


\begin{proof}
  \ref{enumi: 1 prop: second and shape}と
  \ref{enumi: 2 prop: second and shape}は定義より従う。
  \ref{enumi: 3 prop: second and shape}を示す。
  計算すると、
  \begin{align*}
    J^1(h_2) \circ S &= J^1(h_2) \circ J^1(q) \circ \nabla \circ i \\
    &= J^1(j^*) \circ J^1(h) \circ \nabla \circ i \\
    &= J^1(j^*) \circ \nabla^* \circ h \circ i \\
    &= J^1(j^*) \circ \nabla^* \circ p^* \circ h_1 \\
    &= A^* \circ h_1
  \end{align*}
  となる。
  以上で\ref{enumi: 3 prop: second and shape}が示された。
  \ref{enumi: 4 prop: second and shape}を示す。
  計算すると、
  \begin{align*}
    J^1(h_1) \circ A &= J^1(h_1) \circ J^1(p) \circ \nabla \circ j \\
    &= J^1(i^*) \circ J^1(h) \circ \nabla \circ j \\
    &= J^1(i^*) \circ \nabla^* \circ h \circ j \\
    &= J^1(i^*) \circ \nabla^* \circ q^* \circ h_2 \\
    &= S^* \circ h_2
  \end{align*}
  となる。
  以上で\ref{prop: second and shape}の証明を完了する。
\end{proof}


\begin{rem}
  \(\Omega\otimes E_1\subset J^1(E_1)\)と
  \(\Omega\otimes E_2\subset J^1(E_2)\)の成分を見れば、
  \[
  h(S(e_1),e_2) = h(e_1,A^*(e_2))
  \]
  となる。
\end{rem}



\subsection{Gaussの方程式}

\(E\)上の接続\(\nabla\)と計量\(h\)によって定まる\(E_1\)の誘導接続を
\(\nabla^{\top}\dfn J^1(p)\circ \nabla \circ i :E_1 \to J^1(E_1)\)
と表す。

\(E\)上の接続\(\nabla\)の曲率
\(R:E\to J^1(J^1(E))\)の\(E_1\)の成分を
\(E_1\)上の誘導接続\(\nabla^{\top}\)の曲率\(R^{\top}\)と比較したものを
\textbf{Gaussの方程式}と言う。


\begin{lem}\label{lem: nabla E_1}
  \(\nabla\circ i = J^1(i) \circ \nabla^{\top} + J^1(j) \circ S\).
\end{lem}

\begin{proof}
  \(\id_{J^1(E)} = J^1(i)\circ J^1(p) + J^1(j) \circ J^1(q)\)
  であるから、計算すれば
  \begin{align*}
    \nabla\circ i &= \id_{J^1(E)} \circ \nabla \circ i \\
    &= J^1(i)\circ J^1(p) \circ \nabla \circ i
    + J^1(j) \circ J^1(q) \circ \nabla \circ i \\
    &= J^1(i)\circ \nabla^{\top} + J^1(j) \circ S
  \end{align*}
  となる。
\end{proof}


\begin{thm}[Gaussの方程式]
  \label{thm: Gauss eq}
  \(h_2:E_2 \xrightarrow{\sim} E_2^*\)を誘導計量とする。
  このとき、以下の等式が成り立つ:
  \[
  J^1(J^1(i^*)) \circ J^1(J^1(h)) \circ R \circ i
  = J^1(S^*) \circ J^1(h_2) \circ S + J^1(J^1(h_1))\circ R^{\top}.
  \]
\end{thm}

\begin{proof}
  \(\nabla^*\)を双対接続とする。
  計算すると、
  \begin{align*}
    &J^1(J^1(i^*)) \circ J^1(J^1(h)) \circ R \circ i \\
    &= J^1(J^1(i^*)) \circ J^1(J^1(h)) \circ J^1(\nabla) \circ \nabla \circ i \\
    &\overset{\bigstar}{=}
    J^1(J^1(i^*)) \circ J^1(J^1(h)) \circ J^1(\nabla) \circ J^1(i) \circ \nabla^{\top}
    + J^1(J^1(i^*)) \circ J^1(J^1(h)) \circ J^1(\nabla) \circ J^1(j) \circ S \\
    &\overset{\spadesuit}{=}
    J^1(J^1(i^*)) \circ J^1(J^1(h)) \circ J^1(\nabla) \circ J^1(i) \circ \nabla^{\top}
    + J^1(J^1(i^*)) \circ J^1(\nabla^*) \circ J^1(h) \circ J^1(j) \circ S \\
    &\overset{\clubsuit}{=}
    J^1(J^1(h_1)) \circ J^1(J^1(p)) \circ J^1(\nabla) \circ J^1(i) \circ \nabla^{\top}
    + J^1(J^1(i^*)) \circ J^1(\nabla^*) \circ J^1(q^*) \circ J^1(h_2) \circ S \\
    &= J^1(J^1(h_1)) \circ J^1(\nabla^{\top}) \circ \nabla^{\top}
    + J^1(S^*) \circ J^1(h_1) \circ S \\
    &= J^1(J^1(h_1)) \circ R^{\top} + J^1(S^*) \circ J^1(h_1) \circ S
  \end{align*}
  となる。
  ただし\(\bigstar\)の箇所で\autoref{lem: nabla E_1}を用い、
  \(\spadesuit\)の箇所で\autoref{lem: nabla E_1}を用い、
  \(\clubsuit\)の箇所で等式
  \(i^*\circ h = h_1\circ p\)と\(h\circ j = q^* \circ h_2\)を用いた。
  以上で\autoref{thm: Gauss eq}の証明を完了する。
\end{proof}




\end{document}
