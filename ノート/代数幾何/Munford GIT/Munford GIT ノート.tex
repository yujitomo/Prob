\documentclass[uplatex]{jsarticle}

\usepackage{amssymb}
\usepackage{amsmath}
\usepackage{mathrsfs}
\usepackage{amsfonts}
\usepackage{mathtools}

\usepackage{xcolor}
\usepackage[dvipdfmx]{graphicx}



\usepackage{ulem}

\usepackage{braket}

%%%%%ハイパーリンク
%\usepackage[colorlinks=true,urlcolor=blue!70!black,citecolor=blue!60!black,linkcolor=blue!60!black]{hyperref}
%\usepackage{aliascnt} %for creating different biblatex references for different theoremstyles
\usepackage[setpagesize=false,dvipdfmx]{hyperref}
\usepackage{aliascnt}
\hypersetup{
    colorlinks=true,
    citecolor=blue,
    linkcolor=blue,
    urlcolor=blue,
}

\renewcommand{\eqref}[1]{\textcolor{blue}{(\ref{#1})}}

%%%%%%ハイパーリンク


%%%%%図式
%\usepackage{tikz}%%%図
\usepackage{amscd}%%%簡単な図式

\usepackage{tikz}
\usepackage{tikz-cd} %commutative diagrams in TikZ
\usetikzlibrary{calc}
\usetikzlibrary{matrix,arrows}
\usetikzlibrary{decorations.markings}

%%%%%図式



%%%%%%%%%%%%定理環境%%%%%%%%%%%%
%%%%%%%%%%%%定理環境%%%%%%%%%%%%
%%%%%%%%%%%%定理環境%%%%%%%%%%%%

\usepackage{amsthm}

%%%%%%%%%%%%Plain型%%%%%%%%%%%%


%%%%%%%%%%%%definition型%%%%%%%%%%%%

\theoremstyle{definition}

\renewcommand{\sectionautorefname}{Section}

\newtheorem{thm}{Theorem}[section]
\newcommand{\thmautorefname}{Theorem}


\newaliascnt{prop}{thm}%%%カウンター「prop」の定義(thmと同じ)
\newtheorem{prop}[prop]{Proposition}
\aliascntresetthe{prop}
\newcommand{\propautorefname}{Proposition}%%%カウンター名propは「命題」で参照する

\newaliascnt{cor}{thm}
\newtheorem{cor}[cor]{Corollary}
\aliascntresetthe{cor}
\newcommand{\corautorefname}{Corollary}

\newaliascnt{lem}{thm}
\newtheorem{lem}[lem]{Lemma}
\aliascntresetthe{lem}
\newcommand{\lemautorefname}{Lemma}

\newaliascnt{defi}{thm}
\newtheorem{defi}[defi]{Definition}
\aliascntresetthe{defi}
\newcommand{\defiautorefname}{Definition}

\newaliascnt{rem}{thm}
\newtheorem{rem}[rem]{Remark}
\aliascntresetthe{rem}
\newcommand{\remautorefname}{Remark}


\newaliascnt{exam}{thm}
\newtheorem{exam}[exam]{Example}
\aliascntresetthe{exam}
\newcommand{\examautorefname}{Example}

\newaliascnt{reconstruction}{thm}
\newtheorem{reconstruction}[reconstruction]{Reconstruction}
\aliascntresetthe{reconstruction}
\newcommand{\reconstructionautorefname}{Reconstruction}

%%%%%%%番号づけない定理環境
\newtheorem*{exam*}{Example}
\newtheorem*{rem*}{Remark}
\newtheorem*{defi*}{Definition}

%%%%%%%%%%%%定理環境%%%%%%%%%%%%
%%%%%%%%%%%%定理環境%%%%%%%%%%%%
%%%%%%%%%%%%定理環境%%%%%%%%%%%%





%%%%%箇条書き環境
\usepackage[]{enumitem}

\makeatletter
\AddEnumerateCounter{\fnsymbol}{\c@fnsymbol}{9}%%%%fnsymbolという文字をenumerate環境のパラメーターで使えるようにする。
\makeatother

\makeatletter
\renewcommand{\p@enumii}{}
\makeatother

\renewcommand{\theenumi}{(\roman{enumi})}%%%%%itemは(1),(2),(3)で番号付ける。
\renewcommand{\labelenumi}{\theenumi}

\renewcommand{\theenumii}{(\alph{enumii})}%%%%%itemは(1),(2),(3)で番号付ける。
\renewcommand{\labelenumii}{\theenumii}

\usepackage{moreenum}
%%%%%箇条書き環境



\usepackage{mandorasymb}
\usepackage{applekeys}
\renewcommand{\qedsymbol}{\pencilkey}
%\renewcommand{\qedsymbol}{\kinoposymbniko}




\usepackage{latexsym}
\DeclareMathOperator{\Hom}{Hom}
\DeclareMathOperator{\Isom}{Isom}
\DeclareMathOperator{\ISOM}{\mathbf{Isom}}
\DeclareMathOperator{\id}{\mathrm{id}}
\DeclareMathOperator{\im}{\mathrm{Im}}
\DeclareMathOperator{\Spec}{\mathrm{Spec}}
\newcommand{\Supp}{\mathrm{Supp}}
\DeclareMathOperator{\Aut}{\mathrm{Aut}}

\newcommand{\coker}{\mathrm{coker}}

\DeclareMathOperator{\Tor}{\mathrm{Tor}}
\DeclareMathOperator{\Ext}{\mathrm{Ext}}

\DeclareMathOperator{\colim}{\mathrm{colim}}
\DeclareMathOperator{\plim}{\mathrm{lim}}

\newcommand{\Ob}{\mathrm{Ob}}

\newcommand{\rsa}{\rightsquigarrow}
\renewcommand{\coprod}{\amalg}
\renewcommand{\emptyset}{\varnothing}
\newcommand{\ep}{\varepsilon}
\newcommand{\op}{\mathrm{op}}

\newcommand{\dfn}{:\overset{\mbox{{\scriptsize def}}}{=}}
\newcommand{\deff}{:\hspace{-3pt}\overset{\text{def}}{\iff}}
\newcommand{\dl}{\partial}

\newcommand{\Qcoh}{\mathsf{Qcoh}}
\newcommand{\Coh}{\mathsf{Coh}}
\newcommand{\Pic}{\mathrm{Pic}}
\newcommand{\Sym}{\mathrm{Sym}}
\newcommand{\Mod}{\mathsf{Mod}}


\newcommand{\A}{\mathbb{A}}
\newcommand{\C}{\mathbb{C}}
\newcommand{\K}{\mathbb{K}}
\renewcommand{\P}{\mathbb{P}}
\newcommand{\R}{\mathbb{R}}
\newcommand{\Q}{\mathbb{Q}}
\newcommand{\Z}{\mathbb{Z}}
\newcommand{\N}{\mathbb{N}}



\newcommand{\mcA}{\mathcal{A}}
\newcommand{\mcB}{\mathcal{B}}
\newcommand{\mcC}{\mathcal{C}}
\newcommand{\mcD}{\mathcal{D}}
\newcommand{\mcE}{\mathcal{E}}
\newcommand{\mcF}{\mathcal{F}}
\newcommand{\mcG}{\mathcal{G}}
\newcommand{\mcH}{\mathcal{H}}
\newcommand{\mcI}{\mathcal{I}}
\newcommand{\mcJ}{\mathcal{J}}
\newcommand{\mcK}{\mathcal{K}}
\newcommand{\mcL}{\mathcal{L}}
\newcommand{\mcM}{\mathcal{M}}
\newcommand{\mcN}{\mathcal{N}}
\newcommand{\mcO}{\mathcal{O}}
\newcommand{\mcP}{\mathcal{P}}
\newcommand{\mcQ}{\mathcal{Q}}
\newcommand{\mcR}{\mathcal{R}}
\newcommand{\mcS}{\mathcal{S}}
\newcommand{\mcT}{\mathcal{T}}
\newcommand{\mcU}{\mathcal{U}}
\newcommand{\mcV}{\mathcal{V}}
\newcommand{\mcW}{\mathcal{W}}
\newcommand{\mcX}{\mathcal{X}}
\newcommand{\mcY}{\mathcal{Y}}
\newcommand{\mcZ}{\mathcal{Z}}





\newcommand{\bfA}{\mathbf{A}}
\newcommand{\bfB}{\mathbf{B}}
\newcommand{\bfC}{\mathbf{C}}
\newcommand{\bfD}{\mathbf{D}}
\newcommand{\bfE}{\mathbf{E}}
\newcommand{\bfF}{\mathbf{F}}
\newcommand{\bfG}{\mathbf{G}}
\newcommand{\bfH}{\mathbf{H}}
\newcommand{\bfI}{\mathbf{I}}
\newcommand{\bfJ}{\mathbf{J}}
\newcommand{\bfK}{\mathbf{K}}
\newcommand{\bfL}{\mathbf{L}}
\newcommand{\bfM}{\mathbf{M}}
\newcommand{\bfN}{\mathbf{N}}
\newcommand{\bfO}{\mathbf{O}}
\newcommand{\bfP}{\mathbf{P}}
\newcommand{\bfQ}{\mathbf{Q}}
\newcommand{\bfR}{\mathbf{R}}
\newcommand{\bfS}{\mathbf{S}}
\newcommand{\bfT}{\mathbf{T}}
\newcommand{\bfU}{\mathbf{U}}
\newcommand{\bfV}{\mathbf{V}}
\newcommand{\bfW}{\mathbf{W}}
\newcommand{\bfX}{\mathbf{X}}
\newcommand{\bfY}{\mathbf{Y}}
\newcommand{\bfZ}{\mathbf{Z}}

\DeclareMathOperator{\OOO}{\mcO}

\newcommand{\OC}{{\OOO_C}}
\newcommand{\OD}{{\OOO_D}}
\renewcommand{\OE}{{\OOO_E}}
\newcommand{\OF}{{\OOO_F}}
\newcommand{\OH}{{\OOO_H}}
\newcommand{\OM}{{\OOO_N}}
\newcommand{\ON}{{\OOO_N}}
\newcommand{\OS}{{\OOO_S}}
\newcommand{\OT}{{\OOO_T}}
\newcommand{\OU}{{\OOO_U}}
\newcommand{\OV}{{\OOO_V}}
\newcommand{\OW}{{\OOO_W}}
\newcommand{\OX}{{\OOO_X}}
\newcommand{\OY}{{\OOO_Y}}
\newcommand{\OZ}{{\OOO_Z}}

\newcommand{\OO}[1]{\OOO_{#1}}


\newcommand{\resol}{\mathrm{Resol}}
\newcommand{\spectm}{\mathrm{Sp}}




\title{Munford GIT ノート}

\author{ゆじ}

\begin{document}

\maketitle


\section{Definitions}


\begin{defi}[群スキーム]
  \(S\)-スキーム\(G\)が\textbf{群スキーム}であるとは...
\end{defi}

\begin{defi}[代数群]
  体\(k\)上の代数多様体\(G\)が\textbf{代数群}であるとは、
  \(k\)上の群スキームであって、さらに滑らかな\(k\)-多様体であることを言う。
\end{defi}

\begin{defi}[群スキームの作用]
  \(S\)上の群スキーム\(G\)の\(S\)-スキーム\(X\)への\textbf{作用}とは...
\end{defi}


\begin{defi}[orbit]
  \(S\)上の群スキーム\(G\)が\(S\)-スキーム\(X\)に作用しているとする。
  \(T\)-値点\(T\to X\)の\textbf{orbit}とは...
  \textbf{stabilizer}とは...
\end{defi}


\begin{defi}[Categorical Quotient]
  \(S\)上の群スキーム\(G\)が\(S\)-スキーム\(X\)に作用しているとする。
  \(S\)-スキーム\(Y\)と射\(q:X\to Y\)のペア\((Y,q)\)が\textbf{categorical quotient}であるとは、
  次を満たすことを言う:
  \begin{enumerate}
    \item
    以下の図式は可換である:
    \[
    \begin{CD}
      G\times X @>{\text{作用} \sigma}>> X \\
      @V{\text{第二射影} p_2}VV @VV{q}V \\
      X @>{q}>> Y
    \end{CD}
    \]
    \item
    任意の\(S\)-スキーム\(Z\)と射\(r:X\to Z\)に対し、
    \(r\circ p_2 = r \circ \sigma\)なら、
    一意的に\(f:Y\to Z\)が存在して、\(r = f\circ q\)が成り立つ。
  \end{enumerate}
\end{defi}





\begin{defi}[Geometric Quotient]
  \(S\)上の群スキーム\(G\)が\(S\)-スキーム\(X\)に作用しているとする。
  \(S\)-スキーム\(Y\)と射\(q:X\to Y\)のペア\((Y,q)\)が\textbf{geometric quotient}であるとは、
  次を満たすことを言う:
  \begin{enumerate}
    \item
    \(q\circ p_2 = q \circ \sigma\)である。
    \item
    \(q\)は全射であり、
    \((\sigma,p_2): G\times_S X \to X\times_S X\)の像は\(X\times_Y X\)となる。
    \item
    \(q\)は\textbf{submersive}である。
    すなわち、
    \(V\subset Y\)が開であることと\(q^{-1}(Y)\subset X\)が開であることは同値である。
    \item
    \(q^{\#}:\mcO_Y \to q_*\mcO_X\)は単射であり、
    その像は\( = (q_*\mcO_X)^G\)に等しい。
  \end{enumerate}
\end{defi}



\begin{rem*}
  最後の条件以外はどんな\(Y'\to Y\)での基底変換のあとでも満たされる。
  最後の条件は開埋め込みでの基底変換 (\(Y\)の開部分集合の上への制限) のあとでは満たされる。
\end{rem*}




\begin{defi}[Universal Categorical (Geometric) Quotient, Uniform Categorical (Geometric) Quotient]
  \(X\to Y\)は、任意の射\(Y'\to Y\)での基底変換\(X'\to Y'\)が
  categorical (geometric) quotientであるとき、
  \textbf{universal categorical (geometric) quotient}と言う。
  \(X\to Y\)は、任意の平坦射\(Y'\to Y\)での基底変換\(X'\to Y'\)が
  categorical (geometric) quotientであるとき、
  \textbf{uniform categorical (geometric) quotient}と言う。
\end{defi}









\section{First Properties}




\begin{prop}[\textbf{Geom.\(\Rightarrow\)Cat.}]
  \(\sigma\)を\(G/S\)の\(X/S\)への作用とする。
  \(q:X\to Y\)がgeometric quotientであるとき、
  それはcategorical quotientである。
  さらに、\(q\)がuniversal geometric quotientであれば、
  それはuniversal categorical quotientである。
\end{prop}

\begin{proof}
  \(r:X\to Z\)が\(r \circ \sigma = r \circ p_2\)を満たすとする。
  \(Z\)のアフィン開集合\(W = \Spec(C)\subset Z\)を任意にとる。
  \(q\)は全射であり、\(r^{-1}(W)\subset X\)は\(G\)-invariantなので、
  ある部分集合\(V\subset Y\)が存在して (集合として) \(q^{-1}(V) = r^{-1}(W)\)となる。
  ここで\(W\subset Z\)は開なので、\(r^{-1}(W)\)も開である。
  \(q\)がsubmersiveであることから、よって\(V\)は開となる。
  以上より、開部分スキーム\(V\subset Y\)が存在して
  \(U \dfn q^{-1}(V) = r^{-1}(W)\)が成り立つことがわかった。

  次に環の図式
  \[
  \begin{CD}
    C @. \Gamma(V,\mcO_V) \\
    @V{r^{\#}|_W}VV @VV{q^{\#}|_V}V \\
    \Gamma(U,\mcO_U) @= \Gamma(U,\mcO_U)
  \end{CD}
  \]
  について考える。
  \(q\)はgeometric quotientなので、4つめの条件より、
  \(\im(q^{\#}|_V) = \Gamma(U,\mcO_U)^G\)が成り立つ。
  仮定より、\(r^{\#}|_W\)は\(G\)-不変部分を経由するので、
  以上より上の図式を可換にする射\(C\to \Gamma(V,\mcO_V)\)が一意的に存在する。
  これによって射\(V\to W\)を得る。

  最後に\(W\)をより小さいアフィン開集合\(W'\)へと制限することを考えると、
  上の\(C\to \Gamma(V,\mcO_V)\)の一意性は、
  \(V'\to W'\)が\(V\to W\)の制限として得られることを示している。
  以上より、これら\(W\)上で射が貼りあって\(Y\to W\)を得る。
\end{proof}


\begin{rem*}

\end{rem*}




\begin{rem*}
  このRemarkに登場するスキームはすべてネーターであるとする。
  \(S\)が正規であると仮定しなさい。
  このとき、Chevalleyの判定法により、
  \(G\)が\(S\)上普遍開であることは、
  \(G\to S\)が開写像であること、もしくは、
  \(G\to S\)のすべてのfiberの次元が等しいことと同値である。
  このことは、「\(G\to S\)が開写像であって
  \((Y,q)\)がgeometric quotientであれば、
  \(q\)が普遍開写像となる」
  ということをimplyする。
  それを見てみよう。

  \(q\)が開であることを示すために、
  \(U\subset X\)を任意の開集合とする。
  \(G\)は\(S\)上普遍開であるので、
  \(p_2:G\times X \to X\)は普遍開である。
  さらに\(\sigma\)を
  \[
  G\times X \xrightarrow{(p_1,\sigma)} G\times X \xrightarrow{p_2} X
  \]
  と分解すると、
  \((p_1,\sigma)\)は同型であるから、
  \(\sigma\)も普遍開であることが従う。
  \(p_2(G\times U)\subset X\)は開であり、
  \(q(U) = q(p_2(G\times U))\)となるが、
  \(p_2(G\times U)\)は\(G\)-invariantであるので、
  \(p_2(G\times U) = q^{-1}(q(U))\)が成り立つ。
  \(q\)はsubmersiveであるから、
  \(q^{-1}(q(U)) = p_2(G\times U)\)が開であることは
  \(q(U)\)が開であることを意味する。
\end{rem*}

\begin{rem*}
  上のRemarkに関連して、次元に関する別の帰結を述べる。
\end{rem*}

\begin{prop}
  \(X,Y\)を\(S\)-スキーム、
  \(q:X\to Y\)を\(S\)-スキームの射、
  \(G\)を\(S\)上の群スキームで
  \(\sigma:G\times_S X \to X\)により作用しているとする。
  以下が成り立つことを仮定する:
  \begin{enumerate}
    \item \(X,Y\)は正規既約ネータースキームであり、
    \(Y\)の生成点の商体は標数\(0\)である。
    \item \(G\)は\(S\)上有限型普遍開である。
    \item \(q\)は支配的な有限型の射であって、
    \(q\circ \sigma = q\circ p_2\)を満たす。
    \item 任意の代数閉体\(k\)と任意の\(k\)-valued point \(\Spec(k)\to Y\)に対し、
    \(\Spec(k)\)のfiberは高々一つの\(G_k\dfn G\times_S \Spec(k)\)-orbitからなる。
  \end{enumerate}
  このとき\(q\)は普遍開写像であり、
  \((\im(q),q)\)は\(X\)の\(G\)によるgeometric quotientである。
\end{prop}


\begin{proof}

\end{proof}






\end{document}
