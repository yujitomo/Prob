\documentclass[uplatex]{jsarticle}

\usepackage{amssymb}
\usepackage{amsmath}
\usepackage{mathrsfs}
\usepackage{amsfonts}
\usepackage{mathtools}

\usepackage{xcolor}
\usepackage[dvipdfmx]{graphicx}



\usepackage{ulem}

\usepackage{braket}

%%%%%ハイパーリンク
%\usepackage[colorlinks=true,urlcolor=blue!70!black,citecolor=blue!60!black,linkcolor=blue!60!black]{hyperref}
%\usepackage{aliascnt} %for creating different biblatex references for different theoremstyles
\usepackage[setpagesize=false,dvipdfmx]{hyperref}
\usepackage{aliascnt}
\hypersetup{
    colorlinks=true,
    citecolor=blue,
    linkcolor=blue,
    urlcolor=blue,
}

\renewcommand{\eqref}[1]{\textcolor{blue}{(\ref{#1})}}

%%%%%%ハイパーリンク


%%%%%図式
%\usepackage{tikz}%%%図
\usepackage{amscd}%%%簡単な図式

\usepackage{tikz}
\usepackage{tikz-cd} %commutative diagrams in TikZ
\usetikzlibrary{calc}
\usetikzlibrary{matrix,arrows}
\usetikzlibrary{decorations.markings}

%%%%%図式



%%%%%%%%%%%%定理環境%%%%%%%%%%%%
%%%%%%%%%%%%定理環境%%%%%%%%%%%%
%%%%%%%%%%%%定理環境%%%%%%%%%%%%

\usepackage{amsthm}

%%%%%%%%%%%%Plain型%%%%%%%%%%%%


%%%%%%%%%%%%definition型%%%%%%%%%%%%

\theoremstyle{definition}

\renewcommand{\sectionautorefname}{Section}

\newtheorem{thm}{Theorem}[section]
\newcommand{\thmautorefname}{Theorem}


\newaliascnt{prop}{thm}%%%カウンター「prop」の定義(thmと同じ)
\newtheorem{prop}[prop]{Proposition}
\aliascntresetthe{prop}
\newcommand{\propautorefname}{Proposition}%%%カウンター名propは「命題」で参照する

\newaliascnt{cor}{thm}
\newtheorem{cor}[cor]{Corollary}
\aliascntresetthe{cor}
\newcommand{\corautorefname}{Corollary}

\newaliascnt{lem}{thm}
\newtheorem{lem}[lem]{Lemma}
\aliascntresetthe{lem}
\newcommand{\lemautorefname}{Lemma}

%%%%%%%アルファベットで番号づける定理環境
\newtheorem{thmA}{Theorem}[section]
\newcommand{\thmAautorefname}{Theorem}
\renewcommand\thethmA{\Alph{thmA}}

\newtheorem{corA}{Theorem}[section]
\newcommand{\corAautorefname}{Corollary}
\renewcommand\thecorA{\Alph{corA}}

\newaliascnt{defi}{thm}
\newtheorem{defi}[defi]{Definition}
\aliascntresetthe{defi}
\newcommand{\defiautorefname}{Definition}

\newaliascnt{rem}{thm}
\newtheorem{rem}[rem]{Remark}
\aliascntresetthe{rem}
\newcommand{\remautorefname}{Remark}


\newaliascnt{exam}{thm}
\newtheorem{exam}[exam]{Examplpe}
\aliascntresetthe{exam}
\newcommand{\examautorefname}{Examplpe}

%%%%%%%番号づけない定理環境
\newtheorem*{exam*}{Example}
\newtheorem*{rrem*}{Remark}
\newtheorem*{defi*}{Definition}

%%%%%%%%%%%%定理環境%%%%%%%%%%%%
%%%%%%%%%%%%定理環境%%%%%%%%%%%%
%%%%%%%%%%%%定理環境%%%%%%%%%%%%





%%%%%箇条書き環境
\usepackage[]{enumitem}

\makeatletter
\AddEnumerateCounter{\fnsymbol}{\c@fnsymbol}{9}%%%%fnsymbolという文字をenumerate環境のパラメーターで使えるようにする。
\makeatother

\makeatletter
\renewcommand{\p@enumii}{}
\makeatother

\renewcommand{\theenumi}{(\roman{enumi})}%%%%%itemは(1),(2),(3)で番号付ける。
\renewcommand{\labelenumi}{\theenumi}

\renewcommand{\theenumii}{(\alph{enumii})}%%%%%itemは(1),(2),(3)で番号付ける。
\renewcommand{\labelenumii}{\theenumii}

\usepackage{moreenum}
%%%%%箇条書き環境



\usepackage{mandorasymb}
\usepackage{applekeys}
\renewcommand{\qedsymbol}{\pencilkey}
%\renewcommand{\qedsymbol}{\kinoposymbniko}




\usepackage{latexsym}
\DeclareMathOperator{\Hom}{Hom}
\DeclareMathOperator{\Isom}{Isom}
\DeclareMathOperator{\ISOM}{\mathbf{Isom}}
\DeclareMathOperator{\id}{\mathrm{id}}
\DeclareMathOperator{\im}{\mathrm{Im}}
\DeclareMathOperator{\Spec}{\mathrm{Spec}}
\newcommand{\Supp}{\mathrm{Supp}}
\DeclareMathOperator{\Aut}{\mathrm{Aut}}

\newcommand{\coker}{\mathrm{coker}}

\DeclareMathOperator{\Tor}{\mathrm{Tor}}
\DeclareMathOperator{\Ext}{\mathrm{Ext}}

\DeclareMathOperator{\colim}{\mathrm{colim}}
\DeclareMathOperator{\plim}{\mathrm{lim}}
\newcommand{\Lotimes}[1]{\mathop{\otimes^{\mathbf{L}}_{#1}}}
\newcommand{\bLotimes}[1]{\mathop{\bar{\otimes}^{\mathbf{L}}_{#1}}}
\DeclareMathOperator{\RHom}{\mathbf{R}Hom}
\DeclareMathOperator{\bRHom}{\underline{\mathbf{R}Hom}}
\DeclareMathOperator{\inHom}{\mathcal{H}om}
\newcommand{\Ob}{\mathrm{Ob}}
\newcommand{\FP}[1]{\mathsf{FP}_{/#1}}

\newcommand{\rsa}{\rightsquigarrow}
\renewcommand{\coprod}{\amalg}
\renewcommand{\emptyset}{\varnothing}
\newcommand{\ep}{\varepsilon}
\newcommand{\dg}{\mathrm{dg}}
\newcommand{\op}{\mathrm{op}}

\newcommand{\dfn}{:\overset{\mbox{{\scriptsize def}}}{=}}
\newcommand{\deff}{:\hspace{-3pt}\overset{\text{def}}{\iff}}

\newcommand{\univ}[1]{\mathbb{#1}}
\newcommand{\usm}{\(\univ{U}\)-small}
\newcommand{\vsm}{\(\univ{V}\)-small}
\newcommand{\unit}[1]{\mathbf{1}_{\mathcal{#1}}}

\newcommand{\Qcoh}{\mathsf{Qcoh}}
\newcommand{\Coh}{\mathsf{Coh}}
\newcommand{\Pic}{\mathrm{Pic}}
\newcommand{\Sym}{\mathrm{Sym}}
\newcommand{\Mod}{\mathsf{Mod}}


\newcommand{\A}{\mathbb{A}}
\newcommand{\C}{\mathbb{C}}
\renewcommand{\P}{\mathbb{P}}
\newcommand{\R}{\mathbb{R}}
\newcommand{\Q}{\mathbb{Q}}
\newcommand{\Z}{\mathbb{Z}}
\newcommand{\N}{\mathbb{N}}



\newcommand{\mcA}{\mathcal{A}}
\newcommand{\mcB}{\mathcal{B}}
\newcommand{\mcC}{\mathcal{C}}
\newcommand{\mcD}{\mathcal{D}}
\newcommand{\mcE}{\mathcal{E}}
\newcommand{\mcF}{\mathcal{F}}
\newcommand{\mcG}{\mathcal{G}}
\newcommand{\mcH}{\mathcal{H}}
\newcommand{\mcI}{\mathcal{I}}
\newcommand{\mcJ}{\mathcal{J}}
\newcommand{\mcK}{\mathcal{K}}
\newcommand{\mcL}{\mathcal{L}}
\newcommand{\mcM}{\mathcal{M}}
\newcommand{\mcN}{\mathcal{N}}
\newcommand{\mcO}{\mathcal{O}}
\newcommand{\mcP}{\mathcal{P}}
\newcommand{\mcQ}{\mathcal{Q}}
\newcommand{\mcR}{\mathcal{R}}
\newcommand{\mcS}{\mathcal{S}}
\newcommand{\mcT}{\mathcal{T}}
\newcommand{\mcU}{\mathcal{U}}
\newcommand{\mcV}{\mathcal{V}}
\newcommand{\mcW}{\mathcal{W}}
\newcommand{\mcX}{\mathcal{X}}
\newcommand{\mcY}{\mathcal{Y}}
\newcommand{\mcZ}{\mathcal{Z}}



\newcommand{\mfa}{\mathfrak{a}}
\newcommand{\mfb}{\mathfrak{b}}
\newcommand{\mfc}{\mathfrak{c}}
\newcommand{\mfd}{\mathfrak{d}}
\newcommand{\mfe}{\mathfrak{e}}
\newcommand{\mff}{\mathfrak{f}}
\newcommand{\mfg}{\mathfrak{g}}
\newcommand{\mfh}{\mathfrak{h}}
\newcommand{\mfi}{\mathfrak{i}}
\newcommand{\mfj}{\mathfrak{j}}
\newcommand{\mfk}{\mathfrak{k}}
\newcommand{\mfl}{\mathfrak{l}}
\newcommand{\mfm}{\mathfrak{m}}
\newcommand{\mfn}{\mathfrak{n}}
\newcommand{\mfo}{\mathfrak{o}}
\newcommand{\mfp}{\mathfrak{p}}
\newcommand{\mfq}{\mathfrak{q}}
\newcommand{\mfr}{\mathfrak{r}}
\newcommand{\mfs}{\mathfrak{s}}
\newcommand{\mft}{\mathfrak{t}}
\newcommand{\mfu}{\mathfrak{u}}
\newcommand{\mfv}{\mathfrak{v}}
\newcommand{\mfw}{\mathfrak{w}}
\newcommand{\mfx}{\mathfrak{x}}
\newcommand{\mfy}{\mathfrak{y}}
\newcommand{\mfz}{\mathfrak{z}}

\DeclareMathOperator{\OOO}{\mcO}

\newcommand{\OC}{{\OOO_C}}
\newcommand{\OD}{{\OOO_D}}
\renewcommand{\OE}{{\OOO_E}}
\newcommand{\OF}{{\OOO_F}}
\newcommand{\OH}{{\OOO_H}}
\newcommand{\OS}{{\OOO_S}}
\newcommand{\OT}{{\OOO_T}}
\newcommand{\OU}{{\OOO_U}}
\newcommand{\OV}{{\OOO_V}}
\newcommand{\OW}{{\OOO_W}}
\newcommand{\OX}{{\OOO_X}}
\newcommand{\OY}{{\OOO_Y}}
\newcommand{\OZ}{{\OOO_Z}}

\newcommand{\OO}[1]{\OOO_{#1}}



\title{スキーム論速成コース}

\author{ゆじ}

\begin{document}

\maketitle



\section{定義周辺}


\subsection{最低限の可換環論}


\subsubsection{中山の補題}




\begin{lem}
  \(M\)を有限生成\(A\)-加群とすると、
  \(M\)は極大部分加群を持つ、
  すなわち、\(M/N\)が非自明な部分加群を持たないような
  部分加群\(N\subset M\)が存在する。
  とくに、環\(A\)には極大イデアルが存在する。
\end{lem}

\begin{proof}
  生成元を\(m_1,\cdots, m_r\in M\)として
  部分加群の集合
  \[
  \left\{ N \subset M \middle| \exists i, m_i \not\in N\right\}
  \]
  にZornの補題を使えば示せる。
\end{proof}



\begin{thm}[中山の補題]
  \(A\)を局所環、\(k\)を\(A\)の剰余体、
  \(M\)を有限生成\(A\)-加群とする。
  このとき、\(M=0\)であるための必要十分条件は\(M\otimes_A k=0\)である。
\end{thm}

\begin{proof}
  必要性は明らかである。
  十分性は極大部分加群の存在より従う。
\end{proof}


\begin{rem}
  \(M\)が有限生成でない場合は反例がある。
  たとえば\(A\)がDVRで\(M=K\)が商体である場合、
  \(K\otimes_Ak=0\)である。
\end{rem}


\begin{cor}
  \(A\)を環、\(M\)を有限生成加群、
  \(\mfp\)を\(A\)の素イデアルとするとき、
  \(\mfp \in \Supp(M)\)であるための必要十分条件は
  \(M\otimes_A k(\mfp) \neq 0\)である。
\end{cor}






\subsubsection{平坦性}


\begin{defi}
  \(A\)を環とする。
  \begin{itemize}
    \item \(A\)-加群\(M\)が\textbf{平坦}であるとは、
    函手\((-)\otimes_A M\)が完全函手であることを意味する。
    \item 環の射\(A\to B\)が\textbf{平坦}であるとは、
    \(B\)が\(A\)-加群として平坦であることを意味する。
    \item 環の射\(A\to B\)が\textbf{忠実平坦}であるとは、
    \(0\)でない任意の\(A\)-加群\(M\neq 0\)に対し、
    \(M\otimes_AB \neq 0\)となることを意味する。
  \end{itemize}
\end{defi}


\begin{exam}
  \(A\)を環とする。
  \begin{itemize}
    \item
    元\(f\in A\)での局所化\(A\to A_f\)や
    素イデアル\(\mfp\subset A\)での局所化\(A\to A_{\mfp}\)は平坦である。
    \item
    \(0\)は平坦\(A\)-加群である。
    \item
    体の拡大は忠実平坦な環の射である。
  \end{itemize}
\end{exam}





\begin{lem}\label{lem: ff kihon}
  \(\varphi:A\to B\)を環の射とする。
  \begin{enumerate}
    \item \label{enumi: ff0}
    \(\varphi\)が忠実平坦であるとする。
    \(p:M\to N\)を\(A\)-加群の射とする。
    このとき、\(p\otimes \id: M\otimes_A B \to N\otimes_A B\)が単射 (resp. 全射) であれば、
    \(p\)も単射 (resp. 全射) となる。
    とくに、\(A\)-加群の複体が完全であることの必要十分条件は、
    \(B\)への基底変換のあとで完全となることである。
    \item \label{enumi: ff1}
    \(\varphi\)が忠実平坦であるとする。
    このとき、\(\varphi\)は単射である。
    \item \label{enumi: ff2}
    \(\varphi\)が平坦であるとする。
    このとき、\(\varphi\)が忠実平坦であるための必要十分条件は、
    \(\varphi\)が引き起こす射
    \(f:\Spec(B)\to \Spec(A)\)が全射となることである。
  \end{enumerate}
\end{lem}

\begin{proof}
  \ref{enumi: ff0}を示す。
  \(\varphi\)は平坦であるから、自然な射
  \(\ker(p)\otimes_A B \xrightarrow{\sim} \ker(p\otimes \id)\)
  (resp. \(\coker(p\otimes \id) \xrightarrow{\sim} \coker(p)\otimes_A B\))
  は同型射である。
  これと\(\varphi\)が忠実平坦であることから\ref{enumi: ff0}が従う。

  \ref{enumi: ff1}を示す。
  \(\varphi\otimes \id:B\to B\otimes_A B\)
  は掛け算写像\(B\otimes_A B\to B\)というレトラクトを持つので単射である。
  従って、\ref{enumi: ff0}より、
  \(\varphi\)は単射である。

  \ref{enumi: ff2}を示す。
  必要性を示す。\(\varphi\)が忠実平坦であると仮定する。
  \(\mfp\)を\(A\)の素イデアルとすると、
  \(k(\mfp)\neq 0\)なので、
  \(\varphi\)が忠実平坦であることから、
  \(k(\mfp)\otimes_A B \neq 0\)となる。
  これは\(f^{-1}(\mfp)\neq \emptyset\)、すなわち\(f\)の全射性を示している。
  以上で必要性の証明を完了する。
  十分性を示す。\(\varphi\)が平坦であり、さらに\(f\)が全射であると仮定する。
  \(M\neq 0\)を\(0\)でない\(A\)-加群とする。
  \(\varphi\)が忠実平坦であることを示すためには、
  \(M\otimes_A B\neq 0\)を示せばよい。
  \(M\neq 0\)であるから、\(0\)でない有限生成部分加群\(0\neq N \subset M\)が存在する。
  \(N\neq 0\)であるから、\(\Supp(N)\neq \emptyset\)である。
  さらに、\(f\)は全射であるから、
  点\(\mfq\in f^{-1}(\Supp(N))\subset \Spec(B)\)が存在する。
  \(\mfp \dfn f(\mfq) \in \Supp(N)\)と書く。
  \begin{itemize}
    \item \(\mfp\in \Supp(N)\)なので、中山の補題より、\(N\otimes_A k(\mfp)\neq 0\)である。
    \item 体の拡大\(k(\mfp)\subset k(\mfq)\)はいつでも忠実平坦なので、
    \[
    N\otimes_A B \otimes_B k(\mfq) \cong
    N\otimes_A k(\mfp) \otimes_{k(\mfp)} k(\mfq) \neq 0
    \]
    となる。
    \item とくに、\(N\otimes_AB\neq 0\)である。
  \end{itemize}
  \(\varphi\)は平坦なので、射
  \(0\neq N\otimes_A B\hookrightarrow M\otimes_A B\)は単射である。
  従って、\(M\otimes_A B \neq 0\)となる。
  以上ですべての主張の証明が完了した。
\end{proof}


\begin{exam}\label{exam: Zar cov ff}
  \(A\)を環、\(f_1,\cdots, f_r\in A\)を元とする。
  \(\bigcup_{i=1}^r D(f_i) = \Spec(A)\)であると仮定する。
  このとき、\(\Spec(\prod_{i=1}^r A_{f_i}) = \coprod_{i=1}^r \Spec(A_{f_i})\)であるので、
  \autoref{lem: ff kihon} \ref{enumi: ff2}より、自然な射
  \(A\to \prod_{i=1}^r A_{f_i}\)は忠実平坦である。
\end{exam}



\begin{cor}\label{cor: ff exact}
  \(\varphi: A\to B\)を忠実平坦な環の射とする。
  このとき、
  \[
  \begin{tikzpicture}[auto]
    \node (a) at (0, 0) {\(M\)};
    \node (b) at (5, 0) {\(M\otimes_A B\)};
    \node (c) at (10, 0) {\(M\otimes_A B\otimes_A B\)};
    \draw[->] (a) to node {\(\scriptstyle m\mapsto m\otimes 1\)} (b);
    \draw[->, transform canvas={yshift=2pt}]
    (b) to node {\(\scriptstyle m\otimes b\mapsto m\otimes b\otimes 1\)} (c);
    \draw[->, transform canvas={yshift=-2pt}]
    (b) to node[swap] {\(\scriptstyle m\otimes b\mapsto m\otimes 1\otimes b\)} (c);
  \end{tikzpicture}
  \]
  はイコライザーの図式である。
\end{cor}

\begin{proof}
  \autoref{lem: ff kihon} \ref{enumi: ff0}より、
  主張を示すためには、
  \[
  \begin{tikzpicture}[auto]
    \node (a) at (0, 0) {\(M\otimes_AB\)};
    \node (b) at (5, 0) {\(M\otimes_AB\otimes_AB\)};
    \node (c) at (12, 0) {\(M\otimes_AB\otimes_AB\otimes_AB\)};
    \draw[->] (a) to node {\(\scriptstyle m\otimes b\mapsto m\otimes 1\otimes b\)} (b);
    \draw[->, transform canvas={yshift=2pt}]
    (b) to node {\(\scriptstyle m\otimes b_1 \otimes b_2 \mapsto m\otimes b_1\otimes 1 \otimes b_2\)} (c);
    \draw[->, transform canvas={yshift=-2pt}]
    (b) to node[swap] {\(\scriptstyle m\otimes b_1 \otimes b_2 \mapsto m \otimes 1 \otimes b_1\otimes b_2\)} (c);
  \end{tikzpicture}
  \]
  がイコライザーの図式であることを示せば十分である。
  左側の射を\(f\)とおき、右上の射を\(f_1\)、右下の射を\(f_2\)とおく。
  掛け算射を\(\varphi :B \otimes_A B\to B\)とする。
  \(f\)は\(\id_M \otimes \varphi\)というレトラクトを持つので単射である。
  \(\psi \dfn \id_M \otimes \id_B \otimes \varphi\)とおくと、
  \begin{align*}
    \psi(f_1(m\otimes b_1 \otimes b_2))
    &= \psi(m\otimes b_1\otimes 1 \otimes b_2) = m\otimes b_1\otimes b_2, \\
    \psi(f_2(m\otimes b_1 \otimes b_2))
    &= \psi(m\otimes 1 \otimes b_1 \otimes b_2))
    = m\otimes 1 \otimes b_1b_2 = f(m\otimes b_1b_2)
  \end{align*}
  となるので、
  \(f_1(m\otimes b_1\otimes b_2) = f_2(m\otimes b_1\otimes b_2)\)であれば、
  \(m\otimes b_1\otimes b_2 = f(m\otimes b_1b_2)\)となる。
  以上より、上記の図式がイコライザーの図式であることが示された。
\end{proof}




\begin{cor}\label{cor: Zar cov exact}
  \(A\)を環、\(f_1,\cdots, f_r\in A\)を元とする。
  \(\Spec(A) = \bigcup_{i=1}^rD(f_i)\)と仮定する。
  このとき、
  \[
  \begin{tikzpicture}[auto]
    \node (a) at (0, 0) {\(M\)};
    \node (b) at (5, 0) {\(\prod_{i=1}^r M_{f_i}\)};
    \node (c) at (10, 0) {\(\prod_{i,j} M_{f_if_j}\)};
    \draw[->] (a) to node {\(\scriptstyle m\mapsto m/1\)} (b);
    \draw[->, transform canvas={yshift=2pt}]
    (b) to node {\(\scriptstyle m/f_i\mapsto mf_j/(f_if_j)\)} (c);
    \draw[->, transform canvas={yshift=-2pt}]
    (b) to node[swap] {\(\scriptstyle m/f_j\mapsto mf_i/(f_if_j)\)} (c);
  \end{tikzpicture}
  \]
  はイコライザーの図式である。
\end{cor}

\begin{proof}
  \autoref{exam: Zar cov ff}と\autoref{cor: ff exact}より従う。
\end{proof}




\section{スキームの定義と基本性質}



\subsection{スキームの定義}


\begin{lem}\label{lem: B-sh}
  \(X\)を位相空間、
  \(\mcB \dfn \{B_i\subset X\}_{i\in I}\)を (有限交差で閉じる) 開基とする。
  \(\mcB\)は包含関係によって圏とみなす。
  \(F:\mcB^{\op} \to \mathsf{Set}\)を函手とする。
  任意の\(B\in \mcB\)と\(\mcB\)の元による\(B\)の開被覆
  \(\{B_j\in \mcB\}_{j\in J}\)に対し、
  \[
  \begin{tikzpicture}[auto]
    \node (a) at (0, 0) {\(F(B)\)};
    \node (b) at (5, 0) {\(\prod_j F(B_j)\)};
    \node (c) at (10, 0) {\(\prod_{j_1,j_2} F(B_{j_1j_2})\)};
    \draw[->] (a) to (b);
    \draw[->, transform canvas={yshift=2pt}]
    (b) to (c);
    \draw[->, transform canvas={yshift=-2pt}]
    (b) to (c);
  \end{tikzpicture}
  \]
  がイコライザーの図式であるとする。
  このとき、任意の開集合\(U\subset X\)に対して、
  \(U\)に属する\(\mcB\)の元からなる\(\mcB\)の充満部分圏を\(\mcB|_U\)と表すとき、
  \[
  \tilde{F}(U) \dfn \lim_{B\in \mcB|_U}F(B)
  \]
  とおけば、\(\tilde{F}\)は\(X\)上の層となる。
\end{lem}


\begin{proof}
  開集合の包含関係\(U_1\supset U_2\)があれば、
  函手\(\mcB|_{U_2} \to \mcB|_{U_1}\)ができるので、
  これによって\(F\)は前層となる。
  極限どうしの順序交換によって層であることが確認できる。
\end{proof}


\begin{defi}
  \(A\)を環、\(M\)を\(A\)-加群とする。
  \(\mcB\dfn \{D(f)|f\in A\}\)と置く。
  \(f\in A\)に対して、
  \(S_f\dfn \bigcap_{\mfp\in D(f)}(A\setminus \mfp)\)とおくと、
  これは積閉集合である。
  また、\(S_f\)は開集合\(D(f)\)のみにより決定され、\(f\)のとり方によらない。
  \(F(D(f))\dfn (S_f)^{-1}M\)
  とするとき、
  \(F\)は\autoref{lem: B-sh}の仮定を満たし、
  \(\Spec(A)\)上の層を定める。
  とくに\(M=A\)の場合、
  \(\Spec(A)\)上の環の層が定まる。
  これを\textbf{構造層}といい、
  \(\mcO_{\Spec(A)}\)で表す。
  一般の\(M\)に対して上の手続きにより構成される層を\(\tilde{M}\)で表す。
  これは\(\mcO_{\Spec(A)}\)-加群である。
\end{defi}

\begin{rem}
  \begin{itemize}
    \item
    \((S_f)^{-1}M \cong M_f\)である。
    \item
    構成より、\(\tilde{M}(D(f)) \cong M_f\)である。
    とくに、\(\Gamma(\Spec(A),\tilde{M}) \cong M\)である。
    \item
    各点\(\mfp\in \Spec(A)\)に対して、stalkは
    \(\tilde{M}_{\mfp} \cong \colim_{\mfp\in D(f)} M_f \cong M_{\mfp}\)
    となる。
    特に、環つき空間\((\Spec(A),\mcO_{\Spec(A)})\)は局所環つき空間である。
  \end{itemize}
\end{rem}



\begin{defi}
  \((\Spec(A),\mcO_{\Spec(A)})\)と同型な局所環つき空間のことを
  \textbf{アフィンスキーム}という。
  局所環つき空間\((X,\mcO_X)\)が\textbf{スキーム}であるとは、
  ある開被覆\(X = \bigcup_i U_i\)が存在し、
  \((U_i,\mcO_X|_{U_i})\)がアフィンスキームとなることを言う。
  スキーム\(X\)上の\(\mcO_X\)-加群\(F\)が\textbf{準連接層}であるとは、
  各アフィン開集合\(U\subset X\)に対して、
  ある\(\mcO_X(U)\)-加群\(M\)が存在し、
  \(F|_U\cong \tilde{M}\)となることを言う。
  さらにこの\(M\)がいつも有限表示となるとき、
  \(F\)は\textbf{連接層}であると言う。
\end{defi}


\subsection{スキームの張り合わせ}












\end{document}
