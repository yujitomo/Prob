\ifcsname Chap\endcsname\else
\documentclass[uplatex,dvipdfmx]{jsarticle}
\newcommand{\StylePath}{\ifcsname AllKS\endcsname KS-Style/KS-Style.sty\else
\ifcsname Chap\endcsname ../KS-Style/KS-Style.sty\else
../../KS-Style/KS-Style.sty\fi\fi}
\input{\StylePath}

\KSset{1}{9}
\setcounter{section}{\value{KSS}-1}
\begin{document}
\maketitle
\HeaderCommentA
\section{\KSsection{section}}
\setcounter{prob}{\value{KSP}-1}
\fi


\begin{prob}\label{1.9}
  \(\mcC\)をアーベル圏とする。
  \[
  \begin{CD}
    @. X @> f >> Y @> g >> Z @>>> 0 \\
    @. @V \alpha VV @VV \beta V @VV \gamma V \\
    0 @>>> X' @> f' >> Y' @> g' >> Z' @.
  \end{CD}
  \]
  を\(\mcC\)の可換図式で、横向きが完全であるものとする。
  \begin{enumerate}
    \item \label{1.9.1}
    自然な射\(\varphi:\ker(\gamma) \to \coker(\alpha)\)が存在して、
    以下が完全となることを示せ:
    \[
    \ker(\alpha) \to \ker(\beta) \to \ker(\gamma) \xrightarrow{\varphi}
    \coker(\alpha) \to \coker(\beta) \to \coker(\gamma).
    \]
    \item \label{1.9.2}
    以下の図式が可換であることを示せ:
    \[
    \begin{CD}
      @. Y @> g >> Z \\
      @. @AAA @AAA \\
      Y @<<< \ker(\gamma\circ g) @>>> \ker(\gamma) \\
      @VVV @VVV @VV \varphi V \\
      Y' @< f' << X' @>>> \coker(\alpha).
    \end{CD}
    \]
  \end{enumerate}
\end{prob}


\begin{proof}
  \(\varphi\)の構成ができれば、
  \autoref{1.7}によって\ref{1.9.1}は\(\mcC=\Ab\)の場合に帰着され、
  この場合は図式追跡によって初等的に証明できる。
  従って、\ref{1.9.1}を示すためには\(\varphi\)を構成することが十分である。
  以下、\(\varphi\)の構成と\ref{1.9.2}の証明を同時に行う。

  核の普遍性により、以下の図式を可換にするような射
  \(\psi_1:\ker(\gamma \circ g) \to \ker(\gamma)\)が一意的に存在する:
  \[
  \begin{CD}
    0 @>>> \ker(\gamma\circ g) @>>> Y @> \gamma\circ g >> Z' \\
    @. @V \psi_1 VV @VV g V @| \\
    0 @>>> \ker(\gamma) @>>> Y' @> \gamma >> Z'.
  \end{CD}
  \]
  これは\ref{1.9.2}の図式の右上の四角形の可換性を示している。
  また、\autoref{1.8.2}より、\(\psi_1\)はエピである。
  核の普遍性により、以下の図式を可換にするような射
  \(\psi_2:\ker(\gamma\circ g) \to X'\)が一意的に存在する:
  \[
  \begin{CD}
    0 @>>> \ker(\gamma\circ g) @>>> Y @> \gamma\circ g >> Z' \\
    @. @V \psi_2 VV @VV \beta V @| \\
    0 @>>> X' @> f' >> Y' @> g' >> Z'.
  \end{CD}
  \]
  これは\ref{1.9.2}の図式の左下の四角形の可換性を示している。
  自然な射\(X'\to \coker(\alpha)\)と\(\psi_2\)の合成を
  \(\psi_3:\ker(\gamma\circ g) \to \coker(\alpha)\)と置く。
  \(\ker(\psi_1) \to \ker(\gamma\circ g) \xrightarrow{\psi_3} \coker(\alpha)\)
  の合成が\(0\)-射であることが証明できれば、
  \(\psi_1\)がエピであることから、
  \(\psi_3\)は\(\psi_1\)を一意的に経由して、
  図式
  \[
  \begin{CD}
    \ker(\gamma\circ g) @> \psi_1 >> \ker(\gamma) \\
    @V \psi_2 VV @VV \varphi V \\
    X' @>>> \coker(\alpha)
  \end{CD}
  \]
  を可換にする射\(\varphi\)の存在が従う
  (\ref{1.9.2}の図式の右下の四角形の可換性の証明と\(\varphi\)の構成が同時に終わる)。

  \(p:X\to \im(f)\)を自然なエピ射、
  \(j_1:\im(f)\cong \ker(g)\to Y\)を自然なモノ射とする。
  このとき\(f = j_1\circ p\)である。
  核の普遍性により引き起こされる一意的な射を
  \(\alpha' : \im(f)\cong \ker(g)\to X'\)と置く。
  \[
  f'\circ \alpha
  = \beta \circ f
  = \beta \circ j_1 \circ p
  = f'\circ \alpha' \circ p
  \]
  であることと\(f'\)がモノであることから、\(\alpha = \alpha'\circ p\)となる。
  \(q:X'\to \coker(\alpha)\)を自然な射とすると、
  \(q\circ \alpha' \circ p = q\circ \alpha = 0\)となるが、
  \(p\)がエピであることから、\(q\circ \alpha' = 0\)となる。
  \(T\dfn \ker(\psi_1)\)とおき、
  \(i:T\to \ker(\gamma\circ g)\)を自然な射、
  \(j_2:\ker(\gamma\circ g) \to Y\)を自然なモノ射とする。
  \(\psi_1\circ i = 0\)であるから、
  \(g\circ j_2\circ i = 0\)である。
  よって、核の普遍性により、一意的な射
  \(k:T\to \im(f)\)が存在して、
  \(j_1\circ k = j_2\circ i\)となる。
  以上より、
  \[
  f' \circ \psi_2 \circ i
  = \beta\circ j_2\circ i
  = \beta\circ j_1\circ k
  = f'\circ \alpha' \circ k
  \]
  となる。
  \(f'\)はモノなので\(\psi_2\circ i = \alpha'\circ k\)となる。
  従って、\(q\circ \psi_2\circ i = q\circ \alpha' \circ k = 0\)
  となって、示すべき等式を得る。
  以上で\autoref{1.9}の証明を完了する。
\end{proof}



\ifcsname Chap\endcsname\else
\printbibliography
\end{document}
\fi
