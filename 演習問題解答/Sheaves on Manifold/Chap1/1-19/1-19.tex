\ifcsname Chap\endcsname\else
\documentclass[uplatex,dvipdfmx]{jsarticle}
\newcommand{\StylePath}{\ifcsname AllKS\endcsname KS-Style/KS-Style.sty\else
\ifcsname Chap\endcsname ../KS-Style/KS-Style.sty\else
../../KS-Style/KS-Style.sty\fi\fi}
\input{\StylePath}

\KSset{1}{19}
\setcounter{section}{\value{KSS}-1}
\begin{document}
\maketitle
\HeaderCommentA
\section{\KSsection{section}}
\setcounter{prob}{\value{KSP}-1}
\fi


\begin{prob}\label{1.19}
  \(\mcC,\mcC'\)を二つのアーベル圏、
  \(F:\mcC \to \mcC'\)を左完全函手とする。
  さらに\(\mcI\subset \mcC\)を\(F\)-injectiveな部分圏とする。
  対象\(X\in \mcC\)が\textbf{\(F\)-acyclic}であるということを、
  任意の\(k\neq 0\)に対して\(R^kF(X) = 0\)となることとして定義する。
  \(\mcJ\subset \mcC\)を\(F\)-acyclicな対象からなる充満部分圏とする。
  \begin{enumerate}
    \item \label{1.19.1}
    \(\mcJ\)は\(F\)-injectiveであることを示せ。
    \item \label{1.19.2}
    任意の自然数\(n\geq 0\)に対して、
    以下の主張が同値であることを証明せよ:
    \begin{enumerate}
      \item \label{1.19.2.1}
      任意の\(k > n\)と任意の対象\(X\in \mcC\)に対して\(R^kF(X) = 0\)である。
      \item \label{1.19.2.2}
      任意の対象\(X\in \mcC\)に対して、
      完全列
      \[
      0 \to X \to X^0 \to \cdots \to X^n \to 0
      \]
      で各\(0\leq j\leq n\)に対して\(X^j\in \mcJ\)となるものが存在する。
      \item \label{1.19.2.3}
      \(X^0\to \cdots X^n \to 0\)が完全であり、
      任意の\(j < n\)に対して\(X^j\in \mcJ\)であるとき、
      \(X^n\in \mcJ\)である。
    \end{enumerate}
    これらの同値な条件のうちのどれか一つが成立するとき、
    \(F\)は\textbf{コホモロジー次元\(\leq n\)を持つ}と言う。
  \end{enumerate}
\end{prob}

\begin{proof}
  \ref{1.19.1}を示す。
  まず、\(F\)-injectiveな対象は\(F\)-acyclicなので
  (cf. 本文\cite[Proposition 1.8.3]{kashiwara2002sheaves}とその直前の記述)、
  \(\mcI\subset \mcJ\)であり、
  従って\(\mcJ\)は本文条件\cite[Definition 1.8.2 (i)]{kashiwara2002sheaves}を満たす。
  また、\(\mcJ\)に属する対象はすべて\(F\)-acyclicであるから、
  \(\mcJ\)が本文条件\cite[Definition 1.8.2 (ii)]{kashiwara2002sheaves}
  を満たすことは明らかである。
  \(X'\to X\)を\(\mcJ\)に属する対象の間のモノ射として
  \(X'' \dfn X/X'\)とすると、
  各\(i\geq 1\)に対して
  完全列\(R^iF(X)\to R^iF(X/X')\to R^{i+1}F(X')\)を得る。
  \(X,X'\)は\(F\)-acyclicであるから、
  \(R^iF(X)=0, R^{i+1}F(X')=0\)であり、
  従って\(R^iF(X/X')=0\)もわかる。
  これは\(X/X'\)が\(F\)-acyclicであることを示していて、
  \(X/X'\)は\(\mcJ\)に属する。
  よって\(\mcJ\)は本文条件\cite[Definition 1.8.2 (ii)]{kashiwara2002sheaves}を満たし、
  \(\mcJ\)は\(F\)-injectiveである。
  以上で\ref{1.19.1}の証明を完了する。

  \ref{1.19.2}を示す。
  \ref{1.19.2.1}\(\Leftrightarrow\)\ref{1.19.2.2}を示す。
  \ref{1.19.2.1}\(\Leftrightarrow\)\ref{1.19.2.2}
  を示すためには、対象\(X\in \mcC\)を固定して、
  次の二つの主張が同値であることを証明することが十分である:
  \begin{enumerate}[label=(\Alph*)]
    \item \label{1.19.2.p1}
    任意の\(k > n\)に対して\(R^kF(X) = 0\)である。
    \item \label{1.19.2.p2}
    完全列
    \[
    0 \to X \to X^0 \to \cdots \to X^n \to 0
    \]
    で各\(0\leq j\leq n\)に対して\(X^j\in \mcJ\)となるものが存在する。
  \end{enumerate}
  \(n\)に関する帰納法により
  \ref{1.19.2.p1}\(\Leftrightarrow\)\ref{1.19.2.p2}
  を示す。
  \(n=0\)に対して\ref{1.19.2.p1}が成り立つことは、
  \(X\)が\(F\)-acyclicであることと同値であり、
  さらにこれは\(n=0\)に対して\ref{1.19.2.p2}が成り立つことと同値である。
  よって\(n=0\)の場合は明らかに
  \ref{1.19.2.p1}\(\Leftrightarrow\)\ref{1.19.2.p2}
  が成り立つ。
  \(n\geq 1\)として、\(n\)より小さいすべての自然数に対して
  \ref{1.19.2.p1}\(\Leftrightarrow\)\ref{1.19.2.p2}
  が成り立つと仮定する。
  \(\mcJ\)は本文条件\cite[Definition 1.8.2 (i)]{kashiwara2002sheaves}を満たすので、
  モノ射\(d:X\to X^0\)が存在する。
  \(X^0\)は\(F\)-acyclicであるから、
  任意の\(k > n\)に対して
  \(R^{k-1}F(\coker(d)) \cong R^kF(X)\)となる。
  従ってとくに、\(X\)と\(n\)に対して\ref{1.19.2.p1}が成り立つことは、
  \(X=\coker(d)\)と\(n-1\)に対して\ref{1.19.2.p1}が成り立つことと同値である。
  帰納法の仮定により、これは
  \(X=\coker(d)\)と\(n-1\)に対して\ref{1.19.2.p2}が成り立つことと同値である。
  さらに\(\coker(d)\)に対する\ref{1.19.2.p2}の完全列を
  \(X^0\to \coker(d)\)と繋ぐことを考えれば、
  \(X=\coker(d)\)と\(n-1\)に対して\ref{1.19.2.p2}が成り立つことは
  \(X\)と\(n\)に対して\ref{1.19.2.p2}が成り立つことと同値である。
  以上で\ref{1.19.2.p1}\(\Leftrightarrow\)\ref{1.19.2.p2}
  の証明を完了し、
  従って
  \ref{1.19.2.1}\(\Rightarrow\)\ref{1.19.2.2}
  の証明を完了する。

  \ref{1.19.2.1}\(\Rightarrow\)\ref{1.19.2.3}
  を示すためには、
  各対象\(X\in \mcC\)に対して
  次の二つの主張が同値であることを証明することが十分である:
  \begin{enumerate}[label=(\Alph*),start=3]
    \item \label{1.19.2.q1}
    任意の\(k > n\)に対して\(R^kF(X) = 0\)である。
    \item \label{1.19.2.q3}
    完全列
    \[
    0 \to X \to X^0 \to \cdots \to X^n \to 0
    \]
    が条件「各\(j < n\)に対して\(X^j\in \mcJ\)である」を満たせば、
    \(X^n\in \mcJ\)となる。
  \end{enumerate}
  \(n=0\)に対して\ref{1.19.2.q1}が成り立つことは、
  \(X\)が\(F\)-acyclicであることと同値であり、
  これは\(n=0\)に対して\ref{1.19.2.q3}が成り立つことと同値である。
  よって\(n=0\)の場合は明らかに
  \ref{1.19.2.q1}\(\Leftrightarrow\)\ref{1.19.2.q3}
  が成り立つ。
  \(n\geq 1\)として、\(n\)より小さいすべての自然数に対して
  \ref{1.19.2.q1}\(\Leftrightarrow\)\ref{1.19.2.q3}
  が成り立つと仮定する。
  \[
  0 \to X \xrightarrow{d} X^0 \to \cdots \to X^n \to 0
  \]
  を条件「各\(j < n\)に対して\(X^j\in \mcJ\)である」を満たす完全列とする。
  \(X^0\)は\(F\)-acyclicであるから、
  任意の\(k > n\)に対して
  \(R^{k-1}F(\coker(d)) \cong R^kF(X)\)となる。
  よって、\(X\)と\(n\)に対して\ref{1.19.2.q1}が成り立つことは、
  \(X=\coker(d)\)と\(n-1\)に対して\ref{1.19.2.q1}が成り立つことと同値である。
  帰納法の仮定により、これは
  \(X=\coker(d)\)と\(n-1\)に対して\ref{1.19.2.q3}が成り立つことと同値である。
  一方これは明らかに
  \(X\)と\(n\)に対して\ref{1.19.2.q3}が成り立つことと同値であるから、
  よって\ref{1.19.2.q1}\(\Leftrightarrow\)\ref{1.19.2.q3}が従う。
  以上で\ref{1.19.2}の証明を完了し、
  \autoref{1.19}の解答を完了する。
\end{proof}



\ifcsname Chap\endcsname\else
\printbibliography
\end{document}
\fi
