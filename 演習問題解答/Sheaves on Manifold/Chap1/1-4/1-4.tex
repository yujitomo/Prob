\ifcsname Chap\endcsname\else
\documentclass[uplatex,dvipdfmx]{jsarticle}
\newcommand{\StylePath}{\ifcsname AllKS\endcsname KS-Style/KS-Style.sty\else
\ifcsname Chap\endcsname ../KS-Style/KS-Style.sty\else
../../KS-Style/KS-Style.sty\fi\fi}
\input{\StylePath}

\KSset{1}{4}
\setcounter{section}{\value{KSS}-1}
\begin{document}
\maketitle
\HeaderCommentA
\section{\KSsection{section}}
\setcounter{prob}{\value{KSP}-1}
\fi


\begin{prob}\label{1.4}
  \(\mcC\)を加法圏、
  \(X\rightarrow{i_1} Z \rightarrow{p_2} Y\)
  を射の列で、\(p_2\circ i_1 = 0\)を満たすとする。
  このとき、以下の条件が同値であることを示せ:
  \begin{enumerate}
    \item \label{1.4.1}
    任意の対象\(W\in \mcC\)に対して次の列は完全である:
    \[
    0 \to \Hom_{\mcC}(W,X) \to \Hom_{\mcC}(W,Z) \to \Hom_{\mcC}(W,Y) \to 0.
    \]
    \item \label{1.4.2}
    任意の対象\(W\in \mcC\)に対して次の列は完全である:
    \[
    0 \gets \Hom_{\mcC}(X,W) \gets \Hom_{\mcC}(Z,W) \gets \Hom_{\mcC}(Y,W) \gets 0.
    \]
    \item \label{1.4.3}
    射\(i_2:Y\to Z\)と\(p_1:Z\to X\)が存在して、
    \autoref{1.3}の条件を満たす。
  \end{enumerate}
  これらの条件が満たされるとき、
  \[ 0\to X\rightarrow{i_1} Z \rightarrow{p_2} Y \to 0 \]
  は\textbf{分裂する}と言い、
  \(X\)は\(Z\)の直和因子であると言う。
  \begin{enumerate}[label=(\fnsymbol*),start=2]
    \item \label{1.4.4}
    \(\mcC\)がアーベル圏または三角圏であるとする。
    \(i_1:X\to Z, p_1:Z\to X\)が
    \(p_1\circ i_1 = \id_X\)を満たすとき、
    \(X\)は\(Z\)の直和因子となることを示せ。
  \end{enumerate}
\end{prob}

\begin{proof}
  はじめに\ref{1.4.1}, \ref{1.4.2}, \ref{1.4.3}が同値であることを確認する。
  \ref{1.4.1}を仮定して\ref{1.4.3}を証明する。
  \(W=Y\)とすることで、
  \ref{1.4.1}で仮定されている完全性 (のうちの右側の全射性) より、
  \(p_2\circ i_2 = \id_Y\)となる射\(i_2:Y\to Z\)が存在することがわかる。
  \(W=Z\)とすれば、
  \(p_2\circ (\id_Z - i_2\circ p_2) = p_2 - p_2 = 0\)であることと、
  \ref{1.4.1}で仮定されている完全性 (のうちの真ん中の完全性) より、
  \(i_1\circ p_1 = \id_Z - i_2\circ p_2\)
  となる射\(p_1:Z\to X\)が存在することがわかる。
  \(W=X\)として\(p_1\circ i_1 :X\to X\)の行き先を見ると、それは
  \[
  i_1\circ p_1 \circ i_1
  = i_1 - i_2\circ p_2 \circ i_1 = i_1 = i_1 \circ \id_X
  \]
  であるので、
  \ref{1.4.1}で仮定されている完全性 (のうちの左側の単射性) より、
  \(p_1\circ i_1 = \id_X\)であることがわかる。
  \(W=Y\)として\(p_2\circ i_1 : Y\to X\)の行き先を見ると、それは
  \(i_1\circ p_2\circ i_1 = 0\)であるので、
  \ref{1.4.1}で仮定されている完全性 (のうちの左側の単射性) より、
  \(p_2\circ i_1 = 0\)であることがわかる。
  以上で\ref{1.4.1}から\ref{1.4.3}が帰結することがわかった。

  \ref{1.4.2}を仮定すれば\(\mcC^{\op}\)において
  \(Y\xrightarrow{p_2}Z \xrightarrow{i_1}X\)は条件\ref{1.4.1}を満たすので、
  すでに証明したことにより\(\mcC^{\op}\)においての条件\ref{1.4.3}が帰結するが、
  それは\(\mcC\)においての条件\ref{1.4.3}を意味している。
  以上で\ref{1.4.2}から\ref{1.4.3}が帰結することがわかった。

  \ref{1.4.3}を仮定すると、
  \(p_1:Z\to X\)と\(i_2:Y\to Z\)を用いて各\(W\)について
  函手的な直和分解
  \[
  \Hom_{\mcC}(W,Z) \cong \Hom_{\mcC}(W,X)\oplus \Hom_{\mcC}(W,Y)
  \]
  を得るので、これはどんな\(W\)についても
  \ref{1.4.1}の列が分裂完全列であることを意味し、
  \(\mcC^{\op}\)で考えることによって\ref{1.4.2}の列が分裂完全列であることもわかる。
  以上で\ref{1.4.1}, \ref{1.4.2}, \ref{1.4.3}が同値であることが示された。

  \(\mcC\)がアーベル圏であるときに\ref{1.4.4}を証明する。
  任意の\(W\)に対して
  \[
  0\to \Hom_{\mcC}(\coker(i_1),W) \to \Hom_{\mcC}(Z,W) \to \Hom_{\mcC}(X,W)
  \]
  は完全となるが、\(p_1\circ i_1\)であるから、
  \(i_1\)を合成する射\(\Hom_{\mcC}(X,W) \to \Hom_{\mcC}(Z,W)\)は
  一番右の射の分裂を与え、これによって条件\ref{1.4.2}が満たされる。
  以上で\(\mcC\)がアーベル圏である場合は証明された。

  \(\mcC\)が三角圏である場合に\ref{1.4.4}を証明する。
  \(i_1:X\to Z\)を完全三角
  \(X\xrightarrow{i_1} Z\xrightarrow{p_2} Y\to X[1]\)
  に延長すると、任意の\(W\)について長い完全列
  \[
  \begin{CD}
    @>>> \Hom_{\mcC}(W,X) @>>> \Hom_{\mcC}(W,Z) @>>> \Hom_{\mcC}(W,Y) \\
    @>>> \Hom_{\mcC}(W,X[1]) @>>> \Hom_{\mcC}(W,X[1]) @>>> \cdots
  \end{CD}
  \]
  を得る (\(\Hom_{\mcC}(W,-)\)はコホモロジー函手である:
  \cite[Proposition 1.5.3 (ii)]{kashiwara2002sheaves})。
  \(p_1:Z\to X\)を (シフトしてから) 合成することで、
  \(\Hom_{\mcC}(W,X[i]) \to \Hom_{\mcC}(W,Z[i])\)の単射性を得る。
  ここで上の長い列の完全性によって、
  \(\Hom_{\mcC}(W,Z) \to \Hom_{\mcC}(W,Y)\)の全射性が従う。
  これによって条件\ref{1.4.1}が満たされる。
  以上で\(\mcC\)が三角圏である場合も証明された。
  以上で\autoref{1.4}の解答を完了する。
\end{proof}


\ifcsname Chap\endcsname\else
\printbibliography
\end{document}
\fi
