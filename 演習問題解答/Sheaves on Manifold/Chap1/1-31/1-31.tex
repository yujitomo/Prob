\ifcsname Chap\endcsname\else
\documentclass[uplatex,dvipdfmx]{jsarticle}
\newcommand{\StylePath}{\ifcsname AllKS\endcsname KS-Style/KS-Style.sty\else
\ifcsname Chap\endcsname ../KS-Style/KS-Style.sty\else
../../KS-Style/KS-Style.sty\fi\fi}
\input{\StylePath}

\KSset{1}{31}
\setcounter{section}{\value{KSS}-1}
\begin{document}
\maketitle\HeaderCommentA
\section{\KSsection{section}}
\setcounter{prob}{\value{KSP}-1}
\fi


\begin{prob}\label{1.31}
  \(M\in \sfD^b(\Ab)\)とする。
  \begin{enumerate}
    \item \label{1.31.1}
    \(M^* = R\Hom(M,\Z) = 0\)であるとき、\(M=0\)であることを示せ。
    \item \label{1.31.2}
    \(M^*\in \sfD^b_f(\Ab)\)であるとき、\(M\in \sfD^b_f(\Ab)\)であることを示せ。
  \end{enumerate}
\end{prob}

\begin{rem*}
  \ref{1.31.2}は\(M^*\in \sfD^b(\Mod^f(\Z))\)という仮定のもとで
  \(M\in \sfD^b(\Mod^f(\Z))\)を示す問題であったが、
  \(\sfD^b(\Mod^f(\Z))\)は\(\sfD^b(\Ab)\)の部分圏として
  同型で閉じていないので、これはかなり微妙な問題設定であり
  (成り立たないかもしれない)、
  上記の設定がより適切であると思われる。
\end{rem*}

\begin{proof}
  \ref{1.31.1}を示す。
  まず\(M\in \Ab\)である場合に\ref{1.31.1}を証明する。
  \(R\Hom(M\Z)=0\)は\(\Hom(M,\Z) = \Ext^1(M,\Z) = 0\)を意味する。
  このときに\(M=0\)を示す。
  単射\(\Z/n\Z \to M\)を任意にとると
  \(0 = \Ext^1_{\Z}(M,\Z) \to \Ext^1_{\Z}(\Z/n\Z,\Z)\)は全射となるので
  \(\Z/n\Z \cong \Ext^1_{\Z}(\Z/n\Z,\Z) = 0\)となって\(n=1\)となる。
  従って\(M\)はねじれなし群である。
  \(n\neq 0,1,-1\)とすれば\(M/nM\)はねじれ群であるが、
  完全列\(0\to M\xrightarrow{n} M \to M/nM\to 0\)に
  函手\(R\Hom(-,\Z)\)を施すことによって\(R\Hom(M/nM,\Z) = 0\)が従い、
  よって\(M/nM\)はねじれなし群でもある。
  これは\(M/nM=0\)を意味し、従って\(M\)は可除である。
  \(M\)はねじれがないので\(M\to M\otimes \Q\)は単射であり、
  \(M\)は可除なのでこれは全射でもある。
  従って\(M\cong M\otimes \Q\)である。
  もし\(M\neq 0\)なら、\(M\)は\(\Q\)を直和因子として持つ。
  一方、完全列\(0\to \Z\to \Q\to \Q/\Z\to 0\)に
  \(\Hom(\Q/\Z,-)\)を適用することにより、
  \(\hat{\Z} \cong \End(\Q/\Z) \xrightarrow{\sim}\Ext^1(\Q/\Z,\Z)\)
  を得るので、
  同じ完全列に\(\Hom(-,\Z)\)を適用することで
  \(\Ext^1(\Q,\Z) \cong \coker(\Hom(\Z,\Z) \to
  \Ext^1(\Q/\Z,\Z)) \cong \hat{\Z}/\Z \neq 0\)
  が従い、これは\(\Ext^1(M,\Z) = 0\)に反する。
  以上で\(M\in \Ab\)の場合に示された。

  一般の\(M\in \sfD^b(\Ab)\)に対して\ref{1.31.1}を示す。
  \(H^n(M) \neq 0\)となる最大の\(n\)をとる。
  このとき
  \(\tau^{\leq n-1}(M) \to M \to H^n(M)[-n]\xrightarrow{+1}\)
  は完全三角である。
  \(R\Hom(-,\Z)\)を適用してコホモロジーをとることで、
  アーベル群の完全列
  \[
  \begin{CD}
    0 @>>> \Hom(H^n(M),\Z)
    @>>> H^n(R\Hom(M,\Z)) @>>> H^n(R\Hom(\tau^{\leq n-1}(M),\Z)) \\
    @>>> \Ext^1(H^n(M),\Z) @>>> H^{n+1}(R\Hom(M,\Z)) @>>> \cdots
  \end{CD}
  \]
  を得る。
  ここで、\(R\Hom(M,\Z)=0\)であるから、
  \(H^n(R\Hom(M,\Z)) = 0, H^{n+1}(R\Hom(M,\Z)) = 0\)が成り立つ。
  さらに、\autoref{1.21}を\(\tau^{\leq n-1}(M)\)と\(R\Hom(-,\Z)\)に対して適用することによって、
  \(H^n(R\Hom(\tau^{\leq n-1}(M),\Z)) = 0\)であることが従う。
  従って\(\Hom(H^n(M),\Z)=\Ext^1(H^n(M),\Z)=0\)が成り立つ。
  すでに示している\(M\in \Ab\)の場合により\(H^n(M)=0\)が従い、
  これは\(H^n(M)\neq 0\)に矛盾する。
  以上で\ref{1.31.1}の証明を完了する。

  \ref{1.31.2}を示す。
  \ref{1.31.1}の証明と同様に、
  \(H^n(M) \neq 0\)となる最大の\(n\)をとり、
  アーベル群の完全列
  \[
  \begin{CD}
    0 @>>> \Hom(H^n(M),\Z) @>>> H^n(R\Hom(M,\Z))
    @>>> H^n(R\Hom(\tau^{\leq n-1}(M),\Z)) \\
    @>>> \Ext^1(H^n(M),\Z) @>>> H^{n+1}(R\Hom(M,\Z)) @>>>
    H^{n+1}(R\Hom(\tau^{\leq n-1}(M),\Z)) \\
    @>>> 0 @. @.
  \end{CD}
  \]
  について考える
  (\(\Ext^2(H^n(M),\Z)=0\)であることに注意)。
  \autoref{1.21}を\(\tau^{\leq n-1}(M)\)と\(R\Hom(-,\Z)\)に適用することにより、
  \(H^n(R\Hom(\tau^{\leq n-1}(M),\Z)) = 0\)である。
  また、\(R\Hom(M,\Z)\in \sfD_f^b(\Mod(\Z))\)であるので、
  \(H^n(R\Hom(M,\Z)), H^{n+1}(R\Hom(M,\Z))\in \Mod^f(\Z)\)である。
  従って、
  \[\Hom(H^n(M),\Z), \ \Ext^1(H^n(M),\Z), \
  H^{n+1}(R\Hom(\tau^{\leq n-1}(M),\Z)) \ \in \Mod^f(\Z)\]
  である。
  さらに、\(n\)より大きい部分のコホモロジーを見れば、
  \[H^m(R\Hom(\tau^{\leq n-1}(M),\Z))\cong H^m(R\Hom(M,\Z)), \ (\forall m > n+1)\]
  であるので、\(R\Hom(\tau^{\leq n-1}(M),\Z)\in \Mod^f(\Z)\)が従う。
  以上より、帰納的に、
  \ref{1.31.2}を示すためには、
  アーベル群\(M\)が\(\Hom(M,\Z),\Ext^1(M,\Z)\in \Mod^f(\Z)\)を満たすとき
  \(M\in \Mod^f(\Z)\)であることを示すことが十分である。

  \(M\)をアーベル群であって
  \(\Hom(M,\Z)\)と\(\Ext^1(M,\Z)\)がどちらも有限生成であると仮定する。
  ねじれ部分を\(T(M)\subset M\)として、\(F(M)\dfn M/T(M)\)とおく。
  完全列
  \(0\to T(M)\to M\to F(M)\to 0\)に
  \(\Hom(-,\Z)\)を適用することにより、
  全射\(\Ext^1(M,\Z)\to \Ext^1(T(M),\Z)\)を得る。
  従って、\(\Ext^1(T(M),\Z)\)は有限生成アーベル群である。
  完全列\(0\to \Z\to \Q\to \Q/\Z\to 0\)に
  \(\Hom(T(M),-)\)を適用することにより、自然な同型
  \(\Hom(T(M),\Q/\Z)\xrightarrow{\sim}\Ext^1(T(M),\Z)\)を得る。
  \(T(M)\)に離散位相を入れて\(\Q/\Z\)に\(\R/\Z\)の双対位相を入れることにより、
  \(\Hom(T(M),\Q/\Z) = \Hom_{\text{cont.}}(T(M),\Q/\Z)\)を連続準同型のなす位相群とみなすと、
  \(\Hom_{\text{cont.}}(T(M),\Q/\Z)\)は副有限アーベル群である。
  とくにコンパクトハウスドルフである。
  一方、\(\Ext^1(T(M),\Z)\cong \Hom(T(M),\Q/\Z)\)はアーベル群として有限生成であるので、
  \(\Hom_{\text{cont.}}(T(M),\Q/\Z)\)は有限アーベル群であることが従う。
  Pontryagin双対より、
  \(T(M)\cong \Hom_{\text{cont.}}(\Hom_{\text{cont.}}(T(M),\Q/\Z),\Q/\Z)
  = \Hom(\Hom(T(M),\Q/\Z),\Q/\Z)\)
  が成り立つ。
  以上より、\(T(M)\)は有限生成ねじれアーベル群である。

  \(n\)倍写像\(n:F(M)\to F(M)\)を考えると、
  \(F(M)\)はねじれなし群なので、これは単射である。
  従って、\(n\)倍写像
  \[
  \Ext^1(F(M),\Z) \xrightarrow{n}\Ext^1(F(M),\Z)
  \]
  は全射であり、
  \(\Ext^1(F(M),\Z)\)が可除群であることが従う。
  \(n\)-倍写像\(n:M\to M\)を考えると、完全列
  \[
  0\to \Hom(M/nM,\Z) \to \Hom(M,\Z) \xrightarrow{n} \Hom(M,\Z)
  \]
  を得る。
  \(M/nM\)はねじれ群なので\(\Hom(M/nM,\Z)=0\)であり、
  従って\(\Hom(M,\Z)\)はねじれなし群である。
  仮定より、\(\Hom(M,\Z)\)は有限生成なので、
  従って自由アーベル群である。
  ランクを\(r=\mathrm{rank}(\Hom(M,\Z))\)と置く。
  \(r\)に関する帰納法により\(M\)の有限生成性を証明する。

  まず\(r=0\)の場合について考える。
  このとき、\(\Hom(M,\Z)=0\)であり、\(\Ext^1(M,\Z)\)は有限生成である。
  完全列
  \[0\to T(M) \to M\to F(M)\to 0\]
  により得られる完全列
  \[
  \begin{CD}
    0 @>>> \Hom(F(M),\Z) @>>> \Hom(M,\Z) @>>> \Hom(T(M),\Z) @. \\
    @>>> \Ext^1(F(M),\Z) @>>> \Ext^1(M,\Z) @>>> \Ext^1(T(M),\Z) @>>> 0
  \end{CD}
  \]
  について考える。
  \(T(M)\)はねじれ群なので、\(\Hom(T(M),\Z)=0\)が成り立つ。
  よって\(\Ext^1(F(M),\Z)\to \Ext^1(M,\Z)\)は単射である。
  \(\Ext^1(M,\Z)\)は有限生成なので、\(\Ext^1(F(M),\Z)\)も有限生成である。
  一方、\(\Ext^1(F(M),\Z)\)は可除群なので、従って\(\Ext^1(F(M),\Z) = 0\)が成り立つ。
  さらに、\(r=0\)であるという仮定より、\(\Hom(M,\Z)=0\)であるので、
  \(\Hom(F(M),\Z)=0\)が成り立つ。
  ここで\ref{1.31.1}より、\(F(M)=0\)が従う。
  よって\(T(M)\xrightarrow{\sim}M\)は同型射であり、
  既に示した\(T(M)\)の有限生成性より、\(M\)も有限生成である。
  以上で\(r=0\)の場合の証明を完了する。

  \(r>0\)とする。
  \(\Hom(M,\Z)\)のランクが\(r-1\)以下であるような任意の\(M\)について
  主張が成り立つと仮定する。
  この仮定のもとで、
  \(\Hom(M,\Z)\)のランクが\(r\)であるような任意の\(M\)に対して主張を示す。
  \(\Hom(M,\Z)\)のランクが\(r\)であるとする。
  \(\Hom(M,\Z)\neq 0\)であるので、
  \(0\)でない射\(f:M\to \Z\)が存在する.
  \(f\neq 0\)なので、ある\(m\in M\)が存在して\(f(m)\neq 0\)が成り立つ。
  ここで\(1\mapsto m\)により定義される射
  \(\Z\to M\)を考えると、
  任意の\(0\)でない\(n\in \Z\)に対して\(f(nm)=nf(m)\neq 0\)であることから、
  \(\Z\to M\)は単射である。
  この単射の余核を\(M_1\)として、完全列
  \[
  \begin{CD}
    0 @>>> \Z @>{1\mapsto m}>> M @>>> M_1 @>>> 0
  \end{CD}
  \]
  により得られる完全列
  \[
  \begin{CD}
    0 @>>> \Hom(M_1,\Z) @>>> \Hom(M,\Z) @>{f\mapsto f(m)}>> \Hom(\Z,\Z) \\
    @>>> \Ext^1(M_1,\Z) @>>> \Ext^1(M,\Z) @>>> 0
  \end{CD}
  \]
  を考える。
  \(\Hom(M,\Z)\cong \Z^r\)であるので、
  \(\Hom(M_1,\Z)\)はねじれなしであり、
  従って自由アーベル群である。
  また、\(f(m)\neq 0\)であるので、
  \(\Hom(M_1,\Z)\)のランクは\(r-1\)以下である。
  さらに、\(\Ext^1(M,\Z)\)は有限生成であるので、
  \(\Ext^1(M_1,\Z)\)も有限生成である。
  ここで帰納法の仮定より、
  \(M_1\)が有限生成であることが従う。
  よって\(M\)も有限生成である。
  以上で\ref{1.31.2}の証明を完了し
  \autoref{1.31}の解答を完了する。
\end{proof}



\ifcsname Chap\endcsname\else
\printbibliography
\end{document}
\fi
