\ifcsname Chap\endcsname\else
\documentclass[uplatex,dvipdfmx]{jsarticle}
\newcommand{\StylePath}{\ifcsname AllKS\endcsname KS-Style/KS-Style.sty\else
\ifcsname Chap\endcsname ../KS-Style/KS-Style.sty\else
../../KS-Style/KS-Style.sty\fi\fi}
\input{\StylePath}

\KSset{1}{28}
\setcounter{section}{\value{KSS}-1}
\begin{document}
\maketitle\HeaderCommentA
\section{\KSsection{section}}
\setcounter{prob}{\value{KSP}-1}
\fi


\begin{prob}\label{1.28}
  \(A\)を環とする。
  以下の条件が同値であることを証明せよ:
  \begin{enumerate}
    \item \label{1.28.1}
    \(\Mod(A)\)はホモロジー次元\(\leq n\)を持つ。
    \item \label{1.28.2}
    任意の左\(A\)-加群\(M\)は長さ\(n\)以下の入射分解を持つ。
    \item \label{1.28.3}
    任意の左\(A\)-加群\(M\)は長さ\(n\)以下の射影分解を持つ。
  \end{enumerate}
  \(\Mod(A)\)のホモロジー次元か\(\Mod(A^{\op})\)のホモロジー次元のうち
  大きい方を\(A\)の\textbf{大域ホモロジー次元}
  (global homological dimension) と言い、
  \(\gld(A)\)と表す。
\end{prob}

\begin{proof}
  \KSIffAutoref{1.17.1}{1.17.2}より
  \ref{1.28.1} \(\iff\) \ref{1.28.2}であることが従う。
  さらに\(\mcC\)のホモロジー次元は定義より\(\mcC^{\op}\)のホモロジー次元と等しいので、
  \(\Mod(A)^{\op}\)で考えると、再び
  \KSIffAutoref{1.17.1}{1.17.2}より
  \ref{1.28.1} \(\iff\) \ref{1.28.3}であることが従う。
  以上で\autoref{1.28}の解答を完了する。
\end{proof}


\ifcsname Chap\endcsname\else
\printbibliography
\end{document}
\fi
