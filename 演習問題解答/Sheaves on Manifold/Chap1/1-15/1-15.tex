\ifcsname Chap\endcsname\else
\documentclass[uplatex,dvipdfmx]{jsarticle}
\newcommand{\StylePath}{\ifcsname AllKS\endcsname KS-Style/KS-Style.sty\else
\ifcsname Chap\endcsname ../KS-Style/KS-Style.sty\else
../../KS-Style/KS-Style.sty\fi\fi}
\input{\StylePath}

\KSset{1}{15}
\setcounter{section}{\value{KSS}-1}
\begin{document}
\maketitle
\HeaderCommentA
\section{\KSsection{section}}
\setcounter{prob}{\value{KSP}-1}
\fi



\begin{prob}\label{1.15}
  \(\mcC\)を圏とする。
  \(c\in \mcC\)に対して、
  \(h_c^{\mcC} \dfn \Hom_{\mcC}(-,c)\)を
  米田埋め込み\(\mcC\to \sfSet^{\mcC^{\op}}\)による
  \(c\in \mcC\)の像とする
  (\(\mcC\)が明らかな場合は上付き添字の\(\mcC\)を省略してたんに\(h_c\)と表す)。
  \(\Ind(\mcC)\)を\(\sfSet^{\mcC^{\op}}\)の充満部分圏であって、
  ある filtered diagram \(F:\mcI\to \mcC\)に対する
  \(\colim_{i\in \mcI} h_{F(i)}\)と同型な対象たちからなるものとする。

  \(\mcC\)をさらにアーベル圏であるとして、
  \(S_X\)を over category \(\mcC_{X/}\)の充満部分圏であって、
  擬同型\(X\to X'\)たちからなるものとする。
  \begin{enumerate}
    \item \label{1.15.1}
    \(\sigma(X) \dfn \colim_{X'\in S_X}h_{X'}\)
    によって函手
    \(\sigma:\sfD^+(\mcC)\to \Ind(\sfK^+(\mcC))\)
    がwell-definedに定まることを示し、
    \(\sigma\)が忠実充満であることを示せ。
    \item \label{1.15.2}
    \(F:\mcC\to \mcC'\)をアーベル圏の間の左完全函手とする。
    \(T(X)\dfn \colim_{X'\in S_X}h_{F(X')}^{\mcC'}\)
    と定める。
    これによって函手\(T:\sfD^+(\mcC)\to \Ind(\sfK^+(\mcC'))\)
    がwell-definedに定まることを示せ。
    \(F\)が\(X\in \sfD^+(\mcC)\)で\textbf{derivable}であるということを、
    ある対象\(Y\in \sfD^+(\mcC')\)が存在して
    \(T(X) \cong \sigma(Y)\)となることとして定義する。
    このような\(Y\)が (up to isomで) 一意的であることを示せ。
    また、\(F\)がすべての\(X\in \sfD^+(\mcC)\)でderivableであるときに、
    函手\(RF:\sfD^+(\mcC)\to \sfD^+(\mcC')\)で
    \(\sigma\circ RF \cong T\)となるものが (up to isomで) 一意的に存在することを示せ
    (すなわち、\(F\)は右導来函手\(RF\)をadmitsする)。
  \end{enumerate}
\end{prob}

\begin{proof}
  \ref{1.15.1}の函手\(\sigma\)のwell-defined性は
  \ref{1.15.2}の函手\(T\)のwell-defined性の特別な場合 (\(F=\id_{\mcC}\)の場合) であるので、
  まず\ref{1.15.2}の函手\(T\)がwell-definedに定まることを示す。
  \(T\)は函手\(\sfK^+(\mcC)\to \Ind(\sfK^+(\mcC'))\)
  としてはwell-definedに定まっている。
  \(X\in \sfK^+(\mcC)\)を\(0\)と擬同型な対象とする。
  このとき\(0\)-射\(X\to 0\)は圏\(S_X\)の終対象であるので、
  \[T(X) = \colim_{X'\in S_X}h_{F(X)} = h_{F(0)} = h_0 \cong 0\]
  となる。
  よって\(\Ind(\sfK^+(\mcC'))\)において\(T(X)\cong 0\)である。
  従って、本文\cite[Proposition1.6.9 (iii)]{kashiwara2002sheaves}より、
  函手\(T:\sfD^+(\mcC)\to \Ind(\sfK^+(\mcC'))\)がwell-definedに定まる。
  以上で\(T\)が (よって、\(\sigma\)も) well-definedに定まることがわかった。

  函手\(\sigma\)が忠実であることを示す。
  \(X,Y\)を\(\sfD^+(\mcC)\)の対象、
  \(f:X\to Y\)を\(\sfD^+(\mcC)\)の射であって、
  \(\sigma(f) = 0\)であるとする。
  \(f\)は\(\sfK^+(\mcC)\)の図式
  \(X\xrightarrow{f'} Y' \xleftarrow{t} Y\)によって代表される。
  ここで\(t\)は擬同型である。
  \(\sigma(f) = 0\)であることと、
  \(\sigma(t)\)が同型射であることから、
  \(\sigma(f')\)は\(0\)-射である。
  \(\id_X\in h_X(X)\)により代表される元
  \([\id_X]\in \sigma(X)(X) = \colim_{X'\in S_X}h_{X'}(X)\)の
  \(\sigma(f')(X): \sigma(X)(X)\to \sigma(Y')(X)\)での行き先は
  \(f':X\to Y'\)により代表される元
  \([f']\in \sigma(Y')(X) = \colim_{Y''\in S_{Y'}}h_{Y'}(X)\)
  であるが、\(\sigma(f') = 0\)であるから、
  \([f']=0\)である。
  これは、ある\([t':Y'\to Y'']\in S_{Y'}\)が存在して
  \(t'\circ f' = 0\)となることを意味する。
  さらに\(t'\circ f' = 0\)は\(f'\)が\(\sfD^+(\mcC)\)において\(0\)-射であることを意味する。
  よって\(f\)は\(\sfD^+(\mcC)\)において\(0\)-射であることが従う。
  以上より\(\sigma\)は忠実である。

  函手\(\sigma\)が充満であることを示す。
  \(f:\sigma(X)\to \sigma(Y)\)を\(\Ind(\sfK^+(\mcC))\)の射とする。
  \(\id_X:X\to X\)で代表される元
  \([\id_X]\in \sigma(X)(X) = \colim_{X'\in S_X}(h_{X'}(X))\)の
  \(f(X):\sigma(X)(X)\to \sigma(Y)(X)\)での行き先を
  \([f]\in \sigma(Y)(X) = \colim_{Y'\in S_Y}(h_{Y'}(X))\)と置く。
  \(S_Y\)はfilteredであるから、
  ある\([t:Y\to Y']\in S_Y\)とある射\(f':X\to Y'\)が存在して、
  \([f]\)は\(f'\)によって代表される。
  \(\sfK^+(\mcC)\)の図式\(X\xrightarrow{f'}Y' \xleftarrow{t} Y\)
  によって代表される\(\sfD^+(\mcC)\)の射を\(g\)と置くと、
  \(f'\)が\([f]\)を代表することから、
  \(\sigma(g)([\id_X]) = [f]\in \sigma(Y)(X)\)がわかる。
  これは\(\sigma(g) = f\)を意味する。
  以上より\(\sigma\)は充満であり、
  \ref{1.15.1}の証明を完了する。

  \ref{1.15.2}を証明する。
  \(T\)がwell-definedに定義されることは既に示している。
  \(Y\)の (up to isomでの) 一意性は\(\sigma\)が忠実であることから従う。
  すべての\(X\in \sfD^+(\mcC)\)で\(F\)がderivableであれば、
  \(F:\sfD^+(\mcC)\to \Ind(\sfK^+(\mcC'))\)は
  \(\sigma:\sfD^+(\mcC')\to \Ind(\sfK^+(\mcC'))\)の本質的像を一意的に経由するため、
  \(\sigma\)が忠実充満であることから、
  右導来函手\(RF:\sfD^+(\mcC)\to \sfD^+(\mcC')\)であって
  \(\sigma\circ RF \cong T\)となるものが (up to isomで) 一意的に存在する。
  以上で\autoref{1.15}の解答を完了する。
\end{proof}







\ifcsname Chap\endcsname\else
\printbibliography
\end{document}
\fi
