\ifcsname Chap\endcsname\else
\documentclass[uplatex,dvipdfmx]{jsarticle}
\newcommand{\StylePath}{\ifcsname AllKS\endcsname KS-Style/KS-Style.sty\else
\ifcsname Chap\endcsname ../KS-Style/KS-Style.sty\else
../../KS-Style/KS-Style.sty\fi\fi}
\input{\StylePath}

\KSset{1}{34}
\setcounter{section}{\value{KSS}-1}
\begin{document}
\maketitle\HeaderCommentA
\section{\KSsection{section}}
\setcounter{prob}{\value{KSP}-1}
\fi


\begin{prob}\label{1.34}
  \(k\)を体、\(X\in \sfD^b_f(\Mod(k))\)とする。
  \[
  b_i(X) \dfn \dim(H^i(X)), \ \
  b_i^*(X) \dfn (-1)^i \sum_{j\leq i}(-1)^jb_j(X)
  \]
  とおく。
  \(Y\to X\to Z\xrightarrow{+1}\)を\(\sfD^b_f(\Mod(k))\)の完全三角とする。
  以下の式を示せ (\(\chi(X)\)の定義については\autoref{1.32.4}を参照):
  \begin{align*}
    \chi(X) &= \chi(Y) + \chi(Z), \\
    b_i^*(X) &\leq b_i^*(Y) + b_i^*(Z).
  \end{align*}
\end{prob}

\begin{proof}
  一つ目の等式は
  \KSDoubleAutoref{1.32.3}{1.32.4}
  より直ちに従う。
  二つ目の不等式を示す。
  コホモロジーをとると、長完全列
  \[
  \begin{CD}
    @> \delta^{i-1} >> H^{i-1}(Y) @>>> H^{i-1}(X) @>>> H^{i-1}(Z) \\
    @> \delta^i >> H^i(Y) @>>> H^i(X) @>>> H^i(Z) \\
    @> \delta^{i+1} >> \cdots @. @.
  \end{CD}
  \]
  を得る。
  従って、とくに
  \begin{align*}
    0 &\leq \dim(\im(\delta^{i+1})) \\
    &= b_i(Z) - b_i(X) + b_i(Y) - b_{i-1}(Z) + \cdots \\
    &= \sum_{j\leq i}(-1)^{i-j}b_j(Z) - \sum_{j\leq i}(-1)^{i-j}b_j(X)
    + \sum_{j\leq i}(-1)^{i-j}b_j(Y) \\
    &= b_i^*(Z) - b_i^*(X) + b_i^*(Y)
  \end{align*}
  を得る。
  よって二つ目の不等式が従う。
  以上で\autoref{1.34}の解答を完了する。
\end{proof}





\ifcsname Chap\endcsname\else
\printbibliography
\end{document}
\fi
