\ifcsname Chap\endcsname\else
\documentclass[uplatex,dvipdfmx]{jsarticle}
\newcommand{\StylePath}{\ifcsname AllKS\endcsname KS-Style/KS-Style.sty\else
\ifcsname Chap\endcsname ../KS-Style/KS-Style.sty\else
../../KS-Style/KS-Style.sty\fi\fi}
\input{\StylePath}

\KSset{1}{18}
\setcounter{section}{\value{KSS}-1}
\begin{document}
\maketitle
\HeaderCommentA
\section{\KSsection{section}}
\setcounter{prob}{\value{KSP}-1}
\fi


\begin{prob}\label{1.18}
  \(\mcC\)を\(\hd(\mcC) \leq 1\)のアーベル圏とする。
  \(X\in \sfD^b(\mcC)\)を複体とするとき、
  \(\sfD^b(\mcC)\)で
  \[
  X\cong \bigoplus_{k\in \Z}H^k(X)[-k]
  \]
  となることを示せ。
\end{prob}

\begin{proof}
  シフトすることで、\(X^i = 0 ,(\forall i < 0)\)と仮定しても一般性を失わない。
  \(X^n \neq 0\)となる最大の\(n\)に関する帰納法で証明する。
  \(n=0\)であれば主張は自明であるので、
  ある\(n=k\)に対して主張が成立するときに、
  \(n=k+1\)の場合にも成立することを証明する。
  帰納法の仮定より、
  \[
  \tau^{\leq n-1}(X)\cong \bigoplus_{k\in \Z}H^k(\tau^{\leq n-1}(X))[-k]
  \cong \bigoplus_{k\leq n-1}H^k(X)[-k]
  \]
  である。
  従って、所望の同型を証明するためには、
  \(n=1\)の場合、さらに\(d_X^0:X^0\to X^1\)がモノ射である場合に、
  \(\sfD^b(\mcC)\)で\(X\cong \coker(d_X^0)[-1]\)となることを証明することが十分である。

  \(X\in \sfD^b(\mcC)\)は
  \(X^i = 0, (i\in (-\infty,0]\cup (1,+\infty))\)であり、
  さらに\(d_X^0:X^0\to X^1\)がモノ射であるとする。
  \(X^1\to I\)を入射的対象\(I\)へのモノ射とすると、
  \(\hd(\mcC)\leq 1\)であるから、
  \(I/X^0,I/X^1\)はどちらも入射的対象となる。
  複体\(J\)を\(J^0=J^1=I, d_J^0 = \id_I\)で定義し、
  \(J_1\)を\(J_1^0=I/X^0,J_1^1=I/X^1\)で\(d_{J_1}^0\)を自然な射として定義すると、
  \(J_1\)は\(X^1/X^0\)の入射分解であり、
  \(0\to X \to J \to J_1 \to 0\)は\(\Ch(\mcC)\)の完全列である。
  従って、\(X\to J\to J_1\to X[1]\)は\(\sfD(\mcC)\)の完全三角である。
  \(J_1\)は\(X^1/X^0\)の入射分解であるから、
  \(\sfD(\mcC)\)において\(J_1\cong X^1/X^0\)である。
  以上より、\(\sfD(\mcC)\)の完全三角\(X\to J \to X^1/X^0\to X[1]\)を得る。
  さらに、定義より\(\sfD(\mcC)\)において\(J\cong 0\)であるから、
  これは\(\sfD(\mcC)\)において\(X\cong (X^1/X^0)[-1]\)となることを示している。
  以上で\autoref{1.18}の解答を完了する。
\end{proof}


\ifcsname Chap\endcsname\else
\printbibliography
\end{document}
\fi
