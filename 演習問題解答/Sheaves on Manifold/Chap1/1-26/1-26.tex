\ifcsname Chap\endcsname\else
\documentclass[uplatex,dvipdfmx]{jsarticle}
\newcommand{\StylePath}{\ifcsname AllKS\endcsname KS-Style/KS-Style.sty\else
\ifcsname Chap\endcsname ../KS-Style/KS-Style.sty\else
../../KS-Style/KS-Style.sty\fi\fi}
\input{\StylePath}

\KSset{1}{26}
\setcounter{section}{\value{KSS}-1}
\begin{document}
\maketitle\HeaderCommentA
\section{\KSsection{section}}
\setcounter{prob}{\value{KSP}-1}
\fi



\begin{prob}\label{1.26}
  \(\mcC\)をアーベル圏、
  \(X\)を\(\mcC\)の複体で、
  各\(n\)に対して
  \(X^{p,q}\neq 0, p+q=n\)となる\((p,q)\)は高々有限個であるとする。
  さらに\(q_0 < q_1\)が存在して、
  \(q\neq q_0,q_1\)に対して\(\sfD(\mcC)\)において
  \(H_{II}^q(X) \cong 0\)であると仮定する。
  このとき次の三角形が完全であることを示せ:
  \[
  H_{II}^{q_0}(X)[-q_0] \to \Tot(X) \to H_{II}^{q_1}[-q_1]
  \xrightarrow{+1}.
  \]
\end{prob}

\begin{proof}
  \autoref{1.25.1}より、
  \(n\neq q_0,q_1\)に対して
  \(\Tot(\tau_{II}^{\leq n-1}(X)) \to \Tot(\tau_{II}^{\leq n}(X))\)と
  \(\Tot(\tau_{II}^{\geq n}(X)) \to \Tot(\tau_{II}^{\geq n+1}(X))\)は
  どちらも擬同型である。
  従って、\autoref{1.25.3}より、
  任意の\(n < q_0\)に対して\(\sfD(\mcC)\)において
  \(\Tot(\tau_{II}^{\leq n}(X)) \cong 0\)であり、
  任意の\(n > q_1\)に対して\(\sfD(\mcC)\)において
  \(\Tot(\tau_{II}^{\geq n}(X)) \cong 0\)である。
  再び\autoref{1.25.1}を用いると、
  任意の\(q_0 \leq n < q_1\)に対して\(\sfD(\mcC)\)において
  \(\Tot(\tau_{II}^{\leq n}(X)) \cong H_{II}^{q_0}[-q_0]\)であり、
  任意の\(q_0 < n \leq q_1\)に対して\(\sfD(\mcC)\)において
  \(H_{II}^{q_1}[-q_1] \cong \Tot(\tau_{II}^{\geq n}(X))\)であることが従う。
  各\(n\)に対して
  \(\tau_{II}^{\leq n}(X) \to X \to \tau_{II}^{\geq n+1}(X)\xrightarrow{+1}\)
  は\(\sfD(\Ch(\mcC))\)の完全三角なので、
  \[
  \Tot(\tau_{II}^{\leq n}(X)) \to \Tot(X)
  \to \Tot(\tau_{II}^{\geq n+1}(X)) \xrightarrow{+1}
  \]
  は\(\sfD(\mcC)\)の完全三角である。
  \(n=q_0\)とすると、\(n+1\leq q_1\)であるので、従って
  \[
  H_{II}^{q_0}[-q_0] \to \Tot(X) \to H_{II}^{q_1}[-q_1] \xrightarrow{+1}
  \]
  は\(\sfD(\mcC)\)の完全三角である。
  以上で\autoref{1.26}の解答を完了する。
\end{proof}



\ifcsname Chap\endcsname\else
\printbibliography
\end{document}
\fi
