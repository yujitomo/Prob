\ifcsname Chap\endcsname\else
\documentclass[uplatex,dvipdfmx]{jsarticle}
\newcommand{\StylePath}{\ifcsname AllKS\endcsname KS-Style/KS-Style.sty\else
\ifcsname Chap\endcsname ../KS-Style/KS-Style.sty\else
../../KS-Style/KS-Style.sty\fi\fi}
\input{\StylePath}

\KSset{1}{25}
\setcounter{section}{\value{KSS}-1}
\begin{document}
\maketitle\HeaderCommentA
\section{\KSsection{section}}
\setcounter{prob}{\value{KSP}-1}
\fi



\begin{prob}\label{1.25}
  \(\mcC\)をアーベル圏、
  \(X\)を\(\mcC\)の複体で、
  各\(n\)に対して
  \(X^{p,q}\neq 0, p+q=n\)となる\((p,q)\)は高々有限個であるとする。
  \begin{enumerate}
    \item \label{1.25.1}
    以下の三角形が\(\sfD(\mcC)\)において完全であることを示せ:
    \begin{align*}
      \Tot(\tau_{II}^{\leq n-1}(X)) \to
      \Tot(\tau_{II}^{\leq n}(X)) \to
      H_{II}^n(X)[-n] \xrightarrow{+1}, \\
      H_{II}^n(X)[-n] \to
      \Tot(\tau_{II}^{\geq n}(X)) \to
      \Tot(\tau_{II}^{\geq n+1}(X)) \xrightarrow{+1}, \\
    \end{align*}
    \item \label{1.25.2}
    \(k\in \Z\)を固定する。
    自然な射\(H^k(\Tot(\tau_{II}^{\leq n}(X))) \to H^k(\Tot(X))\)
    (resp. \(H^k(\Tot(X))\to H^k(\Tot(\tau_{II}^{\geq n}(X)))\))
    は\(n\gg 0\) (resp. \(n \ll 0\)) に対して同型であることを示せ。
    \item \label{1.25.3}
    \(k\in \Z\)を固定する。
    \(n \ll 0\)に対して\(H^k(\Tot(\tau_{II}^{\leq n}(X))) = 0\)であることと、
    \(n \gg 0\)に対して\(H^k(\Tot(\tau_{II}^{\geq n}(X))) = 0\)であることを示せ。
  \end{enumerate}
\end{prob}

\begin{proof}
  \ref{1.25.1}を示す。
  自然な射
  \(\coker(\tau_{II}^{\leq n-1}(X) \to \tau_{II}^{\leq n}(X))\to H_{II}^n(X)[-n]\)
  に本文\cite[Proposition 1.9.3]{kashiwara2002sheaves}を用いることにより、
  \(\Tot(\coker(\tau_{II}^{\leq n-1}(X) \to \tau_{II}^{\leq n}(X)))\to H_{II}^n(X)[-n]\)
  が擬同型であることが従い、
  これは一つ目の三角形が完全三角であることを示している。
  二つ目の三角形が完全三角であることは
  \(\mcC^{\op}\)において一つ目の三角形が完全三角であることより従う。
  以上で\ref{1.25.1}の証明を完了する。

  \ref{1.25.2}を示す。
  \(X^{p,q}\neq 0, p+q=k,k-1,k+1\)となる\(p\)が存在するような\(q\)のうち
  最大のものを\(n_0\)とすれば、\(n > n_0\)と\(p+q=k,k-1,k+1\)を満たす任意の\(p,q\)に対して
  \((\tau_{II}^{\leq n}(X))^{p,q} = X^{p,q}\)となり、
  \ref{1.25.2}はこれからただちに従う。
  以上で\ref{1.25.2}の証明を完了する。

  \ref{1.25.3}を示す。
  \(X^{p,q}\neq 0, p+q=k,k-1,k+1\)となる\(p\)が存在するような\(q\)のうち
  最小のものを\(n_0\)とすれば、\(n < n_0\)と\(p+q=k,k-1,k+1\)を満たす任意の\(p,q\)に対して
  \((\tau_{II}^{\leq n}(X))^{p,q} = X^{p,q} = 0\)となり、
  \ref{1.25.3}はこれからただちに従う。
  以上で\ref{1.25.3}の証明を完了し、
  \autoref{1.25}の解答を完了する。
\end{proof}




\ifcsname Chap\endcsname\else
\printbibliography
\end{document}
\fi
