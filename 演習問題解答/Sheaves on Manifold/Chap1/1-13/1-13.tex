\ifcsname Chap\endcsname\else
\documentclass[uplatex,dvipdfmx]{jsarticle}
\newcommand{\StylePath}{\ifcsname AllKS\endcsname KS-Style/KS-Style.sty\else
\ifcsname Chap\endcsname ../KS-Style/KS-Style.sty\else
../../KS-Style/KS-Style.sty\fi\fi}
\input{\StylePath}

\KSset{1}{13}
\setcounter{section}{\value{KSS}-1}
\begin{document}
\maketitle
\HeaderCommentA
\section{\KSsection{section}}
\setcounter{prob}{\value{KSP}-1}
\fi



\begin{prob}\label{1.13}
  \(\mcC\)を三角圏、
  \(X_i\to Y_i\to Z_i\to X_i[1], (i=1,2)\)を\(\mcC\)の二つの三角形とする。
  これら二つの三角形が完全三角であるためには、
  三角形
  \[X_1\oplus X_2 \to Y_1\oplus Y_2 \to Z_1\oplus Z_2 \to X_1[1]\oplus X_2[1]\]
  が完全三角であることが必要十分である、ということを示せ。
\end{prob}

\begin{proof}
  必要性を証明する。
  二つの三角形
  \(X_i\to Y_i\to Z_i\to X_i[1], (i=1,2)\)
  が完全三角であるとする。
  \(M\dfn M(X_1\oplus X_2 \to Y_1\oplus Y_2)\)と置く (mapping cone)。
  自然な射\(X_1\oplus X_2\to X_i\)と\(Y_1\oplus Y_2\to Y_i\)により
  可換図式
  \[
  \begin{CD}
    X_1\oplus X_2 @>>> Y_1\oplus Y_2 \\
    @VVV @VVV \\
    X_i @>>> Y_i
  \end{CD}
  \]
  を得る。
  よって、(TR4)より、ある射\(M\to Z_i\)が存在して、
  これらが完全三角の間の射を形成する。
  二つの射\(M\to Z_1,M\to Z_2\)により、
  射\(M\to Z_1\oplus Z_2\)ができて、
  可換図式
  \[
  \begin{CD}
    X_1\oplus X_2 @>>> Y_1\oplus Y_2 @>>> M @>>> X_1[1]\oplus X_2[1] \\
    @| @| @VVV @| \\
    X_1\oplus X_2 @>>> Y_1\oplus Y_2 @>>> Z_1\oplus Z_2 @>>> X_1[1]\oplus X_2[1] \\
  \end{CD}
  \]
  を得る。
  任意に\(P\in \mcC\)を取って、函手\(\Hom(P,-)\)を適用すると、
  各\(\Hom(P,X_i)\to \Hom(P,Y_i)\to \Hom(P,Z_i)\to \Hom(P,X_i[1])\)は完全であるから、
  \[
  \Hom(P,X_1\oplus X_2) \to \Hom(P,Y_1\oplus Y_2) \to
  \Hom(P,Z_1\oplus Z_2) \to \Hom(P,X_1[1]\oplus X_2[1])
  \]
  も完全である。
  よって\autoref{1.12}より
  \[X_1\oplus X_2 \to Y_1\oplus Y_2 \to Z_1\oplus Z_2 \to X_1[1]\oplus X_2[1]\]
  も完全三角であることが従う。
  以上で必要性の証明を完了する。

  十分性を証明する。
  \[X_1\oplus X_2 \to Y_1\oplus Y_2 \to Z_1\oplus Z_2 \to X_1[1]\oplus X_2[1]\]
  が完全三角であると仮定する。
  \(M_i\dfn M(X_i\to Y_i)\)と置く。
  自然な射\(X_i\to X_1\oplus X_2\)と\(Y_i\to Y_1\oplus Y_2\)により
  可換図式
  \[
  \begin{CD}
    X_i @>>> Y_i \\
    @VVV @VVV \\
    X_1\oplus X_2 @>>> Y_1\oplus Y_2
  \end{CD}
  \]
  を得る。
  よって、(TR4)より、ある射\(M_i\to Z_1\oplus Z_2\)が存在して、
  これらが完全三角の間の射を形成する。
  自然な射\(X_1\oplus X_2 \to X_i, Y_1\oplus Y_2 \to Y_i, Z_1\oplus Z_2\to Z_i\)
  と合成することで、可換図式
  \[
  \begin{CD}
    X_i @>>> Y_i @>>> M_i @>>> X_i[1] \\
    @| @| @VVV @| \\
    X_i @>>> Y_i @>>> Z_i @>>> X_i[1]
  \end{CD}
  \]
  を得る。
  任意に\(P\in \mcC\)を取って函手\(\Hom(P,-)\)を適用する。
  \[X_1\oplus X_2 \to Y_1\oplus Y_2 \to Z_1\oplus Z_2 \to X_1[1]\oplus X_2[1]\]
  が完全三角であることから、
  \[
  \Hom(P,X_1\oplus X_2) \to \Hom(P,Y_1\oplus Y_2) \to
  \Hom(P,Z_1\oplus Z_2) \to \Hom(P,X_1[1]\oplus X_2[1])
  \]
  は完全であり、従って各\(i=1,2\)に対して
  \[
  \Hom(P,X_i) \to \Hom(P,Y_i) \to \Hom(P,Z_i) \to \Hom(P,X_i[1])
  \]
  も完全である。
  よって\autoref{1.12}より
  \(X_i\to Y_i\to Z_i\to X_i[1]\)も完全三角であることが従う。
  以上で十分性の証明を完了し、
  \autoref{1.13}の解答を完了する。
\end{proof}




\ifcsname Chap\endcsname\else
\printbibliography
\end{document}
\fi
