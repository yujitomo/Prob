\ifcsname Chap\endcsname\else
\documentclass[uplatex,dvipdfmx]{jsarticle}
\newcommand{\StylePath}{\ifcsname AllKS\endcsname KS-Style/KS-Style.sty\else
\ifcsname Chap\endcsname ../KS-Style/KS-Style.sty\else
../../KS-Style/KS-Style.sty\fi\fi}
\input{\StylePath}

\KSset{1}{38}
\setcounter{section}{\value{KSS}-1}
\begin{document}
\maketitle\HeaderCommentA
\section{\KSsection{section}}
\setcounter{prob}{\value{KSP}-1}
\fi



\begin{prob}\label{1.38}
  \(I,I'\)をfilteredな圏として、\(\varphi:I\to I'\)を函手とする。
  \(\varphi\)が\textbf{cofinal}であるとは、
  以下の条件を満たすことを言う:
  \begin{enumerate}
    \item
    任意の\(i'\in I'\)に対してある\(i\in I\)と射\(i'\to \varphi(i)\)が存在する。
    \item
    任意の\(i\in I\)と\(i'\in I'\)と射\((f:\varphi(i)\to i')\in I'\)に対して
    ある射\((g:i\to i_1)\in I\)と
    \((h:i'\to \varphi(i_1))\in I'\)が存在して\(h\circ f = g\)となる。
  \end{enumerate}
  \(\mcC\)を圏、\(I,I_1\)をfilteredな圏、
  \(F:I\to \mcC, G:I^{\op}\to \mcC\)を函手、
  \(\varphi:I_1\to I\)をcofinalとする。
  自然な射
  \[
  \colim (F\circ \varphi) \to \colim F, \ \
  ``\colim" (F\circ \varphi) \to ``\colim" F, \ \
  \lim G \to \lim (G\circ \varphi), \ \
  ``\lim" G \to ``\lim" (G\circ \varphi)
  \]
  はいずれも同型射であることを示せ。
\end{prob}

\begin{proof}
  \(\colim (F\circ \varphi) \to \colim F\)が同型射であることがわかれば、
  \(\mcC\to \hat{\mcC}=\sfSet^{\mcC^{\op}}\)を合成して
  函手\(I\to \hat{\mcC}\)に対してその事実を適用することにより
  \(``\colim" (F\circ \varphi) \to ``\colim" F\)が同型射であることが従う。
  \(\lim\)に関しても同様である。
  さらに\(\colim (F\circ \varphi) \to \colim F\)が同型射であることがわかれば、
  \(G^{\op}:I\to \mcC^{\op}\)に対してその事実を適用することにより
  \(\lim G \to \lim (G\circ \varphi)\)が同型射であることが従う。
  以上より、\autoref{1.38}を示すためには、
  \(\colim (F\circ \varphi) \to \colim F\)が同型射であることを示すことが十分である。

  \(X\in \mcC\)を任意にとる。
  \(\colim (F\circ \varphi) \to \colim F\)が同型射であることを示すためには、
  米田の補題より、
  自然な射
  \(\Psi:\lim_{i\in I}\Hom_{\mcC}(F(i),X) \to
  \lim_{i_1\in I_1}\Hom_{\mcC}(F(\varphi(i_1)),X)\)
  が全単射であることを示すことが十分である。
  \((f_i),(g_i)\in \lim_{i\in I}\Hom_{\mcC}(F(i),X)\)が
  \(\Psi((f_i)) = \Psi((g_i))\)を満たすとする。
  このとき、各\(i_1\in I_1\)に対して
  \(f_{\varphi(i_1)} = g_{\varphi(i_1)}\)が成り立つ。
  \(i\in I\)を任意にとる。
  \(\varphi:I_1\to I\)はcofinalであるから、一つめの条件より、
  ある\(i_1\in I_1\)と射\(p:i\to \varphi(i_1)\)が存在する。
  \((f_i),(g_i)\)はそれぞれ\(\lim_{i\in I}\Hom_{\mcC}(F(i),X)\)の元であるから、
  \(f_{\varphi(i_1)}\circ F(p) = f_i, g_{\varphi(i_1)} \circ F(p) = g_i\)を満たす。
  \(f_{\varphi(i_1)} = g_{\varphi(i_1)}\)であるので、
  従って\(f_i = g_i\)が成り立つ。
  これは\((f_i)=(g_i)\)を意味し、よって\(\Psi\)は単射である。

  \(\Psi\)が全射であることを示す。
  \((h_{\varphi(i_1)})_{i_1\in I_1}\in \lim_{i_1\in I_1}\Hom_{\mcC}(F(\varphi(i_1)),X)\)
  を任意にとる。
  各\(i\in I\)に対して一つ\(i_1\in I_1\)と射\(p_1:i\to \varphi(i_1)\)を選ぶ
  (\(\varphi\)がcofinalであることの一つめの条件を用いる)。
  \(h_i \dfn h_{i_1}\circ F(p_1)\)と定義する。
  まずこれが\(i_1,p_1\)の取り方に依存しないことを示す。
  そのためには、別の\(p_2:i\to \varphi(i_2)\)に対して
  \(h_{i_1}\circ F(p_1) = h_{i_2}\circ F(p_2)\)が成り立つことが十分である。
  \(I_1\)はfilteredであるから、
  \(i_3\in I_1\)と\(a_1:i_1\to i_3,a_2:i_2\to i_3\)が存在する。
  \(I\)はfilteredであるから、
  二つの並行な射\(\varphi(a_1)\circ p_1, \varphi(a_2)\circ p_2: i \to \varphi(i_3)\)
  に対してある射\(g:\varphi(i_3)\to i'\)が存在して
  \(g\circ \varphi(a_1)\circ p_1 = g\circ \varphi(a_2)\circ p_2\)
  が成り立つ。
  さらに\(\varphi\)はcofinalであるから、
  \(g:\varphi(i_3)\to i'\)に二つめの条件を用いることで、
  ある\((b:i_3\to i_4)\in I_1\)と\((g':i'\to \varphi(i_4))\in I\)が存在して
  \(g' \circ g = \varphi(b)\)が成り立つ。
  このとき
  \[
  \varphi(b\circ a_1) \circ p_1
  = g'\circ g \circ \varphi(a_1) \circ p_1
  = g'\circ g \circ \varphi(a_2) \circ p_2
  = \varphi(b\circ a_2) \circ p_2
  \]
  が成り立つ。
  \(p_4 \dfn \varphi(b\circ a_1) \circ p_1\)とおけば、
  \[
  h_{i_1}\circ F(p_1)
  = h_{i_4} \circ F(\varphi(b\circ a_1)\circ p_1)
  = h_{i_4} \circ F(\varphi(b\circ a_2)\circ p_2)
  = h_{i_2}\circ F(p_2)
  \]
  が成り立つ。
  以上で\(h_i\)の定義が\(p_1:i\to \varphi(i_1)\)の取り方に依存しないことが示された。
  次に\((h_i)_{i\in I}\)が\(\lim\Hom_{\mcC}(F(i),X)\)の元を定めることを示す。
  射\((p:i\to i')\in I\)を任意にとる。
  \(q:i'\to \varphi(i_1)\)を一つ選べば、
  \(h_i\)の定義が\(p_1:i\to \varphi(i_1)\)の取り方に依存しないことから、
  \[
  h_i = h_{i_1}\circ F(q\circ p)
  = h_{i_1}\circ F(q) \circ F(p)
  = h_{i'} \circ F(p)
  \]
  が成り立つ。
  これは\((h_i)_{i\in I}\)が\(F(p)\)たちと両立的であることを意味し、
  従って\((h_i)_{i\in I}\)は\(\lim\Hom_{\mcC}(F(i),X)\)の元を定める。
  各\(i_1\in I\)に対して\(h_{\varphi(i_1)} = h_{i_1}\)であるから、
  \(\Psi((h_i)_{i\in I}) = (h_{i_1})_{i_1\in I_1}\)が成り立つ。
  よって\(\Psi\)は全射である。
  以上で\(\Psi\)が全単射であることが従い、
  \autoref{1.38}の証明を完了する。
\end{proof}





\ifcsname Chap\endcsname\else
\printbibliography
\end{document}
\fi
