\ifcsname Chap\endcsname\else
\documentclass[uplatex,dvipdfmx]{jsarticle}
\newcommand{\StylePath}{\ifcsname AllKS\endcsname KS-Style/KS-Style.sty\else
\ifcsname Chap\endcsname ../KS-Style/KS-Style.sty\else
../../KS-Style/KS-Style.sty\fi\fi}
\input{\StylePath}

\KSset{1}{5}
\setcounter{section}{\value{KSS}-1}
\begin{document}
\maketitle
\HeaderCommentA
\section{\KSsection{section}}
\setcounter{prob}{\value{KSP}-1}
\fi

\begin{prob}\label{1.5}
  \(\mcC\)をアーベル圏、
  \(0\to X\to Z\to Y\to 0\)を完全列とする。
  \(X\)が入射的、または\(Y\)が射影的であれば、
  この完全列は分裂する。
\end{prob}

\begin{proof}
  \(\id_X\)を延長するか、\(\id_Y\)を持ち上げるか、をすれば良いだけ
  (本文の\cite[Definition 1.2.8]{kashiwara2002sheaves}の直後の主張を用いる)。
\end{proof}

\ifcsname Chap\endcsname\else
\printbibliography
\end{document}
\fi
