\ifcsname Chap\endcsname\else
\documentclass[uplatex,dvipdfmx]{jsarticle}
\newcommand{\StylePath}{\ifcsname AllKS\endcsname KS-Style/KS-Style.sty\else
\ifcsname Chap\endcsname ../KS-Style/KS-Style.sty\else
../../KS-Style/KS-Style.sty\fi\fi}
\input{\StylePath}

\KSset{1}{33}
\setcounter{section}{\value{KSS}-1}
\begin{document}
\maketitle\HeaderCommentA
\section{\KSsection{section}}
\setcounter{prob}{\value{KSP}-1}
\fi


\begin{prob}\label{1.33}
  \(k\)を体、\(V\)を\(k\)-線形空間とする。
  自己準同型\(u:V\to V\)が
  \textbf{trace class}
  であるとは、
  ある\(n\)に対して\(\dim(u^n(V)) < \infty\)
  が成り立つことと定義する。
  \(u:V\to V\)が trace class であるとき、
  \(\tr(u) \dfn \tr(u|_{u^n(V)})\)と定義する。
  \begin{enumerate}
    \item \label{1.33.1}
    \(\tr(u)\)の定義は\(n\)に依存しないことを示せ。
    \item \label{1.33.2}
    \(V\xrightarrow{u} W \xrightarrow{v} V\)を\(k\)-線形空間の射の列とする。
    \(u\circ v\)が trace class であることと
    \(v\circ u\)が trace class であることは同値であることを示せ。
    さらにこのとき\(\tr(u\circ v) = \tr(v\circ u)\)が成り立つことを示せ。
    \item \label{1.33.3}
    \(k\)-線形空間の完全列の自己準同型
    \[
    \begin{CD}
      0 @>>> V_1 @>>> V_2 @>>> V_3 @>>> 0 \\
      @. @V{v_1}VV @V{v_2}VV @V{v_3}VV @. \\
      0 @>>> V_1 @>>> V_2 @>>> V_3 @>>> 0
    \end{CD}
    \]
    について、\(v_2\)が trace class であることと
    \(v_1,v_3\)がどちらも trace class であることは同値であることを示せ。
    さらにこのとき、\(\tr(v_2) = \tr(v_1) + \tr(v_3)\)が成り立つことを示せ。
  \end{enumerate}
\end{prob}


\begin{proof}
  \ref{1.33.1}を示す。
  まず\(x\mapsto [u:V\to V]\)により\(V\)を\(k[x]\)-加群と考える。
  十分大きい\(n\)に対して\(\dim(\im(u^n)) < \infty\)であるので、
  \(n \gg 0\)で\(\im(u^n) = \im(u^{n+1})\)となる。
  従って、自然な射\(\im(u^n) \subset V\to V\otimes_k k[x,1/x]\)は
  \(n \gg 0\)で同型射であり、
  とくに\(V\otimes_k k[x,1/x]\)は
  \(k\)-線形空間として有限次元である。
  \(u\)のトレースは\(k\)-線形空間\(V\otimes_k k[x,1/x]\)上への
  \(x\)の作用にしか依存しないため、\(n\)の取り方によらずにwell-definedである。
  以上で\ref{1.33.1}の証明を完了する。

  \ref{1.33.2}を示す。
  \(v\circ (u\circ v)^n\circ u = (v\circ u)^{n+1}\)
  なので\(u\circ v\)が trace class であることと
  \(v\circ u\)が trace class であることは同値である。
  \(v\circ u:V\to V\)と
  \(u\circ v:W\to W\)によって\(V,W\)をそれぞれ\(k[x]\)-加群と考えたとき、
  \(u:V\to W\)と\(v:W\to V\)は\(k[x]\)-加群の射である。
  さらに、\(u\circ v\)か\(v\circ u\)の一方が trace class であれば、
  十分大きい\(n\)に対して
  \(v\circ u:(v\circ u)^n (V) \to (v\circ u)^n (V)\)と
  \(u\circ v:(u\circ v)^n (W) \to (u\circ v)^n (W)\)はいずれも全単射であり、
  とくに\(k[x]\)-加群の同型射である。
  これは\(v\circ u\)と\(u\circ v\)の固有値の和が等しいことを意味する。
  以上で\ref{1.33.2}の証明を完了する。

  \ref{1.33.3}を示す。
  \(v_1,v_2,v_3\)によって\(V_1,V_2,V_3\)を\(k[x]\)-加群とみなす。
  \(v_1,v_2,v_3\)が\(k\)-線形空間の完全列の射を成すことから、
  \[
  \begin{CD}
    0 @>>> V_1 @>>> V_2 @>>> V_3 @>>> 0
  \end{CD}
  \]
  は\(k[x]\)-加群の完全列である。
  \(k[x,1/x]\)をテンソルすると、
  \(k[x,1/x]\)は\(k[x]\)上平坦であるから、
  \(k[x,1/x]\)-加群の完全列
  \[
  \begin{CD}
    0 @>>> V_1\otimes_{k[x]}k[x,1/x] @>>> V_2\otimes_{k[x]}k[x,1/x]
    @>>> V_3\otimes_{k[x]}k[x,1/x] @>>> 0
  \end{CD}
  \]
  を得る。
  \(v_i\)が trace class であることは、
  \(V_i\otimes_{k[x]}k[x,1/x]\)が長さ有限であることと同値であるので、
  以上より\(v_2\)が trace class であることと
  \(v_1,v_3\)がどちらも trace class であることが同値であることが従う。
  \(v_i\)のトレースは\(V_i\otimes_{k[x]}k[x,1/x]\)への\(v_i\)の作用
  (つまり\(x\)の作用)
  のトレースであるから、
  \(V_i\otimes_{k[x]}k[x,1/x]\)たちの成す短完全列を考えることによって、
  \(\tr(v_2) = \tr(v_1) + \tr(v_3)\)であることが従う
  (cf. \autoref{1.32.2}の証明の一番最初の部分など)。
  以上で\ref{1.33.3}の証明を完了し、
  \autoref{1.33}の解答を完了する。
\end{proof}



\ifcsname Chap\endcsname\else
\printbibliography
\end{document}
\fi
