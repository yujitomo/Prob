\ifcsname Chap\endcsname\else
\documentclass[uplatex,dvipdfmx]{jsarticle}
\newcommand{\StylePath}{\ifcsname AllKS\endcsname KS-Style/KS-Style.sty\else
\ifcsname Chap\endcsname ../KS-Style/KS-Style.sty\else
../../KS-Style/KS-Style.sty\fi\fi}
\input{\StylePath}

\KSset{1}{21}
\setcounter{section}{\value{KSS}-1}
\begin{document}
\maketitle\HeaderCommentA
\section{\KSsection{section}}
\setcounter{prob}{\value{KSP}-1}
\fi



\begin{prob}\label{1.21}
  \(F:\mcC\to\mcD\)をアーベル圏の間の左完全函手とする。
  \(F\)-injectiveな\(\mcI\subset \mcC\)が存在すると仮定する。
  \(X\in \sfD^+(\mcC)\)は\(i>0,j\leq j_0\)に対して
  \(R^iF(H^j(X)) = 0\)を満たすとする。
  このとき、\(j\leq j_0\)に対して\(R^jF(X) \cong F(H^j(X))\)
  となることを示せ。
\end{prob}

\begin{proof}
  \(X\in \sfD^+(\mcC)\)であるから、
  \autoref{1.21}を示すためには\(j_0 \geq 0\)であると仮定しても一般性を失わない。
  \(j_0\)に関する帰納法で\autoref{1.21}を示す。
  \(j_0=0\)の場合、
  \(R^0F(X)\cong \ker(F(d_X^0)) \cong F(\ker(d_X^0)) = F(H^0(X))\)
  であるから主張は自明である。

  \(j_0\)未満で\autoref{1.21}が成り立つと仮定する。
  \(Y\dfn \tau^{\leq j_0-1}(X), Z\dfn \tau^{\geq j_0}(X)\)とすると
  \(Y\to X\to Z\)は完全三角であり、
  \(Z[-j_0]\in \sfD^+(\mcC)\)であり、
  帰納法の仮定より、\(j\leq j_0-1\)に対して\(R^jF(Y) \cong F(H^j(Y))\)であり、
  さらに\(\tau^{\geq j_0}(Y) = 0\)であるから\(R^jF(Y) = 0, (j\geq j_0)\)である。
  \(X\)が今の\(j_0\)に対して\autoref{1.21}の仮定を満たすことから、
  \(Z[-j_0]\)は\(j_0=0\)に対して\autoref{1.21}の仮定を満たし、
  すでに示したことによって\(R^0f(Z[-j_0])\cong F(H^0(Z[-j_0]))\)となる。
  従って、\(R^jF(Z) = 0, (j \leq j_0-1)\)かつ
  \(R^{j_0}F(Z) \cong F(H^{j_0}(Z))\)である。
  \(Z = \tau^{\geq j_0}(X)\)なので\(H^{j_0}(Z) \cong H^{j_0}(X)\)であり、
  従って\(R^{j_0}F(Z) \cong F(H^{j_0}(X))\)が従う。

  完全三角\(Y\to X\to Z\to Y[1]\)に\(RF\)を適用して得られる
  完全三角\(RF(Y) \to RF(X) \to RF(Z)\to RF(Y)[1]\)の
  コホモロジーをとることで、長完全列
  \[
  R^jF(Y) \to R^jF(X) \to R^jF(Z) \to R^{j+1}F(Y)
  \]
  を得る。
  ここで\(j\leq j_0-1\)に対して\(R^jF(Y) \cong F(H^j(Y))\)であることと、
  \(j\leq j_0-1\)に対して\(R^jF(Z) = 0\)であることから、
  \(j\leq j_0-1\)に対して\(R^jF(Y)\to R^jF(X)\)は同型射である。
  さらに、\(\tau^{\leq j_0-1}(Y) = Y\)であるから、
  \(R^{j_0}F(Y) = 0\)である。
  従って、\(R^{j_0}F(X)\to R^{j_0}F(Z)\)が同型射となる。
  よって\(R^{j_0}F(X)\cong R^{j_0}F(Z) \cong F(H^{j_0}(X))\)が従う。
  以上で\autoref{1.21}の解答を完了する。
\end{proof}







\ifcsname Chap\endcsname\else
\printbibliography
\end{document}
\fi
