\ifcsname Chap\endcsname\else
\documentclass[uplatex,dvipdfmx]{jsarticle}
\newcommand{\StylePath}{\ifcsname AllKS\endcsname KS-Style/KS-Style.sty\else
\ifcsname Chap\endcsname ../KS-Style/KS-Style.sty\else
../../KS-Style/KS-Style.sty\fi\fi}
\input{\StylePath}

\KSset{1}{17}
\setcounter{section}{\value{KSS}-1}
\begin{document}
\maketitle
\HeaderCommentA
\section{\KSsection{section}}
\setcounter{prob}{\value{KSP}-1}
\fi


\begin{prob}\label{1.17}
  \(\mcC\)をアーベル圏とする。
  \(\mcC\)が\textbf{ホモロジー次元\(\leq n\)である}
  ということを、任意の\(X,Y\in \mcC\)に対して
  \(\Ext^i(X,Y) = 0 ,(\forall i > n)\)となることによって定義する。
  ただし、ここで\(\Ext^i(X,Y) \dfn \Hom_{\sfD(\mcC)}(X,Y[i])\)である。
  自然数\(n\)であって、
  \(\mcC\)がホモロジー次元\(\leq n\)となるもののうち、
  最小のものを\(\hd(\mcC)\)と表し、
  \(\mcC\)の\textbf{ホモロジー次元}と言う。

  \(\mcC\)は十分入射的対象を持つと仮定する。
  このとき、自然数\(n\)に対して、以下の主張が同値であることを示せ:
  \begin{enumerate}
    \item \label{1.17.1}
    \(\hd(\mcC) \leq n\)である。
    \item \label{1.17.2}
    任意の対象\(X\in \mcC\)に対して、
    \(X\)の入射分解\(X\to I\)であって、
    \(i > n\)に対して\(I^i = 0\)となるものが存在する。
  \end{enumerate}
\end{prob}

\begin{proof}
  \ref{1.17.1}\(\Rightarrow\)\ref{1.17.2}を示す。
  \(\hd(\mcC)\leq n\)であるとする。
  任意に対象\(X\in \mcC\)をとり、
  \(X\to I\)を入射分解とする。
  \(Y\in \mcC\)を任意の対象とすると、
  \autoref{1.16.2}より、
  \(H^i(\Hom_{\mcC}(Y,I)) \cong \Hom_{\sfD(\mcC)}(Y,X[i]) = \Ext^i(Y,X)\)
  である。
  \(\hd(\mcC)\leq n\)なので、
  \(H^{n+1}(\Hom_{\mcC}(Y,I)) = 0\)であり、
  従って
  \begin{align*}
    \im(\Hom_{\mcC}(Y,I^n)\to \Hom_{\mcC}(Y,I^{n+1}))
    &\cong \ker(\Hom_{\mcC}(Y,I^{n+1})\to \Hom_{\mcC}(Y,I^{n+2})) \\
    &\cong \Hom_{\mcC}(Y,\ker(d_I^{n+1})) \\
    &\cong \Hom_{\mcC}(Y,\im(d_I^n))
  \end{align*}
  となる。
  よって、完全列
  \[
  \begin{CD}
    0 @>>> \ker(d_I^n) @>>> I^n @>>> \im(d_I^n) @>>> 0
  \end{CD}
  \]
  は任意の\(Y\)に対する
  \(\Hom_{\mcC}(Y,-)\)を適用したあとも完全である。
  従って、\autoref{1.4}より、
  \(I^n\cong \ker(d_I^n) \oplus \im(d_I^n)\)となることがわかる。
  \(I^n\)は入射的対象であるから、
  その直和因子である\(\ker(d_I^n)\)も入射的対象である。
  従って、\(X\to \tau^{\leq n}(I)\)は長さが\(n\)以下の入射分解となる。
  以上で\ref{1.17.1}\(\Rightarrow\)\ref{1.17.2}の証明を完了する。

  \ref{1.17.2}\(\Rightarrow\)\ref{1.17.1}を示す。
  任意に対象\(X\in \mcC\)をとり、
  \(X\to I\)を長さ\(n\)以下の入射分解とする。
  \(Y\in \mcC\)を任意の対象とすると、
  \autoref{1.16.2}より、
  \(H^i(\Hom_{\mcC}(Y,I)) \cong \Hom_{\sfD(\mcC)}(Y,X[i]) = \Ext^i(Y,X)\)
  であるので、\(I^i = 0, (i>n)\)より、\(i>n\)に対して
  \(\Ext^i(Y,X) = 0\)となることがわかる。
  以上で\autoref{1.17}の解答を完了する。
\end{proof}





\ifcsname Chap\endcsname\else
\printbibliography
\end{document}
\fi
