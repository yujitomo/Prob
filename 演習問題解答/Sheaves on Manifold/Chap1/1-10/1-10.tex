\ifcsname Chap\endcsname\else
\documentclass[uplatex,dvipdfmx]{jsarticle}
\newcommand{\StylePath}{\ifcsname AllKS\endcsname KS-Style/KS-Style.sty\else
\ifcsname Chap\endcsname ../KS-Style/KS-Style.sty\else
../../KS-Style/KS-Style.sty\fi\fi}
\input{\StylePath}

\KSset{1}{10}
\setcounter{section}{\value{KSS}-1}
\begin{document}
\maketitle
\HeaderCommentA
\section{\KSsection{section}}
\setcounter{prob}{\value{KSP}-1}
\fi


\begin{prob}\label{1.10}
  \(\mcC\)をアーベル圏とする。
  図式
  \[
  \begin{CD}
    0 @>>> M @>>> M_0 @>>> M_1 @>>> 0 \\
    @. @| @. @. @. \\
    0 @>>> M @>>> M_0' @>>> M_1' @>>> 0
  \end{CD}
  \]
  において、横向きは完全であり、\(M_0,M_0'\)はそれぞれ入射的であるとする。
  同型射\(M_0\oplus M_1' \xrightarrow{\sim} M_0'\oplus M_1\)を構成せよ。
\end{prob}

\begin{proof}
  \(N\dfn M_0\coprod_M M_0'\)とおいて、
  \(j:M_0 \to N \gets M_0':j'\)を自然な射とする。
  \(i,i'\)はモノ射であるから、
  \autoref{1.6.3}より、
  そのpush-outである\(j,j'\)もそれぞれモノ射である。
  従って、\(M_0\)が入射的であることから、
  ある射\(p:N\to M_0\)が存在して\(p\circ j = \id_{M_0}\)となり、
  \(M_0'\)が入射的であることから、
  ある射\(p':N\to M_0'\)が存在して\(p'\circ j' = \id_{M_0'}\)となる。
  図式
  \[
  \begin{CD}
    M @>>> M_0 @>>> M_1 \\
    @VVV @V j VV @VVV \\
    M_0' @> j' >> N @>>> X \\
    @VVV @VVV @. \\
    M_1' @>>> Y @.
  \end{CD}
  \]
  に\(\mcC^{\op}\)で\autoref{1.6.2}を適用すると、
  射\(M_1\to X\)と\(M_1'\to Y\)はそれぞれ同型射であることがわかる。
  以上より、二つの完全列
  \[
  \begin{CD}
    0 @>>> M_0 @> j >> N @>>> M_1' @>>> 0 \\
    0 @>>> M_0' @> j' >> N @>>> M_1 @>>> 0
  \end{CD}
  \]
  を得る。
  \(j,j'\)は分裂モノ射であるから、\autoref{1.4}より、
  同型射\(M_0\oplus M_1' \cong N \cong M_0'\oplus M_1\)を得る。
  以上で\autoref{1.10}の証明を完了する。
\end{proof}





\ifcsname Chap\endcsname\else
\printbibliography
\end{document}
\fi
