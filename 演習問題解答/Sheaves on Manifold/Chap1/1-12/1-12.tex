\ifcsname Chap\endcsname\else
\documentclass[uplatex,dvipdfmx]{jsarticle}
\newcommand{\StylePath}{\ifcsname AllKS\endcsname KS-Style/KS-Style.sty\else
\ifcsname Chap\endcsname ../KS-Style/KS-Style.sty\else
../../KS-Style/KS-Style.sty\fi\fi}
\input{\StylePath}

\KSset{1}{12}
\setcounter{section}{\value{KSS}-1}
\begin{document}
\maketitle
\HeaderCommentA
\section{\KSsection{section}}
\setcounter{prob}{\value{KSP}-1}
\fi



\begin{prob}\label{1.12}
  \(\mcC\)を三角圏とし、
  \[
  \begin{CD}
    X @>>> Y @>>> Z @>>> X[1] \\
    @| @| @VVV @| \\
    X @>>> Y @>>> Z' @>>> X[1]
  \end{CD}
  \]
  を\(\mcC\)の可換図式で、上の列が完全三角であるものとする。
  このとき、以下の条件のうちの一方が成り立つとき、
  下の列も完全三角であることを示せ:
  \begin{enumerate}
    \item \label{1.12.1}
    任意の対象\(P\in \mcC\)に対して、以下の列は完全である:
    \[
    \Hom(P,X) \to \Hom(P,Y) \to \Hom(P,Z') \to \Hom(P,X[1]).
    \]
    \item \label{1.12.2}
    任意の対象\(Q\in \mcC\)に対して、以下の列は完全である:
    \[
    \Hom(X,Q) \gets \Hom(Y,Q) \gets \Hom(Z',Q) \gets \Hom(X[1],Q).
    \]
  \end{enumerate}
\end{prob}


\begin{proof}
  \(\Hom(P,-)\)と\(\Hom(-,Q)\)はそれぞれ cohomological functor であるから、
  \ref{1.12.1}を仮定すれば、射\(\Hom(P,Z) \to \Hom(P,Z')\)は同型射であることが従い、
  \ref{1.12.2}を仮定すれば、射\(\Hom(Z',Q) \to \Hom(Z,Q)\)は同型射であることが従う。
  すると、米田の補題より、
  \ref{1.12.1}と\ref{1.12.2}のいずれかを仮定すれば、射\(Z\to Z'\)は同型射であることが従う。
  \(\mcC\)は三角圏なので、
  \cite[Proposition 1.4.4 (TR0)]{kashiwara2002sheaves}を満たし、
  従って所望の完全性を得る。
\end{proof}




\ifcsname Chap\endcsname\else
\printbibliography
\end{document}
\fi
