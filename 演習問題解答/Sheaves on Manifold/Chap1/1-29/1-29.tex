\ifcsname Chap\endcsname\else
\documentclass[uplatex,dvipdfmx]{jsarticle}
\newcommand{\StylePath}{\ifcsname AllKS\endcsname KS-Style/KS-Style.sty\else
\ifcsname Chap\endcsname ../KS-Style/KS-Style.sty\else
../../KS-Style/KS-Style.sty\fi\fi}
\input{\StylePath}

\KSset{1}{29}
\setcounter{section}{\value{KSS}-1}
\begin{document}
\maketitle\HeaderCommentA
\section{\KSsection{section}}
\setcounter{prob}{\value{KSP}-1}
\fi


\begin{prob}\label{1.29}
  \(A\)を環とする。
  \begin{enumerate}
    \item \label{1.29.1}
    任意の自由加群は射影的であることを示せ。
    \item \label{1.29.2}
    任意の射影加群はある自由加群の直和因子であることを示せ。
    \item \label{1.29.3}
    射影加群は平坦加群であることを示せ。
    \item \label{1.29.4}
    \(n\geq 0\)を自然数とする。
    以下の条件が同値であることを示せ:
    \begin{enumerate}
      \item \label{1.29.4.1}
      任意の右\(A\)-加群\(N\)と任意の左\(A\)-加群\(M\)と任意の\(j>n\)に対して
      \(\Tor_j^A(N,M) = 0\)である。
      \item \label{1.29.4.2}
      任意の左\(A\)-加群\(M\)に対して
      完全列\(0\to P^n \to \cdots \to P^0 \to M\to 0\)
      であって各\(P^i\)が平坦加群となるものが存在する。
      \item \label{1.29.4.3}
      任意の右\(A\)-加群\(M\)に対して
      完全列\(0\to P^n \to \cdots \to P^0 \to M\to 0\)
      であって各\(P^i\)が平坦加群となるものが存在する。
    \end{enumerate}
    これらの同値な条件を満たす最小の\(n\in \N\cup\{\infty\}\)を
    \(\wgld(A)\)と表し、
    \(A\)の\textbf{弱大域次元} (weak global dimension) という。
    \item \label{1.29.5}
    \(\wgld(A) \leq \gld(A)\)であることを示せ。
  \end{enumerate}
\end{prob}

\begin{proof}
  \ref{1.29.1}は函手の同型
  \(\Hom_A(A^{\oplus I},-) \cong \prod_I(-)\)
  より従う。

  \ref{1.29.2}を示す。
  \(P\)を射影加群として
  全射\(p:A^{\oplus I}\to P\)をとる。
  \(P\)が射影加群であることから
  射\(\id_P:P\to P\)がリフトして
  \(p\circ s=\id_P\)となる
  \(s:P\to A^{\oplus I}\)が存在する。
  よって\autoref{1.4.4}より
  \(P\)は\(A^{\oplus I}\)の直和因子である。
  以上で\ref{1.29.2}の証明を完了する。

  \ref{1.29.3}を示す。
  \(P\)を射影加群として、
  \(P\)が直和因子となるように射\(i:P\to A^{\oplus I}\)をとる。
  \(p:A^{\oplus I}\to P\)を\(i\)の左逆射、
  つまり\(p\circ i = \id_P\)となる射とする。
  \(f:M\to N\)を\(A\)-加群の単射とする。
  \ref{1.29.3}を示すためには、
  \(f\otimes_A \id_P\)が単射であることを示すことが十分である。
  可換図式
  \[
  \begin{CD}
    M\otimes_A P @> \id \otimes i >>
    M\otimes_A A^{\oplus I} @> \id \otimes p >> M\otimes_A P \\
    @V f\otimes \id VV @V f\otimes \id VV @VV f\otimes \id V \\
    N\otimes_A P @> \id \otimes i >>
    N\otimes_A A^{\oplus I} @> \id \otimes p >> N\otimes_A P
  \end{CD}
  \]
  において、上と下の合成は\(\id\)であり、
  \(M\otimes_A A^{\oplus I}\cong M^{\oplus I}\)より真ん中は単射である。
  従って両端も単射であることが従う。
  以上で\ref{1.29.3}の証明を完了する。

  \ref{1.29.4}を示す。
  \ref{1.29.4.1} \(\iff\) \ref{1.29.4.2}
  を示すことができれば、
  \(A^{\op}\)に対して
  \ref{1.29.4.1} \(\iff\) \ref{1.29.4.2}
  を適用することで
  \ref{1.29.4.1} \(\iff\) \ref{1.29.4.3}
  が従う。
  残っているのは
  \ref{1.29.4.1} \(\iff\) \ref{1.29.4.2}
  を示すことである。

  \ref{1.29.4.1}が成り立つと仮定する。
  自由分解\(\cdots \to P^n \xrightarrow{d_P^n} \cdots \to P^0 \xrightarrow{d_P^0} M\to 0\)
  を一つとる。
  任意の\(N\)と\(j>n\)に対して\(\Tor_j^A(N,M) = 0\)が成り立つので、
  とくに任意の\(N\)と\(j>n-1\)に対して\(\Tor_j^A(N,\ker(d_P^0))=0\)が成り立つ。
  \(\ker(d_P^0)\cong \im(d_P^1)\)に注意して繰り返すと、
  繰り返して、任意の\(N\)と任意の\(j>0\)に対して
  \(\Tor_j^A(N,\ker(d_P^{n-1})) = 0\)が成り立つ。
  このことは\(\ker(d_P^{n-1})\)が平坦であることを意味していて、
  完全列
  \(0\to \ker(d_P^{n-1}) \to P^{n-1} \to \cdots \to P^0 \to M \to 0\)
  は\(M\)の長さ\(n\)以下の平坦分解である。
  以上で
  \ref{1.29.4.1} \(\Rightarrow\) \ref{1.29.4.2}
  が示された。

  \ref{1.29.4.2}が成り立つと仮定する。
  任意に左\(A\)-加群\(M\)と右\(A\)-加群\(N\)と\(j>n\)をとる。
  仮定より\(M\)の平坦分解
  \(0 \to P^n \xrightarrow{d_P^n} \cdots \to P^0 \xrightarrow{d_P^0} M\to 0\)
  が存在する。
  \(P^n,P^{n-1}\)は平坦であるから、
  完全列\(0\to P^n \to P^{n-1} \to \im(d_P^{n-1})\to 0\)
  に\(N\otimes_A(-)\)を施すことで、
  任意の\(j>1\)に対して\(\Tor_j^A(N,\im(d_P^{n-1}))=0\)であることが従う。
  完全列\(0\to \im(d_P^{n-1})\to P^{n-2}\to \im(d_P^{n-2}) \to 0\)
  に\(N\otimes_A(-)\)を施すことで、
  任意の\(j>2\)に対して\(\Tor_j^A(N,\im(d_P^{n-2}))=0\)であることが従う。
  帰納的に、
  任意の\(j>k\)に対して\(\Tor_j^A(N,\im(d_P^{n-k}))=0\)であることが従う。
  \(n=k\)とすれば所望の結論を得る。
  以上で
  \ref{1.29.4.2} \(\Rightarrow\) \ref{1.29.4.1}
  が示され、\ref{1.29.4}の証明を完了する。

  \ref{1.29.5}は
  \KSIffAutoref{1.28.1}{1.28.3}と
  \ref{1.29.3}より従う。
  以上で\autoref{1.29}の解答を完了する。
\end{proof}


\ifcsname Chap\endcsname\else
\printbibliography
\end{document}
\fi
