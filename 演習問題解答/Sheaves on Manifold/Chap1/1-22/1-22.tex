\ifcsname Chap\endcsname\else
\documentclass[uplatex,dvipdfmx]{jsarticle}
\newcommand{\StylePath}{\ifcsname AllKS\endcsname KS-Style/KS-Style.sty\else
\ifcsname Chap\endcsname ../KS-Style/KS-Style.sty\else
../../KS-Style/KS-Style.sty\fi\fi}
\input{\StylePath}

\KSset{1}{22}
\setcounter{section}{\value{KSS}-1}
\begin{document}
\maketitle\HeaderCommentA
\section{\KSsection{section}}
\setcounter{prob}{\value{KSP}-1}
\fi


\begin{prob}\label{1.22}
  \(\mcC,\mcD,\mcE\)をそれぞれアーベル圏として、
  \(F:\mcC\to \mcD, G:\mcD\to \mcE\)を左完全函手とする。
  \(F\)-injectiveな\(\mcI\subset \mcC\)と
  \(G\)-injectiveな\(\mcJ\subset \mcD\)が存在して、
  \(F(\mcI)\subset \mcJ\)となると仮定する
  (本文\cite[Proposition 1.8.7]{kashiwara2002sheaves}の状況設定)。
  \(X\in \sfD^+(\mcC)\)は\(R^jF(X) = 0, (\forall j<n)\)を満たすと仮定する。
  \(R^n(G\circ F)(X) \cong (G\circ R^nF)(X)\)を示せ。
\end{prob}

\begin{proof}
  本文\cite[Remark 1.8.6]{kashiwara2002sheaves}を\(RF(X)\)と\(G\)に対して適用することで、
  任意の\(j<n\)に対して\(R^jG(RF(X)) = 0\)であり、さらに
  \(R^nG(RF(X)) \cong G(H^n(RF(X))) = G(R^nF(X))\)である。
  また、本文\cite[Proposition 1.8.7]{kashiwara2002sheaves}より
  \(R(G\circ F)(X) \cong RG(RF(X))\)であるので、
  \(n\)番目のコホモロジーをとれば
  \(R^n(G\circ F)(X) \cong R^nG(RF(X))\)が従う。
  よって\(R^n(G\circ F)(X) \cong R^nG(RF(X))\cong G(R^nF(X))\)となり、
  以上で\autoref{1.22}の解答を完了する。
\end{proof}




\ifcsname Chap\endcsname\else
\printbibliography
\end{document}
\fi
