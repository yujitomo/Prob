\ifcsname Chap\endcsname\else
\documentclass[uplatex,dvipdfmx]{jsarticle}
\newcommand{\StylePath}{\ifcsname AllKS\endcsname KS-Style/KS-Style.sty\else
\ifcsname Chap\endcsname ../KS-Style/KS-Style.sty\else
../../KS-Style/KS-Style.sty\fi\fi}
\input{\StylePath}

\KSset{1}{8}
\setcounter{section}{\value{KSS}-1}
\begin{document}
\maketitle
\HeaderCommentA
\section{\KSsection{section}}
\setcounter{prob}{\value{KSP}-1}
\fi


\begin{prob}[The Five Lemma]\label{1.8}
  \(\mcC\)をアーベル圏とする。
  \(\mcC\)の可換図式
  \[
  \begin{CD}
    X^0 @>>> X^1 @>>> X^2 @>>> X^3 @>>> X^4 \\
    @V f_0 VV   @V f_1 VV   @V f_2 VV   @V f_3 VV   @V f_4 VV \\
    Y^0 @>>> Y^1 @>>> Y^2 @>>> Y^3 @>>> Y^4
  \end{CD}
  \]
  について以下の主張を証明せよ。
  ただし横向きは完全であるとする。
  \begin{enumerate}
    \item \label{1.8.1}
    \(f_0\)がエピであり、
    \(f_1,f_3\)がモノであれば、
    \(f_2\)はモノである。
    \item \label{1.8.2}
    \(f_4\)がモノであり、
    \(f_1,f_3\)がエピであれば、
    \(f_2\)はエピである。
  \end{enumerate}
\end{prob}

\begin{proof}
  \autoref{1.7}によって\(\Ab\)での主張と見做して良く、
  この場合、主張は初等的である。
\end{proof}


\ifcsname Chap\endcsname\else
\printbibliography
\end{document}
\fi
