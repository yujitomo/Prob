\ifcsname Chap\endcsname\else
\documentclass[uplatex,dvipdfmx]{jsarticle}
\newcommand{\StylePath}{\ifcsname AllKS\endcsname KS-Style/KS-Style.sty\else
\ifcsname Chap\endcsname ../KS-Style/KS-Style.sty\else
../../KS-Style/KS-Style.sty\fi\fi}
\input{\StylePath}

\KSset{1}{39}
\setcounter{section}{\value{KSS}-1}
\begin{document}
\maketitle
\HeaderCommentA
\section{\KSsection{section}}
\setcounter{prob}{\value{KSP}-1}
\fi


\begin{prob}\label{1.39}
  \(\mcC\)をアーベル圏とする。
  \(X,Y\in \sfD^b(\mcC)\)に対して
  \(\Ext^j(X,Y) \dfn \Hom_{\sfD(\mcC)}(X,Y[j])\)とおく。
  \begin{enumerate}
    \item \label{1.39.1}
    \(X,Y\in \mcC\)として\(n\geq 1\)とする。
    完全列
    \[
    E: 0 \to Y\to Z_n\to Z_{n-1} \to \cdots \to Z_1 \to X\to 0
    \]
    が元\(C(E) \in \Ext^n(X,Y)\)を定めることを示せ。
    この完全列を\textbf{\(X\)の\(Y\)による\(n\)-拡大}という。
    \item \label{1.39.2}
    任意の\(\Ext^n(X,Y)\)の元は\(C(E)\)の形で表すことができることを示せ。
    \item \label{1.39.3}
    \(E':0\to Y\to Z'_n\to \cdots \to Z'_1\to X\to 0\)を別の拡大とする。
    \(C(E) = C(E')\)であるための必要十分条件は、
    ある拡大\(E'': 0\to Y\to Z''_n\to \cdots \to Z''_1\to X\to 0\)
    と以下の可換図式が存在することであるということを示せ:
    \[
    \begin{CD}
      Y @>>> Z_n @>>> \cdots @>>> Z_1 @>>> X \\
      @| @AAA @. @AAA @| \\
      Y @>>> Z''_n @>>> \cdots @>>> Z''_1 @>>> X \\
      @| @VVV @. @VVV @| \\
      Y @>>> Z'_n @>>> \cdots @>>> Z'_1 @>>> X.
    \end{CD}
    \]
  \end{enumerate}
  \(\Ext^n(X,Y)\)をしばしば\textbf{Yoneda extension}という。
\end{prob}


\begin{proof}
  \ref{1.39.1}を示す。
  \(Z^i \dfn Z_{-i+1}\)と定義して、完全列\(E\)から\(X,Y\)を取り除いた複体を
  \[Z = (\cdots 0\to Z^{-n+1} \to \cdots \to Z^0\to 0 \to \cdots)\]
  と表す。
  このとき\(\tau^{\leq n-1}(Z) = Y[n-1]\)であり、
  \(\tau^{\geq 0}(Z) = X\)である。
  また、\(Z\)は\(-(n-2)\)次から\(-1\)次で完全なので、
  \(Y[n-1]\to Z\to X\xrightarrow{+1}\)は完全三角である。
  これに函手\(\Hom_{\sfD(\mcC)}(X,-)\)を適用することにより、
  射\(\Hom_{\sfD(\mcC)}(X,X)\to \Hom_{\sfD(\mcC)}(X,Y[n]) = \Ext^n(X,Y)\)を得る。
  \(\id_X\)の行き先を\(C(E)\)とすれば良い。
  以上で\ref{1.39.1}の証明を完了する。

  \ref{1.39.2}を示す。
  \(f\in \Ext^n(X,Y)\)を一つとる。
  定義より\(\Ext^n(X,Y) = \Hom_{\sfD(\mcC)}(X,Y[n])\)であるので、
  \(f\)は\(\sfD(\mcC)\)の射\(f:X\to Y[n]\)とみなせる。
  \(f\)を\(\sfD(\mcC)\)の完全三角
  \(X\xrightarrow{f} Y[n]\to Z\xrightarrow{+1}\)に伸ばす。
  このとき\(Y[n-1]\to Z[-1]\to X\xrightarrow{+1}\)も完全三角である。
  コホモロジーをとれば、
  \(H^i(Z[-1])\)は\(n-1\)次で\(H^{n-1}(Z[-1])\cong Y\)、
  \(0\)次で\(H^0(Z[-1])\cong X\)、他は\(0\)である。
  従って
  \[E : 0\to [Y\cong H^n(Z)]\to Z^{-n} \to \cdots \to Z^{-1} \to [X\cong H^{-1}(Z)]\to 0\]
  は完全である。
  完全三角\(Y[n-1]\to Z[-1]\to X\xrightarrow{+1}\)は
  \(X\xrightarrow{f} Y[n]\to Z\xrightarrow{+1}\)を
  \(-1\)方向に二つずらした完全三角なので、
  \(\Hom_{\sfD(\mcC)}(X,-)\)に入れると
  \(\id_X\)の行き先は\(f:X\to Y[n]\)に他ならない。
  このことは\(f = C(E)\)を意味している。
  以上で\ref{1.39.2}の証明を完了する。

  \ref{1.39.3}を示す。
  \(E,E'\)から\ref{1.39.1}のように定義した複体をそれぞれ\(Z,Z'\)と表す。
  十分性を示す。
  \(E''\)から\ref{1.39.1}のように定義した複体を\(Z''\)と表す。
  \ref{1.39.1}の証明より、\(\sfD(\mcC)\)の完全三角とその間の射
  \[
  \begin{CD}
    Y[n-1] @>>> Z @>>> X @> +1 >> \\
    @| @AAA @| @. \\
    Y[n-1] @>>> Z'' @>>> X @> +1 >> \\
    @| @VVV @| @. \\
    Y[n-1] @>>> Z' @>>> X @> +1 >>
  \end{CD}
  \]
  を得る。
  これを\(\Hom_{\sfD(\mcC)}(X,-)\)に入れると、
  アーベル群の可換図式
  \[
  \begin{CD}
    \Hom_{\sfD(\mcC)}(X,X) @= \Hom_{\sfD(\mcC)}(X,X) @= \Hom_{\sfD(\mcC)}(X,X) \\
    @V \delta VV @V \delta'' VV @VV \delta' V \\
    \Hom_{\sfD(\mcC)}(X,Y[n]) @= \Hom_{\sfD(\mcC)}(X,Y[n]) @= \Hom_{\sfD(\mcC)}(X,Y[n])
  \end{CD}
  \]
  を得る。
  ここで定義より\(C(E) = \delta(\id_X), C(E') = \delta'(\id_X)\)であるが、
  上の図式が可換であることは\(\delta = \delta'' = \delta'\)を意味するので、
  よって\(C(E) = \delta(\id_X) = \delta'(\id_X) = C(E')\)が成り立つ。
  以上で十分性の証明を完了する。

  必要性を示す。
  \(f = C(E) = C(E')\in \Ext^n(X,Y) = \Hom_{\sfD(\mcC)}(X,Y[n])\)とおく。
  \(f:X\to Y[n]\)を\(\sfD(\mcC)\)の完全三角
  \(X\xrightarrow{f} Y[n]\to Z'' \xrightarrow{+1}\)に伸ばす。
  \ref{1.39.2}の証明と同様に、このとき
  \[
  0\to Y\to {Z''}^{-n} \to \cdots \to {Z''}^{-1} \to X\to 0
  \]
  は完全である。
  \(Z''\)の\(-n-1\)次以下と\(0\)次以上を\(0\)で置き直した複体を再び\(Z''\)で表す。
  すると上の完全列により
  \(Y[n-1]\to Z'' [-1] \to X\xrightarrow{+1}\)が
  \(\sfK(\mcC)\)の完全三角であることが従う。
  \(f=C(E)\)であるから、三角圏の公理
  (本文\cite[Proposition 1.4.4 (TR4)]{kashiwara2002sheaves}) より
  \(\sfK(\mcC)\)の射\(Z''\to Z\)が存在して、
  \(\id_X,\id_{Y[n]}\)によって
  \(X\xrightarrow{f} Y[n]\to Z'' \xrightarrow{+1}\)から
  \(X\xrightarrow{f} Y[n]\to Z \xrightarrow{+1}\)への完全三角の射を形成する。
  同様に、\(f=C(E')\)であるから、完全三角の射を形成するような\(Z''\to Z'\)も存在する。
  よって\(\sfK(\mcC)\)の擬同型からなる図式
  \(Z \gets Z'' \to Z\)を得る。
  これらの射を代表する\(\Ch(\mcC)\)の擬同型からなる図式
  \(Z\gets Z'' \to Z\)を\(\mcC\)の図式として書き直すと、
  可換図式
  \[
  \begin{CD}
    H^{-n}(Z) @>>> Z^{-n} @>>> \cdots @>>> Z^{-1} @>>> H^{-1}(Z) \\
    @AAA @AAA @. @AAA @AAA \\
    H^{-n}(Z'') @>>> {Z''}^{-n} @>>> \cdots @>>> {Z''}^{-1} @>>> H^{-1}(Z'') \\
    @VVV @VVV @. @VVV @VVV \\
    H^{-n}(Z') @>>> {Z'}^{-n} @>>> \cdots @>>> {Z'}^{-1} @>>> H^{-1}(Z').
  \end{CD}
  \]
  を得る。
  \(H^{-n}(Z) \cong H^{-n}(Z'') \cong H^{-n}(Z') \cong Y\)と
  \(H^{-1}(Z) \cong H^{-1}(Z'') \cong H^{-1}(Z') \cong X\)に注意すれば
  所望の可換図式を得る。以上で必要性の証明を完了し、
  \ref{1.39.3}の証明を完了し、
  \autoref{1.39}の解答を完了する。
\end{proof}




\ifcsname Chap\endcsname\else
\printbibliography
\end{document}
\fi
