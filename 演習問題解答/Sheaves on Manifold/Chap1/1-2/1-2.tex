\ifcsname Chap\endcsname\else
\documentclass[uplatex,dvipdfmx]{jsarticle}
\newcommand{\StylePath}{\ifcsname AllKS\endcsname KS-Style/KS-Style.sty\else
\ifcsname Chap\endcsname ../KS-Style/KS-Style.sty\else
../../KS-Style/KS-Style.sty\fi\fi}
\input{\StylePath}

\KSset{1}{2}
\setcounter{section}{\value{KSS}-1}
\begin{document}
\maketitle
\HeaderCommentA
\section{\KSsection{section}}
\setcounter{prob}{\value{KSP}-1}
\fi

\begin{prob}\label{1.2}
  \(\mcC,\mcC'\)を二つの圏、
  \(F:\mcC\to \mcC', G:\mcC'\to \mcC\)を二つの函手とする。
  \begin{enumerate}
    \item \label{1.2.1}
    次の二つの条件は同値であることを示せ:
    \begin{enumerate}
      \item \label{1.2.1.1}
      函手の射\(\alpha:F\circ G \to \id_{\mcC'}, \beta:G\circ F \to \id_{\mcC}\)
      が存在して、任意の\(X\in \mcC, Y\in \mcC'\)に対し
      \[
      \id_{G(Y)} = G(\alpha_Y) \circ \beta_{G(Y)} \ , \
      \id_{F(X)} = \alpha_{F(X)} \circ F(\beta_X)
      \]
      となる。
      \item \label{1.2.1.2}
      \(\Hom_{\mcC'}(F(-),?),\Hom_{\mcC}(-,G(?))\)
      は函手\(\mcC^{\op}\times \mcC' \to \sfSet\)として同型である。
    \end{enumerate}
    \item \label{1.2.2}
    任意の函手\(F:\mcC\to \mcC'\)に対して、
    \(F\)の左 (または右) 随伴は、存在すれば、どの二つも函手として同型となる。
    \item \label{1.2.3}
    任意の函手\(F:\mcC\to \mcC'\)に対して、
    \(F\)の左 (または右) 随伴が存在するための必要十分条件は、
    任意の対象\(Y\in \mcC'\)に対して
    函手\(X\mapsto \Hom_{\mcC'}(F(X),Y)\)
    (または\(X\mapsto \Hom_{\mcC'}(Y,F(X))\))
    が表現可能であることである。
  \end{enumerate}
\end{prob}

\begin{proof}
  \ref{1.2.1}を示す。
  \ref{1.2.1.1}を仮定する。
  \(F:\mcC\to \mcC'\)は函手なので、
  \(\mcC^{\op}\times\mcC\)から\(\sfSet\)への二つの函手の間の射
  \(\Hom_{\mcC}(-,?) \to \Hom_{\mcC'}(F(-),F(?))\)
  を引き起こす。
  これを\(\bar{F}\)と書く。
  同じく\(\bar{G}\)を定義する。
  函手の射
  \begin{align*}
    &P: \Hom_{\mcC'}(F(-),?) \xrightarrow{\bar{G}} \Hom_{\mcC}(G(F(-)),G(?))
    \xrightarrow{(??)\circ \beta_{(-)}} \Hom_{\mcC}(-,G(?)) \\
    &Q: \Hom_{\mcC}(-,G(?)) \xrightarrow{\bar{F}} \Hom_{\mcC'}(F(-),F(G(?)))
    \xrightarrow{\alpha_{(?)} \circ (??)} \Hom_{\mcC'}(F(-),?)
  \end{align*}
  を合成で定義する。
  すると、各対象\((X,Y)\in \mcC^{\op}\times \mcC'\)と
  \(f\in \Hom_{\mcC'}(F(X),Y), g\in \Hom_{\mcC}(X,G(Y))\)に対し、
  \begin{align*}
    Q_{(X,Y)}(P_{(X,Y)}(f))
    &= \alpha_Y \circ F(G(f)\circ \beta_X))
    & P_{(X,Y)}(Q_{(X,Y)}(g))
    &= G(\alpha_Y \circ F(g)) \circ \beta_X \\
    &= \alpha_Y \circ F(G(f)) \circ F(\beta_X)
    &&= G(\alpha_Y) \circ G(F(g)) \circ \beta_X \\
    &\overset{\bigstar}{=} f \circ \alpha_{F(X)} \circ F(\beta_X)
    &&\overset{\bigstar}{=} G(\alpha_Y) \circ \beta_{G(Y)} \circ g \\
    &\overset{*}{=} f \circ \id_{F(X)} \ = f,
    &&\overset{*}{=} \id_{G(Y)} \circ g \ = g
  \end{align*}
  となる。
  ただし、\(\bigstar\)の箇所で\(\alpha,\beta\)が自然変換であることを用い、
  \(*\)の箇所で\ref{1.2.1.1}で仮定されている条件を用いた。
  以上より\(P,Q\)は函手の同型射である。
  よって\ref{1.2.1.2}が示された。

  逆に\ref{1.2.1.2}を仮定する。
  \(P:\Hom_{\mcC'}(F(-),?) \xrightarrow{\sim} \Hom_{\mcC}(-,G(?))\)
  を函手の同型射として、\(Q\dfn P^{-1}\)と置く。
  \(P\)が函手の射であることは、各射
  \([f:F(X) \to Y],[f':F(X')\to Y]\in \mcC',
  [g:Y\to Y']\in \mcC', [h:X\to X']\in \mcC\)
  に対して
  \(P_{(X,Y')}(g\circ f) = G(g)\circ P_{(X,Y)}(f),
  P_{(X',Y)}(f')\circ h = P_{(X,Y)}(f'\circ F(h))\)
  となることを意味する
  (以下の図式が可換である):
  \[
  \begin{CD}
    \Hom_{\mcC'}(F(X'),Y) @> (-)\circ F(h) >>
    \Hom_{\mcC'}(F(X),Y) @> g\circ (-) >>
    \Hom_{\mcC'}(F(X),Y') \\
    @V P_{(X',Y)} VV @V P_{(X,Y)} VV @VV P_{(X,Y')} V \\
    \Hom_{\mcC'}(X',G(Y)) @> (-)\circ h >>
    \Hom_{\mcC}(X,G(Y)) @> G(g) \circ (-) >>
    \Hom_{\mcC}(X,G(Y')).
  \end{CD}
  \]
  \(Q\)についても同様の等式が成立する
  (上の図式で縦向きの射が逆になったものが\(Q\)の場合)。
  次のように自然変換を定義する:
  \begin{align*}
    \alpha_{(?)} &\dfn Q_{(G(?),?)}(\id_{G(?)}): F(G(?)) \to (?), \\
    \beta_{(-)} &\dfn P_{(-,F(-))}(\id_{F(-)}): (-) \to G(F(-)).
  \end{align*}
  すると各\((X,Y)\in \mcC^{\op}\times \mcC'\)に対して
  \begin{align*}
    \alpha_{F(X)} \circ F(\beta_X)
    &= Q_{(G(F(X)),F(X))}(\id_{G(F(X))}) \circ F(\beta_X) \\
    &\overset{\bigstar_l}{=} Q_{(X,F(X))}(\id_{F(X)}) \circ \beta_X) \\
    &= Q_{(X,F(X))}(P_{(X,F(X))}(\id_{F(X)})) \ = \id_{F(X)} \\
    G(\alpha_Y)\circ \beta_{G(Y)}
    &= G(\alpha_Y) \circ P_{(G(Y),F(G(Y)))}(\id_{F(G(Y))})  \\
    &\overset{\bigstar_r}{=} P_{(G(Y),Y)}(\alpha_Y \circ \id_{F(G(Y))})  \\
    &= P_{(G(Y),Y)}(Q_{(G(Y),Y)}(\id_{G(Y)})) \ = \id_{G(Y)}
  \end{align*}
  となる。ただし\(\bigstar_l,\bigstar_r\)の箇所で上の図式の可換性を用いた
  (\(l\)は左、\(r\)は右側の四角形の可換性を用いている)。
  以上で\ref{1.2.1}の証明を完了する。

  \ref{1.2.2}を示す。
  \(G,G'\)がどちらも\(F\)の右随伴であれば、
  \ref{1.2.1}における函手の自然同型を用いて
  \[
  \Hom_{\mcC}(-,G(?)) \cong \Hom_{\mcC'}(F(-),?) \cong \Hom_{\mcC}(-,G'(?))
  \]
  となるので、米田の補題によって\(G\cong G'\)がわかる。
  左随伴についても同様である。

  \ref{1.2.3}を示す。
  各\(Y\in \mcC'\)について
  \(X\mapsto \Hom_{\mcC'}(F(X),Y)\)が表現可能であるとし、
  その表現対象を\(G(Y)\)と置く。
  すると\(X\)について自然な同型
  \(\Hom_{\mcC'}(F(X),Y)\cong \Hom_{\mcC}(X,G(Y))\)を得る。
  \(g:Y\to Y'\)を任意の\(\mcC'\)の射とする。
  すると\(g\)を合成することによって函手の射
  \(\Hom_{\mcC'}(F(-),Y) \to \Hom_{\mcC'}(F(-),Y')\)を得る。
  よって米田の補題によって一意的に射\(G(Y)\to G(Y')\)を得る。
  この射を\(G(g)\)と書く。
  すると\(G\)は函手であり、
  \(X\)について自然な同型\(\Hom_{\mcC'}(F(X),Y)\cong \Hom_{\mcC}(X,G(Y))\)は
  構成から\(Y\)についても自然である。
  これによって\(F\)の右随伴\(G\)を得る。
  左随伴に関しても同様である。
  以上で\autoref{1.2}の解答を完了する。
\end{proof}



\ifcsname Chap\endcsname\else
\printbibliography
\end{document}
\fi
