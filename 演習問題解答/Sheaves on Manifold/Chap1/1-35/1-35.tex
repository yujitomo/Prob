\ifcsname Chap\endcsname\else
\documentclass[uplatex,dvipdfmx]{jsarticle}
\newcommand{\StylePath}{\ifcsname AllKS\endcsname KS-Style/KS-Style.sty\else
\ifcsname Chap\endcsname ../KS-Style/KS-Style.sty\else
../../KS-Style/KS-Style.sty\fi\fi}
\input{\StylePath}

\KSset{1}{35}
\setcounter{section}{\value{KSS}-1}
\begin{document}
\maketitle\HeaderCommentA
\section{\KSsection{section}}
\setcounter{prob}{\value{KSP}-1}
\fi

\begin{prob}\label{1.35}
  \(\hat{\mcC} = \sfSet^{\mcC^{\op}}\)を前層圏とする。
  \begin{enumerate}
    \item \label{1.35.1}
    \(I\)を有向集合、
    \(X_i\)を\(I\)で添字付けられた圏\(\mcC\)の図式とする。
    \(X\mapsto \colim_{i\in I}\Hom_{\mcC}(X,X_i)\)
    により定まる\(\hat{\mcC}\)の対象
    \(``\colim"_{i\in I} X_i\)
    (この記号の定義は\cite[Definition 1.11.4]{kashiwara2002sheaves}を参照) は
    \(I\)で添字付けられた図式
    \(h_{X_i}\in \hat{\mcC}\)の余極限であることを示せ。
    より詳しく、
    \(F\in \hat{\mcC}\)に対して以下の自然な同型を示せ:
    \[
    \Hom_{\hat{\mcC}}(``\colim"_{i\in I} X_i, F) \cong \colim_{i\in I} F(X_i).
    \]
    \item \label{1.35.2}
    \(Y_j\in \mcC\)を有向集合\(J\)で添字付けられた図式とする。
    以下の自然な同型を示せ:
    \[
    \Hom_{\hat{\mcC}}(``\colim"_{i\in I} X_i, ``\colim"_{j\in J} Y_j)
    \cong \lim_{i\in I}\colim_{j\in J}\Hom_{\mcC}(X_i,Y_j).
    \]
  \end{enumerate}
\end{prob}

\begin{rem*}
  本文では\ref{1.35.2}の左辺の右側の\(``\colim"\)が
  たんに\(\colim\)と表記されていたが、これは\(``"\)をつけ忘れた?
\end{rem*}

\begin{proof}
  \ref{1.35.1}を示す。
  函手圏の余極限は各点ごとに計算されるので
  \(``\colim"_{i\in I} X_i\cong \colim_{i\in I}h_{X_i}\)が従う。
  さらにこれがわかると、余極限の定義と米田の補題より、
  \begin{align*}
    \Hom_{\hat{\mcC}}(``\colim"_{i\in I} X_i, F)
    &\cong \Hom_{\hat{\mcC}}(\colim_{i\in I} X_i, F) \\
    &\cong \lim_{i\in I}\Hom_{\hat{\mcC}}(h_{X_i}, F) \\
    &\cong \lim_{i\in I}F(X_i)
  \end{align*}
  が従う。
  以上で\ref{1.35.1}の証明を完了する。

  \ref{1.35.2}を示す。
  素直に計算すると、
  \begin{align*}
    \Hom_{\hat{\mcC}}(``\colim"_{i\in I} X_i, ``\colim"_{j\in J} h_{Y_j})
    &\overset{\star}{\cong}
    \Hom_{\hat{\mcC}}(\colim_{i\in I} h_{X_i}, \colim_{j\in J} h_{Y_j}) \\
    &\overset{*}{\cong}
    \lim_{i\in I}\Hom_{\hat{\mcC}}(h_{X_i}, \colim_{j\in J} h_{Y_j}) \\
    &\overset{\scriptstyle\spadesuit}{\cong}
    \lim_{i\in I}(\colim_{j\in J} h_{Y_j})(X_i) \\
    &\overset{\scriptstyle\clubsuit}{\cong}
    \lim_{i\in I}\colim_{j\in J}\Hom_{\mcC}(X_i,Y_j)
  \end{align*}
  となる。
  ただしここで、\(\star\)の部分に\ref{1.35.1}を用い、
  \(*\)の部分に余極限の定義を用い、
  \(\spadesuit\)の部分に米田の補題を用い、
  \(\clubsuit\)の部分に函手圏での余極限が各点ごとに計算されることを用いた。
  以上で\ref{1.35.2}の証明を完了し、
  \autoref{1.35}の解答を完了する。
\end{proof}




\ifcsname Chap\endcsname\else
\printbibliography
\end{document}
\fi
