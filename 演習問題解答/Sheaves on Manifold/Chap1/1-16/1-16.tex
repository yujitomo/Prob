\ifcsname Chap\endcsname\else
\documentclass[uplatex,dvipdfmx]{jsarticle}
\newcommand{\StylePath}{\ifcsname AllKS\endcsname KS-Style/KS-Style.sty\else
\ifcsname Chap\endcsname ../KS-Style/KS-Style.sty\else
../../KS-Style/KS-Style.sty\fi\fi}
\input{\StylePath}

\KSset{1}{16}
\setcounter{section}{\value{KSS}-1}
\begin{document}
\maketitle
\HeaderCommentA
\section{\KSsection{section}}
\setcounter{prob}{\value{KSP}-1}
\fi


\begin{prob}\label{1.16}
  \(\mcC\)を加法圏とする。
  \begin{enumerate}
    \item \label{1.16.1}
    \(X\in \Ch^-(\mcC), Y\in \Ch^+(\mcC)\)とする。
    以下の等式を証明せよ:
    \begin{align*}
      &Z^0(\Tot(\Hom_{\mcC}(X,Y))) = \Hom_{\Ch(\mcC)}(X,Y), \\
      &B^0(\Tot(\Hom_{\mcC}(X,Y))) = \Ht(X,Y), \\
      &H^0(\Tot(\Hom_{\mcC}(X,Y))) = \Hom_{\sfK(\mcC)}(X,Y).
    \end{align*}
    ただしここで\(\Hom_{\mcC}(X,Y)\)は二重複体とみなしている。
    \item \label{1.16.2}
    さらに\(\mcC\)がアーベル圏であり、
    十分入射的対象を持つか、または十分射影的対象を持つと仮定する。
    \(X\in \sfD^-(\mcC),Y\in \sfD^+(\mcC)\)に対して、
    次の等式を示せ:
    \[
    H^0(R\Hom_{\mcC}(X,Y)) = \Hom_{\sfD(\mcC)}(X,Y).
    \]
  \end{enumerate}
\end{prob}

\begin{proof}
  \ref{1.16.1}を示す。
  \(H^{i,j} \dfn \Hom_{\mcC}(X^{-i},Y^j)\)とおけば、
  \(X\in \Ch^-(\mcC), Y\in \Ch^+(\mcC)\)であることから、
  二重複体\(H^{i,j}\)は本文の条件\cite[(1.9.2)]{kashiwara2002sheaves}を満たし、
  \(\Ch^2_f(\mcC)\)に属する。
  \(f:X\to Y\)を\(\Ch(\mcC)\)の射とすると、
  \(f\)は\(\mcC\)の射の族\(f^n:X^n\to Y^n\)であって
  \(f^n\circ d_X^{n-1} = d_Y^{n-1}\circ f^{n-1}\)を満たすものである。
  よって、とくに\(f\in \bigoplus_{i+j = 0} H^{i,j} = \Tot^0(H^{\bullet,\bullet})\)であり、
  等式\(f^n\circ d_X^{n-1} = d_Y^{n-1}\circ f^{n-1}\)はさらに
  \(f\)が\(Z^0(\Tot(H^{\bullet,\bullet}))\)に属することを意味する。
  以上で\ref{1.16.1}の一つ目の等式が従う。
  \(f:X\to Y\)が homotopic to zero であるとする。
  このとき、ある\(\mcC\)の射の族\(s^n:X^n\to Y^{n-1}\)が存在して
  \(f^n = s^{n+1}\circ d_X^n + d_Y^{n-1} \circ s^n\)となる。
  射の族\(s=(s^n)_{n\in \Z}\)は\(\bigoplus_{i+j=-1}H^{i,j}\)に属し、
  等式\(f^n = s^{n+1}\circ d_X^n + d_Y^{n-1} \circ s^n\)は
  \(\Tot^{-1}(H^{\bullet,\bullet}))\to \Tot^0(H^{\bullet,\bullet}))\)
  での\(s\)の像が\(f\in \Tot^0(H^{\bullet,\bullet}))\)となることを意味する。
  以上で\ref{1.16.1}の二つ目の等式が従う。
  \ref{1.16.1}の三つ目の等式は\ref{1.16.1}の一つ目と二つ目の等式より直ちに従う。
  以上で\ref{1.16.1}の証明を完了する。

  \ref{1.16.2}を示す。
  \(\mcC\)が十分射影的対象を持つ場合、
  \(\mcC^{\op}\)を考えることによって、
  \(\mcC\)が十分入射的対象を持つ場合に帰着される
  (cf. \cite[Remark 1.10.10]{kashiwara2002sheaves})。
  よって、\ref{1.16.2}を示すためには、
  \(\mcC\)が十分入射的対象を持つと仮定しても一般性を失わない。
  \autoref{1.15}の意味で\(S_Y\)という記号を用いる。
  \(\mcC\)は十分入射的対象を持つので、
  \autoref{1.15.1}より
  \(\Hom_{\sfD(\mcC)}(X,Y) \cong \colim_{Y'\in S_Y}\Hom_{\sfK(\mcC)}(X,Y')\)
  となり、また\autoref{1.15.2}より
  \(R\Hom_{\mcC}(X,Y) \cong \colim_{Y'\in S_Y}\Tot(\Hom_{\sfK(\mcC)}(X,Y'))\)
  となる。
  \(H^0\)を取ることで、
  \[
  H^0(R\Hom_{\mcC}(X,Y))
  \cong H^0(\colim_{Y'\in S_Y}\Tot(\Hom_{\sfK(\mcC)}(X,Y')))
  \]
  が従うが、\(S_Y\)はfilteredであるから、余極限は\(H^0\)と可換して、
  \[
  H^0(R\Hom_{\mcC}(X,Y)) \cong
  \colim_{Y'\in S_Y}H^0(\Tot(\Hom_{\sfK(\mcC)}(X,Y')))
  \]
  が従う。
  よって、\ref{1.16.1}の最後の等式で\(Y'\in S_Y\)に渡り余極限をとれば、
  \begin{align*}
    H^0(R\Hom_{\mcC}(X,Y))
    &\cong \colim_{Y'\in S_Y}H^0(\Tot(\Hom_{\sfK(\mcC)}(X,Y'))) \\
    &\cong \colim_{Y'\in S_Y}\Hom_{\sfK(\mcC)}(X,Y') \\
    &\cong \Hom_{\sfD(\mcC)}(X,Y)
  \end{align*}
  が従う。
  以上で\ref{1.16.2}の証明を完了し、
  \autoref{1.16}の解答を完了する。
\end{proof}





\ifcsname Chap\endcsname\else
\printbibliography
\end{document}
\fi
