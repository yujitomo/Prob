\ifcsname Chap\endcsname\else
\documentclass[uplatex,dvipdfmx]{jsarticle}
\newcommand{\StylePath}{\ifcsname AllKS\endcsname KS-Style/KS-Style.sty\else
\ifcsname Chap\endcsname ../KS-Style/KS-Style.sty\else
../../KS-Style/KS-Style.sty\fi\fi}
\input{\StylePath}

\KSset{1}{1}

\setcounter{section}{\value{KSS}-1}

\begin{document}
\maketitle
\HeaderCommentA
\section{\KSsection{section}}
\setcounter{prob}{\value{KSP}-1}
\fi

\begin{prob}\label{1.1}
  \(\mcC\)を加法圏とする。
  このとき、合成が双線型となるような各\(\Hom_{\mcC}\)に入るアーベル群の構造は一意的であることを示せ。
\end{prob}

\begin{proof}
  \(\mcC\)の加法圏の定義による\(\Hom\)の加法をたんに\(+\)で表し、
  問いの性質を満たす加法を\(+_1\)で表すことにする。
  \(X,Y\in \mcC\)とする。
  合成\(\Hom_{\mcC}(X,0)\times \Hom_{\mcC}(0,Y)\to \Hom_{\mcC}(X,Y)\)
  は\(+_1\)に関して (\(+\)に関しても) 双線型であるから、
  その像として定まる合成射\(X\to 0 \to Y\)は
  \(+_1\)に関する (同様に、\(+\)に関する) 単位元である。
  よって\(0\)は\(+_1\)に関する単位元である。
  自然な同型と加法
  \[
  \Hom_{\mcC}(X,Y\times Y) \xrightarrow{\sim}
  \Hom_{\mcC}(X,Y) \times \Hom_{\mcC}(X,Y) \xrightarrow{+} \Hom_{\mcC}(X,Y)
  \]
  により射\(m:Y\times Y\to Y\)を得る。
  二つの射\(f,g\in \Hom_{\mcC}(X,Y)\)に対して、
  \((f,g)\in \Hom_{\mcC}(X,Y\times Y)\)と\(m\)を合成すると、
  \(m\)の定義により\(m\circ (f,g) = f+g\)である。
  一方、\(m\)を合成する射
  \(\Hom_{\mcC}(X,Y\times Y) \to \Hom_{\mcC}(X,Y)\)は、
  \(+_1\)の定義により\(+_1\)と可換する。
  \(0\)が\(+_1\)の単位元であることから、\((f,g) = (f,0) +_1 (0,g)\)であるので、
  従って、
  \[
  f+g = m\circ (f,g) = m\circ [(f,0)+_1(0,g)]
  = m\circ (f,0) +_1 m\circ (0,g)
  = f+_1 g
  \]
  がわかる。
  以上で\autoref{1.1}の証明を完了する。
\end{proof}

\ifcsname Chap\endcsname\else
\printbibliography
\end{document}
\fi
