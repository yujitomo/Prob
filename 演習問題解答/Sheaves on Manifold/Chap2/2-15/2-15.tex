\ifcsname Chap\endcsname\else
\documentclass[uplatex,dvipdfmx]{jsarticle}
\newcommand{\StylePath}{\ifcsname AllKS\endcsname KS-Style/KS-Style.sty\else
\ifcsname Chap\endcsname ../KS-Style/KS-Style.sty\else
../../KS-Style/KS-Style.sty\fi\fi}
\input{\StylePath}

\KSset{2}{15}
\setcounter{section}{\value{KSS}-1}
\begin{document}
\maketitle
\HeaderCommentA
\section{\KSsection{section}}
\setcounter{prob}{\value{KSP}-1}

\fi



\begin{prob}\label{2.15}
  \begin{enumerate}
    \item \label{2.15.1}
    \(F^{\bullet}\)を下に有界な\(X\)上の層の複体とする。
    自然な射
    \(H^j(\Gamma(X,F^{\bullet})) \to H^j(R\Gamma(X,F^{\bullet}))\)
    を構成せよ。
    \item \label{2.15.2}
    \(\mcU = \{U_i\}_i\)を\(X\)の開被覆として、
    \(F\)を\(X\)上の層とする。
    自然な射\(H^j(C^{\bullet}(\mcU,F)) \to H^j(X,F)\)を構成せよ。
  \end{enumerate}
\end{prob}

\begin{proof}
  \ref{2.15.1}を示す。
  入射的層からなる複体へのモノな擬同型\(F^{\bullet}\xrightarrow{\text{qis}}I^{\bullet}\)
  をとれば複体の射
  \(\Gamma(X,F^{\bullet})\to \Gamma(X,I^{\bullet})\cong R\Gamma(X,F^{\bullet})\)
  \(\Gamma(X,F^{\bullet})\to \Gamma_(X,I^{\bullet})\cong R\Gamma(X,F^{\bullet})\)
  が得られるので、
  \(j\)次コホモロジーをとることによって射
  \(H^j(\Gamma(X,F^{\bullet})) \to H^j(R\Gamma(X,F^{\bullet}))\)
  を得る。
  以上で\ref{2.15.1}の証明を完了する。

  \ref{2.15.2}を示す。
  \(F\)を\(0\)次だけが\(F\)で他が\(0\)である自明な複体とみなすと、
  本文\cite[Proposition 2.8.4]{kashiwara2002sheaves}より、augmentation map
  \(\delta:F\xrightarrow{\text{qis}} \mcC^{\bullet}(\mcU,F)\)
  は擬同型である。
  よって\(\sfD^+(\Ab(X))\)の同型射
  \(R\Gamma(X,\delta):R\Gamma(X,F)\xrightarrow{\sim}
  R\Gamma(X,\mcC^{\bullet}(\mcU,F))\)を得る。
  \ref{2.15.1}を\(F^{\bullet} = \mcC^{\bullet}(\mcU,F)\)に対して適用すると、
  \(\Gamma(X,\mcC^{\bullet}(\mcU,F)) \cong C^{\bullet}(\mcU,F)\)
  であるので、射
  \(H^j(C^{\bullet}(\mcU,F)) \to H^j(R\Gamma(X,\mcC^{\bullet}(\mcU,F))\)
  を得る。
  これに\(H^j(R\Gamma(X,\delta)^{-1})\)を合成することで
  射\(H^j(C^{\bullet}(\mcU,F))\to H^j(X,F)\)を得る。
  以上で\ref{2.15.2}の証明を完了し、
  \autoref{2.15}の解答を完了する。
\end{proof}




\ifcsname Chap\endcsname\else
\printbibliography
\end{document}
\fi
