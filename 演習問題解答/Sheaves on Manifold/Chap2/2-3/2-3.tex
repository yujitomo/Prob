\ifcsname Chap\endcsname\else
\documentclass[uplatex,dvipdfmx]{jsarticle}
\newcommand{\StylePath}{\ifcsname AllKS\endcsname KS-Style/KS-Style.sty\else
\ifcsname Chap\endcsname ../KS-Style/KS-Style.sty\else
../../KS-Style/KS-Style.sty\fi\fi}
\input{\StylePath}

\KSset{2}{3}

\setcounter{section}{\value{KSS}-1}

\begin{document}
\maketitle
\HeaderCommentA
\section{\KSsection{section}}
\setcounter{prob}{\value{KSP}-1}
\fi

\begin{prob}\label{2.3}
  \begin{enumerate}
    \item \label{2.3.1}
    \(U\subset X\)を開部分集合、
    \(x\in\bar{U}\setminus U\)として、
    層\(\Z_U\)について考えることによって
    \(\inHom(F,G)_x\cong \Hom(F_x,G_x)\)は一般には正しくないことを示せ。
    \item \label{2.3.2}
    次を満たす\(X\)上の層\(F\)と閉部分集合\(Z\subset X\)と開部分集合\(U\)の例を与えよ:
    \(Z\cap U=\emptyset\)であり、
    \(R\Gamma_Z(F_U)\neq 0\)である。
    \(\Gamma_Z(F_U)\)であることを確認し、
    このことから、一般に合成函手の導来函手が
    導来函手の合成とは異なることを帰結せよ。
  \end{enumerate}
\end{prob}

\begin{proof}
  \ref{2.3.1}を示す。
  \(F=\Z_U\)とおき、\(G\)は任意の層とする。
  \(x\not\in U\)なので\(F_x=0\)であり、
  従って、このとき、\(\Hom(F_x,G_x)=0\)が成り立つ。
  また、各開集合\(V\subset X\)に対して自然に
  \[
  \inHom(F,G)(V) = \inHom(\Z_U,G)(V) = \Hom(\Z_U|_V,G|_V)
  = \Hom(\Z_{U\cap V},G|_V) \cong G(U\cap V)
  \]
  が成り立つので、
  \(\inHom(F,G)\cong \Gamma_U(G)\)が成り立つ。
  従って、たとえば\(X=\R,U=\R\setminus \{0\},x=0,G=\Z\)とすると、
  \[
  \inHom(\Z_U,\Z)_x \cong \Gamma_U(G)_x
  \cong \Z\oplus\Z \neq 0 = \Hom(F_x,G_x)
  \]
  である。
  以上で\ref{2.3.1}の証明を完了する。

  \ref{2.3.2}を示す。
  まず一般に\(Z\cap U=\emptyset\)であれば、
  \(F_U\)の各切断は\(U\)の中に台を持つので
  \(\Gamma_Z(F_U)=0\)が成り立つ。
  \(X=\R_{\geq 0}, U=\R_{>0}\subset X, Z=\{0\}\)
  として\(F=\Z_X\)を定数層とする。
  このとき、\(F\)は定数層なので、
  各開集合\(V\subset X\)に対して
  \(s\in F(V)\)で\(Z\)の外で\(0\)となるものは\(0\)しかない
  (\(0\in V\)であり、\(s_0=n\neq 0\)であれば、
  \(0\)を含む\(V\)の連結成分の上で\(s=n\)である)。
  従って\(\Gamma_Z(F)=0\)である。
  完全列\(0\to F_U\to F\to F_Z\to 0\)に函手\(\Gamma_Z(-)\)を施すことにより、
  同型射
  \(\Z_x \cong \Gamma_Z(F_Z)\xrightarrow{\sim} H^1_Z(F_U)\)を得る。
  従ってこれが所望の例を与える。
  以上で\ref{2.3.2}の証明を完了し、
  \autoref{2.3}の解答を完了する。
\end{proof}





\ifcsname Chap\endcsname\else
\printbibliography
\end{document}
\fi
