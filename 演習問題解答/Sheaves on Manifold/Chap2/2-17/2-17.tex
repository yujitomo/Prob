\ifcsname Chap\endcsname\else
\documentclass[uplatex,dvipdfmx]{jsarticle}
\newcommand{\StylePath}{\ifcsname AllKS\endcsname KS-Style/KS-Style.sty\else
\ifcsname Chap\endcsname ../KS-Style/KS-Style.sty\else
../../KS-Style/KS-Style.sty\fi\fi}
\input{\StylePath}

\KSset{2}{17}
\setcounter{section}{\value{KSS}-1}
\begin{document}
\maketitle
\HeaderCommentA
\section{\KSsection{section}}
\setcounter{prob}{\value{KSP}-1}

\fi

\begin{prob}\label{2.17}
  \(X\)を局所コンパクト空間、
  \(\mcR\)を\(X\)上の (可換) 環の層で\(\wgld(\mcR)<\infty\)であるものとし、
  \(Z_1,Z_2\subset X\)を局所閉部分集合とする。
  \begin{enumerate}
    \item \label{2.17.1}
    \(F_1,F_2\in \sfD^+(\mcR)\)に対し、
    自然な射
    \[
    R\Gamma_{Z_1}(F_1)\otimes_{\mcR}^L R\Gamma_{Z_2}(F_2)
    \to R\Gamma_{Z_1\cap Z_2}(F_1\otimes_{\mcR}^L F_2)
    \]
    を構成せよ。
    \item \label{2.17.2}
    \(A\)を可換環として、\(\mcR=A_X\)であると仮定せよ。
    \(F_1,F_2\in \sfD^+(\mcR)\)に対し、
    自然な射
    \[
    R\Gamma_{Z_1}(X,F_1)\otimes_A^L R\Gamma_{Z_2}(X,F_2)
    \to R\Gamma_{Z_1\cap Z_2}(X,F_1\otimes_A^L F_2)
    \]
    を構成し、各\(p,q\in \Z\)に対して
    \[
    H^p_{Z_1}(X,F_1)\otimes_A H^q_{Z_2}(X,F_2)
    \to H^{p+q}_{Z_1\cap Z_2}(X,F_1\otimes_A^L F_2)
    \]
    を構成せよ。
    最後の射は\textbf{cup積}と呼ばれる。
  \end{enumerate}
\end{prob}

\begin{proof}
  \ref{2.17.1}を示す。
  本文\cite[同型 (2.6.9)]{kashiwara2002sheaves}より、自然に
  \(R\Gamma_{Z_i}(F_i) \cong R\inHom_{\mcR}(\mcR_{Z_i},F_i)\)が成り立つ。
  また、本文\cite[射 (2.6.11)]{kashiwara2002sheaves}より、自然な射
  \begin{align*}
    R\inHom_{\mcR}(\mcR_{Z_1},F_1) \otimes_{\mcR}^L R\inHom_{\mcR}(\mcR_{Z_2},F_2)
    &\to R\inHom_{\mcR}(\mcR_{Z_1},F_1\otimes_{\mcR}^L R\inHom_{\mcR}(\mcR_{Z_2},F_2)) \\
    &\to R\inHom_{\mcR}(\mcR_{Z_1},R\inHom_{\mcR}(\mcR_{Z_2},F_1\otimes_{\mcR}^L F_2))
  \end{align*}
  を得る。
  さらに本文\cite[Proposition 2.6.3 (ii)]{kashiwara2002sheaves}より、
  自然な同型
  \[
  R\inHom_{\mcR}(\mcR_{Z_1},R\inHom_{\mcR}(\mcR_{Z_2},F_1\otimes_{\mcR}^L F_2))
  \cong R\inHom_{\mcR}(\mcR_{Z_1}\otimes_{\mcR}^L \mcR_{Z_2},F_1\otimes_{\mcR}^L F_2))
  \]
  を得る。
  ここで\(\mcR_{Z_i}\)は \(\mcR\)-flat であるので、自然に
  \(\mcR_{Z_1}\otimes_{\mcR}^L \mcR_{Z_2} \cong
  \mcR_{Z_1}\otimes_{\mcR} \mcR_{Z_2} \cong \mcR_{Z_1\cap Z_2}\)
  が成り立つ。
  再び本文\cite[同型 (2.6.9)]{kashiwara2002sheaves}を用いることで、自然に
  \(R\Gamma_{Z_1\cap Z_2}(F_1\otimes_{\mcR}^L F_2) \cong
  R\inHom_{\mcR}(\mcR_{Z_1\cap Z_2},F_1\otimes_{\mcR}^L F_2)\)
  が成り立つので、
  これらを組み合わせることによって所望の射を得る。
  以上で\ref{2.17.1}の証明を完了する。

  \ref{2.17.2}を示す。
  \(f:X\to \{\mathrm{pt}\}\)を自明な射とする。
  まず、\(Z_1=Z_2=X\)の場合に証明する。
  この場合、\(R\Gamma_{Z_i}(X,-)\cong Rf_*(-)\)が成り立つ。
  今、\(\mcR = A_X = f^{-1}A\)は定数層であるので、
  従って、本文\cite[Proposition 2.6.4 (ii)]{kashiwara2002sheaves}より、自然に
  \[
  \Hom_{\sfD^+(A)}(Rf_*F_1\otimes_A^L Rf_*F_2, Rf_*(F_1\otimes_{A_X}^L F_2))
  \cong \Hom_{\sfD^+(A_X)}(f^{-1}(Rf_*F_1\otimes_A^L Rf_*F_2),F_1\otimes_{A_X}^L F_2)
  \]
  が成り立つ。
  また、本文\cite[Proposition 2.6.5]{kashiwara2002sheaves}より、自然に
  \(f^{-1}(Rf_*F_1\otimes_A^L Rf_*F_2)\cong
  (f^{-1}Rf_*F_1)\otimes_{A_X}^L (f^{-1}Rf_*F_2)\)
  が成り立つ。
  本文\cite[射 (2.6.17)]{kashiwara2002sheaves}より、自然な射
  \((f^{-1}Rf_*F_1)\otimes_{A_X}^L (f^{-1}Rf_*F_2)\to F_1\otimes_{A_X}^LF_2\)
  があり、以上より射
  \begin{align*}
    \Hom_{\sfD^+(A_X)}(F_1\otimes_{A_X}^LF_2,F_1\otimes_{A_X}^L F_2)
    &\to \Hom_{\sfD^+(A_X)}((f^{-1}Rf_*F_1)\otimes_{A_X}^L (f^{-1}Rf_*F_2)
    ,F_1\otimes_{A_X}^L F_2) \\
    &\cong \Hom_{\sfD^+(A_X)}(f^{-1}(Rf_*F_1\otimes_A^L Rf_*F_2), F_1\otimes_{A_X}^L F_2) \\
    &\cong \Hom_{\sfD^+(A)}(Rf_*F_1\otimes_A^L Rf_*F_2, Rf_*(F_1\otimes_{A_X}^L F_2))
  \end{align*}
  を得る。
  \(\id\)の行き先が射
  \(Rf_*F_1\otimes_A^L Rf_*F_2\to Rf_*(F_1\otimes_{A_X}^L F_2)\)を与える。
  以上で\(Z_1=Z_2=X\)の場合の1つ目の射の構成を完了する。
  一般の場合、\ref{2.17.1}の自然な射に対して\(R\Gamma(X,-)\)を適用し、
  \(Z_1=Z_2=X\)の場合に得られた射と合成することによって、射
  \begin{align*}
    R\Gamma_{Z_1}(X,F_1)\otimes_A^L R\Gamma_{Z_2}(X,F_2)
    &\cong R\Gamma(X,R\Gamma_{Z_1}(F_1))\otimes_A^L R\Gamma(X,R\Gamma_{Z_2}(F_2)) \\
    &\to R\Gamma(X,R\Gamma_{Z_1}(F_1)\otimes_{A_X}^L R\Gamma_{Z_2}(F_2)) \\
    &\to R\Gamma(X,R\Gamma_{Z_1\cap Z_2}(F_1\otimes_{A_X}^L F_2)) \\
    &\cong R\Gamma_{Z_1\cap Z_2}(X,F_1\otimes_{A_X}^L F_2)
  \end{align*}
  を得る。
  以上で一般の場合の1つ目の射の構成を完了する。
  二つ目の射は、\autoref{1.24.1}で
  \(F=\otimes_A, X=R\Gamma_{Z_1}(X,F_1), Y=R\Gamma_{Z_2}(X,F_2)\)
  とすることにより、自然な射
  \begin{align*}
    H^p_{Z_1}(X,F_1)\otimes_A H^q_{Z_2}(X,F_2)
    &\to H^{p+q}(R\Gamma_{Z_1}(X,F_1)\otimes_A^L R\Gamma_{Z_2}(X,F_2)) \\
    &\to H^{p+q}(R\Gamma_{Z_1\cap Z_2}(X,F_1\otimes_{A_X}^L F_2))
  \end{align*}
  を得る。これが所望の射である。
  以上で\ref{2.17.2}の証明を完了し、
  \autoref{2.17}の解答を完了する。
\end{proof}







\ifcsname Chap\endcsname\else
\printbibliography
\end{document}
\fi
