\documentclass[uplatex,dvipdfmx]{jsarticle}

\usepackage{amssymb}
\usepackage{amsmath}
\usepackage{mathrsfs}
\usepackage{amsfonts}
\usepackage{mathtools}

\usepackage{xcolor}
\usepackage[dvipdfmx]{graphicx}


%%%%%ハイパーリンク
\usepackage[setpagesize=false]{hyperref}
\usepackage{aliascnt}
\hypersetup{
    colorlinks=true,
    citecolor=blue,
    linkcolor=blue,
    urlcolor=blue,
}
%%%%%ハイパーリンク




%%%%%図式
\usepackage{tikz}%%%図
\usetikzlibrary{arrows}
\usepackage{amscd}%%%簡単な図式
%%%%%図式


%%%%%%%%%%%%定理環境%%%%%%%%%%%%
%%%%%%%%%%%%定理環境%%%%%%%%%%%%
%%%%%%%%%%%%定理環境%%%%%%%%%%%%


\usepackage{amsthm}
\theoremstyle{definition}
\newtheorem{thm}{定理}[section]
\newcommand{\thmautorefname}{定理}

\newaliascnt{prop}{thm}%%%カウンター「prop」の定義(thmと同じ)
\newtheorem{prop}[prop]{命題}
\aliascntresetthe{prop}
\newcommand{\propautorefname}{命題}%%%カウンター名propは「命題」で参照する

\newaliascnt{cor}{thm}
\newtheorem{cor}[cor]{系}
\aliascntresetthe{cor}
\newcommand{\corautorefname}{系}

\newaliascnt{lem}{thm}
\newtheorem{lem}[lem]{補題}
\aliascntresetthe{lem}
\newcommand{\lemautorefname}{補題}

\newaliascnt{rem}{thm}
\newtheorem{rem}[rem]{注意}
\aliascntresetthe{rem}
\newcommand{\remautorefname}{注意}

\newaliascnt{defi}{thm}
\newtheorem{defi}[defi]{定義}
\aliascntresetthe{defi}
\newcommand{\defiautorefname}{定義}

\newaliascnt{eg}{thm}
\newtheorem{eg}[eg]{例}
\aliascntresetthe{eg}
\newcommand{\egautorefname}{例}

\newaliascnt{obs}{thm}
\newtheorem{obs}[obs]{観察}
\aliascntresetthe{obs}
\newcommand{\obsautorefname}{観察}


\newaliascnt{prob}{thm}
\newtheorem{prob}[prob]{問題}
\aliascntresetthe{prob}
\newcommand{\probautorefname}{問題}


%%%%%%%番号づけない定理環境
\newtheorem*{exam*}{例}
\newtheorem*{recall*}{Recall}
\newtheorem*{rem*}{注意}
\newtheorem*{kansou*}{感想}
\newtheorem*{question*}{疑問}
\newtheorem*{defi*}{定義}


%%%証明環境の調整
\makeatletter
\renewenvironment{proof}[1][\proofname]{
  \pushQED{\qed}%
  \normalfont \topsep6\p@\@plus6\p@\relax
  \trivlist
  \item[\hskip\labelsep
    #1\@addpunct{\textbf{.}}]\ignorespaces
}{%
  \popQED\endtrivlist\@endpefalse
}
\makeatother
\providecommand{\proofname}{証明}


%%%%%%%%%%%%定理環境%%%%%%%%%%%%
%%%%%%%%%%%%定理環境%%%%%%%%%%%%
%%%%%%%%%%%%定理環境%%%%%%%%%%%%




%%%%%箇条書き環境
\usepackage[]{enumitem}

\makeatletter
\AddEnumerateCounter{\fnsymbol}{\c@fnsymbol}{9}%%%%fnsymbolという文字をenumerate環境のパラメーターで使えるようにする。
\makeatother

\renewcommand{\theenumi}{(\arabic{enumi})}%%%%%itemは(1),(2),(3)で番号付ける。
\renewcommand{\theenumii}{(\roman{enumii})}
\renewcommand{\labelenumi}{\theenumi}
\renewcommand{\labelenumii}{\theenumii}
%%%%%箇条書き環境




\usepackage{latexsym}
\DeclareMathOperator{\Hom}{\mathrm{Hom}}
\newcommand{\inHom}{\mathcal{H}om}
\DeclareMathOperator{\End}{\mathrm{End}}
\DeclareMathOperator{\Isom}{Isom}
\DeclareMathOperator{\ISOM}{\mathbf{Isom}}
\DeclareMathOperator{\id}{\mathrm{id}}
\DeclareMathOperator{\im}{\mathrm{Im}}
\newcommand{\op}{\mathrm{op}}
\newcommand{\Ob}{\mathrm{Ob}}
\newcommand{\hd}{\mathrm{hd}}
\DeclareMathOperator{\Tot}{\mathrm{Tot}}
\DeclareMathOperator{\Ext}{\mathrm{Ext}}
\DeclareMathOperator{\Tor}{\mathrm{Tor}}

\DeclareMathOperator{\coker}{\mathrm{coker}}
\DeclareMathOperator{\colim}{\mathrm{colim}}
\DeclareMathOperator{\plim}{\mathrm{lim}}
\DeclareMathOperator{\rank}{\mathrm{rank}}
\DeclareMathOperator{\tr}{\mathrm{tr}}
\DeclareMathOperator{\codim}{\mathrm{codim}}
\DeclareMathOperator{\Ind}{\mathrm{Ind}}
\DeclareMathOperator{\Int}{\mathrm{Int}}

\DeclareMathOperator{\Spec}{\mathrm{Spec}}
\DeclareMathOperator{\Supp}{\mathrm{Supp}}
\DeclareMathOperator{\Proj}{\mathrm{Proj}}
\DeclareMathOperator{\Sym}{\mathrm{Sym}}
\DeclareMathOperator{\Sing}{\mathrm{Sing}}
\DeclareMathOperator{\red}{\mathrm{red}}
\DeclareMathOperator{\length}{\mathrm{length}}
\DeclareMathOperator{\hight}{\mathrm{ht}}
\DeclareMathOperator{\trdeg}{\mathrm{trdeg}}



\newcommand{\Ch}{\mathsf{Ch}}
\newcommand{\Sh}{\mathsf{Sh}}
\newcommand{\Ab}{\mathsf{Ab}}
\newcommand{\Mod}{\mathsf{Mod}}
\newcommand{\sfK}{\mathsf{K}}
\newcommand{\sfD}{\mathsf{D}}
\newcommand{\Ht}{\mathrm{Ht}}


\newcommand\C{\mathbb{C}}
\newcommand\R{\mathbb{R}}
\newcommand\Q{\mathbb{Q}}
\newcommand\Z{\mathbb{Z}}
\newcommand\N{\mathbb{N}}
\renewcommand\P{\mathbb{P}}
\newcommand\A{\mathbb{A}}
\renewcommand\L{\mathbb{L}}


\newcommand\sfSet{\mathsf{Set}}
\newcommand\sfTop{\mathsf{Top}}
\newcommand\sfRing{\mathsf{Ring}}



\newcommand\mcA{\mathcal{A}}
\newcommand\mcB{\mathcal{B}}
\newcommand\mcC{\mathcal{C}}
\newcommand\mcD{\mathcal{D}}
\newcommand\mcE{\mathcal{E}}
\newcommand\mcF{\mathcal{F}}
\newcommand\mcG{\mathcal{G}}
\newcommand\mcH{\mathcal{H}}
\newcommand\mcI{\mathcal{I}}
\newcommand\mcJ{\mathcal{J}}
\newcommand\mcK{\mathcal{K}}
\newcommand\mcM{\mathcal{M}}
\newcommand\mcN{\mathcal{N}}
\newcommand\mcO{\mathcal{O}}
\newcommand\mcP{\mathcal{P}}
\newcommand\mcQ{\mathcal{Q}}
\newcommand\mcR{\mathcal{R}}
\newcommand\mcS{\mathcal{S}}
\newcommand\mcT{\mathcal{T}}
\newcommand\mcU{\mathcal{U}}
\newcommand\mcW{\mathcal{W}}
\newcommand\mcX{\mathcal{X}}
\newcommand\mcY{\mathcal{Y}}



\newcommand\mfa{\mathfrak{a}}
\newcommand\mfb{\mathfrak{b}}
\newcommand\mfm{\mathfrak{m}}
\newcommand\mfn{\mathfrak{n}}
\newcommand\mfp{\mathfrak{p}}
\newcommand\mfq{\mathfrak{q}}
\newcommand\mfr{\mathfrak{r}}


\DeclareMathOperator{\OOO}{\mcO}

\newcommand\OA{\OOO_A}
\newcommand\OB{\OOO_B}
\newcommand\OC{\OOO_C}
\newcommand\OD{\OOO_D}
\renewcommand\OE{\OOO_E}
\newcommand\OF{\OOO_F}
\newcommand\OG{\OOO_G}
\newcommand\OH{\OOO_H}
\newcommand\OI{\OOO_I}
\newcommand\OJ{\OOO_J}
\newcommand\OK{\OOO_K}
\newcommand\OL{\OOO_L}
\newcommand\OM{\OOO_M}
\newcommand\ON{\OOO_N}
\newcommand\OP{\OOO_P}
\newcommand\OQ{\OOO_Q}
\newcommand\OR{\OOO_R}
\newcommand\OS{\OOO_S}
\newcommand\OT{\OOO_T}
\newcommand\OU{\OOO_U}
\newcommand\OV{\OOO_V}
\newcommand\OX{\OOO_X}
\newcommand\OY{\OOO_Y}
\newcommand\OZ{\OOO_Z}
\newcommand{\OO}[1]{\OOO_{#1}}
\newcommand{\OOP}[1]{\OO{\P({#1})}}
\newcommand{\OOA}[1]{\OO{\A({#1})}}

\newcommand{\loc}{\mathrm{loc}}


\newcommand{\gld}{\mathrm{gld}}
\newcommand{\wgld}{\mathrm{wgld}}

\newcommand{\rsa}{\rightsquigarrow}
\newcommand{\dto}{\dashrightarrow}
\renewcommand{\emptyset}{\varnothing}

\def\dfn{:\overset{\mbox{{\scriptsize def}}}{=}}
\newcommand{\deff}{:\hspace{-3pt}\overset{\text{def}}{\iff}}




%%%%%%%%%タイトル
\title{Sheaves on Manifolds 解答}
\author{ゆじとも}
\date{\today}

\begin{document}


\maketitle
\setcounter{subsection}{-1}


現時点 (=2021.01.28) で解いていない (\(\fallingdotseq\)解けていない) 問題:
\autoref{1.30} \ref{1.30.2}、\autoref{2.16} \ref{2.16.2}、
\autoref{2.21} 以降。


\begin{itemize}
  \item
  解きっぱなしで見直してないのでいっぱいミスがあると思います。
  参考にする場合は注意してください。
  ミスを発見した方は指摘していただければ幸いです。
  \item
  自然そうな仮定が本文中に書かれていないように見えた場合は、
  そのような仮定を置いた上で解いています。
  なので、その場合は問題文を少し変更して書いて、
  問題文の直後に Remark を置くようにしました。
  そのような問題であって、
  本文の指示通りの仮定で解くことができるものがあれば、
  指摘していただければ幸いです。
  \item
  私はこの分野の専門家ではありませんので、
  重ねて申し上げますが、
  本当にヤバいミスをしたまま放置している可能性は十分あります。
  また、解答も「最適解」から程遠いものもたくさんあるかと思います。
  そのようなもののうち、
  あまりに目に余るものがあれば、
  指摘していただければ幸いです。
  \item
  TeX的な問題点 (ハイパーリンクがPDFファイルで機能してないのとか)
  に関しては現時点 (=2021.02.01) では放置しています。
  そのような問題と向き合う機運が高まったら修正します。
  \item
  「本文」は柏原先生とSchapira先生
  (私はSchapira先生という方について全く存じ上げておりませんので
  「先生」という敬称が正しいものかわかりませんが)
  の共著である「Sheaves on Manifolds」を指しています。
  現時点 (=2021.02.01) では「本文定理hoge」のような形で参照していますが、
  ちゃんとTeX的にただしい参照の方法に変更する機運が高まったら修正します。
\end{itemize}



\newpage

\section{Homological Algebra}


\begin{prob}\label{prob: 1.1}
  \(\mcC\)を加法圏とする。
  このとき、合成が双線型となるような各\(\Hom_{\mcC}\)に入るアーベル群の構造は一意的であることを示せ。
\end{prob}

\begin{proof}
  \(\mcC\)の加法圏の定義による\(\Hom\)の加法をたんに\(+\)で表し、
  問いの性質を満たす加法を\(+_1\)で表すことにする。
  \(X,Y\in \mcC\)とする。
  合成\(\Hom_{\mcC}(X,0)\times \Hom_{\mcC}(0,Y)\to \Hom_{\mcC}(X,Y)\)
  は\(+_1\)に関して (\(+\)に関しても) 双線型であるから、
  その像として定まる合成射\(X\to 0 \to Y\)は
  \(+_1\)に関する (同様に、\(+\)に関する) 単位元である。
  よって\(0\)は\(+_1\)に関する単位元である。
  自然な同型と加法
  \[
  \Hom_{\mcC}(X,Y\times Y) \xrightarrow{\sim}
  \Hom_{\mcC}(X,Y) \times \Hom_{\mcC}(X,Y) \xrightarrow{+} \Hom_{\mcC}(X,Y)
  \]
  により射\(m:Y\times Y\to Y\)を得る。
  二つの射\(f,g\in \Hom_{\mcC}(X,Y)\)に対して、
  \((f,g)\in \Hom_{\mcC}(X,Y\times Y)\)と\(m\)を合成すると、
  \(m\)の定義により\(m\circ (f,g) = f+g\)である。
  一方、\(m\)を合成する射
  \(\Hom_{\mcC}(X,Y\times Y) \to \Hom_{\mcC}(X,Y)\)は、
  \(+_1\)の定義により\(+_1\)と可換する。
  \(0\)が\(+_1\)の単位元であることから、\((f,g) = (f,0) +_1 (0,g)\)であるので、
  従って、
  \[
  f+g = m\circ (f,g) = m\circ [(f,0)+_1(0,g)]
  = m\circ (f,0) +_1 m\circ (0,g)
  = f+_1 g
  \]
  がわかる。
  以上で示された。
\end{proof}



\begin{prob}\label{prob: 1.2}
  \(\mcC,\mcC'\)を二つの圏、
  \(F:\mcC\to \mcC', G:\mcC'\to \mcC\)を二つの函手とする。
  \begin{enumerate}
    \item \label{enumi: 1.2.1}
    次の二つの条件は同値であることを示せ:
    \begin{enumerate}
      \item \label{enumii: 1.2.1.1}
      函手の射\(\alpha:F\circ G \to \id_{\mcC'}, \beta:G\circ F \to \id_{\mcC}\)
      が存在して、任意の\(X\in \mcC, Y\in \mcC'\)に対し
      \[
      \id_{G(Y)} = G(\alpha_Y) \circ \beta_{G(Y)} \ , \
      \id_{F(X)} = \alpha_{F(X)} \circ F(\beta_X)
      \]
      となる。
      \item \label{enumii: 1.2.1.2}
      \(\Hom_{\mcC'}(F(-),?),\Hom_{\mcC}(-,G(?))\)
      は函手\(\mcC^{\op}\times \mcC' \to \sfSet\)として同型である。
    \end{enumerate}
    \item \label{enumi: 1.2.2}
    任意の函手\(F:\mcC\to \mcC'\)に対して、
    \(F\)の左 (または右) 随伴は、存在すれば、どの二つも函手として同型となる。
    \item \label{enumi: 1.2.3}
    任意の函手\(F:\mcC\to \mcC'\)に対して、
    \(F\)の左 (または右) 随伴が存在するための必要十分条件は、
    任意の対象\(Y\in \mcC'\)に対して
    函手\(X\mapsto \Hom_{\mcC'}(F(X),Y)\)
    (または\(X\mapsto \Hom_{\mcC'}(Y,F(X))\))
    が表現可能であることである。
  \end{enumerate}
\end{prob}

\begin{proof}
  \ref{enumi: 1.2.1}を示す。
  \ref{enumii: 1.2.1.1}を仮定する。
  \(F:\mcC\to \mcC'\)は函手なので、
  \(\mcC^{\op}\times\mcC\)から\(\sfSet\)への二つの函手の間の射
  \(\Hom_{\mcC}(-,?) \to \Hom_{\mcC'}(F(-),F(?))\)
  を引き起こす。
  これを\(\bar{F}\)と書く。
  同じく\(\bar{G}\)を定義する。
  函手の射
  \begin{align*}
    &P: \Hom_{\mcC'}(F(-),?) \xrightarrow{\bar{G}} \Hom_{\mcC}(G(F(-)),G(?))
    \xrightarrow{(??)\circ \beta_{(-)}} \Hom_{\mcC}(-,G(?)) \\
    &Q: \Hom_{\mcC}(-,G(?)) \xrightarrow{\bar{F}} \Hom_{\mcC'}(F(-),F(G(?)))
    \xrightarrow{\alpha_{(?)} \circ (??)} \Hom_{\mcC'}(F(-),?)
  \end{align*}
  を合成で定義する。
  すると、各対象\((X,Y)\in \mcC^{\op}\times \mcC'\)と
  \(f\in \Hom_{\mcC'}(F(X),Y), g\in \Hom_{\mcC}(X,G(Y))\)に対し、
  %\begin{align*}
    %Q_{(X,Y)}(P_{(X,Y)}(f))
    %&= \alpha_Y \circ F(G(f)\circ \beta_X))  \\
    %&= \alpha_Y \circ F(G(f)) \circ F(\beta_X) \\
    %&\overset{\bigstar}{=} f \circ \alpha_{F(X)} \circ F(\beta_X) \\
    %&\overset{*}{=} f \circ \id_{F(X)} \ = f, \\
    %& P_{(X,Y)}(Q_{(X,Y)}(g))
    %&= G(\alpha_Y \circ F(g)) \circ \beta_X \\
    %&= G(\alpha_Y) \circ G(F(g)) \circ \beta_X \\
    %&\overset{\bigstar}{=} G(\alpha_Y) \circ \beta_{G(Y)} \circ g \\
    %&\overset{*}{=} \id_Y \circ g \ = g \\
  %\end{align*}
  \begin{align*}
    Q_{(X,Y)}(P_{(X,Y)}(f))
    &= \alpha_Y \circ F(G(f)\circ \beta_X))
    & P_{(X,Y)}(Q_{(X,Y)}(g))
    &= G(\alpha_Y \circ F(g)) \circ \beta_X \\
    &= \alpha_Y \circ F(G(f)) \circ F(\beta_X)
    &&= G(\alpha_Y) \circ G(F(g)) \circ \beta_X \\
    &\overset{\bigstar}{=} f \circ \alpha_{F(X)} \circ F(\beta_X)
    &&\overset{\bigstar}{=} G(\alpha_Y) \circ \beta_{G(Y)} \circ g \\
    &\overset{*}{=} f \circ \id_{F(X)} \ = f,
    &&\overset{*}{=} \id_{G(Y)} \circ g \ = g
  \end{align*}
  となる。
  ただし、\(\bigstar\)の箇所で\(\alpha,\beta\)が自然変換であることを用い、
  \(*\)の箇所で\ref{enumii: 1.2.1.1}で仮定されている条件を用いた。
  以上より\(P,Q\)は函手の同型射である。
  よって\ref{enumii: 1.2.1.2}が示された。

  逆に\ref{enumii: 1.2.1.2}を仮定する。
  \(P:\Hom_{\mcC'}(F(-),?) \xrightarrow{\sim} \Hom_{\mcC}(-,G(?))\)
  を函手の同型射として、\(Q\dfn P^{-1}\)と置く。
  \(P\)が函手の射であることは、各射
  \([f:F(X) \to Y],[f':F(X')\to Y]\in \mcC',
  [g:Y\to Y']\in \mcC', [h:X\to X']\in \mcC\)
  に対して
  \(P_{(X,Y')}(g\circ f) = G(g)\circ P_{(X,Y)}(f),
  P_{(X',Y)}(f')\circ h = P_{(X,Y)}(f'\circ F(h))\)
  となることを意味する
  (以下の図式が可換である):
  \[
  \begin{CD}
    \Hom_{\mcC'}(F(X'),Y) @> (-)\circ F(h) >>
    \Hom_{\mcC'}(F(X),Y) @> g\circ (-) >>
    \Hom_{\mcC'}(F(X),Y') \\
    @V P_{(X',Y)} VV @V P_{(X,Y)} VV @VV P_{(X,Y')} V \\
    \Hom_{\mcC'}(X',G(Y)) @> (-)\circ h >>
    \Hom_{\mcC}(X,G(Y)) @> G(g) \circ (-) >>
    \Hom_{\mcC}(X,G(Y')).
  \end{CD}
  \]
  \(Q\)についても同様の等式が成立する
  (上の図式で縦向きの射が逆になったものが\(Q\)の場合)。
  次のように自然変換を定義する:
  \begin{align*}
    \alpha_{(?)} &\dfn Q_{(G(?),?)}(\id_{G(?)}): F(G(?)) \to (?), \\
    \beta_{(-)} &\dfn P_{(-,F(-))}(\id_{F(-)}): (-) \to G(F(-)).
  \end{align*}
  すると各\((X,Y)\in \mcC^{\op}\times \mcC'\)に対して
  \begin{align*}
    \alpha_{F(X)} \circ F(\beta_X)
    &= Q_{(G(F(X)),F(X))}(\id_{G(F(X))}) \circ F(\beta_X) \\
    &\overset{\bigstar_l}{=} Q_{(X,F(X))}(\id_{F(X)}) \circ \beta_X) \\
    &= Q_{(X,F(X))}(P_{(X,F(X))}(\id_{F(X)})) \ = \id_{F(X)} \\
    G(\alpha_Y)\circ \beta_{G(Y)}
    &= G(\alpha_Y) \circ P_{(G(Y),F(G(Y)))}(\id_{F(G(Y))})  \\
    &\overset{\bigstar_r}{=} P_{(G(Y),Y)}(\alpha_Y \circ \id_{F(G(Y))})  \\
    &= P_{(G(Y),Y)}(Q_{(G(Y),Y)}(\id_{G(Y)})) \ = \id_{G(Y)}
  \end{align*}
  となる。ただし\(\bigstar_l,\bigstar_r\)の箇所で上の図式の可換性を用いた
  (\(l\)は左、\(r\)は右側の四角形の可換性を用いている)。
  以上で\ref{enumi: 1.2.1}の証明を完了する。

  \ref{enumi: 1.2.2}を示す。
  \(G,G'\)がどちらも\(F\)の右随伴であれば、
  \ref{enumi: 1.2.1}における函手の自然同型を用いて
  \[
  \Hom_{\mcC}(-,G(?)) \cong \Hom_{\mcC'}(F(-),?) \cong \Hom_{\mcC}(-,G'(?))
  \]
  となるので、米田の補題によって\(G\cong G'\)がわかる。
  左随伴についても同様である。

  \ref{enumi: 1.2.3}を示す。
  各\(Y\in \mcC'\)について
  \(X\mapsto \Hom_{\mcC'}(F(X),Y)\)が表現可能であるとし、
  その表現対象を\(G(Y)\)と置く。
  すると\(X\)について自然な同型
  \(\Hom_{\mcC'}(F(X),Y)\cong \Hom_{\mcC}(X,G(Y))\)を得る。
  \(g:Y\to Y'\)を任意の\(\mcC'\)の射とする。
  すると\(g\)を合成することによって函手の射
  \(\Hom_{\mcC'}(F(-),Y) \to \Hom_{\mcC'}(F(-),Y')\)を得る。
  よって米田の補題によって一意的に射\(G(Y)\to G(Y')\)を得る。
  この射を\(G(g)\)と書く。
  すると\(G\)は函手であり、
  \(X\)について自然な同型\(\Hom_{\mcC'}(F(X),Y)\cong \Hom_{\mcC}(X,G(Y))\)は
  構成から\(Y\)についても自然である。
  これによって\(F\)の右随伴\(G\)を得る。
  左随伴に関しても同様である。
  以上ですべての解答を完了する。
\end{proof}





\begin{prob}\label{prob: 1.3}
  \(\mcC\)を各\(\Hom\)がアーベル群の構造を持ち、
  \(0\)対象を持ち、
  さらに任意の二つの対象に対する積を持つ圏とする
  (本文定義1.2.1における条件(i),(ii),(iii))。
  このとき、\(Z\in \mcC\)が函手
  \(W\mapsto \Hom_{\mcC}(X,W)\oplus \Hom_{\mcC}(Y,W)\)
  の表現対象であるための必要十分条件は、
  射\(i_1:X\to Z, i_2:Y\to Z, p_1: Z\to X, p_2:Z\to Y\)が存在し、
  \begin{align*}
    p_2\circ i_1 = 0, \ p_1 \circ i_2 = 0, \
    p_1\circ i_1 = \id_X, \ p_2\circ i_2 = \id_Y, \
    i_1\circ p_1 + i_2 \circ p_2 = \id_Z
  \end{align*}
  となることである。
\end{prob}

\begin{proof}
  必要性を示す。
  \(Z\in \mcC\)が函手
  \(W\mapsto \Hom_{\mcC}(X,W)\oplus \Hom_{\mcC}(Y,W)\)
  の表現対象であると仮定する。自然な全単射
  \(\Hom_{\mcC}(Z,Z) \xrightarrow{\sim} \Hom_{\mcC}(X,Z)\oplus \Hom_{\mcC}(Y,Z)\)
  による\(\id_Z\)の送り先を\((i_1,i_2)\)とする。
  自然な全単射
  \(\Hom_{\mcC}(Z,X) \xrightarrow{\sim} \Hom_{\mcC}(X,X)\oplus \Hom_{\mcC}(Y,X)\)
  により\((\id_X,0)\)へと写る射を\(p_1:Z\to X\)とし、
  自然な全単射
  \(\Hom_{\mcC}(Z,Y) \xrightarrow{\sim} \Hom_{\mcC}(X,Y)\oplus \Hom_{\mcC}(Y,Y)\)
  により\((0,\id_Y)\)へと写る射を\(p_2:Z\to Y\)とする。
  このとき、\(i_1,i_2,p_1,p_2\)の定義より、
  \[
  p_1\circ i_1 = \id_X, \ p_1\circ i_2 = 0, \
  p_2\circ i_1 = 0, \ p_2\circ i_2 = \id_Y
  \]
  であることがわかる。
  また、
  \begin{align*}
    &(i_1\circ p_1 + i_2\circ p_2) \circ i_1
    = i_1\circ p_1 \circ i_1 + i_2\circ p_2 \circ i_1
    = i_1 + 0 = i_1, \\
    &(i_1\circ p_1 + i_2\circ p_2) \circ i_2
    = i_1\circ p_1 \circ i_2 + i_2\circ p_2 \circ i_2
    = 0 + i_2 = i_2
  \end{align*}
  であるが、このような性質を満たす射\(Z\to Z\)は
  \(Z\)の普遍性によって\(\id_Z\)に限られる。
  従って\(i_1\circ p_1 + i_2\circ p_2 = \id_Z\)もわかる。
  以上で必要性の証明を完了する。

  十分性を示す。
  問いの条件を満たす射\(i_1,i_2,p_1,p_2\)が存在すると仮定する。
  \(i_1,i_2,p_1,p_2\)を合成することにより、
  \(W\)について自然な射
  \begin{align*}
    &\varphi: \Hom_{\mcC}(X,W) \oplus \Hom_{\mcC}(Y,W) \to \Hom_{\mcC}(Z,W),
    &&\varphi(f,g) \dfn f\circ p_1 + g\circ p_2, \\
    &\psi: \Hom_{\mcC}(Z,W) \to \Hom_{\mcC}(X,W) \oplus \Hom_{\mcC}(Y,W),
    &&\psi(h) \dfn (h\circ i_1, h\circ i_2) \\
  \end{align*}
  を得る。
  各\(f:X\to W, g:Y\to W, h:Z\to W\)について
  \begin{align*}
    \varphi(\psi(h)) &= \varphi(h\circ i_1, h\circ i_2)
    = h\circ i_1 \circ p_1 + h\circ i_2\circ p_2
    = h\circ (i_1 \circ p_1 + i_2 \circ p_2) = h\circ \id_Z = h \\
    \psi(\varphi(f,g)) &= \psi(f\circ p_1 + g\circ p_2)
    = ((f\circ p_1 + g\circ p_2)\circ i_1, (f\circ p_1 + g\circ p_2)\circ i_2) \\
    &= (f\circ p_1\circ i_1 + g\circ p_2\circ i_1,
    f\circ p_1\circ i_2 + g\circ p_2 \circ i_2)
    = (f,g)
  \end{align*}
  となるので、\(\varphi,\psi\)は全単射である。
  これは\(Z\)が所望の表現対象であることを示している。
  以上で解答を完了する。
\end{proof}







\begin{prob}\label{1.4}
  \(\mcC\)を加法圏、
  \(X\rightarrow{i_1} Z \rightarrow{p_2} Y\)
  を射の列で、\(p_2\circ i_1 = 0\)を満たすとする。
  このとき、以下の条件が同値であることを示せ:
  \begin{enumerate}
    \item \label{1.4.1}
    任意の対象\(W\in \mcC\)に対して次の列は完全である:
    \[
    0 \to \Hom_{\mcC}(W,X) \to \Hom_{\mcC}(W,Z) \to \Hom_{\mcC}(W,Y) \to 0.
    \]
    \item \label{1.4.2}
    任意の対象\(W\in \mcC\)に対して次の列は完全である:
    \[
    0 \gets \Hom_{\mcC}(X,W) \gets \Hom_{\mcC}(Z,W) \gets \Hom_{\mcC}(Y,W) \gets 0.
    \]
    \item \label{1.4.3}
    射\(i_2:Y\to Z\)と\(p_1:Z\to X\)が存在して、
    \autoref{prob: 1.3}の条件を満たす。
  \end{enumerate}
  これらの条件が満たされるとき、
  \[ 0\to X\rightarrow{i_1} Z \rightarrow{p_2} Y \to 0 \]
  は\textbf{分裂する}と言い、
  \(X\)は\(Z\)の直和因子であると言う。
  \begin{enumerate}[label=(\fnsymbol*),start=2]
    \item \label{1.4.4}
    \(\mcC\)がアーベル圏または三角圏であるとする。
    \(i_1:X\to Z, p_1:Z\to X\)が
    \(p_1\circ i_1 = \id_X\)を満たすとき、
    \(X\)は\(Z\)の直和因子となることを示せ。
  \end{enumerate}
\end{prob}

\begin{proof}
  はじめに\ref{1.4.1}, \ref{1.4.2}, \ref{1.4.3}が同値であることを確認する。
  \ref{1.4.1}を仮定して\ref{1.4.3}を証明する。
  \(W=Y\)とすることで、
  \ref{1.4.1}で仮定されている完全性 (のうちの右側の全射性) より、
  \(p_2\circ i_2 = \id_Y\)となる射\(i_2:Y\to Z\)が存在することがわかる。
  \(W=Z\)とすれば、
  \(p_2\circ (\id_Z - i_2\circ p_2) = p_2 - p_2 = 0\)であることと、
  \ref{1.4.1}で仮定されている完全性 (のうちの真ん中の完全性) より、
  \(i_1\circ p_1 = \id_Z - i_2\circ p_2\)
  となる射\(p_1:Z\to X\)が存在することがわかる。
  \(W=X\)として\(p_1\circ i_1 :X\to X\)の行き先を見ると、それは
  \[
  i_1\circ p_1 \circ i_1
  = i_1 - i_2\circ p_2 \circ i_1 = i_1 = i_1 \circ \id_X
  \]
  であるので、
  \ref{1.4.1}で仮定されている完全性 (のうちの左側の単射性) より、
  \(p_1\circ i_1 = \id_X\)であることがわかる。
  \(W=Y\)として\(p_2\circ i_1 : Y\to X\)の行き先を見ると、それは
  \(i_1\circ p_2\circ i_1 = 0\)であるので、
  \ref{1.4.1}で仮定されている完全性 (のうちの左側の単射性) より、
  \(p_2\circ i_1 = 0\)であることがわかる。
  以上で\ref{1.4.1}から\ref{1.4.3}が帰結することがわかった。

  \ref{1.4.2}を仮定すれば\(\mcC^{\op}\)において
  \(Y\xrightarrow{p_2}Z \xrightarrow{i_1}X\)は条件\ref{1.4.1}を満たすので、
  すでに証明したことにより\(\mcC^{\op}\)においての条件\ref{1.4.3}が帰結するが、
  それは\(\mcC\)においての条件\ref{1.4.3}を意味している。
  以上で\ref{1.4.2}から\ref{1.4.3}が帰結することがわかった。

  \ref{1.4.3}を仮定すると、
  \(p_1:Z\to X\)と\(i_2:Y\to Z\)を用いて各\(W\)について
  函手的な直和分解
  \[
  \Hom_{\mcC}(W,Z) \cong \Hom_{\mcC}(W,X)\oplus \Hom_{\mcC}(W,Y)
  \]
  を得るので、これはどんな\(W\)についても
  \ref{1.4.1}の列が分裂完全列であることを意味し、
  \(\mcC^{\op}\)で考えることによって\ref{1.4.2}の列が分裂完全列であることもわかる。
  以上で\ref{1.4.1}, \ref{1.4.2}, \ref{1.4.3}が同値であることが示された。

  \(\mcC\)がアーベル圏であるときに\ref{1.4.4}を証明する。
  任意の\(W\)に対して
  \[
  0\to \Hom_{\mcC}(\coker(i_1),W) \to \Hom_{\mcC}(Z,W) \to \Hom_{\mcC}(X,W)
  \]
  は完全となるが、\(p_1\circ i_1\)であるから、
  \(i_1\)を合成する射\(\Hom_{\mcC}(X,W) \to \Hom_{\mcC}(Z,W)\)は
  一番右の射の分裂を与え、これによって条件\ref{1.4.2}が満たされる。
  以上で\(\mcC\)がアーベル圏である場合は証明された。

  \(\mcC\)が三角圏である場合に\ref{1.4.4}を証明する。
  \(i_1:X\to Z\)を完全三角
  \(X\xrightarrow{i_1} Z\xrightarrow{p_2} Y\to X[1]\)
  に延長すると、任意の\(W\)について長い完全列
  \[
  \begin{CD}
    @>>> \Hom_{\mcC}(W,X) @>>> \Hom_{\mcC}(W,Z) @>>> \Hom_{\mcC}(W,Y) \\
    @>>> \Hom_{\mcC}(W,X[1]) @>>> \Hom_{\mcC}(W,X[1]) @>>> \cdots
  \end{CD}
  \]
  を得る (\(\Hom_{\mcC}(W,-)\)はコホモロジー函手である:本文命題1.5.3 (ii))。
  \(p_1:Z\to X\)を (シフトしてから) 合成することで、
  \(\Hom_{\mcC}(W,X[i]) \to \Hom_{\mcC}(W,Z[i])\)の単射性を得る。
  ここで上の長い列の完全性によって、
  \(\Hom_{\mcC}(W,Z) \to \Hom_{\mcC}(W,Y)\)の全射性が従う。
  これによって条件\ref{1.4.1}が満たされる。
  以上で\(\mcC\)が三角圏である場合も証明された。
  以上で解答を完了する。
\end{proof}




\begin{prob}\label{1.5}
  \(\mcC\)をアーベル圏、
  \(0\to X\to Z\to Y\to 0\)を完全列とする。
  \(X\)が入射的、または\(Y\)が射影的であれば、
  この完全列は分裂する。
\end{prob}

\begin{proof}
  \(\id_X\)を延長するか、\(\id_Y\)を持ち上げるか、をすれば良いだけ
  (本文の定義1.2.8の直後の主張を用いる)。
\end{proof}






\begin{prob}\label{1.6}
  \(\mcC\)をアーベル圏とする。
  \begin{enumerate}
    \item \label{1.6.1}
    \(f:X\to Z,g: Y\to Z\)を射とする。
    このとき、\(\ker(X\oplus Y\to Z)\)は函手
    \[
    W\mapsto \Hom(W,X) \times_{\Hom(W,Z)}\Hom(W,Y)
    \]
    の表現対象であることを示せ。
    この対象を\(X\times_ZY\)と表す。

    同様に、二つの射\(f:Z\to X, g:Z\to Y\)が与えられているとき、
    \(\coker(Z\to X\oplus Y)\)は函手
    \[
    W\mapsto \Hom(X,W)\times_{\Hom(Z,W)}\Hom(Y,W)
    \]
    の表現対象であることを示せ。
    この対象を\(X\coprod_ZY\)と表す。
    \item \label{1.6.2}
    \ref{1.6.1}の状況下で、
    \(f':X\times_ZY\to Y, g':X\times_ZY\to X\)を射影とするとき、
    自然な射\(\ker(f')\to \ker(f), \ker(g')\to \ker(g)\)が存在して
    それぞれ同型であることを示せ。
    \item \label{1.6.3}
    \[
    \begin{CD}
      X' @> f' >>Y' \\
      @V g' VV @VV g V \\
      X @> f >> Y
    \end{CD}
    \]
    を可換図式とする。
    このとき、以下の条件は同値であることを示せ:
    \begin{enumerate}
      \item \label{1.6.3.1}
      自然な射\(X'\to X'\times_YY'\)はエピである。
      \item \label{1.6.3.2}
      \(X\coprod_{X'}Y'\to Y\)はモノである。
      \item \label{1.6.3.3}
      次はすべて完全である:
      \[
      \begin{CD}
        0 @>>> \ker(f')\times_{X'}\ker(g') @>>> \ker(g') @>>> \ker(g) @>>> 0, \\
        0 @>>> \ker(f')\times_{X'}\ker(g') @>>> \ker(f') @>>> \ker(f) @>>> 0, \\
        0 @>>> \coker(f') @>>> \coker(f) @>>> \coker(f)\coprod_Y\coker(g) @>>> 0, \\
        0 @>>> \coker(g') @>>> \coker(g) @>>> \coker(f)\coprod_Y\coker(g) @>>> 0.
      \end{CD}
      \]
    \end{enumerate}
    \item \label{1.6.4}
    \(f:X\to Y\)を\(\Ch(\mcC)\)の射とする。
    各\(n\)について、可換図式
    \[
    \begin{CD}
      \coker(d^{n-1}_X) @>>> X^{n+1} \\
      @VVV @VVV \\
      \coker(d^{n-1}_Y) @>>> Y^{n+1}
    \end{CD}
    \]
    が\ref{1.6.3}の同値な条件を満たすとする。
    このとき\(f\)は擬同型であることを示せ。
  \end{enumerate}
\end{prob}


\begin{proof}
  \ref{1.6.1}は自明である。

  \ref{1.6.2}を\(f\)側のみ示す。
  核の普遍性から自然な射\(\ker(f')\to \ker(f)\)が存在する。
  これが同型であることを示せば良い。
  まず\(\mcC\)がアーベル群の圏である場合に
  自然な射\(\ker(f')\to \ker(f)\)が同型射であることを示す。
  \((x,y)\in X\times_ZY\)が\(f'(x,y)=0\)を満たし、さらに\(\ker(f)\)での像
  (これは\(x\)に等しい)が\(0\)であれば、
  \(0 = f'(x,y) = y\)であるから\((x,y) = (0,0)\)がわかり、
  従って\(\ker(f')\to \ker(f)\)は単射である。
  任意の\(x\in \ker(f)\)に対して
  \((x,0)\in X\times Y\)は\(X\times_ZY\)に属しているので、
  \(\ker(f')\to \ker(f)\)は全射である。
  以上より\(\mcC\)がアーベル群の圏である場合には主張が示された。

  一般のアーベル圏\(\mcC\)の場合に
  \(\ker(f')\to \ker(f)\)が同型射であることを証明をする。
  \(W\)を任意にとり、
  \(f_W:\Hom(W,X)\to \Hom(W,Z), f'_W:\Hom(W,X\times_ZY)\to \Hom(W,Y)\)
  を\(f,f'\)を合成することにより得られる射とする。
  このとき\(\Hom(W,X\times_ZY)\cong \Hom(W,X)\times_{\Hom(W,Z)}\Hom(W,Y)\)であるから、
  \(\mcC\)がアーベル群の場合の結果より、
  自然な射\(\ker(f'_W)\to \ker(f_W)\)は同型射である。
  従って、米田の補題により、\(\ker(f')\to \ker(f)\)も同型射である。
  以上で一般のアーベル圏の場合も証明ができた。

  \ref{1.6.3}を証明する。
  まず\ref{1.6.3.2}を仮定して\ref{1.6.3.1}を証明する。
  \(Z\dfn X\coprod_{X'}Y\)とおく。
  射\(Z\to Y\xleftarrow{g} Y'\)があるので、
  \(Z'\dfn Z\times_YY'\)ができる。
  このとき、pull-backの普遍性により、\(Z'\times_Z X \cong X\times_YY'\)となる:
  \[
  \begin{CD}
    Z'\times_Z X @>>> Z' @>>> Y' \\
    @VVV @VVV @VVV \\
    X @>>> Z @>>> Y.
  \end{CD}
  \]
  上の可換図式の左側に\ref{1.6.2}を用いることで、
  \(\ker(Z'\to Y')\cong \ker(Z\to Y)\)であることがわかる。
  仮定より\(\ker(Z\to Y) = 0\)であるから、
  \(Z'\to Y'\)はモノ射である。
  \(Z \dfn X\coprod_{X'}Y'\)であるから、
  射\(Y'\to Z\)があり、これによって\(Z'\to Y'\)のレトラクト\(Y'\to Z'\)を得る
  (合成\(Y'\to Z'\to Y'\)は\(\id\))。
  \(Z'\to Y'\)がモノ射であることから、レトラクトの存在より、
  \(Z'\to Y'\)は同型射でなければならない。
  よって、射\(X'\to X\times_YY'\)がエピであることを示すためには、
  \(X'\to X\times_ZY'\)がエピであることを示すことが十分である。
  しかし、構成より、
  \[
  X\times_ZY \cong \ker(X\oplus Y \to Z)
  \cong \ker(X\oplus Y \to \coker(X' \to X\oplus Y))
  \cong \im(X' \to X\oplus Y)
  \]
  となる。
  これは\(X'\to X\oplus_ZY\)がエピであることを示している。
  以上で
  \ref{1.6.3.2}\(\Rightarrow\)\ref{1.6.3.1}
  が証明された。
  \(\mcC^{\op}\)で考えることにより、
  \ref{1.6.3.1}\(\Rightarrow\)\ref{1.6.3.2}
  がわかる。
  以上で
  \ref{1.6.3.1}\(\Leftrightarrow\)\ref{1.6.3.2}
  がわかった。

  \ref{1.6.3.1}と\ref{1.6.3.2}を仮定して
  \ref{1.6.3.3}を証明する。
  \(\varphi:X'\to X\times_YY'\)と置く。
  任意に射\(h:W\to X'\)をとる。
  \(X\times_YY'\)の定義より、
  \begin{itemize}
    \item[ \ ]
    \(\varphi\circ h = 0\).
    \item[\(\Leftrightarrow\)]
    \(f'\circ h = 0\)かつ\(g'\circ h = 0\).
    \item[\(\Leftrightarrow\)]
    \(h\)は\(\ker(f')\)と\(\ker(g')\)の両方を経由する。
    \item[\(\Leftrightarrow\)]
    \(h\)は\(\ker(f')\times_{X'}\ker(g')\)を経由する。
  \end{itemize}
  となる。
  従って自然に\(\ker(\varphi) \cong \ker(f')\times_{X'}\ker(g')\)となる。
  \(\mcC^{\op}\)で考えることで、自然に
  \(\coker(X\coprod_{X'}Y'\to Y)\cong \coker(f)\coprod_Y\coker(g)\)
  となることがわかる。

  図式
  \[
  \begin{CD}
    X' @> \varphi >> X\times_Y Y' \\
    @V g' VV @VVV \\
    X @= X
  \end{CD}
  \]
  について考える。
  \ref{1.6.2}を用いることで、
  \(\ker(X\times_YY'\to X) \cong \ker(g)\)であることがわかる。
  二重に添字付けられた図式の極限が交換することにより、
  \(\ker\)が交換して、
  \[
  \ker(f')\times_{X'}\ker(g')\cong \ker(\varphi)
  \cong \ker(\ker(\varphi)\to 0) \cong \ker(\ker(g')\to \ker(g))
  \]
  がわかる。
  \(f\)側でも同じことをすることによって、
  \[
  \begin{CD}
    0 @>>> \ker(f')\times_{X'}\ker(g') @>>> \ker(g') @>>> \ker(g),  \\
    0 @>>> \ker(f')\times_{X'}\ker(g') @>>> \ker(f') @>>> \ker(f),
  \end{CD}
  \]
  が完全であることがわかった。
  \(\mcC^{\op}\)で考えることで、
  \[
  \begin{CD}
    \coker(f') @>>> \coker(f) @>>> \coker(f)\coprod_Y\coker(g) @>>> 0, \\
    \coker(g') @>>> \coker(g) @>>> \coker(f)\coprod_Y\coker(g) @>>> 0,
  \end{CD}
  \]
  が完全であることがわかる
  (ここまで\ref{1.6.3.1}と\ref{1.6.3.2}を使っていない)。

  \(\ker(g')\to \ker(g)\)がエピであることを証明すれば、
  \(g\)と\(f\)を入れ替えることによって
  \(\ker(f')\to \ker(f)\)がエピであることがわかり、
  \(\mcC^{\op}\)で考えることで\(\coker\)の方のモノ性も従う。
  なので\(\ker(g')\to \ker(g)\)がエピであることを示すことが残っていることである。
  \ref{1.6.2}より\(\ker(g)\cong \ker(X\times_YY'\to X)\)であるから、
  \(\ker(g')\to \ker(g)\)がエピであることを示すためには、
  可換図式
  \[
  \begin{CD}
    X' @>>> X\times_YY' \\
    @V g' VV @VVV \\
    X @= X
  \end{CD}
  \]
  で\(\ker(g')\to \ker(X\times_YY'\to X)\)
  が\(\ker(g')\to \ker(g)\)がエピであることを示すことが十分である。
  よって\(Y=X,f=\id_X\)であり、\(f'\)はエピであると仮定しても一般性を失わない。
  また、\(\im(g)\)をとっても\(\ker\)は変わらないので、
  \(g\)もエピであると仮定しても一般性を失わない。
  このとき\(X'\)の部分対象として\(\ker(f')\subset \ker(g')\)であるので、
  \(X',\ker(g')\)を\(\ker(f')\)で割ることによって、
  完全列の間の射
  \[
  \begin{CD}
    0 @>>> \ker(g')/\ker(f') @>>> X'/\ker(f') @> g' >> X @>>> 0 \\
    @. @VVV @VVV @| @. \\
    0 @>>> \ker(g) @>>> Y' @>>> X @>>> 0
  \end{CD}
  \]
  を得る。
  ここで真ん中の射\(X'/\ker(f')\to Y'\)は\(f'\)がエピであることによってエピ射である。
  従って縦向き真ん中の射と縦向き右端の射が同型であることがわかった。
  射の圏において同型な二つの射の核は当然同型であるから、
  縦向き左端の射が同型であることがわかる。
  これは\(\ker(g')\to \ker(g)\)がエピであることを意味する。
  以上で\ref{1.6.3.1}と\ref{1.6.3.2}を仮定することで
  \ref{1.6.3.3}が従うことがわかった。

  \ref{1.6.3.3}を仮定して\ref{1.6.3.1}を示す部分が残っている。
  \ref{1.6.3.3}を仮定する。
  \(\ker(f')\times_{X'}\ker(g')\cong \ker(\varphi:X'\to X\times_YY')\)
  となることはすでに示している。
  \(\ker(\varphi:X'\to X\times_YY')\)で\(X'\)を割ることで、
  \(\ker(f')\times_{X'}\ker(g') = 0\)であると仮定しても一般性を失わない。
  \(p:X\times_YY'\to Y'\)を自然な射影とする。
  可換図式
  \[
  \begin{CD}
    X\times_YY' @> p >> Y' \\
    @VVV @VV g V \\
    X @> f >> Y
  \end{CD}
  \]
  にすでに示した「\ref{1.6.3.1}\(\Rightarrow\)\ref{1.6.3.3}」を適用することで、
  自然な射\(\coker(p)\to \coker(f)\)はモノ射であることがわかる。
  また、図式
  \[
  \begin{CD}
    X' @> f' >> Y' \\
    @V \varphi VV   @| \\
    X\times_YY' @> p >> Y'
  \end{CD}
  \]
  が可換であることから、
  \[\coker(\coker(f') \to \coker(p)) \cong
  \coker(\coker(\varphi) \to \coker(\id_{Y'})) = 0\]
  となるので、\(\coker(f')\to \coker(p)\)はエピである。
  今\ref{1.6.3.3}を仮定しているので、
  合成\(\coker(f')\to \coker(p) \to \coker(f)\)はモノ射であり、
  従ってとくに\(\coker(f')\to \coker(p)\)もモノ射である。
  このことは、\(\coker(f')\to \coker(p)\)が同型射であることを意味する。
  従って図式
  \[
  \begin{CD}
    Y' @>>> \coker(f') \\
    @| @VV \cong V \\
    Y' @>>> \coker(p)
  \end{CD}
  \]
  はCartesianであり、\ref{1.6.2}を適用することで、
  自然な射
  \[\im(f') = \ker(Y'\to \coker(f')) \xrightarrow{\sim} \ker(Y'\to\coker(p)) = \im(p)\]
  は同型射であることがわかる。
  \ref{1.6.2}より、自然な射\(\ker(f)\xrightarrow{\sim} \ker(p)\)は同型であるので、
  以下の可換図式を得る:
  \[
  \begin{CD}
    0 @>>> \ker(f') @> i >> X' @> f' >> \im(f') @>>> 0  \\
    @.   @V \cong VV   @V \varphi VV   @VV \cong V    @. \\
    0 @>>> \ker(p) @> j >> X\times_YY' @> p >> \im(p) @>>> 0,
  \end{CD}
  \]
  ただしここで横向きはすべて完全であり、
  \(i:\ker(f')\to X', j:\ker(p)\to X\times_YY'\)は自然なモノ射である。
  \(\bar{\varphi}:\im(f')\xrightarrow{\sim} \im(p)\)と置く。
  任意に射\(h:X\times_YY' \to Z\)を取って、\(h\circ\varphi = 0\)であると仮定する。
  \(\varphi\)がエピであることを示すには、\(h=0\)を証明することが十分である。
  このとき\(h\circ\varphi \circ i = 0\)であることと、
  \(\ker(f')\cong \ker(p)\)であることと、
  上の図式が可換であることにより、
  \(h\circ j = 0\)がわかる。
  従って\(h = h'\circ p\)となる射\(h':\im(p)\to Z\)が存在する。
  \(h'\circ \bar{\varphi}\circ f' = h'\circ p\circ \varphi = h\circ \varphi = 0\)
  であることと、\(f'\)がエピであることから、\(h'\circ \bar{\varphi} = 0\)であるが、
  \(\bar{\varphi}\)は同型射であるので、\(h'=0\)がわかる。
  以上より\(h = h'\circ p = 0\)である。
  以上で\ref{1.6.3}の証明を完了する。

  \ref{1.6.4}を示す。
  まず\(\coker(d_X^{n-1})\cong X^n/\im(d_X^{n-1})\)であることから
  \(H^n(X) \cong \ker(\coker(d_X^{n-1})\to X^{n+1})\)である。
  可換図式
  \[
  \begin{CD}
    \coker(d_X^{n-1}) @>>> X^{n+1} \\
    @VVV @VV f^{n+1} V \\
    \coker(d_Y^{n-1}) @>>> Y^{n+1}
  \end{CD}
  \]
  に\ref{1.6.3} \ref{1.6.3.3}を使うことで、
  \(H^n(f):H^n(X)\to H^n(Y)\)は各\(n\)でエピであることがわかる。
  また、\(\coker(\coker(d_X^{n-1})\to X^{n+1}) \cong \coker(d_X^n)\)であるから、
  上の可換図式に再び\ref{1.6.3} \ref{1.6.3.3}を使うことで、
  \(\coker(d_X^n)\to \coker(d_Y^n)\)は各\(n\)でモノであることがわかる。
  特に\(\coker(d_X^{n-1})\to \coker(d_Y^{n-1})\)もモノであり、
  従って上の可換図式に再び\ref{1.6.3} \ref{1.6.3.3}を使うと
  \[
  \ker(H^n(f)) \cong
  \ker(\ker(\coker(d_X^{n-1})\to \coker(d_Y^{n-1})) \to \ker(f^{n+1})) = 0
  \]
  がわかる。
  以上より\(H^n(f)\)は各\(n\)でモノ射である。
  \(H^n(f)\)はエピだったので、\(H^n(f)\)は同型射となる。
  このことは\(f\)が擬同型であることを意味する。
  以上で\autoref{1.6}の証明を完了する。
\end{proof}












\begin{prob}\label{1.7}
  \(\mcC\)をアーベル圏とする。
  \begin{enumerate}
    \item \label{1.7.1}
    \(Z\in \mcC\)を対象とする。
    圏\(\mcP(Z)\)を次で定義する:
    \begin{itemize}
      \item 対象はエピ射\(f:X\to Z\)である。
      \item 二つの対象\(f:X\to Z\)と\(g:Y\to Z\)の間の射\((f:X\to Z)\to (g:Y\to Z)\)は
      \(\mcC\)のエピ射\(h:X\to Y\)であって\(f\circ h = g\)となるものである。
      \item 合成は\(\mcC\)の合成によって定義する。
    \end{itemize}
    このとき、\(\mcP(Z)\)はcofilteredであることを示せ。
    \item \label{1.7.2}
    対象\(X\in \mcC\)に対し、
    \(\tilde{h}_Z(X) \dfn \colim_{Z'\in \mcP(Z)}\Hom_{\mcC}(Z',X)\)
    とおく。
    以下を示せ:
    \begin{enumerate}
      \item \label{1.7.2.1}
      函手\(\tilde{h}_Z:\mcC \to \mathsf{Ab}\)は完全函手である。
      \item \label{1.7.2.2}
      \(f,f'\in \Hom_{\mcC}(X,X')\)を二つの射とする。
      任意の\(Z\in \mcC\)に対して
      \(\tilde{h}_Z(f) = \tilde{h}_Z(f')\)が成り立つとき、
      \(f=f'\)である。
      \item \label{1.7.2.3}
      すべての対象\(Z\in \mcC\)に対する\(\tilde{h}_Z\)での像が
      \(\mathsf{Ab}\)において完全であるような\(\mcC\)の列は完全である。
    \end{enumerate}
  \end{enumerate}
\end{prob}



\begin{rem*}
  第一版では、\ref{1.7.1}の問題文は次のように表記されている (引用):

  For an object \(Z\) of \(\mcC\),
  let \(\mathscr{P}(Z)\) be the category
  whose objects are the epimorphisms \(f:Z'\to Z\),
  a morphism \((f:Z'\to Z) \to (f':Z''\to Z)\)
  being defined by \(h:Z'\to Z''\) with \(f'\circ h = f\).
  Prove that \(\mathscr{P}(Z)\) is cofiltrant,
  that is, \(\mathscr{P}(Z)^{\circ}\) is filtrant.

  この文章をそのまま読むと、圏\(\mathscr{P}(Z)\)は、
  \(Z\)への射がエピとなるものたちからなる圏\(\mcC_{/Z}\)の充満部分圏であると読める
  (というか、この文章は\(h\)もエピであることが想定されているようには読めないと思う)。
  しかし、このように読むと、
  \(\mathscr{P}(Z)\)はcofilteredにはならない。
  たとえば、\(k\)を標数が\(2\)でない体、
  \(\mcC\)を\(k\)-線形空間の圏、
  \(Z=k\)として、
  \(\mcC_{/k}\)の対象として
  \(p\dfn \id_k: X\dfn k\to Z\)と
  \(q\dfn \mathrm{pr}_1:Y \dfn k\times k \to Z\)
  を考え、\(p,q\)の間の射として
  \(f_1:X\to Y\)を\(f_1(a) = (a,a)\)で定め、
  \(f_2:X\to Y\)を\(f_2(a) = (a,-a)\)で定める。
  このとき、線形空間\(V\)と射\(g:V\to X\)が
  \(f_1\circ g = f_2\circ g\)を満たせば、
  \(g\)が\(0\)-射であることが容易に従う (標数が\(2\)でないことを用いる)。
  従って、とくに、\(g\)はエピとはならず、
  従って、\(g:V\to k\)は\(\mathscr{P}(Z)\)の対象となることは決してない。
  このことは\(\mathscr{P}(Z)^{\op}\)が
  本文定義1.11.2条件(1.11.2)を満たさない (とくにcofilteredではない) ことを示している。
\end{rem*}

\begin{proof}
  \ref{1.7.1}を示す。
  \(\mcP(Z)\)の図式
  \(h_1: (f_1:X_1\to Z) \to (g:Y\to Z) \gets (f_2:X_2\to Z) : h_2\)
  を任意にとって、fiber積\(X_1\times_Y X_2\)を考える。
  \(p_i: X_1\times_Y X_2\to X_i , (i=1,2)\)を射影とする。
  このとき、\(f_1\circ p_1 = g\circ h_1\circ p_1 = g\circ h_2\circ p_2 = f_2\circ p_2\)
  であるから、
  \(f\dfn f_1\circ p_1\)とすれば、
  \(f:X_1\times_Y X_2\to Z\)は圏\(\mcC_{/Z}\)におけるfiber積となる。
  \(\mcP(Z)\)は終対象\(\id_Z:Z\to Z\)を持つので、
  従って、\(\mcP(Z)\)がcofilteredであることを示すためには、
  \(f:X_1\times_Y X_2 \to Z\)がエピ射であることを示すことが十分である。
  \autoref{1.6}\ref{1.6.3}より、エピ射のpull-backはエピ射であるから、
  \(p_i\)はエピ射であり、
  エピ射の合成はエピ射であるから、
  \(f = f_1\circ p_1\)もエピ射である。
  以上で\ref{1.7.1}の解答を完了する。

  \ref{1.7.2} \ref{1.7.2.1}を示す。
  集合の間の写像の圏において、単射のfilered colimitは単射である。
  従って\(\tilde{h}_Z\)は左完全函手である。
  残っているのは\(\tilde{h}_Z\)の右完全性を証明することである。
  \(g:X_1\to X_3\)を\(\mcC\)のエピ射とし、
  \(\tilde{r}_3\in \tilde{h}_Z(X_3)\)を任意にとる。
  \(\tilde{r}_3\)の代表元を\(r_3:Z_3\to X_3\)とする。
  ここで\(Z_3\)はある\(\mcP(Z)\)の対象\(z_3:Z_3\to Z\)のdomainであり、
  \(r_3:Z_3\to X_3\)は\(\mcC\)の射である。
  図式\(r_3:Z_3\to X_3\gets X_1: g\)のfiber積を\(Z_1\)とし、
  射影を\(h:Z_1\to Z_3, r_1:Z_1\to X_1\)とする。
  エピ射のpull-backはエピ射であるから、\(h\)はエピである。
  従って、\(z_1\dfn z_3\circ h:Z_1\to Z\)は\(\mcP(Z)\)の対象であり、
  \(h\)は\(\mcP(Z)\)の射である。
  さらに、\(g\circ r_1 = r_3\circ h\)であるから、
  \(r_1:Z_1\to X_1\)により代表される元\(\tilde{r}_1\in \tilde{h}_Z(X_1)\)は
  射\(\tilde{h}_Z(X_1)\to \tilde{h}_Z(X_3)\)により\(\tilde{r}_3\)へと写る。
  従って\(\tilde{h}_Z(X_1)\to \tilde{h}_Z(X_3)\)は全射である。
  以上で\ref{1.7.2} \ref{1.7.2.1}の解答を完了する。

  \ref{1.7.2} \ref{1.7.2.2}を示す。
  \(f,f':X\to X'\)が任意の\(Z\in \mcC\)に対して
  \(\tilde{h}_Z(f) = \tilde{h}_Z(f')\)を満たしていると仮定する。
  \(Z=X\)として、
  \(\id_X:X\to X\)により代表される元を\(\tilde{i}\in \tilde{h}_Z(X)\)、
  \(f,f':X\to X'\)により代表される元を\(\tilde{f},\tilde{f}'\in \tilde{h}_Z(X')\)とする。
  このとき、
  \[
  \tilde{f} = \tilde{h}_Z(f)\circ \tilde{i}
  = \tilde{h}_Z(f')\circ \tilde{i} = \tilde{f}'
  \]
  となる。
  \(\mcP(Z)\)の各射はエピなので、
  自然な射\(\Hom_{\mcC}(Z,X)\to \tilde{h}_Z(X)\)は単射である。
  従って、等式\(\tilde{f}=\tilde{f}'\)は\(f=f'\)であることを意味する。
  以上で\ref{1.7.2} \ref{1.7.2.2}の解答を完了する。

  \ref{1.7.2} \ref{1.7.2.3}を示す。
  \(\mcC\)を離散圏 (射が\(\id\)しかない圏) とみなした圏を\(\bar{\mcC}\)とおく。
  \(\bar{\mcC}\)から\(\Ab\)への(加法的とは限らない)函手のなす圏
  \([\bar{\mcC},\Ab]\)はアーベル圏である。
  \(\tilde{h}:\mcC\to [\bar{\mcC},\Ab], X\mapsto [Z\mapsto \tilde{h}_Z(X)]\)
  はアーベル圏の間の加法的函手である。
  \ref{1.7.2} \ref{1.7.2.1}より、各\(\tilde{h}_Z\)は完全函手であるから、
  \(\tilde{h}\)も完全函手である。
  \ref{1.7.2} \ref{1.7.2.2}より\(\tilde{h}\)は忠実である。
  従って、\ref{1.7.2} \ref{1.7.2.3}を示すためには、
  アーベル圏の間の忠実な完全函手\(F:\mcC\to \mcD\)と
  \(\mcC\)の射の列\(X\xrightarrow{f} Y \xrightarrow{g} Z\)に対して、
  \(X\xrightarrow{f} Y \xrightarrow{g} Z\)が\(\mcC\)で完全であることと
  \(F(X)\xrightarrow{F(f)}F(Y)\xrightarrow{F(g)}F(Z)\)が\(\mcD\)で完全であることが
  同値であることを証明することが十分である。
  \(F\)は忠実なので、\(g\circ f = 0\)であることと\(F(g)\circ F(f) = 0\)であることは同値である。
  \(F\)は完全函手なので、\(\im(F(f))\)と\(F(\im(f))\)は自然に同型であり、
  \(\ker(F(g))\)と\(F(\ker(g))\)も自然に同型である。
  さらに\(F\)は忠実なので、
  自然な射\(\im(f)\to \ker(g)\)が同型であることは
  \(F\)での像\(F(\im(f))\to F(\ker(g))\)が同型であることと同値である。
  よって、\(X\xrightarrow{f} Y \xrightarrow{g} Z\)が\(\mcC\)で完全であることと
  \(F(X)\xrightarrow{F(f)}F(Y)\xrightarrow{F(g)}F(Z)\)が\(\mcD\)で完全であることは
  同値である。
  以上で\autoref{1.7}の証明を完了する。
\end{proof}











\begin{prob}[The Five Lemma]\label{1.8}
  \(\mcC\)をアーベル圏とする。
  \(\mcC\)の可換図式
  \[
  \begin{CD}
    X^0 @>>> X^1 @>>> X^2 @>>> X^3 @>>> X^4 \\
    @V f_0 VV   @V f_1 VV   @V f_2 VV   @V f_3 VV   @V f_4 VV \\
    Y^0 @>>> Y^1 @>>> Y^2 @>>> Y^3 @>>> Y^4
  \end{CD}
  \]
  について以下の主張を証明せよ。
  ただし横向きは完全であるとする。
  \begin{enumerate}
    \item \label{1.8.1}
    \(f_0\)がエピであり、
    \(f_1,f_3\)がモノであれば、
    \(f_2\)はモノである。
    \item \label{1.8.2}
    \(f_4\)がモノであり、
    \(f_1,f_3\)がエピであれば、
    \(f_2\)はエピである。
  \end{enumerate}
\end{prob}

\begin{proof}
  \autoref{1.7}によって\(\Ab\)での主張と見做して良く、
  この場合、主張は初等的である。
\end{proof}




\begin{prob}\label{1.9}
  \(\mcC\)をアーベル圏とする。
  \[
  \begin{CD}
    @. X @> f >> Y @> g >> Z @>>> 0 \\
    @. @V \alpha VV @VV \beta V @VV \gamma V \\
    0 @>>> X' @> f' >> Y' @> g' >> Z' @.
  \end{CD}
  \]
  を\(\mcC\)の可換図式で、横向きが完全であるものとする。
  \begin{enumerate}
    \item \label{1.9.1}
    自然な射\(\varphi:\ker(\gamma) \to \coker(\alpha)\)が存在して、
    以下が完全となることを示せ:
    \[
    \ker(\alpha) \to \ker(\beta) \to \ker(\gamma) \xrightarrow{\varphi}
    \coker(\alpha) \to \coker(\beta) \to \coker(\gamma).
    \]
    \item \label{1.9.2}
    以下の図式が可換であることを示せ:
    \[
    \begin{CD}
      @. Y @> g >> Z \\
      @. @AAA @AAA \\
      Y @<<< \ker(\gamma\circ g) @>>> \ker(\gamma) \\
      @VVV @VVV @VV \varphi V \\
      Y' @< f' << X' @>>> \coker(\alpha).
    \end{CD}
    \]
  \end{enumerate}
\end{prob}


\begin{proof}
  \(\varphi\)の構成ができれば、
  \autoref{1.7}によって\ref{1.9.1}は\(\mcC=\Ab\)の場合に帰着され、
  この場合は図式追跡によって初等的に証明できる。
  従って、\ref{1.9.1}を示すためには\(\varphi\)を構成することが十分である。
  以下、\(\varphi\)の構成と\ref{1.9.2}の証明を同時に行う。

  核の普遍性により、以下の図式を可換にするような射
  \(\psi_1:\ker(\gamma \circ g) \to \ker(\gamma)\)が一意的に存在する:
  \[
  \begin{CD}
    0 @>>> \ker(\gamma\circ g) @>>> Y @> \gamma\circ g >> Z' \\
    @. @V \psi_1 VV @VV g V @| \\
    0 @>>> \ker(\gamma) @>>> Y' @> \gamma >> Z'.
  \end{CD}
  \]
  これは\ref{1.9.2}の図式の右上の四角形の可換性を示している。
  また、\autoref{1.8} \ref{1.8.2}より、\(\psi_1\)はエピである。
  核の普遍性により、以下の図式を可換にするような射
  \(\psi_2:\ker(\gamma\circ g) \to X'\)が一意的に存在する:
  \[
  \begin{CD}
    0 @>>> \ker(\gamma\circ g) @>>> Y @> \gamma\circ g >> Z' \\
    @. @V \psi_2 VV @VV \beta V @| \\
    0 @>>> X' @> f' >> Y' @> g' >> Z'.
  \end{CD}
  \]
  これは\ref{1.9.2}の図式の左下の四角形の可換性を示している。
  自然な射\(X'\to \coker(\alpha)\)と\(\psi_2\)の合成を
  \(\psi_3:\ker(\gamma\circ g) \to \coker(\alpha)\)と置く。
  \(\ker(\psi_1) \to \ker(\gamma\circ g) \xrightarrow{\psi_3} \coker(\alpha)\)
  の合成が\(0\)-射であることが証明できれば、
  \(\psi_1\)がエピであることから、
  \(\psi_3\)は\(\psi_1\)を一意的に経由して、
  図式
  \[
  \begin{CD}
    \ker(\gamma\circ g) @> \psi_1 >> \ker(\gamma) \\
    @V \psi_2 VV @VV \varphi V \\
    X' @>>> \coker(\alpha)
  \end{CD}
  \]
  を可換にする射\(\varphi\)の存在が従う
  (\ref{1.9.2}の図式の右下の四角形の可換性の証明と\(\varphi\)の構成が同時に終わる)。

  \(p:X\to \im(f)\)を自然なエピ射、
  \(j_1:\im(f)\cong \ker(g)\to Y\)を自然なモノ射とする。
  このとき\(f = j_1\circ p\)である。
  核の普遍性により引き起こされる一意的な射を
  \(\alpha' : \im(f)\cong \ker(g)\to X'\)と置く。
  \[
  f'\circ \alpha
  = \beta \circ f
  = \beta \circ j_1 \circ p
  = f'\circ \alpha' \circ p
  \]
  であることと\(f'\)がモノであることから、\(\alpha = \alpha'\circ p\)となる。
  \(q:X'\to \coker(\alpha)\)を自然な射とすると、
  \(q\circ \alpha' \circ p = q\circ \alpha = 0\)となるが、
  \(p\)がエピであることから、\(q\circ \alpha' = 0\)となる。
  \(T\dfn \ker(\psi_1)\)とおき、
  \(i:T\to \ker(\gamma\circ g)\)を自然な射、
  \(j_2:\ker(\gamma\circ g) \to Y\)を自然なモノ射とする。
  \(\psi_1\circ i = 0\)であるから、
  \(g\circ j_2\circ i = 0\)である。
  よって、核の普遍性により、一意的な射
  \(k:T\to \im(f)\)が存在して、
  \(j_1\circ k = j_2\circ i\)となる。
  以上より、
  \[
  f' \circ \psi_2 \circ i
  = \beta\circ j_2\circ i
  = \beta\circ j_1\circ k
  = f'\circ \alpha' \circ k
  \]
  となる。
  \(f'\)はモノなので\(\psi_2\circ i = \alpha'\circ k\)となる。
  従って、\(q\circ \psi_2\circ i = q\circ \alpha' \circ k = 0\)
  となって、示すべき等式を得る。
  以上で\autoref{1.9}の証明を完了する。
\end{proof}







\begin{prob}\label{1.10}
  \(\mcC\)をアーベル圏とする。
  図式
  \[
  \begin{CD}
    0 @>>> M @>>> M_0 @>>> M_1 @>>> 0 \\
    @. @| @. @. @. \\
    0 @>>> M @>>> M_0' @>>> M_1' @>>> 0
  \end{CD}
  \]
  において、横向きは完全であり、\(M_0,M_0'\)はそれぞれ入射的であるとする。
  同型射\(M_0\oplus M_1' \xrightarrow{\sim} M_0'\oplus M_1\)を構成せよ。
\end{prob}

\begin{proof}
  \(N\dfn M_0\coprod_M M_0'\)とおいて、
  \(j:M_0 \to N \gets M_0':j'\)を自然な射とする。
  \(i,i'\)はモノ射であるから、
  \autoref{1.6}\ref{1.6.3}より、
  そのpush-outである\(j,j'\)もそれぞれモノ射である。
  従って、\(M_0\)が入射的であることから、
  ある射\(p:N\to M_0\)が存在して\(p\circ j = \id_{M_0}\)となり、
  \(M_0'\)が入射的であることから、
  ある射\(p':N\to M_0'\)が存在して\(p'\circ j' = \id_{M_0'}\)となる。
  図式
  \[
  \begin{CD}
    M @>>> M_0 @>>> M_1 \\
    @VVV @V j VV @VVV \\
    M_0' @> j' >> N @>>> X \\
    @VVV @VVV @. \\
    M_1' @>>> Y @.
  \end{CD}
  \]
  に\(\mcC^{\op}\)で\autoref{1.6}\ref{1.6.2}を適用すると、
  射\(M_1\to X\)と\(M_1'\to Y\)はそれぞれ同型射であることがわかる。
  以上より、二つの完全列
  \[
  \begin{CD}
    0 @>>> M_0 @> j >> N @>>> M_1' @>>> 0 \\
    0 @>>> M_0' @> j' >> N @>>> M_1 @>>> 0
  \end{CD}
  \]
  を得る。
  \(j,j'\)は分裂モノ射であるから、\autoref{1.4}より、
  同型射\(M_0\oplus M_1' \cong N \cong M_0'\oplus M_1\)を得る。
  以上で\autoref{1.10}の証明を完了する。
\end{proof}








\begin{prob}\label{1.11}
  \(\mcC\)をアーベル圏、
  \(X\in \Ch(\mcC)\)を複体であって、
  任意の\(Y\in \mcC\)に対して
  アーベル群の複体\(\Hom_{\mcC}(Y,X)\)が完全であるものとする。
  このとき\(X\)は\(\sfK(\mcC)\)で\(0\)であることを示せ。
\end{prob}


\begin{proof}
  \(\Hom_{\mcC}(Y,-)\)は左完全函手であるから、
  任意の\(n\)に対して、自然に
  \[
  \Hom_{\mcC}(Y,\ker(d_X^n)) \xrightarrow{\sim}
  \ker(d_X^n\circ (-): \Hom_{\mcC}(Y,X^n)\to \Hom_{\mcC}(Y,X^{n+1}))
  \]
  となる。
  \(\Hom_{\mcC}(Y,X)\)は完全であるから、任意の\(n\)に対して、自然に
  \[
  \Hom_{\mcC}(Y,\im(d_X^n))
  \cong \Hom_{\mcC}(Y,\ker(d_X^{n+1}))
  \cong \ker(d_X^{n+1}\circ (-))
  \cong \im(d_X^n\circ (-))
  \]
  となる。
  従って、任意の\(n\)に対して、自然な射
  \(\im(d_X^n\circ (-))\to \Hom_{\mcC}(Y,\im(d_X^n))\)
  は同型射であり、
  任意の\(n\)に対して、完全列
  \[
  \begin{CD}
    0 @>>> \ker(d_X^n) @>>> X^n @>>> \im(d_X^n) @>>> 0
  \end{CD}
  \]
  に\(\Hom_{\mcC}(Y,-)\)を施した後のアーベル群の列も完全である。
  よって\autoref{1.4}より、任意の\(n\)に対して、
  \(X^n \cong \im(d_X^n)\oplus \ker(d_X^n)\)となることが従う。

  \(X\)が\(\sfK(\mcC)\)において\(0\)であるためには、
  \(\id_X:X\to X\)が homotopic to zero であることが十分である。
  \(s^n:X^n\to X^{n-1}\)を、
  \(\ker(d_X^n)\to X^n\)の分裂\(p^n:X^n\to \ker(d_X^n)\)と、
  同型射\(l^n:\im(d_X^{n-1})\xrightarrow{\sim}\ker(d_X^n)\)の逆射と、
  \(X^{n-1}\to \im(d_X^{n-1})\)の分裂\(i^{n-1}:\im(d_X^{n-1})\to X^{n-1}\)の、
  三つの射の合成射として\(s^n \dfn i^{n-1}\circ (l^n)^{-1}\circ p^n\)と定める。
  このとき、\(s^{n+1}\circ d_X^n:X^n\to X^n\)は
  自然なエピ射\(X^n\to \im(d_X^n)\)と\(i^n:\im(d_X^n)\to X^n\)の合成射に等しく、
  \(d_X^{n-1}\circ s^n:X^n\to X^n\)は
  \(p^n:X^n\to \ker(d_X^n)\)と自然なモノ射\(\ker(d_X^n)\to X^n\)の合成射に等しい。
  従って\(\id_{X^n} = s^{n+1}\circ d_X^n + d_X^{n-1}\circ s^n\)となり、
  \(\id_X\)は homotopic to zero であることがわかる。
  以上で\autoref{1.11}の解答を完了する。
\end{proof}






\begin{prob}\label{1.12}
  \(\mcC\)を三角圏とし、
  \[
  \begin{CD}
    X @>>> Y @>>> Z @>>> X[1] \\
    @| @| @VVV @| \\
    X @>>> Y @>>> Z' @>>> X[1]
  \end{CD}
  \]
  を\(\mcC\)の可換図式で、上の列が完全三角であるものとする。
  このとき、以下の条件のうちの一方が成り立つとき、
  下の列も完全三角であることを示せ:
  \begin{enumerate}
    \item \label{1.12.1}
    任意の対象\(P\in \mcC\)に対して、以下の列は完全である:
    \[
    \Hom(P,X) \to \Hom(P,Y) \to \Hom(P,Z') \to \Hom(P,X[1]).
    \]
    \item \label{1.12.2}
    任意の対象\(Q\in \mcC\)に対して、以下の列は完全である:
    \[
    \Hom(X,Q) \gets \Hom(Y,Q) \gets \Hom(Z',Q) \gets \Hom(X[1],Q).
    \]
  \end{enumerate}
\end{prob}


\begin{proof}
  \(\Hom(P,-)\)と\(\Hom(-,Q)\)はそれぞれ cohomological functor であるから、
  \ref{1.12.1}を仮定すれば、射\(\Hom(P,Z) \to \Hom(P,Z')\)は同型射であることが従い、
  \ref{1.12.2}を仮定すれば、射\(\Hom(Z',Q) \to \Hom(Z,Q)\)は同型射であることが従う。
  すると、米田の補題より、
  \ref{1.12.1}と\ref{1.12.2}のいずれかを仮定すれば、射\(Z\to Z'\)は同型射であることが従う。
  \(\mcC\)は三角圏なので、本文命題1.4.4の(TR0)を満たし、
  従って所望の完全性を得る。
\end{proof}







\begin{prob}\label{1.13}
  \(\mcC\)を三角圏、
  \(X_i\to Y_i\to Z_i\to X_i[1], (i=1,2)\)を\(\mcC\)の二つの三角形とする。
  これら二つの三角形が完全三角であるためには、
  三角形
  \[X_1\oplus X_2 \to Y_1\oplus Y_2 \to Z_1\oplus Z_2 \to X_1[1]\oplus X_2[1]\]
  が完全三角であることが必要十分である、ということを示せ。
\end{prob}

\begin{proof}
  必要性を証明する。
  二つの三角形
  \(X_i\to Y_i\to Z_i\to X_i[1], (i=1,2)\)
  が完全三角であるとする。
  \(M\dfn M(X_1\oplus X_2 \to Y_1\oplus Y_2)\)と置く (mapping cone)。
  自然な射\(X_1\oplus X_2\to X_i\)と\(Y_1\oplus Y_2\to Y_i\)により
  可換図式
  \[
  \begin{CD}
    X_1\oplus X_2 @>>> Y_1\oplus Y_2 \\
    @VVV @VVV \\
    X_i @>>> Y_i
  \end{CD}
  \]
  を得る。
  よって、(TR4)より、ある射\(M\to Z_i\)が存在して、
  これらが完全三角の間の射を形成する。
  二つの射\(M\to Z_1,M\to Z_2\)により、
  射\(M\to Z_1\oplus Z_2\)ができて、
  可換図式
  \[
  \begin{CD}
    X_1\oplus X_2 @>>> Y_1\oplus Y_2 @>>> M @>>> X_1[1]\oplus X_2[1] \\
    @| @| @VVV @| \\
    X_1\oplus X_2 @>>> Y_1\oplus Y_2 @>>> Z_1\oplus Z_2 @>>> X_1[1]\oplus X_2[1] \\
  \end{CD}
  \]
  を得る。
  任意に\(P\in \mcC\)を取って、函手\(\Hom(P,-)\)を適用すると、
  各\(\Hom(P,X_i)\to \Hom(P,Y_i)\to \Hom(P,Z_i)\to \Hom(P,X_i[1])\)は完全であるから、
  \[
  \Hom(P,X_1\oplus X_2) \to \Hom(P,Y_1\oplus Y_2) \to
  \Hom(P,Z_1\oplus Z_2) \to \Hom(P,X_1[1]\oplus X_2[1])
  \]
  も完全である。
  よって\autoref{1.12}より
  \[X_1\oplus X_2 \to Y_1\oplus Y_2 \to Z_1\oplus Z_2 \to X_1[1]\oplus X_2[1]\]
  も完全三角であることが従う。
  以上で必要性の証明を完了する。

  十分性を証明する。
  \[X_1\oplus X_2 \to Y_1\oplus Y_2 \to Z_1\oplus Z_2 \to X_1[1]\oplus X_2[1]\]
  が完全三角であると仮定する。
  \(M_i\dfn M(X_i\to Y_i)\)と置く。
  自然な射\(X_i\to X_1\oplus X_2\)と\(Y_i\to Y_1\oplus Y_2\)により
  可換図式
  \[
  \begin{CD}
    X_i @>>> Y_i \\
    @VVV @VVV \\
    X_1\oplus X_2 @>>> Y_1\oplus Y_2
  \end{CD}
  \]
  を得る。
  よって、(TR4)より、ある射\(M_i\to Z_1\oplus Z_2\)が存在して、
  これらが完全三角の間の射を形成する。
  自然な射\(X_1\oplus X_2 \to X_i, Y_1\oplus Y_2 \to Y_i, Z_1\oplus Z_2\to Z_i\)
  と合成することで、可換図式
  \[
  \begin{CD}
    X_i @>>> Y_i @>>> M_i @>>> X_i[1] \\
    @| @| @VVV @| \\
    X_i @>>> Y_i @>>> Z_i @>>> X_i[1]
  \end{CD}
  \]
  を得る。
  任意に\(P\in \mcC\)を取って函手\(\Hom(P,-)\)を適用する。
  \[X_1\oplus X_2 \to Y_1\oplus Y_2 \to Z_1\oplus Z_2 \to X_1[1]\oplus X_2[1]\]
  が完全三角であることから、
  \[
  \Hom(P,X_1\oplus X_2) \to \Hom(P,Y_1\oplus Y_2) \to
  \Hom(P,Z_1\oplus Z_2) \to \Hom(P,X_1[1]\oplus X_2[1])
  \]
  は完全であり、従って各\(i=1,2\)に対して
  \[
  \Hom(P,X_i) \to \Hom(P,Y_i) \to \Hom(P,Z_i) \to \Hom(P,X_i[1])
  \]
  も完全である。
  よって\autoref{1.12}より
  \(X_i\to Y_i\to Z_i\to X_i[1]\)も完全三角であることが従う。
  以上で十分性の証明を完了し、
  \autoref{1.13}の解答を完了する。
\end{proof}





\begin{prob}\label{1.14}
  \(\mcC\)を圏、\(S\)を積閉系とする。
  対象\(X\in \mcC\)に対して、圏\(S_X\)を次で定義する:
  \begin{itemize}
    \item
    \(S_X\)の対象は\(S\)に属する射\(s:X'\to X\)である。
    \item
    対象\(s:X'\to X\)から対象\(s':X''\to X\)への射は、
    \(\mcC\)の射\(h:X''\to X'\)であって
    \(s'=s\circ h\)となるものとして定義する
    (\(\mcC\)の射の向きとは逆向きであることに注意)。
  \end{itemize}
  このとき、以下を証明せよ:
  \begin{enumerate}
    \item \label{1.14.1}
    \(S_X\)はfilteredである。
    \item \label{1.14.2}
    \(X,Y\in \mcC\)を対象とする。
    このとき、以下が成り立つ:
    \[\Hom_{\mcC_S}(X,Y) = \colim_{X'\in S_X}\Hom_{\mcC}(X',Y).\]
    \item \label{1.14.3}
    射を逆向きにすることで圏\(S_Y^a\)を定義し、次が成り立つことを示せ:
    \[\Hom_{\mcC_S}(X,Y) = \colim_{Y'\in S_Y^a}\Hom_{\mcC}(X,Y').\]
  \end{enumerate}
\end{prob}

\begin{rem*}
  第一版の本文では、\ref{1.14.1}は、
  \((S_X)^{\op}\)がfilteredであることを示す問題となっているが、
  これは誤植であると思われる。
  なお、\((S_X)^{\op}\)がfilteredであることは、
  終対象\(\id_X:X\to X\)を持つことから自明である。
\end{rem*}

\begin{proof}
  \ref{1.14.1}を示す。
  \(S_X^{\op}\)がcofilteredであることを証明すれば良い。
  \([s_1:X_1\to X], [s_2:X_2\to X]\)を\(S_X^{\op}\)の対象とする。
  すると本文定義1.6.1 (S3)より、
  ある\(S\)に属する射\(t:W\to X_1\)と\(\mcC\)の射\(f:W\to X_2\)が存在して
  \(s_1\circ t = s_2\circ f\)となる。
  \(s_1,t\in S\)なので、本文定義1.6.1 (S2)より、\(u\dfn s_1\circ t\in S\)である。
  従って、\([u:W\to X]\)は\(S_X^{\op}\)の対象であり、
  \(f,t\)は\(S_X^{\op}\)の射である。
  よって\(S_X\)は本文定義1.11.2の条件(1.11.1)を満たす。

  次に、\([s_1:X_1\to X], [s_2:X_2\to X]\)を\(S_X^{\op}\)の対象とし、
  \(f_1,f_2:s_1\to s_2\)を\(S_X^{\op}\)の二つの射とする。
  このとき、\(s_2\circ f_1 = s_2\circ f_2\)であるから、
  本文定義1.6.1 (S4)より、
  \(S\)に属するある射\(t:Y\to X_1\)が存在して、
  \(f_1\circ t = f_2\circ t\)となる。
  \(u\dfn s_1\circ t\)とすれば、本文定義1.6.1 (S2)より\(u\in S\)であるから、
  \([u:Y\to X]\)は\(S_X^{\op}\)の対象であり、
  \(t:u\to s_1\)は\(S_X^{\op}\)の射である。
  よって\(S_X\)は本文定義1.11.2の条件(1.11.2)を満たす。
  以上で\ref{1.14.1}の証明を完了する。

  \ref{1.14.2}を示す。
  \[
  T \dfn \left\{ (X',s,f) \middle| X'\in \mcC, [s:X'\to X]\in S, f:X'\to Y\right\}
  \]
  と置く (本文定義1.6.2のHom集合の定義式の割る前の集合) と、
  \[T = \coprod_{[s:X'\to X]\in S_X}\Hom_{\mcC}(X',Y)\]
  である。
  また、\(f\in \Hom_{\mcC}(X',Y), g\in \Hom_{\mcC}(X'',Y)\)に対して
  本文定義1.6.2で定義されている関係は、
  ある\(S_X\)の射\(X'\to X'''\gets X''\)が存在して、
  \(f,g\)は\(\Hom_{\mcC}(X''',Y)\)において等しい、ということを意味している。
  従って、集合の圏における余極限の具体的な構成を思い出すと、
  \(\Hom_{\mcC_S}(X,Y) = \colim_{X'\in S_X}\Hom_{\mcC}(X',Y)\)となることがわかる。
  以上で\ref{1.14.2}の証明を完了する。

  \ref{1.14.3}を示す。
  \(S_Y^a\dfn ((S^{\op})_Y)^{\op}\)とおく。
  ただし\(S^{\op}\)は\(S\)に対応する圏\(\mcC^{\op}\)の積閉系である。
  \ref{1.14.1}より\((S^{\op})_Y\)はcofilteredであるから、
  \(((S^{\op})_Y)^{\op}\)はfilteredである。
  また\ref{1.14.2}より
  \[\Hom_{\mcC^{\op}_{S^{\op}}}(Y,X) =
  \colim_{Y'\in (S^{\op})_Y}\Hom_{\mcC^{\op}}(Y',X)\]
  である。
  \(\op\)をとれば、
  \[\Hom_{(\mcC^{\op}_{S^{\op}})^{\op}}(X,Y) =
  \colim_{Y'\in S_Y^a}\Hom_{\mcC}(X,Y')\]
  であることが従う。
  \((\mcC^{\op}_{S^{\op}})^{\op} = \mcC_S\)であること
  (cf. 本文 Remark 1.6.4)
  に注意すれば、所望の等式を得る。
  以上で\ref{1.14.3}の証明を完了し、
  \autoref{1.14}の解答を完了する。
\end{proof}








\begin{prob}\label{1.15}
  \(\mcC\)を圏とする。
  \(c\in \mcC\)に対して、
  \(h_c^{\mcC} \dfn \Hom_{\mcC}(-,c)\)を
  米田埋め込み\(\mcC\to \sfSet^{\mcC^{\op}}\)による
  \(c\in \mcC\)の像とする
  (\(\mcC\)が明らかな場合は上付き添字の\(\mcC\)を省略してたんに\(h_c\)と表す)。
  \(\Ind(\mcC)\)を\(\sfSet^{\mcC^{\op}}\)の充満部分圏であって、
  ある filtered diagram \(F:\mcI\to \mcC\)に対する
  \(\colim_{i\in \mcI} h_{F(i)}\)と同型な対象たちからなるものとする。

  \(\mcC\)をさらにアーベル圏であるとして、
  \(S_X\)を over category \(\mcC_{X/}\)の充満部分圏であって、
  擬同型\(X\to X'\)たちからなるものとする。
  \begin{enumerate}
    \item \label{1.15.1}
    \(\sigma(X) \dfn \colim_{X'\in S_X}h_{X'}\)
    によって函手
    \(\sigma:\sfD^+(\mcC)\to \Ind(\sfK^+(\mcC))\)
    がwell-definedに定まることを示し、
    \(\sigma\)が忠実充満であることを示せ。
    \item \label{1.15.2}
    \(F:\mcC\to \mcC'\)をアーベル圏の間の左完全函手とする。
    \(T(X)\dfn \colim_{X'\in S_X}h_{F(X')}^{\mcC'}\)
    と定める。
    これによって函手\(T:\sfD^+(\mcC)\to \Ind(\sfK^+(\mcC'))\)
    がwell-definedに定まることを示せ。
    \(F\)が\(X\in \sfD^+(\mcC)\)で\textbf{derivable}であるということを、
    ある対象\(Y\in \sfD^+(\mcC')\)が存在して
    \(T(X) \cong \sigma(Y)\)となることとして定義する。
    このような\(Y\)が (up to isomで) 一意的であることを示せ。
    また、\(F\)がすべての\(X\in \sfD^+(\mcC)\)でderivableであるときに、
    函手\(RF:\sfD^+(\mcC)\to \sfD^+(\mcC')\)で
    \(\sigma\circ RF \cong T\)となるものが (up to isomで) 一意的に存在することを示せ
    (すなわち、\(F\)は右導来函手\(RF\)をadmitsする)。
  \end{enumerate}
\end{prob}

\begin{proof}
  \ref{1.15.1}の函手\(\sigma\)のwell-defined性は
  \ref{1.15.2}の函手\(T\)のwell-defined性の特別な場合 (\(F=\id_{\mcC}\)の場合) であるので、
  まず\ref{1.15.2}の函手\(T\)がwell-definedに定まることを示す。
  \(T\)は函手\(\sfK^+(\mcC)\to \Ind(\sfK^+(\mcC'))\)
  としてはwell-definedに定まっている。
  \(X\in \sfK^+(\mcC)\)を\(0\)と擬同型な対象とする。
  このとき\(0\)-射\(X\to 0\)は圏\(S_X\)の終対象であるので、
  \[T(X) = \colim_{X'\in S_X}h_{F(X)} = h_{F(0)} = h_0 \cong 0\]
  となる。
  よって\(\Ind(\sfK^+(\mcC'))\)において\(T(X)\cong 0\)である。
  従って、本文命題1.6.9 (iii)より、
  函手\(T:\sfD^+(\mcC)\to \Ind(\sfK^+(\mcC'))\)がwell-definedに定まる。
  以上で\(T\)が (よって、\(\sigma\)も) well-definedに定まることがわかった。

  函手\(\sigma\)が忠実であることを示す。
  \(X,Y\)を\(\sfD^+(\mcC)\)の対象、
  \(f:X\to Y\)を\(\sfD^+(\mcC)\)の射であって、
  \(\sigma(f) = 0\)であるとする。
  \(f\)は\(\sfK^+(\mcC)\)の図式
  \(X\xrightarrow{f'} Y' \xleftarrow{t} Y\)によって代表される。
  ここで\(t\)は擬同型である。
  \(\sigma(f) = 0\)であることと、
  \(\sigma(t)\)が同型射であることから、
  \(\sigma(f')\)は\(0\)-射である。
  \(\id_X\in h_X(X)\)により代表される元
  \([\id_X]\in \sigma(X)(X) = \colim_{X'\in S_X}h_{X'}(X)\)の
  \(\sigma(f')(X): \sigma(X)(X)\to \sigma(Y')(X)\)での行き先は
  \(f':X\to Y'\)により代表される元
  \([f']\in \sigma(Y')(X) = \colim_{Y''\in S_{Y'}}h_{Y'}(X)\)
  であるが、\(\sigma(f') = 0\)であるから、
  \([f']=0\)である。
  これは、ある\([t':Y'\to Y'']\in S_{Y'}\)が存在して
  \(t'\circ f' = 0\)となることを意味する。
  さらに\(t'\circ f' = 0\)は\(f'\)が\(\sfD^+(\mcC)\)において\(0\)-射であることを意味する。
  よって\(f\)は\(\sfD^+(\mcC)\)において\(0\)-射であることが従う。
  以上より\(\sigma\)は忠実である。

  函手\(\sigma\)が充満であることを示す。
  \(f:\sigma(X)\to \sigma(Y)\)を\(\Ind(\sfK^+(\mcC))\)の射とする。
  \(\id_X:X\to X\)で代表される元
  \([\id_X]\in \sigma(X)(X) = \colim_{X'\in S_X}(h_{X'}(X))\)の
  \(f(X):\sigma(X)(X)\to \sigma(Y)(X)\)での行き先を
  \([f]\in \sigma(Y)(X) = \colim_{Y'\in S_Y}(h_{Y'}(X))\)と置く。
  \(S_Y\)はfilteredであるから、
  ある\([t:Y\to Y']\in S_Y\)とある射\(f':X\to Y'\)が存在して、
  \([f]\)は\(f'\)によって代表される。
  \(\sfK^+(\mcC)\)の図式\(X\xrightarrow{f'}Y' \xleftarrow{t} Y\)
  によって代表される\(\sfD^+(\mcC)\)の射を\(g\)と置くと、
  \(f'\)が\([f]\)を代表することから、
  \(\sigma(g)([\id_X]) = [f]\in \sigma(Y)(X)\)がわかる。
  これは\(\sigma(g) = f\)を意味する。
  以上より\(\sigma\)は充満であり、
  \ref{1.15.1}の証明を完了する。

  \ref{1.15.2}を証明する。
  \(T\)がwell-definedに定義されることは既に示している。
  \(Y\)の (up to isomでの) 一意性は\(\sigma\)が忠実であることから従う。
  すべての\(X\in \sfD^+(\mcC)\)で\(F\)がderivableであれば、
  \(F:\sfD^+(\mcC)\to \Ind(\sfK^+(\mcC'))\)は
  \(\sigma:\sfD^+(\mcC')\to \Ind(\sfK^+(\mcC'))\)の本質的像を一意的に経由するため、
  \(\sigma\)が忠実充満であることから、
  右導来函手\(RF:\sfD^+(\mcC)\to \sfD^+(\mcC')\)であって
  \(\sigma\circ RF \cong T\)となるものが (up to isomで) 一意的に存在する。
  以上で\autoref{1.15}の解答を完了する。
\end{proof}








\begin{prob}\label{1.16}
  \(\mcC\)を加法圏とする。
  \begin{enumerate}
    \item \label{1.16.1}
    \(X\in \Ch^-(\mcC), Y\in \Ch^+(\mcC)\)とする。
    以下の等式を証明せよ:
    \begin{align*}
      &Z^0(\Tot(\Hom_{\mcC}(X,Y))) = \Hom_{\Ch(\mcC)}(X,Y), \\
      &B^0(\Tot(\Hom_{\mcC}(X,Y))) = \Ht(X,Y), \\
      &H^0(\Tot(\Hom_{\mcC}(X,Y))) = \Hom_{\sfK(\mcC)}(X,Y).
    \end{align*}
    ただしここで\(\Hom_{\mcC}(X,Y)\)は二重複体とみなしている。
    \item \label{1.16.2}
    さらに\(\mcC\)がアーベル圏であり、
    十分入射的対象を持つか、または十分射影的対象を持つと仮定する。
    \(X\in \sfD^-(\mcC),Y\in \sfD^+(\mcC)\)に対して、
    次の等式を示せ:
    \[
    H^0(R\Hom_{\mcC}(X,Y)) = \Hom_{\sfD(\mcC)}(X,Y).
    \]
  \end{enumerate}
\end{prob}

\begin{proof}
  \ref{1.16.1}を示す。
  \(H^{i,j} \dfn \Hom_{\mcC}(X^{-i},Y^j)\)とおけば、
  \(X\in \Ch^-(\mcC), Y\in \Ch^+(\mcC)\)であることから、
  二重複体\(H^{i,j}\)は本文の条件(1.9.2)を満たし、
  \(\Ch^2_f(\mcC)\)に属する。
  \(f:X\to Y\)を\(\Ch(\mcC)\)の射とすると、
  \(f\)は\(\mcC\)の射の族\(f^n:X^n\to Y^n\)であって
  \(f^n\circ d_X^{n-1} = d_Y^{n-1}\circ f^{n-1}\)を満たすものである。
  よって、とくに\(f\in \bigoplus_{i+j = 0} H^{i,j} = \Tot^0(H^{\bullet,\bullet})\)であり、
  等式\(f^n\circ d_X^{n-1} = d_Y^{n-1}\circ f^{n-1}\)はさらに
  \(f\)が\(Z^0(\Tot(H^{\bullet,\bullet}))\)に属することを意味する。
  以上で\ref{1.16.1}の一つ目の等式が従う。
  \(f:X\to Y\)が homotopic to zero であるとする。
  このとき、ある\(\mcC\)の射の族\(s^n:X^n\to Y^{n-1}\)が存在して
  \(f^n = s^{n+1}\circ d_X^n + d_Y^{n-1} \circ s^n\)となる。
  射の族\(s=(s^n)_{n\in \Z}\)は\(\bigoplus_{i+j=-1}H^{i,j}\)に属し、
  等式\(f^n = s^{n+1}\circ d_X^n + d_Y^{n-1} \circ s^n\)は
  \(\Tot^{-1}(H^{\bullet,\bullet}))\to \Tot^0(H^{\bullet,\bullet}))\)
  での\(s\)の像が\(f\in \Tot^0(H^{\bullet,\bullet}))\)となることを意味する。
  以上で\ref{1.16.1}の二つ目の等式が従う。
  \ref{1.16.1}の三つ目の等式は\ref{1.16.1}の一つ目と二つ目の等式より直ちに従う。
  以上で\ref{1.16.1}の証明を完了する。

  \ref{1.16.2}を示す。
  \(\mcC\)が十分射影的対象を持つ場合、
  \(\mcC^{\op}\)を考えることによって、
  \(\mcC\)が十分入射的対象を持つ場合に帰着される (cf. 本文 Remark 1.10.10)。
  よって、\ref{1.16.2}を示すためには、
  \(\mcC\)が十分入射的対象を持つと仮定しても一般性を失わない。
  \autoref{1.15}の意味で\(S_Y\)という記号を用いる。
  \(\mcC\)は十分入射的対象を持つので、
  \autoref{1.15} \ref{1.15.1}より
  \(\Hom_{\sfD(\mcC)}(X,Y) \cong \colim_{Y'\in S_Y}\Hom_{\sfK(\mcC)}(X,Y')\)
  となり、また\autoref{1.15} \ref{1.15.2}より
  \(R\Hom_{\mcC}(X,Y) \cong \colim_{Y'\in S_Y}\Tot(\Hom_{\sfK(\mcC)}(X,Y'))\)
  となる。
  \(H^0\)を取ることで、
  \[
  H^0(R\Hom_{\mcC}(X,Y))
  \cong H^0(\colim_{Y'\in S_Y}\Tot(\Hom_{\sfK(\mcC)}(X,Y')))
  \]
  が従うが、\(S_Y\)はfilteredであるから、余極限は\(H^0\)と可換して、
  \[
  H^0(R\Hom_{\mcC}(X,Y)) \cong
  \colim_{Y'\in S_Y}H^0(\Tot(\Hom_{\sfK(\mcC)}(X,Y')))
  \]
  が従う。
  よって、\ref{1.16.1}の最後の等式で\(Y'\in S_Y\)に渡り余極限をとれば、
  \begin{align*}
    H^0(R\Hom_{\mcC}(X,Y))
    &\cong \colim_{Y'\in S_Y}H^0(\Tot(\Hom_{\sfK(\mcC)}(X,Y'))) \\
    &\cong \colim_{Y'\in S_Y}\Hom_{\sfK(\mcC)}(X,Y') \\
    &\cong \Hom_{\sfD(\mcC)}(X,Y)
  \end{align*}
  が従う。
  以上で\ref{1.16.2}の証明を完了し、
  \autoref{1.16}の解答を完了する。
\end{proof}






\begin{prob}\label{1.17}
  \(\mcC\)をアーベル圏とする。
  \(\mcC\)が\textbf{ホモロジー次元\(\leq n\)である}
  ということを、任意の\(X,Y\in \mcC\)に対して
  \(\Ext^i(X,Y) = 0 ,(\forall i > n)\)となることによって定義する。
  ただし、ここで\(\Ext^i(X,Y) \dfn \Hom_{\sfD(\mcC)}(X,Y[i])\)である。
  自然数\(n\)であって、
  \(\mcC\)がホモロジー次元\(\leq n\)となるもののうち、
  最小のものを\(\hd(\mcC)\)と表し、
  \(\mcC\)の\textbf{ホモロジー次元}と言う。

  \(\mcC\)は十分入射的対象を持つと仮定する。
  このとき、自然数\(n\)に対して、以下の主張が同値であることを示せ:
  \begin{enumerate}
    \item \label{1.17.1}
    \(\hd(\mcC) \leq n\)である。
    \item \label{1.17.2}
    任意の対象\(X\in \mcC\)に対して、
    \(X\)の入射分解\(X\to I\)であって、
    \(i > n\)に対して\(I^i = 0\)となるものが存在する。
  \end{enumerate}
\end{prob}

\begin{proof}
  \ref{1.17.1}\(\Rightarrow\)\ref{1.17.2}を示す。
  \(\hd(\mcC)\leq n\)であるとする。
  任意に対象\(X\in \mcC\)をとり、
  \(X\to I\)を入射分解とする。
  \(Y\in \mcC\)を任意の対象とすると、
  \autoref{1.16} \ref{1.16.2}より、
  \(H^i(\Hom_{\mcC}(Y,I)) \cong \Hom_{\sfD(\mcC)}(Y,X[i]) = \Ext^i(Y,X)\)
  である。
  \(\hd(\mcC)\leq n\)なので、
  \(H^{n+1}(\Hom_{\mcC}(Y,I)) = 0\)であり、
  従って
  \begin{align*}
    \im(\Hom_{\mcC}(Y,I^n)\to \Hom_{\mcC}(Y,I^{n+1}))
    &\cong \ker(\Hom_{\mcC}(Y,I^{n+1})\to \Hom_{\mcC}(Y,I^{n+2})) \\
    &\cong \Hom_{\mcC}(Y,\ker(d_I^{n+1})) \\
    &\cong \Hom_{\mcC}(Y,\im(d_I^n))
  \end{align*}
  となる。
  よって、完全列
  \[
  \begin{CD}
    0 @>>> \ker(d_I^n) @>>> I^n @>>> \im(d_I^n) @>>> 0
  \end{CD}
  \]
  は任意の\(Y\)に対する
  \(\Hom_{\mcC}(Y,-)\)を適用したあとも完全である。
  従って、\autoref{1.4}より、
  \(I^n\cong \ker(d_I^n) \oplus \im(d_I^n)\)となることがわかる。
  \(I^n\)は入射的対象であるから、
  その直和因子である\(\ker(d_I^n)\)も入射的対象である。
  従って、\(X\to \tau^{\leq n}(I)\)は長さが\(n\)以下の入射分解となる。
  以上で\ref{1.17.1}\(\Rightarrow\)\ref{1.17.2}の証明を完了する。

  \ref{1.17.2}\(\Rightarrow\)\ref{1.17.1}を示す。
  任意に対象\(X\in \mcC\)をとり、
  \(X\to I\)を長さ\(n\)以下の入射分解とする。
  \(Y\in \mcC\)を任意の対象とすると、
  \autoref{1.16} \ref{1.16.2}より、
  \(H^i(\Hom_{\mcC}(Y,I)) \cong \Hom_{\sfD(\mcC)}(Y,X[i]) = \Ext^i(Y,X)\)
  であるので、\(I^i = 0, (i>n)\)より、\(i>n\)に対して
  \(\Ext^i(Y,X) = 0\)となることがわかる。
  以上で\autoref{1.17}の解答を完了する。
\end{proof}






\begin{prob}\label{1.18}
  \(\mcC\)を\(\hd(\mcC) \leq 1\)のアーベル圏とする。
  \(X\in \sfD^b(\mcC)\)を複体とするとき、
  \(\sfD^b(\mcC)\)で
  \[
  X\cong \bigoplus_{k\in \Z}H^k(X)[-k]
  \]
  となることを示せ。
\end{prob}

\begin{proof}
  シフトすることで、\(X^i = 0 ,(\forall i < 0)\)と仮定しても一般性を失わない。
  \(X^n \neq 0\)となる最大の\(n\)に関する帰納法で証明する。
  \(n=0\)であれば主張は自明であるので、
  ある\(n=k\)に対して主張が成立するときに、
  \(n=k+1\)の場合にも成立することを証明する。
  帰納法の仮定より、
  \[
  \tau^{\leq n-1}(X)\cong \bigoplus_{k\in \Z}H^k(\tau^{\leq n-1}(X))[-k]
  \cong \bigoplus_{k\leq n-1}H^k(X)[-k]
  \]
  である。
  従って、所望の同型を証明するためには、
  \(n=1\)の場合、さらに\(d_X^0:X^0\to X^1\)がモノ射である場合に、
  \(\sfD^b(\mcC)\)で\(X\cong \coker(d_X^0)[-1]\)となることを証明することが十分である。

  \(X\in \sfD^b(\mcC)\)は
  \(X^i = 0, (i\in (-\infty,0]\cup (1,+\infty))\)であり、
  さらに\(d_X^0:X^0\to X^1\)がモノ射であるとする。
  \(X^1\to I\)を入射的対象\(I\)へのモノ射とすると、
  \(\hd(\mcC)\leq 1\)であるから、
  \(I/X^0,I/X^1\)はどちらも入射的対象となる。
  複体\(J\)を\(J^0=J^1=I, d_J^0 = \id_I\)で定義し、
  \(J_1\)を\(J_1^0=I/X^0,J_1^1=I/X^1\)で\(d_{J_1}^0\)を自然な射として定義すると、
  \(J_1\)は\(X^1/X^0\)の入射分解であり、
  \(0\to X \to J \to J_1 \to 0\)は\(\Ch(\mcC)\)の完全列である。
  従って、\(X\to J\to J_1\to X[1]\)は\(\sfD(\mcC)\)の完全三角である。
  \(J_1\)は\(X^1/X^0\)の入射分解であるから、
  \(\sfD(\mcC)\)において\(J_1\cong X^1/X^0\)である。
  以上より、\(\sfD(\mcC)\)の完全三角\(X\to J \to X^1/X^0\to X[1]\)を得る。
  さらに、定義より\(\sfD(\mcC)\)において\(J\cong 0\)であるから、
  これは\(\sfD(\mcC)\)において\(X\cong (X^1/X^0)[-1]\)となることを示している。
  以上で\autoref{1.18}の解答を完了する。
\end{proof}




\begin{prob}\label{1.19}
  \(\mcC,\mcC'\)を二つのアーベル圏、
  \(F:\mcC \to \mcC'\)を左完全函手とする。
  さらに\(\mcI\subset \mcC\)を\(F\)-injectiveな部分圏とする。
  対象\(X\in \mcC\)が\textbf{\(F\)-acyclic}であるということを、
  任意の\(k\neq 0\)に対して\(R^kF(X) = 0\)となることとして定義する。
  \(\mcJ\subset \mcC\)を\(F\)-acyclicな対象からなる充満部分圏とする。
  \begin{enumerate}
    \item \label{1.19.1}
    \(\mcJ\)は\(F\)-injectiveであることを示せ。
    \item \label{1.19.2}
    任意の自然数\(n\geq 0\)に対して、
    以下の主張が同値であることを証明せよ:
    \begin{enumerate}
      \item \label{1.19.2.1}
      任意の\(k > n\)と任意の対象\(X\in \mcC\)に対して\(R^kF(X) = 0\)である。
      \item \label{1.19.2.2}
      任意の対象\(X\in \mcC\)に対して、
      完全列
      \[
      0 \to X \to X^0 \to \cdots \to X^n \to 0
      \]
      で各\(0\leq j\leq n\)に対して\(X^j\in \mcJ\)となるものが存在する。
      \item \label{1.19.2.3}
      \(X^0\to \cdots X^n \to 0\)が完全であり、
      任意の\(j < n\)に対して\(X^j\in \mcJ\)であるとき、
      \(X^n\in \mcJ\)である。
    \end{enumerate}
    これらの同値な条件のうちのどれか一つが成立するとき、
    \(F\)は\textbf{コホモロジー次元\(\leq n\)を持つ}と言う。
  \end{enumerate}
\end{prob}

\begin{proof}
  \ref{1.19.1}を示す。
  まず、\(F\)-injectiveな対象は\(F\)-acyclicなので
  (cf. 本文命題1.8.3とその直前の記述)、
  \(\mcI\subset \mcJ\)であり、
  従って\(\mcJ\)は本文定義1.8.2の条件(i)を満たす。
  また、\(\mcJ\)に属する対象はすべて\(F\)-acyclicであるから、
  \(\mcJ\)が本文定義1.8.2の条件(iii)を満たすことは明らかである。
  \(X'\to X\)を\(\mcJ\)に属する対象の間のモノ射として
  \(X'' \dfn X/X'\)とすると、
  各\(i\geq 1\)に対して
  完全列\(R^iF(X)\to R^iF(X/X')\to R^{i+1}F(X')\)を得る。
  \(X,X'\)は\(F\)-acyclicであるから、
  \(R^iF(X)=0, R^{i+1}F(X')=0\)であり、
  従って\(R^iF(X/X')=0\)もわかる。
  これは\(X/X'\)が\(F\)-acyclicであることを示していて、
  \(X/X'\)は\(\mcJ\)に属する。
  よって\(\mcJ\)は本文定義1.8.2の条件(ii)を満たし、
  \(\mcJ\)は\(F\)-injectiveである。
  以上で\ref{1.19.1}の証明を完了する。

  \ref{1.19.2}を示す。
  \ref{1.19.2}\ref{1.19.2.1}\(\Leftrightarrow\)\ref{1.19.2}\ref{1.19.2.2}を示す。
  \ref{1.19.2}\ref{1.19.2.1}\(\Leftrightarrow\)\ref{1.19.2}\ref{1.19.2.2}
  を示すためには、対象\(X\in \mcC\)を固定して、
  次の二つの主張が同値であることを証明することが十分である:
  \begin{enumerate}
    \item \label{1.19.2.p1}
    任意の\(k > n\)に対して\(R^kF(X) = 0\)である。
    \item \label{1.19.2.p2}
    完全列
    \[
    0 \to X \to X^0 \to \cdots \to X^n \to 0
    \]
    で各\(0\leq j\leq n\)に対して\(X^j\in \mcJ\)となるものが存在する。
  \end{enumerate}
  \(n\)に関する帰納法により
  \ref{1.19.2.p1}\(\Leftrightarrow\)\ref{1.19.2.p2}
  を示す。
  \(n=0\)に対して\ref{1.19.2.p1}が成り立つことは、
  \(X\)が\(F\)-acyclicであることと同値であり、
  さらにこれは\(n=0\)に対して\ref{1.19.2.p2}が成り立つことと同値である。
  よって\(n=0\)の場合は明らかに
  \ref{1.19.2.p1}\(\Leftrightarrow\)\ref{1.19.2.p2}
  が成り立つ。
  \(n\geq 1\)として、\(n\)より小さいすべての自然数に対して
  \ref{1.19.2.p1}\(\Leftrightarrow\)\ref{1.19.2.p2}
  が成り立つと仮定する。
  \(\mcJ\)は本文定義1.8.2の条件(i)を満たすので、
  モノ射\(d:X\to X^0\)が存在する。
  \(X^0\)は\(F\)-acyclicであるから、
  任意の\(k > n\)に対して
  \(R^{k-1}F(\coker(d)) \cong R^kF(X)\)となる。
  従ってとくに、\(X\)と\(n\)に対して\ref{1.19.2.p1}が成り立つことは、
  \(X=\coker(d)\)と\(n-1\)に対して\ref{1.19.2.p1}が成り立つことと同値である。
  帰納法の仮定により、これは
  \(X=\coker(d)\)と\(n-1\)に対して\ref{1.19.2.p2}が成り立つことと同値である。
  さらに\(\coker(d)\)に対する\ref{1.19.2.p2}の完全列を
  \(X^0\to \coker(d)\)と繋ぐことを考えれば、
  \(X=\coker(d)\)と\(n-1\)に対して\ref{1.19.2.p2}が成り立つことは
  \(X\)と\(n\)に対して\ref{1.19.2.p2}が成り立つことと同値である。
  以上で\ref{1.19.2.p1}\(\Leftrightarrow\)\ref{1.19.2.p2}
  の証明を完了し、
  従って
  \ref{1.19.2}\ref{1.19.2.1}\(\Rightarrow\)\ref{1.19.2}\ref{1.19.2.2}
  の証明を完了する。

  \ref{1.19.2}\ref{1.19.2.1}\(\Rightarrow\)\ref{1.19.2}\ref{1.19.2.3}
  を示すためには、
  各対象\(X\in \mcC\)に対して
  次の二つの主張が同値であることを証明することが十分である:
  \begin{enumerate}
    \item \label{1.19.2.q1}
    任意の\(k > n\)に対して\(R^kF(X) = 0\)である。
    \item \label{1.19.2.q3}
    完全列
    \[
    0 \to X \to X^0 \to \cdots \to X^n \to 0
    \]
    が条件「各\(j < n\)に対して\(X^j\in \mcJ\)である」を満たせば、
    \(X^n\in \mcJ\)となる。
  \end{enumerate}
  \(n=0\)に対して\ref{1.19.2.q1}が成り立つことは、
  \(X\)が\(F\)-acyclicであることと同値であり、
  これは\(n=0\)に対して\ref{1.19.2.q3}が成り立つことと同値である。
  よって\(n=0\)の場合は明らかに
  \ref{1.19.2.q1}\(\Leftrightarrow\)\ref{1.19.2.q3}
  が成り立つ。
  \(n\geq 1\)として、\(n\)より小さいすべての自然数に対して
  \ref{1.19.2.q1}\(\Leftrightarrow\)\ref{1.19.2.q3}
  が成り立つと仮定する。
  \[
  0 \to X \xrightarrow{d} X^0 \to \cdots \to X^n \to 0
  \]
  を条件「各\(j < n\)に対して\(X^j\in \mcJ\)である」を満たす完全列とする。
  \(X^0\)は\(F\)-acyclicであるから、
  任意の\(k > n\)に対して
  \(R^{k-1}F(\coker(d)) \cong R^kF(X)\)となる。
  よって、\(X\)と\(n\)に対して\ref{1.19.2.q1}が成り立つことは、
  \(X=\coker(d)\)と\(n-1\)に対して\ref{1.19.2.q1}が成り立つことと同値である。
  帰納法の仮定により、これは
  \(X=\coker(d)\)と\(n-1\)に対して\ref{1.19.2.q3}が成り立つことと同値である。
  一方これは明らかに
  \(X\)と\(n\)に対して\ref{1.19.2.q3}が成り立つことと同値であるから、
  よって\ref{1.19.2.q1}\(\Leftrightarrow\)\ref{1.19.2.q3}が従う。
  以上で\ref{1.19.2}の証明を完了し、
  \autoref{1.19}の解答を完了する。
\end{proof}





\begin{prob}\label{1.20}
  \(\mcC,\mcD,\mcE\)をそれぞれアーベル圏として、
  \(F:\mcC\to \mcD, G:\mcD\to \mcE\)を左完全函手とする。
  \(F\)-injectiveな\(\mcI\subset \mcC\)と
  \(G\)-injectiveな\(\mcJ\subset \mcD\)が存在して、
  \(F(\mcI)\subset \mcJ\)となると仮定する (本文命題1.8.7の状況設定)。
  さらに、\(F\)はコホモロジー次元\(\leq r\)を持ち、
  \(G\)はコホモロジー次元\(\leq s\)を持つとする。
  このとき、\(G\circ F\)はコホモロジー次元\(\leq r+s\)を持つことを示せ。
\end{prob}

\begin{proof}
  \(X\in \mcC\)を任意にとる。
  \(F\)はコホモロジー次元\(\leq r\)を持つので、
  ある擬同型\(X\xrightarrow{\text{qis.}} I\)で、
  各\(k\)について\(I^k\)は\(F\)-acyclicであり、
  さらに\(\tau^{\leq r}(I) = I\)となるものがある。
  このとき、\(RF(X) \cong RF(I)\)であるが、
  本文命題1.8.3と \autoref{1.19} \ref{1.19.1} より、
  さらに\(RF(I)\cong F(I)\)となる。
  ただしここで\(F(I)\)は各\(I^k\)を\(F\)で送ることによって得られる複体
  (つまり\(\sfK^+(F)(I)\)) を表している。
  従って、本文命題1.8.7より、
  \(R(G\circ F)(X)\cong RG(RF(X)) \cong RG(F(I))\)が従う。
  \(G\)のコホモロジー次元が\(\leq s\)であることと、
  \(\tau^{\leq r}(F(I)) = F(I)\)であることから、
  \autoref{1.20}を示すためには、次の主張を証明することが十分である:
  \begin{enumerate}[label=(\fnsymbol*),start=2]
    \item \label{1.20.p}
    \(F:\mcC\to \mcD\)をアーベル圏の間の左完全函手とする。
    \(F\)はコホモロジー次元\(\leq r\)を持ち、
    さらに\(F\)-injectiveな\(\mcI\subset \mcC\)が存在すると仮定する。
    \(n\geq 0\)を自然数とする。
    このとき、\(\tau^{\leq n}(X) = X\)が成り立つ
    任意の\(X\in \Ch^+(\mcC)\)に対して、
    自然な射\(\tau^{\leq n+r}(RF(X)) \to RF(X)\)は同型射である。
  \end{enumerate}
  \(n\)に関する帰納法により\ref{1.20.p}を示す。
  \(n=0\)の場合は \autoref{1.19} \ref{1.19.2} \ref{1.19.2.2} より従う。
  \(n-1\)以下で\ref{1.20.p}が成立すると仮定する。
  \(Y\dfn \tau^{\leq n-1}(X), Z\dfn \coker(d_X^{n-1})\in \mcC\)と置く。
  このとき、\(Y\to X\)のconeは\(Z[n]\)と擬同型である。
  また、\(\tau^{\leq n-1}(Y)=Y\)であるから、帰納法の仮定より
  \(\tau^{\leq n-1+r}(RF(\tau^{\leq n-1}(X))) \cong RF(\tau^{\leq n-1}(X))\)
  であり、
  \(\tau^{\leq 0}(Z) = Z\)であるから、
  すでに示されている\(n=0\)の場合より、
  \(\tau^{\leq n+r}(RF(Z[n])) = \tau^{\leq r}(RF(Z))[n]
  cong RF(Z)[n] = RF(Z[n])\)である。
  完全三角
  \(Y\to X \to Z[n] \to Y[1]\)
  に\(RF\)を適用して得られる完全三角
  \(RF(Y) \to RF(X) \to RF(Z[n]) \to RF(Y[1])\)
  に\(\tau^{\leq n+r}\)を適用すれば、
  完全三角
  \[
  \tau^{\leq n+r}(RF(Y)) \to \tau^{\leq n+r}(RF(X)) \to
  \tau^{\leq n+r}(RF(Z[n])) \to \tau^{\leq n+r}(RF(Y[1]))
  \]
  を得る。
  \(\tau^{\leq n+r}(RF(Y)) \cong RF(Y),
  \tau^{\leq n+r}(RF(Z[n])) \cong RF(Z[n])\)より、完全三角
  \[
  RF(Y) \to \tau^{\leq n+r}(RF(X)) \to Z[n] \to Y[1]
  \]
  を得る。
  以上で\ref{1.20.p}の証明を完了し、
  \autoref{1.20}の解答を完了する。
\end{proof}







\begin{prob}\label{1.21}
  \(F:\mcC\to\mcD\)をアーベル圏の間の左完全函手とする。
  \(F\)-injectiveな\(\mcI\subset \mcC\)が存在すると仮定する。
  \(X\in \sfD^+(\mcC)\)は\(i>0,j\leq j_0\)に対して
  \(R^iF(H^j(X)) = 0\)を満たすとする。
  このとき、\(j\leq j_0\)に対して\(R^jF(X) \cong F(H^j(X))\)
  となることを示せ。
\end{prob}

\begin{proof}
  \(X\in \sfD^+(\mcC)\)であるから、
  \autoref{1.21}を示すためには\(j_0 \geq 0\)であると仮定しても一般性を失わない。
  \(j_0\)に関する帰納法で\autoref{1.21}を示す。
  \(j_0=0\)の場合、
  \(R^0F(X)\cong \ker(F(d_X^0)) \cong F(\ker(d_X^0)) = F(H^0(X))\)
  であるから主張は自明である。

  \(j_0\)未満で\autoref{1.21}が成り立つと仮定する。
  \(Y\dfn \tau^{\leq j_0-1}(X), Z\dfn \tau^{\geq j_0}(X)\)とすると
  \(Y\to X\to Z\)は完全三角であり、
  \(Z[-j_0]\in \sfD^+(\mcC)\)であり、
  帰納法の仮定より、\(j\leq j_0-1\)に対して\(R^jF(Y) \cong F(H^j(Y))\)であり、
  さらに\(\tau^{\geq j_0}(Y) = 0\)であるから\(R^jF(Y) = 0, (j\geq j_0)\)である。
  \(X\)が今の\(j_0\)に対して\autoref{1.21}の仮定を満たすことから、
  \(Z[-j_0]\)は\(j_0=0\)に対して\autoref{1.21}の仮定を満たし、
  すでに示したことによって\(R^0f(Z[-j_0])\cong F(H^0(Z[-j_0]))\)となる。
  従って、\(R^jF(Z) = 0, (j \leq j_0-1)\)かつ
  \(R^{j_0}F(Z) \cong F(H^{j_0}(Z))\)である。
  \(Z = \tau^{\geq j_0}(X)\)なので\(H^{j_0}(Z) \cong H^{j_0}(X)\)であり、
  従って\(R^{j_0}F(Z) \cong F(H^{j_0}(X))\)が従う。

  完全三角\(Y\to X\to Z\to Y[1]\)に\(RF\)を適用して得られる
  完全三角\(RF(Y) \to RF(X) \to RF(Z)\to RF(Y)[1]\)の
  コホモロジーをとることで、長完全列
  \[
  R^jF(Y) \to R^jF(X) \to R^jF(Z) \to R^{j+1}F(Y)
  \]
  を得る。
  ここで\(j\leq j_0-1\)に対して\(R^jF(Y) \cong F(H^j(Y))\)であることと、
  \(j\leq j_0-1\)に対して\(R^jF(Z) = 0\)であることから、
  \(j\leq j_0-1\)に対して\(R^jF(Y)\to R^jF(X)\)は同型射である。
  さらに、\(\tau^{\leq j_0-1}(Y) = Y\)であるから、
  \(R^{j_0}F(Y) = 0\)である。
  従って、\(R^{j_0}F(X)\to R^{j_0}F(Z)\)が同型射となる。
  よって\(R^{j_0}F(X)\cong R^{j_0}F(Z) \cong F(H^{j_0}(X))\)が従う。
  以上で\autoref{1.21}の解答を完了する。
\end{proof}








\begin{prob}\label{1.22}
  \(\mcC,\mcD,\mcE\)をそれぞれアーベル圏として、
  \(F:\mcC\to \mcD, G:\mcD\to \mcE\)を左完全函手とする。
  \(F\)-injectiveな\(\mcI\subset \mcC\)と
  \(G\)-injectiveな\(\mcJ\subset \mcD\)が存在して、
  \(F(\mcI)\subset \mcJ\)となると仮定する (本文命題1.8.7の状況設定)。
  \(X\in \sfD^+(\mcC)\)は\(R^jF(X) = 0, (\forall j<n)\)を満たすと仮定する。
  \(R^n(G\circ F)(X) \cong (G\circ R^nF)(X)\)を示せ。
\end{prob}

\begin{proof}
  本文 Remark 1.8.6 を\(RF(X)\)と\(G\)に対して適用することで、
  任意の\(j<n\)に対して\(R^jG(RF(X)) = 0\)であり、さらに
  \(R^nG(RF(X)) \cong G(H^n(RF(X))) = G(R^nF(X))\)である。
  また、本文命題1.8.7より\(R(G\circ F)(X) \cong RG(RF(X))\)であるので、
  \(n\)番目のコホモロジーをとれば
  \(R^n(G\circ F)(X) \cong R^nG(RF(X))\)が従う。
  よって\(R^n(G\circ F)(X) \cong R^nG(RF(X))\cong G(R^nF(X))\)となり、
  以上で\autoref{1.22}の解答を完了する。
\end{proof}








\begin{prob}\label{1.23}
  \(\mcC\)をアーベル圏、\(\mcI\subset \mcC\)を充満部分圏とする。
  \(\mcI\)が本文の条件(1.7.5),(1.7.6)を満たすとし、
  さらに次を条件を満たすと仮定せよ:
  \begin{enumerate}[label=(\fnsymbol*),start=2]
    \item \label{1.23.c}
    \(0\to X'\to X\to X''\to 0\)を\(\mcC\)の完全列であって、
    \(X'\in \mcI\)であると仮定する。
    このとき、\(X\in \mcI\)であることは\(X''\in \mcI\)であることと同値である。
  \end{enumerate}
  \(*=\empty,b,-,+\)とする。
  \begin{enumerate}
    \item \label{1.23.1}
    任意の対象\(X\in \Ch^*(\mcC)\)は
    ある\(Y\in \Ch^*(\mcI)\)と擬同型であることを示せ。
    \item \label{1.23.2}
    \(\mcD\)を別のアーベル圏、\(F:\mcC\to \mcD\)を左完全函手とする。
    \(\mcI\)が\(F\)-injectiveであると仮定する。
    このとき\(F\)の右導来函手\(RF:\sfD^*(\mcC)\to \sfD^*(\mcD)\)が存在することを示せ。
    \item \label{1.23.3}
    \(\mcE\)をさらに別のアーベル圏、
    \(G:\mcC\times \mcD \to \mcE\)を左完全な双函手とする。
    各\(Y\in \mcD\)に対して\(\mcI\)は\(G(-,Y)\)-injectiveであるとし、
    さらに各\(I\in \mcI\)と\(0\)と擬同型な\(Y\in \Ch^{\star}(\mcD)\)に対して
    \(G(I,Y)\)は\(0\)と擬同型であるとする。
    このとき、\((*,\star) = (-,-)\)の場合と\((*,\star) = (*,b)\)の場合で、
    \(G\)の右導来函手
    \(RG:\sfD^*(\mcC)\times \sfD^{\star}(\mcD) \to \sfD^*(\mcE)\)
    が存在することを示せ。
  \end{enumerate}
\end{prob}

\begin{rem*}
  少なくとも第一版では\ref{1.23.3}の仮定に
  「各\(I\in \mcI\)と\(0\)と擬同型な\(Y\in \Ch^{\star}(\mcD)\)に対して
  \(G(I,Y)\)は\(0\)と擬同型である」
  というものはなかった。
  なくても証明できるのか?
  本文系1.10.5とパラレルであることを想定すればこの仮定がなければ微妙になると思うが...
\end{rem*}

\begin{proof}
  \ref{1.23.1}を示す。
  \(*=b\)の場合は本文の系1.7.8、
  \(*=+\)の場合は本文の命題1.7.7より従う。
  \(*=\emptyset\)の場合を証明すれば、
  本文の系1.7.8と全く同様の議論により、\(*=-\)の場合が従う。
  残っているのは、\(*=\emptyset\)の場合に\ref{1.23.1}を示すことである。
  \(*=+\)の場合の構成を詳細に見るため、
  \(*=+\)の場合の証明を思い出す。
  \(Z\in \Ch(\mcC)\)に対して、
  複体\(\tilde{\tau}^{\leq n}(Z)\)を次で定義する:
  \[
  \cdots \to Z^i \to \cdots \to Z^n \xrightarrow{d_Z^n} Z^{n+1}
  \to \coker(d_Z^n) \to 0 \to \cdots \to 0 \to \cdots.
  \]
  このとき、自然な全射の列
  \(Z\to \tilde{\tau}^{\leq n}(Z)\to \tilde{\tau}^{\leq n-1}(Z)\)
  が存在して、
  合成\(\tau^{\leq n}(Z) \to Z \to \tilde{\tau}^{\leq n}(Z)\)は擬同型であり、
  さらに自然な射
  \(Z \xrightarrow{\sim} \lim_{n\to \infty}\tilde{\tau}^{\leq n}(Z)\)
  は同型射である (極限が存在することに注意)。
  また、\(Z^{\leq n}\)を次で定義する (\(Z^{\geq n}\)も同様に定義する):
  \[
  \cdots \to Z^i \to \cdots \to Z^n \to 0 \to \cdots \to 0 \to \cdots.
  \]
  このとき、自然な複体のモノ射
  \(\tilde{\tau}^{\leq n-1}(Z) \to Z^{\leq n+1}\)
  が存在する。
  \(X\in \Ch^+(\mcC)\)とする。
  ある\(n\)に対して次の条件が成り立つと仮定する:
  任意の\(m\leq n\)に対して、
  \begin{itemize}
    \item 複体\(I_m\in \Ch^+(\mcI)\)であって\(I_n = I_n^{\leq n}\)を満たすもの、
    \item 複体のモノ射\(f_m:X^{\leq m}\to I_m\)であって
    \(\tilde{\tau}^{\leq m-1}(f_m):
    \tilde{\tau}^{\leq m-1}(X) \to \tilde{\tau}^{\leq m-1}(I_m)\)
    がモノな擬同型となるもの、
  \end{itemize}
  が存在し、任意の\(m_1\leq m_2\leq n\)に対して
  \(I_{m_2}^{\leq m_1} = I_{m_1}, f_{m_1}^i = f_{m_2}^i, (\forall i\leq m_1)\)
  を満たすと仮定する。
  この条件は十分小さい\(n\ll 0\)に対してはつねに満たされる
  (\(n\ll 0\)に対しては\(X^{\leq n} = 0\)なので\(I_n=0\)とすれば良い)。
  \(Y_{n+1} \dfn \tilde{\tau}^{\leq n-1}(I_n)
  \coprod_{\tilde{\tau}^{\leq n-1}(X)} X^{\leq n+1}\)とおくと、
  自然な射\(X^{\leq n+1}\to Y_{n+1}\)は
  モノな擬同型によるpush-outであるのでモノな擬同型である。
  さらに各\(i\leq n\)に対して
  \((\tilde{\tau}^{\leq n}(X))^i \to (X^{\leq n+1})^i\)は同型射なので、
  各\(i\leq n\)に対して\(I_n^i \to Y_{n+1}^i\)も同型射である。
  \(\mcI\)は本文の条件(1.7.5)を満たすので、
  あるモノ射\(Y_{n+1}^{n+1} \to J\)が存在する。
  複体\(I_{n+1}\)を、
  各\(i\leq n\)に対して\(I_{n+1}^i \dfn Y_{n+1}^i \cong I_n^i\)、
  \(i > n+1\)に対して\(I_{n+1}\dfn 0\)、
  \(I_{n+1}^{n+1} \dfn J\)と定めることで、
  モノ射の列\(X^{\leq n+1}\to Y_{n+1} \to I_{n+1}\)を得る。
  この合成を\(f_{n+1}\)とおく。
  すると構成より各\(i\leq n\)に対して\(f_{n+1}^i = f_n\)である。
  また、図式
  \[
  \begin{CD}
    \tilde{\tau}^{\leq n-1}(X) @>>> \tilde{\tau}^{\leq n-1}(I_n) \\
    @VVV @VVV \\
    X^{\leq n+1} @>>> I_{n+1}
  \end{CD}
  \]
  で\autoref{1.6} \ref{1.6.3} \ref{1.6.4}を用いることで
  \(\tilde{\tau}^{\leq n}(f_{n+1})\)がモノな擬同型であることが従う。
  こうして任意の\(n\)に対してモノ射
  \(f_n:X^{\leq n}\to I_n, I_n=I_n^{\leq n} \in \Ch^+(\mcI)\)であって
  \(\tilde{\tau}^{\leq n-1}(f_n)\)がモノな擬同型となるものが存在することがわかったので、
  あとは\(\lim_{n\to \infty}\tilde{\tau}^{\leq n}(f_n)\)をとれば
  所望の擬同型\(f:X\to I, I\in \Ch^+(\mcI)\)を得る。
  構成から、\(f\)はモノ射であり、
  \(X^i = 0\)なら\(f^i = 0\)、となるようにとれる。

  \(\mcI\)が自然数\(d\geq 0\)に対して本文の条件(1.7.6)を満たすとする。
  複体\(Z\)と各\(n\)に対して、
  モノな擬同型\(Z_{\geq n}\to I, (I\in \Ch(\mcI))\)
  をとって\(Z_{\leq -n-1}\)と繋げることで、
  新しい複体\(Z'\)と
  モノな擬同型\(f_0:Z\to Z'\)であって
  \((Z')^i=Z^i,f_0^i = \id_{Z^i},(\forall i < n)\)
  であり、さらに\((Z')^i\in \mcI, (\forall i \geq n)\)
  となるものが存在することが従う。
  この複体\(Z'\)を\(I_n(Z)\)で表す。

  複体\(Z\)が、ある\(i_0\)以上の全ての\(i\geq i_0\)に対して
  \(Z^i\in \mcI\)を満たすと仮定する。
  \(Z_1 \dfn I_{i_0-1}(Z)\)とおくと、
  \(C\dfn \coker(Z\to Z_1)\)は完全であり、
  さらに\(\mcI\)が条件\ref{1.23.c}を満たすことより、
  \(C^i\in \mcI,(\forall i \geq i_0)\)である。
  また、\(C\)が完全であることと
  \(\mcI\)が\(d\)に対して本文の条件(1.7.6)を満たすことより、
  各\(i \geq i_0+d-1\)に対して\(\im(d_C^i) \in \mcI\)である。
  従って、\(C\)が完全であることより、
  \(\tau^{\leq i_0+d}(C)\)は任意の\(i \geq i_0\)に対して
  \((\tau^{\leq i_0+d}(C))^i\in \mcI\)を満たす。
  \(J_{i_0}(Z)\dfn Z_1\times_C\tau^{\leq i_0+d}(C)\)
  (複体の圏でのfiber積) とおくと、
  \(Z\to Z_1\)がモノであることから、
  \[
  \begin{CD}
    0 @>>> Z @>>> J_{i_0}(Z) @>>> \tau^{\leq i_0+d}(C) @>>> 0
  \end{CD}
  \]
  は (複体の圏で) 完全である。
  任意の\(i \geq i_0\)に対して
  \((\tau^{\leq i_0+d}(C))^i\in \mcI\)を満たすことと
  \(\mcI\)は条件\ref{1.23.c}を満たすことより、
  任意の\(i\geq i_0\)に対して\((J_{i_0}(Z)^i\in \mcI\)である。
  さらに\(i > i_0+d\)に対して\((\tau^{\leq i_0+d}(C))^i=0\)なので、
  \(i > i_0+d\)に対して\(Z^i\xrightarrow{\sim}(J_{i_0}(Z))^i\)であり、
  \(i < i_0+d\)に対して\((\tau^{\leq i_0+d}(C))^i\xrightarrow{\sim}C^i\)なので、
  \(i < i_0+d\)に対して\((J_{i_0}(Z))^i\xrightarrow{\sim}Z_1^i\)である。
  従って、とくに\(J_{i_0}(Z)^{i_0-1}\cong Z_1^{i_0-1}\in \mcI\)が従う。
  まとめると、モノな擬同型\(Z\to J_{i_0}(Z)\)であって、
  \(i > i_0+d\)に対して\(Z^i\xrightarrow{\sim}(J_{i_0}(Z))^i\)であり、
  \(i \geq i_0-1\)に対して\((J_{i_0}(Z))^i \in \mcI\)となるものが存在する。

  \(I_0,J_n\)を用いて\ref{1.23.1}の証明を行う。
  複体\(X\in \Ch(\mcC)\)を任意にとる。
  \(X_0\dfn I_0(X)\)とおく。
  各\(n < 0\)に対して、
  \(X_n\dfn J_{n+1}(X_{n+1})\)と定義して、
  モノな擬同型\(X_{n+1}\to X_n\)たちの余極限をとる。
  このとき、各\(i\geq n+d\)に対して\(X_n^i \to X_{n-1}^i\)は同型射であるから、
  \(Y \dfn \colim_{n\to -\infty}X_n\)は圏\(\Ch(\mcC)\)に存在して、
  各\(i\in \Z\)に対して\(Y^i\cong X_{-|i|-d}^i\in \mcI\)となる。
  さらに各\(X_n\to X_{n-1}\)と\(X\to X_0\)はすべて擬同型であるから、
  \(X\to Y\)も擬同型である。
  よって所望の複体と擬同型が構成できた。
  以上で\ref{1.23.1}の証明を完了する。

  \ref{1.23.2}を示す。
  \(\mcN\)を\(\sfK^*(\mcC)\)の充満部分三角圏であって
  \(0\)と擬同型な複体すべてからなるものとする。
  \(\mcN'\dfn \mcN \cap \sfK^*(\mcI)\)とおく。
  すると\ref{1.23.1}より、自然な射
  \(\sfK^*(\mcI)/\mcN' \to \sfD^*(\mcC)\)
  は圏同値である。
  また、\(\mcI\)が\(F\)-injectiveであることから、
  \(\sfK^*(\mcI)\)の\(0\)と擬同型な対象は\(F\)によってacyclicな対象へと写される。
  従って、\(\sfK^*(\mcI) \subset \sfK^*(\mcC)\)と
  \(F:\sfK^*(\mcC)\to \sfK^*(\mcD)\)の合成は
  \(\sfK^*(\mcI)/\mcN'\cong \sfD^*(\mcC)\)を一意的に経由する。
  このことは右導来函手\(RF\)が存在することを意味する。
  以上で\ref{1.23.2}の証明を完了する。

  \ref{1.23.3}を示す。
  \((*,\star)\)は\((*,b)\)または\((-,-)\)を表すとする。
  \(\mcI\)は各\(Y\in \mcD\)に対して\(G(-,Y)\)-injectiveであるから、
  \(I\in \Ch^*(\mcI)\)が完全な複体であれば、
  \(Y\in \Ch^{\star}(\mcD)\)と各\(i\in \Z\)に対して
  \(G(I,Y^i)\in \Ch^*(\mcE)\)も完全な複体であり、
  従って\(G(I,Y)\)は一つ目の添字に関して完全な二重複体となる。
  よって本文定理1.9.3より、
  \(Y\in \Ch^{\star}(\mcD)\)と\(I\in \Ch^*(\mcI)\)の一方が
  \(0\)と擬同型 (完全) な複体であれば、
  \(\Tot(G(I,Y))\)も\(0\)と擬同型となる。
  このことは、
  \(\Tot(G(-,-)):\sfK^*(\mcI) \times \sfK^{\star}(\mcD) \to \sfD^*(\mcE)\)
  が\(\sfD^*(\mcI) \times \sfD^{\star}(\mcD)\)を一意的に経由することを意味する。
  従って三角函手\(\sfD^*(\mcI) \times \sfD^{\star}(\mcD)\to \sfD^*(\mcE)\)を得る。
  ここで\ref{1.23.1}より
  \(\sfD^*(\mcI)\xrightarrow{\sim} \sfD^*(\mcC)\)は圏同値であるので、
  こうして得られた三角函手
  \(\sfD^*(\mcC)\times \sfD^{\star}(\mcD)\to \sfD^*(\mcE)\)は
  所望の右導来函手であることが従う。
  以上で\autoref{1.23}の解答を完了する。
\end{proof}









\begin{prob}\label{1.24}
  \
  \begin{enumerate}
    \item \label{1.24.1}
    \(F:\mcC\to \mcD\)をアーベル圏の間の左完全函手、
    \(\mcI\subset \mcC\)を\(F\)-injectiveな充満部分圏として、
    \(X\in \sfD^+(\mcC)\)を対象とする。
    各\(j\in \Z\)に対して自然な射\(H^j(RF(X)) \to F(H^j(X))\)を構成せよ。
    \item \label{1.24.2}
    \(\mcC,\mcD,\mcE\)をアーベル圏、
    \(F:\mcC\times \mcD\to \mcE\)を加法的な双函手、
    \(X\in \sfD^*(\mcC), Y\in \sfD^*(\mcD)\)を対象とする。
    ここで\(*\)は\(+\)か\(-\)であるとする。
    \begin{enumerate}
      \item \label{1.24.2.1}
      \(F\)が左完全で\(*=+\)
      (resp. \(F\)が右完全で\(*=-\))
      であるとせよ。
      各\(p,q\in \Z\)に対して自然な射\(H^{p+q}(RF(X,Y)) \to F(H^p(X),H^q(Y))\)
      (resp. \(F(H^p(X),H^q(Y)) \to H^{p+q}(LF(X,Y))\)
      を構成せよ。
      \item \label{1.24.2.2}
      \(F\)が完全であるとせよ。
      各\(n\in \Z\)に対して
      以下の同型を示せ:
      \[
      H^n(F(X,Y)) \cong \bigoplus_{p+q=n}F(H^p(X),H^q(Y)).
      \]
    \end{enumerate}
  \end{enumerate}
\end{prob}

\begin{rem*}
  \(\mcI\)のような部分圏の存在に関して本文中では全く仮定がなかったが、
  右導来函手の存在のみから証明できることなんだろうか。
  もしそうなら、\autoref{1.21}でも仮定する必要がなかったはずだけど...
\end{rem*}

\begin{proof}
  \ref{1.24.1}を示す。
  余核の普遍性によって自然な射\(\coker(F(d_X^j)) \to F(\coker(d_X^j))\)を得る。
  さらに核の普遍性によって自然な射
  \(H^j(F(X)) \to \ker(F(\coker(d_X^{j-1}))\to F(X^j))\)を得る。
  ここで\(F\)は左完全であるから、自然な同型
  \(\ker(F(\coker(d_X^{j-1}))\to F(X^j)) \cong
  F(\ker(\coker(d_X^{j-1})\to X^j)) \cong F(H^j(X))\)を得る。
  以上より、自然な射
  \(H^j(F(X)) \to F(H^j(X))\)を得る
  (自然、の意味は、複体\(X\)に対して函手的、という意味。
  余核の間の射も核の間の射も核を\(F\)の中に入れる同型射もすべて\(X\)について函手的)。
  本文の命題1.7.7または\autoref{1.23} \ref{1.23.1}より、
  モノな擬同型\(X\to I, (I\in \sfK^+(\mcI))\)が存在する。
  \(RF(I)\cong F(I)\)が成り立つので、
  自然な射
  \[R^jF(X) \cong R^jF(I) \cong H^j(F(I)) \to F(H^j(I)) \cong F(H^j(X))\]
  を得る。
  以上で\ref{1.24.1}が示された。

  \ref{1.24.2}を示す。
  \ref{1.24.2.1}を示す。
  \(*=+\)で\(F\)が左完全である場合を証明できれば、
  \(\mcC^{\op},\mcD^{\op}\)を考えることによって
  \(*=-\)で\(F\)が右完全である場合も正しいことが従う。
  よって、\ref{1.24.2.1}を示すためには、
  \(*=+\)で\(F\)が左完全であると仮定しても一般性を失わない。
  \ref{1.24.1}の証明と同様に、
  各\(Y^q\)について自然な\(\mcE\)の射
  \(H_I^p(F(X,Y^q)) \to F(H^p(X),Y^q)\)を得る。
  これらを\(q\)に関する複体と考えることで、
  \ref{1.24.1}の証明と同様に、
  各\(p,q\)について自然な\(\mcE\)の可換図式
  \[
  \begin{CD}
    H^q(H_I^p(F(X,Y))) @>>> H^q(F(H^p(X),Y)) \\
    @VVV @VVV \\
    H^p(F(X,H^q(Y))) @>>> F(H^p(X),H^q(Y))
  \end{CD}
  \]
  を得る。
  \(Z\in \Ch^{2,+}(\mcE)\)を二重複体とする。
  複体の射\(\Tot(Z) \to Z^q_{II}[-p]\)で\(n=p+q\)次のコホモロジーをとれば
  \(\mcE\)の射\(H^n(\Tot(Z)) \to H_I^p(Z)\)を得る。
  \(H_I^p(Z^{\bullet,*})\)は\(*\)に関して複体を成し、
  合成\(H^n(\Tot(Z)) \to H_I^p(Z^{\bullet,q})\to H_I^p(Z^{\bullet,q+1})\)は\(0\)-射である。
  従って、\(\mcE\)の射\(H^n(\Tot(Z)) \to H^q(H_I^p(Z))\)を得る。
  よって、もとの二重複体\(F(X,Y)\)に対しても、
  \(\mcE\)の射
  \(H^{p+q}(F(X,Y)) \to H^q(H_I^p(F(X,Y)))\)を得る。
  以上より自然な射\(H^{p+q}(F(X,Y))\to F(H^p(X),H^q(Y))\)を得る。
  ここで擬同型\(X\to I,I\in\sfK^+(\mcI)\)をとれば、
  \(\sfD^+(\mcE)\)において
  \(RF(X,Y)\cong F(I,Y)\)であるため、
  よって自然な射
  \[H^{p+q}(RF(X,Y)) \cong H^{p+q}(F(X,Y)) \to F(H^p(I),H^q(Y)) \cong F(H^p(X),H^q(Y))\]
  を得る。
  以上で\ref{1.24.2.1}の証明を完了する。

  \ref{1.24.2.2}を示す。
  \ref{1.24.2.1}の証明と同様にして、
  \(H_{II}^q(H_I^p(F(X,Y)))\cong F(H^p(X),H^q(Y))\)であることが従うので、
  \ref{1.24.2.2}を示すためには、
  \(\mcE\)の二重複体\(Z\)であって
  \(\tau_I^{\leq n}(Z)=\tau_{II}^{\leq n}(Z) = 0, (\forall n \ll 0)\)
  を満たすものに対して、
  \(H^n(\Tot(Z)) \cong \bigoplus_{p+q=n}H_{II}^q(H_I^p(Z))\)
  であることを証明することが十分である。
  しかしこれは、\(Z\)として\(\tau^{\leq n}(Z)\)をとることで
  任意の\(n\ll 0\)に対して成立し、さらに
  \autoref{1.25} \ref{1.25.1}を用いることで帰納的に任意の\(n\)に対する
  \(\tau^{\leq n}(Z)\)に対して成立するので、
  \(n\to \infty\)の極限をとることで\(Z\)に対して成立することが従う。
  以上で\ref{1.24.2.2}の証明を完了し、
  \ref{1.24.2}の証明を完了し、
  \autoref{1.24}の解答を完了する。
\end{proof}










\begin{prob}\label{1.25}
  \(\mcC\)をアーベル圏、
  \(X\)を\(\mcC\)の複体で、
  各\(n\)に対して
  \(X^{p,q}\neq 0, p+q=n\)となる\((p,q)\)は高々有限個であるとする。
  \begin{enumerate}
    \item \label{1.25.1}
    以下の三角形が\(\sfD(\mcC)\)において完全であることを示せ:
    \begin{align*}
      \Tot(\tau_{II}^{\leq n-1}(X)) \to
      \Tot(\tau_{II}^{\leq n}(X)) \to
      H_{II}^n(X)[-n] \xrightarrow{+1}, \\
      H_{II}^n(X)[-n] \to
      \Tot(\tau_{II}^{\geq n}(X)) \to
      \Tot(\tau_{II}^{\geq n+1}(X)) \xrightarrow{+1}, \\
    \end{align*}
    \item \label{1.25.2}
    \(k\in \Z\)を固定する。
    自然な射\(H^k(\Tot(\tau_{II}^{\leq n}(X))) \to H^k(\Tot(X))\)
    (resp. \(H^k(\Tot(X))\to H^k(\Tot(\tau_{II}^{\geq n}(X)))\))
    は\(n\gg 0\) (resp. \(n \ll 0\)) に対して同型であることを示せ。
    \item \label{1.25.3}
    \(k\in \Z\)を固定する。
    \(n \ll 0\)に対して\(H^k(\Tot(\tau_{II}^{\leq n}(X))) = 0\)であることと、
    \(n \gg 0\)に対して\(H^k(\Tot(\tau_{II}^{\geq n}(X))) = 0\)であることを示せ。
  \end{enumerate}
\end{prob}

\begin{proof}
  \ref{1.25.1}を示す。
  自然な射
  \(\coker(\tau_{II}^{\leq n-1}(X) \to \tau_{II}^{\leq n}(X))\to H_{II}^n(X)[-n]\)
  に本文命題1.9.3を用いることにより、
  \(\Tot(\coker(\tau_{II}^{\leq n-1}(X) \to \tau_{II}^{\leq n}(X)))\to H_{II}^n(X)[-n]\)
  が擬同型であることが従い、
  これは一つ目の三角形が完全三角であることを示している。
  二つ目の三角形が完全三角であることは
  \(\mcC^{\op}\)において一つ目の三角形が完全三角であることより従う。
  以上で\ref{1.25.1}の証明を完了する。

  \ref{1.25.2}を示す。
  \(X^{p,q}\neq 0, p+q=k,k-1,k+1\)となる\(p\)が存在するような\(q\)のうち
  最大のものを\(n_0\)とすれば、\(n > n_0\)と\(p+q=k,k-1,k+1\)を満たす任意の\(p,q\)に対して
  \((\tau_{II}^{\leq n}(X))^{p,q} = X^{p,q}\)となり、
  \ref{1.25.2}はこれからただちに従う。
  以上で\ref{1.25.2}の証明を完了する。

  \ref{1.25.3}を示す。
  \(X^{p,q}\neq 0, p+q=k,k-1,k+1\)となる\(p\)が存在するような\(q\)のうち
  最小のものを\(n_0\)とすれば、\(n < n_0\)と\(p+q=k,k-1,k+1\)を満たす任意の\(p,q\)に対して
  \((\tau_{II}^{\leq n}(X))^{p,q} = X^{p,q} = 0\)となり、
  \ref{1.25.3}はこれからただちに従う。
  以上で\ref{1.25.3}の証明を完了し、
  \autoref{1.25}の解答を完了する。
\end{proof}







\begin{prob}\label{1.26}
  \(\mcC\)をアーベル圏、
  \(X\)を\(\mcC\)の複体で、
  各\(n\)に対して
  \(X^{p,q}\neq 0, p+q=n\)となる\((p,q)\)は高々有限個であるとする。
  さらに\(q_0 < q_1\)が存在して、
  \(q\neq q_0,q_1\)に対して\(\sfD(\mcC)\)において
  \(H_{II}^q(X) \cong 0\)であると仮定する。
  このとき次の三角形が完全であることを示せ:
  \[
  H_{II}^{q_0}(X)[-q_0] \to \Tot(X) \to H_{II}^{q_1}[-q_1]
  \xrightarrow{+1}.
  \]
\end{prob}

\begin{proof}
  \autoref{1.25} \ref{1.25.1}より、
  \(n\neq q_0,q_1\)に対して
  \(\Tot(\tau_{II}^{\leq n-1}(X)) \to \Tot(\tau_{II}^{\leq n}(X))\)と
  \(\Tot(\tau_{II}^{\geq n}(X)) \to \Tot(\tau_{II}^{\geq n+1}(X))\)は
  どちらも擬同型である。
  従って、\autoref{1.25} \ref{1.25.3}より、
  任意の\(n < q_0\)に対して\(\sfD(\mcC)\)において
  \(\Tot(\tau_{II}^{\leq n}(X)) \cong 0\)であり、
  任意の\(n > q_1\)に対して\(\sfD(\mcC)\)において
  \(\Tot(\tau_{II}^{\geq n}(X)) \cong 0\)である。
  再び\autoref{1.25} \ref{1.25.1}を用いると、
  任意の\(q_0 \leq n < q_1\)に対して\(\sfD(\mcC)\)において
  \(\Tot(\tau_{II}^{\leq n}(X)) \cong H_{II}^{q_0}[-q_0]\)であり、
  任意の\(q_0 < n \leq q_1\)に対して\(\sfD(\mcC)\)において
  \(H_{II}^{q_1}[-q_1] \cong \Tot(\tau_{II}^{\geq n}(X))\)であることが従う。
  各\(n\)に対して
  \(\tau_{II}^{\leq n}(X) \to X \to \tau_{II}^{\geq n+1}(X)\xrightarrow{+1}\)
  は\(\sfD(\Ch(\mcC))\)の完全三角なので、
  \[
  \Tot(\tau_{II}^{\leq n}(X)) \to \Tot(X)
  \to \Tot(\tau_{II}^{\geq n+1}(X)) \xrightarrow{+1}
  \]
  は\(\sfD(\mcC)\)の完全三角である。
  \(n=q_0\)とすると、\(n+1\leq q_1\)であるので、従って
  \[
  H_{II}^{q_0}[-q_0] \to \Tot(X) \to H_{II}^{q_1}[-q_1] \xrightarrow{+1}
  \]
  は\(\sfD(\mcC)\)の完全三角である。
  以上で\autoref{1.26}の解答を完了する。
\end{proof}






\begin{prob}\label{1.27}
  \(\mcC\)をアーベル圏 (resp. 三角圏) とする。
  \[K(\mcC)\dfn \left(\bigoplus_{X\in \mcC} \Z\cdot [X]\right)/([X]=[X']+[X''])\]
  と定義する。
  ただしここで\([X]\)は\(\mcC\)の対象の同型類を表し、
  商はすべての完全列\(0\to X'\to X\to X''\to 0\)
  (resp. 完全三角\(X'\to X \to X'' \xrightarrow{+1}\))
  に渡ってとるものとする。
  \(K(\mcC)\)を\(\mcC\)の\textbf{Grothendieck群}と言う。
  \(\mcC\)をアーベル圏とする。
  \(i:\mcC \to \sfD^b(\mcC)\)は群の同型
  \(K(\mcC) \xrightarrow{\sim} K(\sfD^b(X))\)
  を引き起こすことを示せ。
  また、逆射が
  \(\varphi:X\mapsto \sum_j(-1)^j[H^j(X)]\)により与えられることを示せ。
\end{prob}

\begin{proof}
  \(\mcC\)の完全列\(0\to X'\to X\to X''\to 0\)を\(i\)で送れば
  \(\sfD^b(\mcC)\)の完全三角
  \(X'\to X\to X''\xrightarrow{+1}\)を得るので、
  \([X]\mapsto [i(X)]\)によって
  \(K(\mcC)\to K(\sfD^b(\mcC))\)がwell-definedに定義される。
  さらに\(X'\to X\to X''\xrightarrow{+1}\)が\(\sfD^b(\mcC)\)の完全三角であれば、
  コホモロジーをとることで長い完全列
  \[\cdots \to H^i(X')\to H^i(X) \to H^i(X'') \to \cdots\]
  を得るので、
  従って\(\sum_j(-1)^j[H^j(X)] = \sum_j(-1)^j[H^j(X')] + \sum_j(-1)^j[H^j(X'')]\)
  が従い、\(\varphi\)もwell-definedである。
  \(\varphi\circ i = \id_{K(\mcC)}\)は明らかであるから、
  \(i\circ \varphi = \id_{K(\sfD^b(\mcC))}\)であることを確認する。
  一般に、完全三角\(X'\to X\to X'' \xrightarrow{+1}\)に対して
  三角形\(X\to X''\to X'[1] \xrightarrow{+1}\)も完全であることから
  \([X''] = [X]+[X'[1]]\)かつ\([X] = [X']+[X'']\)であることが従い、
  \([X'[1]] = [X'']-[X] = -([X]-[X'']) = -[X']\)であることが従う。
  よって任意の\(X\in \sfD^b(\mcC)\)に対して\([X[1]] = -[X]\)である。
  ある\(n\)で
  \((i\circ \varphi)([\tau^{\leq n}(X)]) = [\tau^{\leq n}(X)]\)
  が成り立つと仮定する
  (これは十分小さい\(n\)に対して明らかに成り立つ)。
  三角形
  \(\tau^{\leq n}(X) \to \tau^{\leq n+1}(X) \to H^{n+1}(X)[-n-1]\xrightarrow{+1}\)
  が完全であることから、
  \begin{align*}
    [\tau^{\leq n+1}(X)] &= [i(H^{n+1}(X))[-n-1]] + [\tau^{\leq n}(X)] \\
    &= (-1)^{n+1}i([H^{n+1}(X)]) + (i\circ \varphi)([\tau^{\leq n}(X)]) \\
    &= (-1)^{n+1}i([H^{n+1}(X)]) + \sum_j(-1)^ji([H^j(\tau^{\leq n}(X))]) \\
    &= (-1)^{n+1}i([H^{n+1}(X)]) + \sum_{j\leq n}(-1)^ji([H^j(X)]) \\
    &= \sum_{j\leq n+1}(-1)^ji([H^j(X)]) \\
    &= \sum_j(-1)^ji([H^j(\tau^{\leq n+1}(X))]) \\
    &= (i\circ \varphi)([\tau^{\leq n+1}(X)])
  \end{align*}
  が従う。
  帰納法により、任意の\(n\)で
  \((i\circ \varphi)([\tau^{\leq n}(X)]) = [X]\)
  であることが従う。
  \(X\in \sfD^b(\mcC)\)であるので、
  十分大きい\(n\)を考えることで
  \((i\circ\varphi)([X]) = [X]\)が従う。
  以上で\(i\circ \varphi = \id_{K(\sfD^b(\mcC))}\)
  であることが従い、
  \autoref{1.27}の証明を完了する。
\end{proof}






\begin{prob}\label{1.28}
  \(A\)を環とする。
  以下の条件が同値であることを証明せよ:
  \begin{enumerate}
    \item \label{1.28.1}
    \(\Mod(A)\)はホモロジー次元\(\leq n\)を持つ。
    \item \label{1.28.2}
    任意の左\(A\)-加群\(M\)は長さ\(n\)以下の入射分解を持つ。
    \item \label{1.28.3}
    任意の左\(A\)-加群\(M\)は長さ\(n\)以下の射影分解を持つ。
  \end{enumerate}
  \(\Mod(A)\)のホモロジー次元か\(\Mod(A^{\op})\)のホモロジー次元のうち
  大きい方を\(A\)の\textbf{大域ホモロジー次元}
  (global homological dimension) と言い、
  \(\gld(A)\)と表す。
\end{prob}

\begin{proof}
  \autoref{1.17} \ref{1.17.1} \(\iff\) \ref{1.17.2}より
  \ref{1.28.1} \(\iff\) \ref{1.28.2}であることが従う。
  さらに\(\mcC\)のホモロジー次元は定義より\(\mcC^{\op}\)のホモロジー次元と等しいので、
  \(\Mod(A)^{\op}\)で考えると、再び
  \autoref{1.17} \ref{1.17.1} \(\iff\) \ref{1.17.2}より
  \ref{1.28.1} \(\iff\) \ref{1.28.3}であることが従う。
  以上で\autoref{1.28}の解答を完了する。
\end{proof}




\begin{prob}\label{1.29}
  \(A\)を環とする。
  \begin{enumerate}
    \item \label{1.29.1}
    任意の自由加群は射影的であることを示せ。
    \item \label{1.29.2}
    任意の射影加群はある自由加群の直和因子であることを示せ。
    \item \label{1.29.3}
    射影加群は平坦加群であることを示せ。
    \item \label{1.29.4}
    \(n\geq 0\)を自然数とする。
    以下の条件が同値であることを示せ:
    \begin{enumerate}
      \item \label{1.29.4.1}
      任意の右\(A\)-加群\(N\)と任意の左\(A\)-加群\(M\)と任意の\(j>n\)に対して
      \(\Tor_j^A(N,M) = 0\)である。
      \item \label{1.29.4.2}
      任意の左\(A\)-加群\(M\)に対して
      完全列\(0\to P^n \to \cdots \to P^0 \to M\to 0\)
      であって各\(P^i\)が平坦加群となるものが存在する。
      \item \label{1.29.4.3}
      任意の右\(A\)-加群\(M\)に対して
      完全列\(0\to P^n \to \cdots \to P^0 \to M\to 0\)
      であって各\(P^i\)が平坦加群となるものが存在する。
    \end{enumerate}
    これらの同値な条件を満たす最小の\(n\in \N\cup\{\infty\}\)を
    \(\wgld(A)\)と表し、
    \(A\)の\textbf{弱大域次元} (weak global dimension) という。
    \item \label{1.29.5}
    \(\wgld(A) \leq \gld(A)\)であることを示せ。
  \end{enumerate}
\end{prob}

\begin{proof}
  \ref{1.29.1}は函手の同型
  \(\Hom_A(A^{\oplus I},-) \cong \prod_I(-)\)
  より従う。

  \ref{1.29.2}を示す。
  \(P\)を射影加群として
  全射\(p:A^{\oplus I}\to P\)をとる。
  \(P\)が射影加群であることから
  射\(\id_P:P\to P\)がリフトして
  \(p\circ s=\id_P\)となる
  \(s:P\to A^{\oplus I}\)が存在する。
  よって\autoref{1.4} \ref{1.4.4}より
  \(P\)は\(A^{\oplus I}\)の直和因子である。
  以上で\ref{1.29.2}の証明を完了する。

  \ref{1.29.3}を示す。
  \(P\)を射影加群として、
  \(P\)が直和因子となるように射\(i:P\to A^{\oplus I}\)をとる。
  \(p:A^{\oplus I}\to P\)を\(i\)の左逆射、
  つまり\(p\circ i = \id_P\)となる射とする。
  \(f:M\to N\)を\(A\)-加群の単射とする。
  \ref{1.29.3}を示すためには、
  \(f\otimes_A \id_P\)が単射であることを示すことが十分である。
  可換図式
  \[
  \begin{CD}
    M\otimes_A P @> \id \otimes i >>
    M\otimes_A A^{\oplus I} @> \id \otimes p >> M\otimes_A P \\
    @V f\otimes \id VV @V f\otimes \id VV @VV f\otimes \id V \\
    N\otimes_A P @> \id \otimes i >>
    N\otimes_A A^{\oplus I} @> \id \otimes p >> N\otimes_A P
  \end{CD}
  \]
  において、上と下の合成は\(\id\)であり、
  \(M\otimes_A A^{\oplus I}\cong M^{\oplus I}\)より真ん中は単射である。
  従って両端も単射であることが従う。
  以上で\ref{1.29.3}の証明を完了する。

  \ref{1.29.4}を示す。
  \ref{1.29.4} \ref{1.29.4.1} \(\iff\) \ref{1.29.4} \ref{1.29.4.2}
  を示すことができれば、
  \(A^{\op}\)に対して
  \ref{1.29.4} \ref{1.29.4.1} \(\iff\) \ref{1.29.4} \ref{1.29.4.2}
  を適用することで
  \ref{1.29.4} \ref{1.29.4.1} \(\iff\) \ref{1.29.4} \ref{1.29.4.3}
  が従う。
  残っているのは
  \ref{1.29.4} \ref{1.29.4.1} \(\iff\) \ref{1.29.4} \ref{1.29.4.2}
  を示すことである。

  \ref{1.29.4} \ref{1.29.4.1}が成り立つと仮定する。
  自由分解\(\cdots \to P^n \xrightarrow{d_P^n} \cdots \to P^0 \xrightarrow{d_P^0} M\to 0\)
  を一つとる。
  任意の\(N\)と\(j>n\)に対して\(\Tor_j^A(N,M) = 0\)が成り立つので、
  とくに任意の\(N\)と\(j>n-1\)に対して\(\Tor_j^A(N,\ker(d_P^0))=0\)が成り立つ。
  \(\ker(d_P^0)\cong \im(d_P^1)\)に注意して繰り返すと、
  繰り返して、任意の\(N\)と任意の\(j>0\)に対して
  \(\Tor_j^A(N,\ker(d_P^{n-1})) = 0\)が成り立つ。
  このことは\(\ker(d_P^{n-1})\)が平坦であることを意味していて、
  完全列
  \(0\to \ker(d_P^{n-1}) \to P^{n-1} \to \cdots \to P^0 \to M \to 0\)
  は\(M\)の長さ\(n\)以下の平坦分解である。
  以上で
  \ref{1.29.4} \ref{1.29.4.1} \(\Rightarrow\) \ref{1.29.4} \ref{1.29.4.2}
  が示された。

  \ref{1.29.4} \ref{1.29.4.2}が成り立つと仮定する。
  任意に左\(A\)-加群\(M\)と右\(A\)-加群\(N\)と\(j>n\)をとる。
  仮定より\(M\)の平坦分解
  \(0 \to P^n \xrightarrow{d_P^n} \cdots \to P^0 \xrightarrow{d_P^0} M\to 0\)
  が存在する。
  \(P^n,P^{n-1}\)は平坦であるから、
  完全列\(0\to P^n \to P^{n-1} \to \im(d_P^{n-1})\to 0\)
  に\(N\otimes_A(-)\)を施すことで、
  任意の\(j>1\)に対して\(\Tor_j^A(N,\im(d_P^{n-1}))=0\)であることが従う。
  完全列\(0\to \im(d_P^{n-1})\to P^{n-2}\to \im(d_P^{n-2}) \to 0\)
  に\(N\otimes_A(-)\)を施すことで、
  任意の\(j>2\)に対して\(\Tor_j^A(N,\im(d_P^{n-2}))=0\)であることが従う。
  帰納的に、
  任意の\(j>k\)に対して\(\Tor_j^A(N,\im(d_P^{n-k}))=0\)であることが従う。
  \(n=k\)とすれば所望の結論を得る。
  以上で
  \ref{1.29.4} \ref{1.29.4.2} \(\Rightarrow\) \ref{1.29.4} \ref{1.29.4.1}
  が示され、\ref{1.29.4}の証明を完了する。

  \ref{1.29.5}は
  \autoref{1.28} \ref{1.28.1} \(\iff\) \ref{1.28.3}と
  \ref{1.29.3}より従う。
  以上で\autoref{1.29}の解答を完了する。
\end{proof}




\begin{prob}\label{1.30}
  \(A\)を可換環とする。
  \(X\in \sfD^b(\Mod(A))\)が\textbf{perfect}であるとは、
  有限生成射影加群からなる有界な複体と擬同型であることを言う。
  \begin{enumerate}
    \item \label{1.30.1}
    \(X\to Y\to Z\xrightarrow{+1}\)が\(\sfD^b(\Mod(A))\)の完全三角であるとする。
    \(X,Y\)がperfectであるとき、\(Z\)もperfectであることを示せ。
    \item \label{1.30.2}
    \(P\)がperfectであるとき、\(P\)の直和因子もperfectであることを示せ。
    \item \label{1.30.3}
    \(X\in \sfD^b(\Mod(A))\)がperfectであるとする。
    \(X^*\dfn R\Hom(X,A)\)とおくと、\(X^*\)もperfectであり、
    自然な射\(X\to X^{**}\)は同型射であることを示せ。
    \item \label{1.30.4}
    \(A\)をネーター環で\(\gld(A) < \infty\)と仮定する。
    \(\Mod^f(A)\)を有限生成\(A\)-加群のなすアーベル圏とする。
    \(\sfD^b(\Mod^f(A))\)の任意の対象はperfectであることを示せ。
    \item \label{1.30.5}
    \(A\)をネーター環で\(\gld(A) < \infty\)と仮定する。
    すべてのコホモロジーが\(\Mod^f(A)\)に属する複体からなる
    充満部分圏を\(\sfD^b_f(\Mod(A))\subset \sfD^b(\Mod(A))\)で表す。
    このとき自然な射\(\sfD^b(\Mod^f(A)) \to \sfD^b_f(\Mod(A))\)
    は圏同値であることを示せ。
  \end{enumerate}
\end{prob}

\begin{proof}
  \ref{1.30.1}を示す。
  \(X,Y\)はperfectなので、
  \ref{1.30.1}を示すためには、
  \(X,Y\)はどちらも有限生成射影加群からなる有界複体であると仮定しても一般性を失わない。
  このとき\(Z\)は\(i\)次が\(Y^i\oplus X^{i+1}\)である複体と擬同型であり、
  \(Y^i,X^{i+1}\)はどちらも有限生成射影加群なので\(Y^i\oplus X^{i+1}\)も有限生成射影加群である。
  従って、\(Z\)もperfectである。
  以上で\ref{1.30.1}の証明を完了する。

  \ref{1.30.2}より先に\ref{1.30.3}を示す。
  \(X\)はperfectであるから、\ref{1.30.3}を示すためには、
  各\(i\)に対して\(X^i\)は有限生成射影加群であり、
  \(X^i=0, (|i| \gg 0)\)であると仮定しても一般性を失わない。
  このとき本文命題1.10.4より、\(\sfD^b(\Mod(A))\)において
  \(R\Hom(X,A) \cong \Hom(X,A)\)である。
  各\(i\)に対して\(\Hom(X,A)^i = \Hom(X^{-i},A)\)は射影加群であり、
  \(|i|\gg 0\)となる\(i\)に対して\(\Hom(X,A)^i = \Hom(X^{-i},A) = 0\)であるから、
  \(\Hom(X,A)\)はperfectであり、従って\(R\Hom(X,A)\)もperfectである。
  有限生成射影加群\(X\)に対して
  自然な射\(X\to \Hom(\Hom(X,A),A)\)が同型射であることから
  自然な射\(X\to \Hom(\Hom(X,A),A)\)は複体の同型射であり、
  従って\(\sfD^b(\Mod(A))\)においても同型射である。
  \(\Hom(X,A)\cong R\Hom(X,A)\)はperfectであるので、
  \(X^{**}\cong \Hom(X^*,A)\cong \Hom(\Hom(X,A),A)\)であり、
  従って自然な射\(X\to X^{**}\)は同型射である。
  以上で\ref{1.30.3}の証明を完了する。

  \ref{1.30.4}を示す。
  任意の有限生成加群は有限生成自由加群のある商と同型であるから、
  アーベル圏\(\Mod^f(A)^{\op}\)は
  \(\mcI\)を有限生成射影加群からなる充満部分圏とするときに本文の条件(1.7.5)を満たす。
  また\(\gld(A) < \infty\)であるから、\autoref{1.28}より、
  アーベル圏\(\Mod^f(A)^{\op}\)は同じ\(\mcI\)に対して本文の条件(1.7.6)を満たす。
  従って本文の系1.7.8より\ref{1.30.4}が従う。
  以上で\ref{1.30.4}の証明を完了する。

  \ref{1.30.5}は\(\Mod(A)^{\op}\)とその
  thick full abelian subcategory
  \(\Mod^f(A)^{\op}\subset \Mod(A)^{\op}\)に対して
  本文命題1.7.11を適用することにより直ちに従う
  (\(\Mod^f(A)^{\op}\)が本文命題1.7.11の条件を満たすことは容易に確認できる)。
\end{proof}




\begin{prob}\label{1.31}
  \(M\in \sfD^b(\Ab)\)とする。
  \begin{enumerate}
    \item \label{1.31.1}
    \(M^* = R\Hom(M,\Z) = 0\)であるとき、\(M=0\)であることを示せ。
    \item \label{1.31.2}
    \(M^*\in \sfD^b_f(\Ab)\)であるとき、\(M\in \sfD^b_f(\Ab)\)であることを示せ。
  \end{enumerate}
\end{prob}

\begin{rem*}
  \ref{1.31.2}は\(M^*\in \sfD^b(\Mod^f(\Z))\)という仮定のもとで
  \(M\in \sfD^b(\Mod^f(\Z))\)を示す問題であったが、
  \(\sfD^b(\Mod^f(\Z))\)は\(\sfD^b(\Ab)\)の部分圏として
  同型で閉じていないので、これはかなり微妙な問題設定であり
  (成り立たないかもしれない)、
  上記の設定がより適切であると思われる。
\end{rem*}

\begin{proof}
  \ref{1.31.1}を示す。
  まず\(M\in \Ab\)である場合に\ref{1.31.1}を証明する。
  \(R\Hom(M\Z)=0\)は\(\Hom(M,\Z) = \Ext^1(M,\Z) = 0\)を意味する。
  このときに\(M=0\)を示す。
  単射\(\Z/n\Z \to M\)を任意にとると
  \(0 = \Ext^1_{\Z}(M,\Z) \to \Ext^1_{\Z}(\Z/n\Z,\Z)\)は全射となるので
  \(\Z/n\Z \cong \Ext^1_{\Z}(\Z/n\Z,\Z) = 0\)となって\(n=1\)となる。
  従って\(M\)はねじれなし群である。
  \(n\neq 0,1,-1\)とすれば\(M/nM\)はねじれ群であるが、
  完全列\(0\to M\xrightarrow{n} M \to M/nM\to 0\)に
  函手\(R\Hom(-,\Z)\)を施すことによって\(R\Hom(M/nM,\Z) = 0\)が従い、
  よって\(M/nM\)はねじれなし群でもある。
  これは\(M/nM=0\)を意味し、従って\(M\)は可除である。
  \(M\)はねじれがないので\(M\to M\otimes \Q\)は単射であり、
  \(M\)は可除なのでこれは全射でもある。
  従って\(M\cong M\otimes \Q\)である。
  もし\(M\neq 0\)なら、\(M\)は\(\Q\)を直和因子として持つ。
  一方、完全列\(0\to \Z\to \Q\to \Q/\Z\to 0\)に
  \(\Hom(\Q/\Z,-)\)を適用することにより、
  \(\hat{\Z} \cong \End(\Q/\Z) \xrightarrow{\sim}\Ext^1(\Q/\Z,\Z)\)
  を得るので、
  同じ完全列に\(\Hom(-,\Z)\)を適用することで
  \(\Ext^1(\Q,\Z) \cong \coker(\Hom(\Z,\Z) \to
  \Ext^1(\Q/\Z,\Z)) \cong \hat{\Z}/\Z \neq 0\)
  が従い、これは\(\Ext^1(M,\Z) = 0\)に反する。
  以上で\(M\in \Ab\)の場合に示された。

  一般の\(M\in \sfD^b(\Ab)\)に対して\ref{1.31.1}を示す。
  \(H^n(M) \neq 0\)となる最大の\(n\)をとる。
  このとき
  \(\tau^{\leq n-1}(M) \to M \to H^n(M)[-n]\xrightarrow{+1}\)
  は完全三角である。
  \(R\Hom(-,\Z)\)を適用してコホモロジーをとることで、
  アーベル群の完全列
  \[
  \begin{CD}
    0 @>>> \Hom(H^n(M),\Z)
    @>>> H^n(R\Hom(M,\Z)) @>>> H^n(R\Hom(\tau^{\leq n-1}(M),\Z)) \\
    @>>> \Ext^1(H^n(M),\Z) @>>> H^{n+1}(R\Hom(M,\Z)) @>>> \cdots
  \end{CD}
  \]
  を得る。
  ここで\(R\Hom(M,\Z)=0\)であるから、
  \(H^n(R\Hom(M,\Z)) = 0, H^{n+1}(R\Hom(M,\Z)) = 0\)である。
  さらに\autoref{1.21}を\(\tau^{\leq n-1}(M)\)と\(R\Hom(-,\Z)\)に適用することにより、
  \(H^n(R\Hom(\tau^{\leq n-1}(M),\Z)) = 0\)である。
  従って\(\Hom(H^n(M),\Z)=\Ext^1(H^n(M),\Z)=0\)が従う。
  すでに示している\(M\in \Ab\)の場合により\(H^n(M)=0\)が従い、
  これは\(H^n(M)\neq 0\)に矛盾する。
  以上で\ref{1.31.1}の証明を完了する。

  \ref{1.31.2}を示す。
  \ref{1.31.1}の証明と同様に、
  \(H^n(M) \neq 0\)となる最大の\(n\)をとり、
  アーベル群の完全列
  \[
  \begin{CD}
    0 @>>> \Hom(H^n(M),\Z) @>>> H^n(R\Hom(M,\Z))
    @>>> H^n(R\Hom(\tau^{\leq n-1}(M),\Z)) \\
    @>>> \Ext^1(H^n(M),\Z) @>>> H^{n+1}(R\Hom(M,\Z)) @>>>
    H^{n+1}(R\Hom(\tau^{\leq n-1}(M),\Z)) \\
    @>>> 0 @. @.
  \end{CD}
  \]
  について考える
  (\(\Ext^2(H^n(M),\Z)=0\)であることに注意)。
  \autoref{1.21}を\(\tau^{\leq n-1}(M)\)と\(R\Hom(-,\Z)\)に適用することにより、
  \(H^n(R\Hom(\tau^{\leq n-1}(M),\Z)) = 0\)である。
  また、\(R\Hom(M,\Z)\in \sfD_f^b(\Mod(\Z))\)であるので、
  \(H^n(R\Hom(M,\Z)), H^{n+1}(R\Hom(M,\Z))\in \Mod^f(\Z)\)である。
  従って、
  \[\Hom(H^n(M),\Z), \ \Ext^1(H^n(M),\Z), \
  H^{n+1}(R\Hom(\tau^{\leq n-1}(M),\Z)) \ \in \Mod^f(\Z)\]
  である。
  さらに、\(n\)より大きい部分のコホモロジーを見れば、
  \[H^m(R\Hom(\tau^{\leq n-1}(M),\Z))\cong H^m(R\Hom(M,\Z)), \ (\forall m > n+1)\]
  であるので、\(R\Hom(\tau^{\leq n-1}(M),\Z)\in \Mod^f(\Z)\)が従う。
  以上より、帰納的に、
  \ref{1.31.2}を示すためには、
  アーベル群\(M\)が\(\Hom(M,\Z),\Ext^1(M,\Z)\in \Mod^f(\Z)\)を満たすとき
  \(M\in \Mod^f(\Z)\)であることを示すことが十分である。

  \(M\)をアーベル群であって
  \(\Hom(M,\Z)\)と\(\Ext^1(M,\Z)\)がどちらも有限生成であると仮定する。
  ねじれ部分を\(T(M)\subset M\)として、\(F(M)\dfn M/T(M)\)とおく。
  自然な射\(M\to \Hom(\Hom(M,\Z),\Z)\)の像は\(F(M)\)と同型であり、
  \(\Hom(M,\Z)\)が有限生成であることから
  \(\Hom(\Hom(M,\Z),\Z)\)も有限生成であるため、
  \(F(M)\)は有限生成アーベル群である。
  従って、\ref{1.31.2}を示すためには、
  \(T(M)\)が有限生成であることを示すことが十分である。
  完全列
  \(0\to T(M)\to M\to F(M)\to 0\)に
  \(\Hom(-,\Z)\)を適用することにより、
  全射\(\Ext^1(M,\Z)\to \Ext^1(T(M),\Z)\)を得る。
  従って、\(\Ext^1(T(M),\Z)\)は有限生成アーベル群である。
  完全列\(0\to \Z\to \Q\to \Q/\Z\to 0\)に
  \(\Hom(T(M),-)\)を適用することにより、自然な同型
  \(\Hom(T(M),\Q/\Z)\xrightarrow{\sim}\Ext^1(T(M),\Z)\)を得る。
  \(T(M)\)に離散位相を入れて\(\Q/\Z\)に\(\R/\Z\)の双対位相を入れることにより、
  \(\Hom(T(M),\Q/\Z) = \Hom_{\text{cont.}}(T(M),\Q/\Z)\)を連続準同型のなす位相群とみなすと、
  \(\Hom_{\text{cont.}}(T(M),\Q/\Z)\)は副有限アーベル群である。
  とくにコンパクトハウスドルフである。
  一方、\(\Ext^1(T(M),\Z)\cong \Hom(T(M),\Q/\Z)\)はアーベル群として有限生成であるので、
  \(\Hom_{\text{cont.}}(T(M),\Q/\Z)\)は有限アーベル群であることが従う。
  Pontryagin双対により、
  \(T(M)\cong \Hom_{\text{cont.}}(\Hom_{\text{cont.}}(T(M),\Q/\Z),\Q/\Z)
  = \Hom(\Hom(T(M),\Q/\Z),\Q/\Z)\)
  であり、これは有限生成ねじれアーベル群である。
  以上で\ref{1.31.2}の証明を完了し
  \autoref{1.31}の解答を完了する。
\end{proof}






\begin{prob}\label{1.32}
  \(k\)を体、\(X\in \sfD^b(\Mod(k))\)とする。
  \(X^*\dfn R\Hom(X,k)\)とおく
  (微分が本文Remark 1.8.11で与えられることを思い出そう)。
  \begin{enumerate}
    \item \label{1.32.1}
    \(X\in \sfD^b_f(\Mod(k))\)と仮定する。
    以下の自然な同型が存在することを示せ:
    \[
    X\xrightarrow{\sim}X^{**} \ \ , \ \
    X^*\otimes X \xrightarrow{\sim} R\Hom(X,X).
    \]
    さらに、\((X^n)^*\otimes X^n \to k\)の直和として
    射\(X^*\otimes X\to k\)を構成せよ。
    \item \label{1.32.2}
    \(X\in \sfD^b_f(\Mod(k))\)と\(v\in \Hom(X,X)\)に対して
    \[
    \tr(v) \dfn \sum_j (-1)^j\tr(H^j(v))
    \]
    と定義する。
    ここで\(\tr(H^j(v))\)は自己準同型\(H^j(v):H^j(X)\to H^j(X)\)の
    トレースである。
    \(Y\in \sfK^b(\Mod^f(k))\)として、
    \(v\in \Hom_{\sfK^b(\Mod^f(k))}(Y,Y)\)とする。
    以下の等式を示せ:
    \[
    \tr(v) = \sum_j (-1)^j \tr(v^j).
    \]
    \item \label{1.32.3}
    \(\sfD^b_f(\Mod(k))\)の完全三角の間の自己射
    \[
    \begin{CD}
      X' @>>> X @>>> X'' @>>> \\
      @V{v'}VV @V{v}VV @V{v''}VV @. \\
      X' @>>> X @>>> X'' @>>> \\
    \end{CD}
    \]
    に対して、\(\tr(v) = \tr(v') + \tr(v'')\)が成り立つことを示せ。
    \item \label{1.32.4}
    \ref{1.32.2}の状況設定において、
    \(\tr(v)\)が\(v\)の
    \[
    H^0(R\Hom(X,X)) \cong H^0(X^*\otimes X) \to k
    \]
    による像と一致することを示せ。
    \(X\in \sfD^b_f(\Mod(k))\)に対して
    \[
    \chi(X) \dfn \sum_j (-1)^j\dim H^j(X)
    \]
    とおく。
    \(k\)において\(\chi(X) = \tr(\id_X)\)が成り立つ。
  \end{enumerate}
\end{prob}

\begin{proof}
  \ref{1.32.1}を示す。
  \(k\)は体なので、\autoref{1.30} \ref{1.30.5}より\(X\)はperfectであり、
  従って、一つ目の同型は\autoref{1.30} \ref{1.30.3}より従う。
  自然な同型射
  \((X^{-m})^*\otimes X^n \xrightarrow{\sim} \Hom(X^{-m},X^n)\)
  を並べることによって、二重複体の同型射
  \(X^*\otimes X\xrightarrow{\sim}\Hom(X,X)\)
  を得る。
  \(\Tot\)を取ることによって複体の同型射
  \(X^*\otimes X \xrightarrow{\sim} R\Hom(X,X)\)を得る。
  これが二つ目の同型である。
  最後の自然な射を構成する。
  \(0\)次の部分は各\(n\in \Z\)に対して自然な射
  \((X^*)^{-n}\otimes X^n = (X^n)^*\otimes X^n \to k\)
  を直和することにより得られる射\((X^*\otimes X)^0\to k\)で、
  他の次数は\(0\)射とすることにより、
  複体の射\(X^*\otimes X \to k\)がwell-definedに定義されることを示す。
  そのためには、これらの射が複体\(X^*\otimes X\)と\(k\)
  (これは\(0\)次部分のみに\(k\)があり他で\(0\)となる複体を表す)
  の微分と可換することを示すことが十分である。
  射
  \begin{align*}
    &(X^*)^{-n}\otimes X^{n-1} \to (X^*)^{-n}\otimes X^n \to k \\
    &(X^*)^{-n}\otimes X^{n-1} \to (X^*)^{-n+1}\otimes X^{n-1} \to k
  \end{align*}
  について考える。
  ただしここで、
  最後の\(k\)への射は\((f,x)\mapsto f(x)\)により与えられる自然な射
  (\((X^*)^{-n}=(X^n)^*\)に注意せよ) であり、
  はじめの射は本文式(1.9.3)により定義される、
  \(\Tot\)の微分を与える射である。
  上の二つの射の合成は\((f,x)\in (X^*)^{-n}\otimes X^{n-1}\)が
  \[(f,x)\mapsto (f,(-1)^{-n}d^{n-1}(x)) \mapsto (-1)^nf(d^{n-1}(x))\]
  と写る射である。
  微分\((X^*)^{-n}\to (X^*)^{-n+1}\)は
  \((-1)^{n-1}d^{n-1}:X^{n-1}\to X^n\)を合成することにより与えられているので
  (cf. 本文Remark 1.8.11)、
  下の二つの射の合成は\((f,x)\in (X^*)^{-n}\otimes X^{n-1}\)が
  \[(f,x)\mapsto ((-1)^{n-1}(f\circ d^{n-1}),x) \mapsto (-1)^{n-1}f(d^{n-1}(x))\]
  と写る射である。
  \((-1)^nf(d^{n-1}(x))+(-1)^{n-1}f(d^{n-1}(x)) = 0\)
  であるため、従って、\(X^*\otimes X\to k\)は複体の射である。
  以上で\ref{1.32.1}の証明を完了する。

  \ref{1.32.2}を示す。
  有限次元\(k\)-線形空間の完全列の自己準同型
  \[
  \begin{CD}
    0 @>>> V_1 @>>> V_2 @>>> V_3 @>>> 0 \\
    @. @V f_1 VV @V f_2 VV @V f_3 VV @. \\
    0 @>>> V_1 @>>> V_2 @>>> V_3 @>>> 0
  \end{CD}
  \]
  があると、
  \(f_1,f_3\)の上三角化を与える\(V_1,V_3\)の基底により
  \(f_2\)の上三角化が与えられる。
  従って\(\tr(f_2) = \tr(f_1) + \tr(f_3)\)が成り立つ。
  完全列の射
  \[
  \begin{CD}
    0 @>>> H^n(Y) @>>> \coker(d_Y^{n-1}) @>>> \ker(d_Y^{n+1}) @>>> H^{n+1}(Y) @>>> 0 \\
    @. @V{H^n(v)}VV @V{C^{n-1}(v)}VV @VV{Z^{n+1}(v)}V @VV{H^{n+1}(v)}V @. \\
    0 @>>> H^n(Y) @>>> \coker(d_Y^{n-1}) @>>> \ker(d_Y^{n+1}) @>>> H^{n+1}(Y) @>>> 0
  \end{CD}
  \]
  にこれを適用することで、
  \(\tr(C^{n-1}(v)) - \tr(H^n(v)) = \tr(Z^{n+1}(v)) - \tr(H^{n+1}(v))\)を得る。
  ただし\(C^n(v)\)は余核の間に引き起こされる自然な射である。
  完全列の間の射
  \[
  \begin{CD}
    0 @>>> Z_Y^n @>>> Y^n @>>> \im(d_Y^n) @>>> 0 \\
    @. @V{Z^n(v)}VV @V{v^n}VV @VV{B^n(v)}V @. \\
    0 @>>> Z_Y^n @>>> Y^n @>>> \im(d_Y^n) @>>> 0
  \end{CD}
  \]
  に適用することにより、
  \(\tr(B^n(v)) + \tr(Z^n(v)) = \tr(v^n)\)を得る。
  完全列の間の射
  \[
  \begin{CD}
    0 @>>> H^n(Y) @>>> \coker(d_Y^{n-1}) @>>> \im(d_Y^n) @>>> 0 \\
    @. @V{H^n(v)}VV @V{C^{n-1}(v)}VV @VV{B^n(v)}V @. \\
    0 @>>> H^n(Y) @>>> \coker(d_Y^{n-1}) @>>> \im(d_Y^n) @>>> 0
  \end{CD}
  \]
  に適用することにより、
  \(\tr(B^n(v)) = \tr(C^{n-1}(v)) - \tr(H^n(v))\)を得る。
  ただし\(C^n(v)\)は余核の間に引き起こされる自然な射である。
  従って、
  \begin{align*}
    \sum_j (-1)^j\tr(v^j)
    &= \sum_j (-1)^j\left( \tr(B^j(v)) + \tr(Z^j(v)) \right) \\
    &= \sum_j (-1)^j\left( \tr(C^{j-1}(v)) - \tr(H^j(v)) + \tr(Z^j(v)) \right) \\
    &= \sum_j (-1)^j\left( \tr(C^{j-1}(v)) + \tr(C^{j-2}(v)) - \tr(H^{j-1}(v)) \right) \\
    &= \sum_j (-1)^{j+1}\tr(H^{j-1}(v)) \\
    &= \sum_j (-1)^j\tr(H^j(v)) = \tr(v)
  \end{align*}
  が成り立つ。
  以上で\ref{1.32.2}の証明を完了する。

  \ref{1.32.3}はコホモロジーをとることによって得られる長完全列を
  短完全列に分解して\ref{1.32.2}の証明の最初で示した等式を用いると証明できる。

  \ref{1.32.4}を示す。
  \(f:X\to Y\)の定める\(H^0(R\Hom(X,Y))\)の元は、
  各\(i\)について\(f^i:X^i\to Y^i\)の定める\(\Hom(X^i,Y^i)\)の元を
  \(((-1)^if^i)\in \bigoplus_i \Hom(X^i,Y^i)\)と並べた元である。
  実際、\(R\Hom(X,Y)\)の微分は、
  第一変数に関しては\((-1)^id_X^i\)を合成することによって与えられるので、
  \(d_Y^i\circ ((-1)^if^i = f^{i+1}\circ ((-1)^id_X^i)\)が成り立ち、
  \(((-1)^if^i)\)は\(H^0(R\Hom(X,Y))\)の元を定める。
  \(v:X\to X\)を複体の自己射とする。
  各\(j\)に対する\(v^j:X^j \to X^j\)のトレースは
  \(\Hom(X^j,X^j) \cong (X^j)^*\otimes X^j \to k\)
  による\(v^j\)の像が定める\(k\)の元と一致する。
  従って、\(v\)の定める\(H^0(R\Hom(X,X))\)の元、
  すなわち\(((-1)^iv^i)\in \bigoplus_i \Hom(X,X)\)の
  自然な射\(H^0(R\Hom(X,X))\to k\)による像は
  \(\sum_j(-1)^j\tr(v^j)\)に他ならない。
  よって\ref{1.32.4}の最初の主張が従う。
  また、\(\dim(V) = \tr(\id_V)\)であるので、
  \ref{1.32.2}より
  \(\chi(X) = \sum_j(-1)^j\dim(H^j(X))\)が従う。
  以上で\ref{1.32.4}の証明を完了し、
  \autoref{1.32}の解答を完了する。
\end{proof}






\begin{prob}\label{1.33}
  \(k\)を体、\(V\)を\(k\)-線形空間とする。
  自己準同型\(u:V\to V\)が
  \textbf{trace class}
  であるとは、
  ある\(n\)に対して\(\dim(u^n(V)) < \infty\)
  が成り立つことと定義する。
  \(u:V\to V\)が trace class であるとき、
  \(\tr(u) \dfn \tr(u|_{u^n(V)})\)と定義する。
  \begin{enumerate}
    \item \label{1.33.1}
    \(\tr(u)\)の定義は\(n\)に依存しないことを示せ。
    \item \label{1.33.2}
    \(V\xrightarrow{u} W \xrightarrow{v} V\)を\(k\)-線形空間の射の列とする。
    \(u\circ v\)が trace class であることと
    \(v\circ u\)が trace class であることは同値であることを示せ。
    さらにこのとき\(\tr(u\circ v) = \tr(v\circ u)\)が成り立つことを示せ。
    \item \label{1.33.3}
    \(k\)-線形空間の完全列の自己準同型
    \[
    \begin{CD}
      0 @>>> V_1 @>>> V_2 @>>> V_3 @>>> 0 \\
      @. @V{v_1}VV @V{v_2}VV @V{v_3}VV @. \\
      0 @>>> V_1 @>>> V_2 @>>> V_3 @>>> 0
    \end{CD}
    \]
    について、\(v_2\)が trace class であることと
    \(v_1,v_3\)がどちらも trace class であることは同値であることを示せ。
    さらにこのとき、\(\tr(v_2) = \tr(v_1) + \tr(v_3)\)が成り立つことを示せ。
  \end{enumerate}
\end{prob}


\begin{proof}
  \ref{1.33.1}を示す。
  まず\(x\mapsto [u:V\to V]\)により\(V\)を\(k[x]\)-加群と考える。
  十分大きい\(n\)に対して\(\dim(\im(u^n)) < \infty\)であるので、
  \(n \gg 0\)で\(\im(u^n) = \im(u^{n+1})\)となる。
  従って、自然な射\(\im(u^n) \subset V\to V\otimes_k k[x,1/x]\)は
  \(n \gg 0\)で同型射であり、
  とくに\(V\otimes_k k[x,1/x]\)は
  \(k\)-線形空間として有限次元である。
  \(u\)のトレースは\(k\)-線形空間\(V\otimes_k k[x,1/x]\)上への
  \(x\)の作用にしか依存しないため、\(n\)の取り方によらずにwell-definedである。
  以上で\ref{1.33.1}の証明を完了する。

  \ref{1.33.2}を示す。
  \(v\circ (u\circ v)^n\circ u = (v\circ u)^{n+1}\)
  なので\(u\circ v\)が trace class であることと
  \(v\circ u\)が trace class であることは同値である。
  \(v\circ u:V\to V\)と
  \(u\circ v:W\to W\)によって\(V,W\)をそれぞれ\(k[x]\)-加群と考えたとき、
  \(u:V\to W\)と\(v:W\to V\)は\(k[x]\)-加群の射である。
  さらに、\(u\circ v\)か\(v\circ u\)の一方が trace class であれば、
  十分大きい\(n\)に対して
  \(v\circ u:(v\circ u)^n (V) \to (v\circ u)^n (V)\)と
  \(u\circ v:(u\circ v)^n (W) \to (u\circ v)^n (W)\)はいずれも全単射であり、
  とくに\(k[x]\)-加群の同型射である。
  これは\(v\circ u\)と\(u\circ v\)の固有値の和が等しいことを意味する。
  以上で\ref{1.33.2}の証明を完了する。

  \ref{1.33.3}を示す。
  \(v_1,v_2,v_3\)によって\(V_1,V_2,V_3\)を\(k[x]\)-加群とみなす。
  \(v_1,v_2,v_3\)が\(k\)-線形空間の完全列の射を成すことから、
  \[
  \begin{CD}
    0 @>>> V_1 @>>> V_2 @>>> V_3 @>>> 0
  \end{CD}
  \]
  は\(k[x]\)-加群の完全列である。
  \(k[x,1/x]\)をテンソルすると、
  \(k[x,1/x]\)は\(k[x]\)上平坦であるから、
  \(k[x,1/x]\)-加群の完全列
  \[
  \begin{CD}
    0 @>>> V_1\otimes_{k[x]}k[x,1/x] @>>> V_2\otimes_{k[x]}k[x,1/x]
    @>>> V_3\otimes_{k[x]}k[x,1/x] @>>> 0
  \end{CD}
  \]
  を得る。
  \(v_i\)が trace class であることは、
  \(V_i\otimes_{k[x]}k[x,1/x]\)が長さ有限であることと同値であるので、
  以上より\(v_2\)が trace class であることと
  \(v_1,v_3\)がどちらも trace class であることが同値であることが従う。
  \(v_i\)のトレースは\(V_i\otimes_{k[x]}k[x,1/x]\)への\(v_i\)の作用
  (つまり\(x\)の作用)
  のトレースであるから、
  \(V_i\otimes_{k[x]}k[x,1/x]\)たちの成す短完全列を考えることによって、
  \(\tr(v_2) = \tr(v_1) + \tr(v_3)\)であることが従う
  (cf. \autoref{1.32} \ref{1.32.2}の証明の一番最初の部分など)。
  以上で\ref{1.33.3}の証明を完了し、
  \autoref{1.33}の解答を完了する。
\end{proof}




\begin{prob}\label{1.34}
  \(k\)を体、\(X\in \sfD^b_f(\Mod(k))\)とする。
  \[
  b_i(X) \dfn \dim(H^i(X)), \ \
  b_i^*(X) \dfn (-1)^i \sum_{j\leq i}(-1)^jb_j(X)
  \]
  とおく。
  \(Y\to X\to Z\xrightarrow{+1}\)を\(\sfD^b_f(\Mod(k))\)の完全三角とする。
  以下の式を示せ (\(\chi(X)\)の定義については\autoref{1.32} \ref{1.32.4}を参照):
  \begin{align*}
    \chi(X) &= \chi(Y) + \chi(Z), \\
    b_i^*(X) &\leq b_i^*(Y) + b_i^*(Z).
  \end{align*}
\end{prob}

\begin{proof}
  一つ目の等式は
  \autoref{1.32} \ref{1.32.3} \ref{1.32.4}
  より直ちに従う。
  二つ目の不等式を示す。
  コホモロジーをとると、長完全列
  \[
  \begin{CD}
    @> \delta^{i-1} >> H^{i-1}(Y) @>>> H^{i-1}(X) @>>> H^{i-1}(Z) \\
    @> \delta^i >> H^i(Y) @>>> H^i(X) @>>> H^i(Z) \\
    @> \delta^{i+1} >> \cdots @. @.
  \end{CD}
  \]
  を得る。
  従って、とくに
  \begin{align*}
    0 &\leq \dim(\im(\delta^{i+1})) \\
    &= b_i(Z) - b_i(X) + b_i(Y) - b_{i-1}(Z) + \cdots \\
    &= \sum_{j\leq i}(-1)^{i-j}b_j(Z) - \sum_{j\leq i}(-1)^{i-j}b_j(X)
    + \sum_{j\leq i}(-1)^{i-j}b_j(Y) \\
    &= b_i^*(Z) - b_i^*(X) + b_i^*(Y)
  \end{align*}
  を得る。
  よって二つ目の不等式が従う。
  以上で\autoref{1.34}の解答を完了する。
\end{proof}





\begin{prob}\label{1.35}
  \(\hat{\mcC} = \sfSet^{\mcC^{\op}}\)を前層圏とする。
  \begin{enumerate}
    \item \label{1.35.1}
    \(I\)を有向集合、
    \(X_i\)を\(I\)で添字付けられた圏\(\mcC\)の図式とする。
    \(X\mapsto \colim_{i\in I}\Hom_{\mcC}(X,X_i)\)
    により定まる\(\hat{\mcC}\)の対象
    \("\colim"_{i\in I} X_i\)
    (この記号の定義は本文定義1.11.4を参照) は
    \(I\)で添字付けられた図式
    \(h_{X_i}\in \hat{\mcC}\)の余極限であることを示せ。
    より詳しく、
    \(F\in \hat{\mcC}\)に対して以下の自然な同型を示せ:
    \[
    \Hom_{\hat{\mcC}}("\colim"_{i\in I} X_i, F) \cong \colim_{i\in I} F(X_i).
    \]
    \item \label{1.35.2}
    \(Y_j\in \mcC\)を有向集合\(J\)で添字付けられた図式とする。
    以下の自然な同型を示せ:
    \[
    \Hom_{\hat{\mcC}}("\colim"_{i\in I} X_i, "\colim"_{j\in J} Y_j)
    \cong \lim_{i\in I}\colim_{j\in J}\Hom_{\mcC}(X_i,Y_j).
    \]
  \end{enumerate}
\end{prob}

\begin{rem*}
  本文では\ref{1.35.2}の左辺の右側の\("\colim"\)が
  たんに\(\colim\)と表記されていたが、これは\(""\)をつけ忘れた?
\end{rem*}

\begin{proof}
  \ref{1.35.1}を示す。
  函手圏の余極限は各点ごとに計算されるので
  \("\colim"_{i\in I} X_i\cong \colim_{i\in I}h_{X_i}\)が従う。
  さらにこれがわかると、余極限の定義と米田の補題より、
  \begin{align*}
    \Hom_{\hat{\mcC}}("\colim"_{i\in I} X_i, F)
    &\cong \Hom_{\hat{\mcC}}(\colim_{i\in I} X_i, F) \\
    &\cong \lim_{i\in I}\Hom_{\hat{\mcC}}(h_{X_i}, F) \\
    &\cong \lim_{i\in I}F(X_i)
  \end{align*}
  が従う。
  以上で\ref{1.35.1}の証明を完了する。

  \ref{1.35.2}を示す。
  素直に計算すると、
  \begin{align*}
    \Hom_{\hat{\mcC}}("\colim"_{i\in I} X_i, "\colim"_{j\in J} h_{Y_j})
    &\overset{\star}{\cong}
    \Hom_{\hat{\mcC}}(\colim_{i\in I} h_{X_i}, \colim_{j\in J} h_{Y_j}) \\
    &\overset{*}{\cong}
    \lim_{i\in I}\Hom_{\hat{\mcC}}(h_{X_i}, \colim_{j\in J} h_{Y_j}) \\
    &\overset{\scriptstyle\spadesuit}{\cong}
    \lim_{i\in I}(\colim_{j\in J} h_{Y_j})(X_i) \\
    &\overset{\scriptstyle\clubsuit}{\cong}
    \lim_{i\in I}\colim_{j\in J}\Hom_{\mcC}(X_i,Y_j)
  \end{align*}
  となる。
  ただしここで、\(\star\)の部分に\ref{1.35.1}を用い、
  \(*\)の部分に余極限の定義を用い、
  \(\spadesuit\)の部分に米田の補題を用い、
  \(\clubsuit\)の部分に函手圏での余極限が各点ごとに計算されることを用いた。
  以上で\ref{1.35.2}の証明を完了し、
  \autoref{1.35}の解答を完了する。
\end{proof}




\begin{prob}\label{1.36}
  \(A\)をネーター環、\(\Mod^f(A)\)を有限生成\(A\)-加群の圏とする。
  \((X_i,\rho_{ij})\)を有向集合で添字付けられた\(\Mod^f(A)\)の図式とする。
  \(\Mod(A)\)での余極限\(\colim_{i\in I}X_i\)が
  \(\Mod^f(A)\)に属すると仮定すると、
  それは\("\colim"_{i\in I}X_i\)の表現対象であることを示せ。
\end{prob}

\begin{rem*}
  もとの文を引用するとこうである (第一版):

  Let \(A\) be a Noetherian ring, and
  let \(\mathfrak{Mod}^f(A)\) be the category of finitely generated \(A\)-modules.
  Let \(\{X_i,\rho_{i,j}\}\) be an inductive system in this category,
  indexed by a directed ordered set \(I\).
  Prove that if \(\varinjlim_{j} X_j\) exists in \(\mathfrak{Mod}^f(A)\),
  then it represents \("\varinjlim_j" X_j\).

  これをそのまま読むと、仮定されていることは
  「\(\Mod^f(A)\)で余極限\(\colim_{i\in I}X_i\)が存在する」
  ということである。
  しかし、だとすると、\autoref{1.36}は本当に正しいだろうか。
  たとえば\(A=\Z\)として、
  有向集合として\(\Z\)のイデアルのなす集合を包含関係の逆向きで順序を入れたものを考え、
  \(X_i\dfn (\frac{1}{n}\Z)/\Z\)と定義して、
  \(\rho_{n,nm}\)を自然な包含射とする。
  \(M\)を有限生成加群、\(f_n:X_n\to M\)を\(\rho_{n,nm}\)たちと両立的な射とする。
  このとき、\(f_n(1/n)\)は\(M\)の可除元を与える。
  \(M\)は有限生成加群であるから、従って\(f_n(1/n)=0\)である。
  これは\(f_n=0\)を意味する。
  従って、どのような\(\rho_{n,nm}\)たちと両立的な射の族\(f_n:X_n\to M\)も
  \(0\)を一意的に経由する。
  これは\(\Mod^f(A)\)における図式\(X_n\)の余極限が\(0\)であることを意味している。
  とくに\(\Mod^f(A)\)における図式\(X_n\)には余極限が存在する。
  一方で\("\colim"_n X_n\)は自明な前層ではないので\(0\)はその表現対象ではない。
  これは問われていることに反する。
  従って、本当に仮定すべきことは、
  「\(\Mod(A)\)における余極限が\(\Mod^f(A)\)に属する」
  ということであろう。
  実際には、(ここで示すように)
  \(\Mod(A)\)における余極限を\(X\)としたとき、
  \(\Mod^f(A)\)上の前層として
  \("\colim"_j X_j \cong \Hom_A(-,X)\)が成り立つ
  (標語的に言えば、有限表示加群は加群の圏のコンパクト対象である、ということ)。
\end{rem*}

\begin{proof}
  \(X\dfn \colim_{i\in I}X_i\) (\(\Mod(A)\)における余極限) とおいて、
  \(\rho_i:X_i\to X\)を自然な射とする。
  \autoref{1.36}を示すためには、
  \(M\)を有限生成\(A\)-加群として、
  自然な射\(\varphi: \colim_{i\in I}\Hom_A(M,X_i) \to \Hom_A(M,X)\)
  が全単射であることを示すことが十分である。

  まず\(\varphi\)が単射であることを示す。
  \(\varphi\)で送って\(0\)である
  \(\colim_{i\in I}\Hom_A(M,X_i)\)の元をとり、
  \(f_i:M\to X_i\)を、その元を代表する射とする。
  \(\varphi\)で送って\(0\)であるので、
  \(\rho_i\circ f_i = 0\)である。
  \(M\)は有限生成なので、
  有限個の\(m_1,\cdots, m_r\in M\)によって生成される。
  \(f_i(m_k)\in X_i, (k=1,\cdots r)\)は\(\rho_i\)で送って\(0\)になるので、
  ある\(i_k \geq i\)が存在して\(X_{i_k}\)において
  \(\rho_{i,i_k}(f_i(m_k)) = 0\)である。
  \(I\)は有向集合であるから、
  \(i_1,\cdots, i_r\)の上界\(j\)が存在する。
  このとき\(\rho_{i,j}(f_i(m_k)) = 0, (\forall k =1,\cdots,r)\)
  であるので、\(\rho_{i,j}\circ f_i = 0\)である。
  \(f_i:M\to X_i\)によって代表される\(\colim_{i\in I}\Hom_A(M,X_i)\)の元は
  \(\rho_{i,j}\circ f_i:M\to X_j\)によって代表される元でもあるので、
  これは\(0\)である。
  以上より\(\varphi\)が単射であることが従う
  (ここまで\(A\)のネーター性は必要ない)。

  \(\varphi\)が全射であることを示す。
  \(f:M\to X\)を\(A\)-加群の射とする。
  全射\(p:A^r\to M\)を一つとる。
  \(A\)はネーターであるので、\(\ker(p)\)は有限生成である
  (ネーター性が本質的に必要なのはこの部分、すなわち、有限生成加群が有限表示であるという部分)。
  \(e_k\in A^r\)を\(k\)番目の座標のみ\(1\)で他が\(0\)となる元とする。
  \(X\)は\(X_i\)たちの filtered colimit であるから、
  \(f(p(e_k))\in X\)に対して
  ある\(i_k\in I\)が存在して
  \(f(p(e_k)) \in \rho_{i_k}(X_{i_k})\)が成り立つ。
  \(I\)は有向集合であるので、
  \(i_1,\cdots, i_r \leq j\)となる\(j\in I\)が存在する。
  このとき\(f(p(A^r)) \subset \rho_j(X_j)\)が成り立つ。
  \(A^r\)は射影加群であるから、
  全射\(\rho_j:X_j\to \rho_j(X_j)\)に沿って
  \(f\circ p:A^r\to \rho_j(X_j)\)がリフトして、
  \(f\circ p = \rho_j \circ g\)となる射\(g:A^r\to X_j\)が存在する。
  \(\ker(p)\)の生成元を\(a_1,\cdots,a_s\in \ker(p)\)とする。
  \(a_k\in \ker(p)\)であるから、
  \(\rho_j(g(a_k)) = f(p(a_k)) = f(0) = 0\)
  が成り立つ。
  従って、各\(k = 1,\cdots, s\)に対してある\(i'_k\geq j\)が存在して、
  \(X_{i'_k}\)で\(\rho_{j,i'_k}(g(a_k)) = 0\)が成り立つ。
  \(I\)は有向集合であるので、
  \(i'_1,\cdots, i'_s \leq j'\)となる\(j'\in I\)が存在する。
  このとき\(\rho_{j,j'}(g(a_k)) = 0, (\forall k = 1,\cdots ,s)\)が成り立つ。
  従って、\(\rho_{j,j'}\circ g: A^r\to X_{j'}\)は
  \(p:A^r\to M\)を一意的に経由して、
  \(\rho_{j,j'}\circ g = h\circ p\)となる
  射\(h:M\to X_{j'}\)を引き起こす。
  このとき
  \[
  \rho_{j'}\circ h \circ p = \rho_{j'}\circ \rho_{j,j'}\circ g = \rho_j\circ g
  = f\circ p
  \]
  が成り立つ。
  \(p\)はエピなので、\(\rho_{j'}\circ h = f\)が成り立つ。
  従って、\(h\)により代表される\(\colim_{i\in I}\Hom_A(M,X_i)\)の元を\([h]\)と書けば、
  \(\varphi([h]) = f\)が成り立つ。
  以上で\(\varphi\)が全射であることの証明を完了し、
  \autoref{1.36}の解答を完了する。
\end{proof}





\begin{prob}\label{1.37}
  \(\mcC\)を加法圏とする。
  \(\End(\mcC)\)を\(\id_{\mcC}:\mcC\to \mcC\)の自己射のなす集合とする。
  すなわち、\(\End(\mcC) \dfn \Hom_{[\mcC,\mcC]}(\id_{\mcC})\)とする。
  \begin{enumerate}
    \item \label{1.37.1}
    \(\End(\mcC)\)は可換環であることを示せ。
    \item \label{1.37.2}
    \(A\)を環とする。
    \(\End(\Mod(A))\)は\(A\)の中心\(Z(A)\)と同型であることを示せ。
    \item \label{1.37.3}
    \(A\)を可換環として、
    環準同型\(A\to \End(\mcC)\)が与えられているとする。
    このとき加法圏\(\mcC\)を\textbf{\(A\)上の加法圏}という。
    \(\mcC\)が\(A\)上の加法圏であるとき、
    \(\Hom_{\mcC}(X,Y)\)は合成が双線型となるような\(A\)-加群の構造を持つことを示せ。
    \item \label{1.37.4}
    \(A\)をネーター環、
    \(\mcC\)を\(A\)上の\textbf{アーベル}圏とする。
    \begin{enumerate}
      \item \label{1.37.4.1}
      \(M\in \Mod^f(A)\)と\(X\in \mcC\)に対して
      函手\(Y\mapsto \Hom_A(M,\Hom_{\mcC}(X,Y)), (Y\in \mcC)\)
      は表現可能であることを示せ。
      この表現対象を\(X\otimes_A M\)と書く。
      \item \label{1.37.4.2}
      \(\otimes_A:\mcC\times\Mod^f(A)\to \mcC\)は右完全な双函手であることを示せ。
      \item \label{1.37.4.3}
      \(\otimes_A\)は左導来函手
      \(\otimes_A^L:\sfD^-(\mcC)\times \sfD^-(\Mod^f(A))\to \sfD^-(\mcC)\)
      を持つことを示せ。
      \item \label{1.37.4.4}
      \(\Hom_A(-,-):\Mod^f(A)^{\op}\times \mcC\to \mcC\)
      についても同様の議論を行え。
    \end{enumerate}
  \end{enumerate}
\end{prob}


\begin{proof}
  \ref{1.37.1}を示す。
  \(f:\id_{\mcC}\to \id_{\mcC}\)は
  各\(M\in \mcC\)に対する自己射\(f_M:M\to M\)の族で
  \(g:M\to N\)に対して\(g\circ f_M = f_N\circ g\)を満たすものである。
  従って、二つの\(f^1,f^2:\id_{\mcC}\to \id_{\mcC}\)に対して
  族\((f^1_M+f^2_M)_{M\in \mcC}\)は\(\id_{\mcC}\)の自己射となるので、
  これによって加法が定義される。
  乗法を合成によって定義すると、\(\mcC\)が加法圏であること、
  すなわち合成が双線型であることから、\(\End(\mcC)\)は環の公理を満たす。
  可換であることを示すことが残っている。
  \(f,g:\id_{\mcC}\to \id_{\mcC}\)と
  \(M\in \mcC\)を任意にとると、
  \(g\)が自然変換であることから、
  射\(f_M:M\to M\)に対して等式\(f_M\circ g_M = g_M\circ f_M\)を満たす。
  従って\(f\circ g = g\circ f\)が成り立ち、
  \(\End(\mcC)\)は合成を乗法として可換である。
  以上で\ref{1.37.1}の証明を完了する。

  \ref{1.37.2}を示す。
  \(a\in Z(A)\)に対して、
  一斉に\(a\)倍をする射\(M\to M\)は
  任意の\(g:M\to N\)と\(m\in M\)に対して
  \(g(am) = ag(m)\)を満たすので
  \(\End(\Mod(A))\)の元を定める。
  こうして写像\(Z(A)\to \End(\Mod(A))\)ができる。
  この写像は明らかに環準同型である。
  単射であることを示すために、\(a\in Z(A)\)が\(\End(\Mod(A))\)で\(0\)であると仮定する。
  すると\(a\)倍写像\(A\to A\)が\(0\)射であるため、\(a=0\)が従う。
  よって\(Z(A)\to \End(\Mod(A))\)は単射である。
  全射であることを示すために、
  \(f:\id_{\Mod(A)}\to \id_{\Mod(A)}\)を任意にとる。
  \(f_A:A\to A\)によって\(a\dfn f_A(1)\)とおく。
  \(f_M\)が\(a\)倍写像であることを示せば、
  \(Z(A)\to \End(\Mod(A))\)が全射であることが従う。
  \(M\in \Mod(A)\)を任意にとる。
  全射\(p:A^{\oplus I}\to M\)をひとつ選ぶ。
  \(f:\id_{\Mod(A)}\to \id_{\Mod(A)}\)が自然変換であることから、
  \(f_{A^{\oplus I}}\)は各座標ごとに\(f_A\)が並んでいる射であり、
  それは\(a\)倍写像に他ならない。
  また\(f_M\circ p = p\circ f_{A^{\oplus I}} = p(a\text{倍}) = ap\)
  が成り立つ。
  ここで\(p\)はエピなので、\(f_M\)も\(a\)-倍写像であることが従う。
  以上で\(Z(A)\to \End(\Mod(A))\)が全射であることが従い、
  \ref{1.37.2}の証明を完了する。

  \ref{1.37.3}を示す。
  \(\varphi:A\to \End(\mcC)\)を環準同型とする。
  \(a\in A\)に対して自然変換\(\varphi(a):\id_{\mcC}\to \id_{\mcC}\)が対応している。
  \(\Hom_{\mcC}(X,Y)\)に\(\varphi(a)_Y:Y\to Y\)を合成することによって
  \(A\)-加群の構造を入れる
  (これが\(\Hom_{\mcC}(X,Y)\)の加法と両立的であることは明らかである)。
  このとき、\(f:X\to Y\)に対して
  \(f\circ \varphi(a)_X = \varphi(a)_Y\circ f\)であるから、
  この\(A\)-加群の構造は\(\varphi(a)_X:X\to X\)を合成することによる\(A\)-加群の構造と等しい。
  また、\(X,Y,Z\in \mcC\)と
  \(a\in A, f\in \Hom_{\mcC}(X,Y), g\in \Hom_{\mcC}(Y,Z)\)に対して、
  \[
  g\circ (a\cdot f) = g\circ (f\circ \varphi(a)_X)
  = (g\circ f)\circ \varphi(a)_X = a\cdot (g\circ f)
  \]
  が成り立つので、
  \(g\)を合成する射\(\Hom_{\mcC}(X,Y) \to \Hom_{\mcC}(X,Z)\)
  は\(A\)-加群の構造と両立的である。
  同じく
  \[
  (a\cdot g)\circ f = (\varphi(a)_Z\circ g) \circ f
  = \varphi(a)_Z\circ (g\circ f) = a\cdot (g\circ f)
  \]
  が成り立つので、\(f\)を合成する射\(\Hom_{\mcC}(Y,Z) \to \Hom_{\mcC}(X,Z)\)
  は\(A\)-加群の構造と両立的である。
  以上より\(\mcC\)の合成は\(A\)-双線型であり、
  \ref{1.37.3}の証明を完了する。

  \ref{1.37.4}を示す。
  \ref{1.37.4.1}を示す。
  \(M=A\)のときは自然に
  \(\Hom_A(A,\Hom_{\mcC}(X,Y)) \cong \Hom_{\mcC}(X,Y)\)
  であるから明らかにこの函手が表現可能であり\(X\otimes_A A\cong X\)が成り立つ。
  \(M\)が\(A\)の有限直和の場合も同様にして
  \(\Hom_A(A^n,\Hom_{\mcC}(X,Y))\cong \Hom_{\mcC}(X,Y)^n \cong \Hom_{\mcC}(X^n,Y)\)
  が成り立つので、この函手は表現可能であり\(X\otimes_A A^n \cong X^n\)が成り立つ。
  一般の有限生成加群\(M\)に対して、所望の表現可能性を証明する。
  \(A\)はネーターであるから、完全列
  \(A^n\to A^m\to M\to 0\)が存在する。
  このとき
  \[
  0 \to \Hom_A(M,\Hom_{\mcC}(X,Y)) \to \Hom_A(A^m,\Hom_{\mcC}(X,Y))
  \to \Hom_A(A^n,\Hom_{\mcC}(X,Y))
  \]
  も完全である。
  \(Y\)に関して函手的に
  \(\Hom_A(A^m,\Hom_{\mcC}(X,Y)) \cong \Hom_{\mcC}(X^m,Y)\)
  が成り立つので、
  \(A\)-加群の完全列
  \[
  0\to \Hom_A(M,\Hom_{\mcC}(X,Y)) \to \Hom_{\mcC}(X^m,Y)\to \Hom_{\mcC}(X^n,Y)
  \]
  を得る。
  従って\(Y\)に関して函手的に
  \(\Hom_A(M,\Hom_{\mcC}(X,Y)) \cong \Hom_{\mcC}(\coker(X^m\to X^n),Y)\)
  が成り立つ。
  よって\(Y\mapsto \Hom_A(M,\Hom_{\mcC}(X,Y))\)は表現可能であることが従う。
  以上で\ref{1.37.4.1}の証明を完了する。

  \ref{1.37.4.2}を示す。
  \(M,X\)に関しての双函手
  \[
  \mcC\times \Mod^f(A) \to \Hom(\mcC,\Mod(A)), \ \
  (X,M)\mapsto [Y\mapsto \Hom_A(M,\Hom_{\mcC}(X,Y))]\]
  の表現対象として\(X\otimes_A M\)が定義されているので、
  米田の補題より\(\otimes_A:\mcC\times \Mod^f(A)\to \mcC\)
  は双函手である。
  さらに\(\Hom_A(-,*)\)が左完全であることと\(\Hom_{\mcC}(-,Y)\)が左完全であることから、
  \(\otimes_A\)はいずれの成分についても右完全であることが従う。
  以上で\ref{1.37.4.2}の証明を完了する。

  \ref{1.37.4.3}を示す。
  \(\mcP\subset \Mod^f(A)^{\op}\)を射影加群からなる部分圏とする。
  \(X\in \mcC\)とする。
  \(\mcP\)が\((X\otimes_A (-))^{\op}:\Mod^f(A)^{\op}\to \mcC^{\op}\)
  に対してinjectiveであることを示す。
  まず\(\mcP\subset \Mod^f(A)^{\op}\)は明らかに本文の条件(1.7.5)
  (=本文定義1.8.2の条件(i)) を満たす。
  次に有限生成加群の完全列\(0\to M_1\to M_2\to M_3\to 0\)で
  \(M_2,M_3\)が射影加群であるとき、
  この完全列は分裂して\(M_1\)は射影加群\(M_2\)の直和因子となるので\(M_1\)も射影加群である。
  従って\(\mcP\)は本文定義1.8.2の条件(ii) を満たす。
  \(0\to P_1\to P_2\to P_3\to 0\)を射影加群の完全列とする。
  これは分裂するので、
  各\(Y\)に対して
  \[
  0\to \Hom_A(P_3,\Hom_{\mcC}(X,Y))
  \to \Hom_A(P_2,\Hom_{\mcC}(X,Y))
  \to \Hom_A(P_1,\Hom_{\mcC}(X,Y)) \to 0
  \]
  も分裂完全列である。
  従って、
  \[0\to X\otimes_A P_1\to X\otimes_A P_2\to X\otimes_A P_3\to 0\]
  も分裂完全列であり、\(\mcP\)は本文定義1.8.2の条件(iii) を満たす。
  以上より\(\mcP\)は\((X\otimes_A (-))^{\op}:\Mod^f(A)^{\op}\to \mcC^{\op}\)
  に対してinjectiveな\(\Mod^f(A)\)の部分圏である。
  \(X\in \Ch^-(\mcC)\)を\(0\)と擬同型な複体、
  \(P\)を射影加群とする。
  \(P\)は\(A^n\)の直和因子であるとする。
  すると\(X\otimes_A P\)は\(X^n\)の直和因子であるから、
  \(X\)が\(0\)と擬同型であることから、\(X\otimes_A P\)も\(0\)と擬同型である。
  従って、函手\((\otimes_A)^{\op}:\mcC^{\op}\times \Mod^f(A)^{\op}\to \mcC^{\op}\)
  の引き起こす三角函手
  \(\sfK^+(\mcC^{\op}) \times \sfK^+(\Mod^f(A)^{\op})\to \sfK^+(\mcC^{\op})\)
  と\(\mcI = \sfK^+(\mcP)\subset \sfK^+(\Mod^f(A)^{\op})\)
  に対して本文系1.10.5を用いることにより、
  \((\otimes_A)^{\op}\)の右導来函手
  \(\sfD^+(\mcC^{\op}) \times \sfD^+(\Mod^f(A)^{\op})\to \sfD^+(\mcC^{\op})\)
  が存在することが従う。
  よって\(\otimes_A\)の左導来函手
  \(\otimes_A^L:\sfD^-(\mcC) \times \sfD^-(\Mod^f(A))\to \sfD^-(\mcC)\)
  が存在することが従い、
  \ref{1.37.4.3}の証明を完了する。

  \ref{1.37.4.4}を示す。
  \(M\in \Mod^f(A)^{\op}\)と\(X\in \mcC\)に対して
  \(\mcC^{\op}\to \Mod(A), Y\mapsto \Hom_A(M,\Hom_{\mcC}(Y,X))\)
  が表現可能であることを示す。
  まず\(M=A\)のときは明らかに\(X\)が表現対象であり、
  \(M=A^n\)の場合も明らかに\(X^n\)が表現対象である。
  一般の\(M\)に対して完全列\(A^n\to A^m\to M\to 0\)をとって
  完全列
  \[
  0\to \Hom_A(M,\Hom_{\mcC}(Y,X))\to \Hom_A(A^m,\Hom_{\mcC}(Y,X))\to
  \Hom_A(A^n,\Hom_{\mcC}(Y,X))
  \]
  作ると、完全列
  \[
  0\to \Hom_A(M,\Hom_{\mcC}(Y,X)) \to \Hom_{\mcC}(Y,X^m)\to \Hom_{\mcC}(Y,X^n)
  \]
  を得るので、\(\Hom\)の左完全性より
  \(Y\)についての自然な同型
  \(\Hom_A(M,\Hom_{\mcC}(Y,X))\cong \Hom_{\mcC}(Y,\ker(X^m\to X^n))\)
  を得る。
  従って\(\mcC^{\op}\to \Mod(A), Y\mapsto \Hom_A(M,\Hom_{\mcC}(Y,X))\)
  は表現可能函手である。
  この表現対象を\(\Hom_A(M,X)\)と表す。
  \(X,Y,M\)についての自然な同型
  \(\Hom_{\mcC}(X\otimes_AM,Y) \cong \Hom_{\mcC}(X,\Hom_A(M,Y))\)
  が存在するので、
  \(\Hom_A(M,-)\)は\((-)\otimes_AM\)の右随伴函手であり、
  従って左完全である。
  また\(X,Y\)についての自然な同型
  \(\Hom_{\mcC}(X,\Hom_A(-,Y))\cong \Hom_A(-,\Hom_{\mcC}(X,Y))\)
  は\(\Hom_A(-,Y)\)の左完全性を示している。
  従って\(\Hom_A(-,-)\)は左完全な双函手である。
  射影加群のなす部分圏\(\mcP\subset \Mod^f(A)^{\op}\)が
  本文定義1.8.2の条件(i),(ii)を満たすことはすでに\ref{1.37.4.3}の証明の中で確認している。
  射影加群の完全列は分裂するので、
  それを\(\Hom_A(-,\Hom_{\mcC}(X,Y))\)に入れて得られる列も分裂完全列である。
  従って\(\Hom_A(-,Y)\)に射影加群の完全列を入れると分裂完全列が得られる。
  このことは\(\mcP\)が本文定義1.8.2の条件(iii)を\(\Hom_A(-,Y)\)に対して満たすことを意味している。
  従って\(\mcP\subset \Mod^f(A)^{\op}\)は\(\Hom_A(-,Y)\)-injectiveである。
  さらに\(Y\)が\(0\)と擬同型で\(P\)が射影加群であるとき、
  \(P\subset A^n\)が直和因子であるとすれば、
  \(\Hom_A(P,Y)\subset Y^n\)も直和因子であるから、
  \(Y\)が\(0\)と擬同型であることから、\(\Hom_A(P,Y)\)も\(0\)と擬同型であることが従う。
  よって本文系1.10.5を適用することで、
  \(\Hom_A(-,-)\)の右導来函手\(R\Hom_A(-,-)\)が存在することが従う。
  以上で\ref{1.37.4.4}の証明を完了し、
  \ref{1.37.4}の証明を完了し、
  \autoref{1.37}の解答を完了する。
\end{proof}





\begin{prob}\label{1.38}
  \(I,I'\)をfilteredな圏として、\(\varphi:I\to I'\)を函手とする。
  \(\varphi\)が\textbf{cofinal}であるとは、
  以下の条件を満たすことを言う:
  \begin{enumerate}
    \item
    任意の\(i'\in I'\)に対してある\(i\in I\)と射\(i'\to \varphi(i)\)が存在する。
    \item
    任意の\(i\in I\)と\(i'\in I'\)と射\((f:\varphi(i)\to i')\in I'\)に対して
    ある射\((g:i\to i_1)\in I\)と
    \((h:i'\to \varphi(i_1))\in I'\)が存在して\(h\circ f = g\)となる。
  \end{enumerate}
  \(\mcC\)を圏、\(I,I_1\)をfilteredな圏、
  \(F:I\to \mcC, G:I^{\op}\to \mcC\)を函手、
  \(\varphi:I_1\to I\)をcofinalとする。
  自然な射
  \[
  \colim (F\circ \varphi) \to \colim F, \ \
  ``\colim'' (F\circ \varphi) \to ``\colim'' F, \ \
  \lim G \to \lim (G\circ \varphi), \ \
  ``\lim'' G \to ``\lim'' (G\circ \varphi)
  \]
  はいずれも同型射であることを示せ。
\end{prob}

\begin{proof}
  \(\colim (F\circ \varphi) \to \colim F\)が同型射であることがわかれば、
  \(\mcC\to \hat{\mcC}=\sfSet^{\mcC^{\op}}\)を合成して
  函手\(I\to \hat{\mcC}\)に対してその事実を適用することにより
  \(``\colim'' (F\circ \varphi) \to ``\colim'' F\)が同型射であることが従う。
  \(\lim\)に関しても同様である。
  さらに\(\colim (F\circ \varphi) \to \colim F\)が同型射であることがわかれば、
  \(G^{\op}:I\to \mcC^{\op}\)に対してその事実を適用することにより
  \(\lim G \to \lim (G\circ \varphi)\)が同型射であることが従う。
  以上より、\autoref{1.38}を示すためには、
  \(\colim (F\circ \varphi) \to \colim F\)が同型射であることを示すことが十分である。

  \(X\in \mcC\)を任意にとる。
  \(\colim (F\circ \varphi) \to \colim F\)が同型射であることを示すためには、
  米田の補題より、
  自然な射
  \(\Psi:\lim_{i\in I}\Hom_{\mcC}(F(i),X) \to
  \lim_{i_1\in I_1}\Hom_{\mcC}(F(\varphi(i_1)),X)\)
  が全単射であることを示すことが十分である。
  \((f_i),(g_i)\in \lim_{i\in I}\Hom_{\mcC}(F(i),X)\)が
  \(\Psi((f_i)) = \Psi((g_i))\)を満たすとする。
  このとき、各\(i_1\in I_1\)に対して
  \(f_{\varphi(i_1)} = g_{\varphi(i_1)}\)が成り立つ。
  \(i\in I\)を任意にとる。
  \(\varphi:I_1\to I\)はcofinalであるから、一つめの条件より、
  ある\(i_1\in I_1\)と射\(p:i\to \varphi(i_1)\)が存在する。
  \((f_i),(g_i)\)はそれぞれ\(\lim_{i\in I}\Hom_{\mcC}(F(i),X)\)の元であるから、
  \(f_{\varphi(i_1)}\circ F(p) = f_i, g_{\varphi(i_1)} \circ F(p) = g_i\)を満たす。
  \(f_{\varphi(i_1)} = g_{\varphi(i_1)}\)であるので、
  従って\(f_i = g_i\)が成り立つ。
  これは\((f_i)=(g_i)\)を意味し、よって\(\Psi\)は単射である。

  \(\Psi\)が全射であることを示す。
  \((h_{\varphi(i_1)})_{i_1\in I_1}\in \lim_{i_1\in I_1}\Hom_{\mcC}(F(\varphi(i_1)),X)\)
  を任意にとる。
  各\(i\in I\)に対して一つ\(i_1\in I_1\)と射\(p_1:i\to \varphi(i_1)\)を選ぶ
  (\(\varphi\)がcofinalであることの一つめの条件を用いる)。
  \(h_i \dfn h_{i_1}\circ F(p_1)\)と定義する。
  まずこれが\(i_1,p_1\)の取り方に依存しないことを示す。
  そのためには、別の\(p_2:i\to \varphi(i_2)\)に対して
  \(h_{i_1}\circ F(p_1) = h_{i_2}\circ F(p_2)\)が成り立つことが十分である。
  \(I_1\)はfilteredであるから、
  \(i_3\in I_1\)と\(a_1:i_1\to i_3,a_2:i_2\to i_3\)が存在する。
  \(I\)はfilteredであるから、
  二つの並行な射\(\varphi(a_1)\circ p_1, \varphi(a_2)\circ p_2: i \to \varphi(i_3)\)
  に対してある射\(g:\varphi(i_3)\to i'\)が存在して
  \(g\circ \varphi(a_1)\circ p_1 = g\circ \varphi(a_2)\circ p_2\)
  が成り立つ。
  さらに\(\varphi\)はcofinalであるから、
  \(g:\varphi(i_3)\to i'\)に二つめの条件を用いることで、
  ある\((b:i_3\to i_4)\in I_1\)と\((g':i'\to \varphi(i_4))\in I\)が存在して
  \(g' \circ g = \varphi(b)\)が成り立つ。
  このとき
  \[
  \varphi(b\circ a_1) \circ p_1
  = g'\circ g \circ \varphi(a_1) \circ p_1
  = g'\circ g \circ \varphi(a_2) \circ p_2
  = \varphi(b\circ a_2) \circ p_2
  \]
  が成り立つ。
  \(p_4 \dfn \varphi(b\circ a_1) \circ p_1\)とおけば、
  \[
  h_{i_1}\circ F(p_1)
  = h_{i_4} \circ F(\varphi(b\circ a_1)\circ p_1)
  = h_{i_4} \circ F(\varphi(b\circ a_2)\circ p_2)
  = h_{i_2}\circ F(p_2)
  \]
  が成り立つ。
  以上で\(h_i\)の定義が\(p_1:i\to \varphi(i_1)\)の取り方に依存しないことが示された。
  次に\((h_i)_{i\in I}\)が\(\lim\Hom_{\mcC}(F(i),X)\)の元を定めることを示す。
  射\((p:i\to i')\in I\)を任意にとる。
  \(q:i'\to \varphi(i_1)\)を一つ選べば、
  \(h_i\)の定義が\(p_1:i\to \varphi(i_1)\)の取り方に依存しないことから、
  \[
  h_i = h_{i_1}\circ F(q\circ p)
  = h_{i_1}\circ F(q) \circ F(p)
  = h_{i'} \circ F(p)
  \]
  が成り立つ。
  これは\((h_i)_{i\in I}\)が\(F(p)\)たちと両立的であることを意味し、
  従って\((h_i)_{i\in I}\)は\(\lim\Hom_{\mcC}(F(i),X)\)の元を定める。
  各\(i_1\in I\)に対して\(h_{\varphi(i_1)} = h_{i_1}\)であるから、
  \(\Psi((h_i)_{i\in I}) = (h_{i_1})_{i_1\in I_1}\)が成り立つ。
  よって\(\Psi\)は全射である。
  以上で\(\Psi\)が全単射であることが従い、
  \autoref{1.38}の証明を完了する。
\end{proof}





\begin{prob}\label{1.39}
  \(\mcC\)をアーベル圏とする。
  \(X,Y\in \sfD^b(\mcC)\)に対して
  \(\Ext^j(X,Y) \dfn \Hom_{\sfD(\mcC)}(X,Y[j])\)とおく。
  \begin{enumerate}
    \item \label{1.39.1}
    \(X,Y\in \mcC\)として\(n\geq 1\)とする。
    完全列
    \[
    E: 0 \to Y\to Z_n\to Z_{n-1} \to \cdots \to Z_1 \to X\to 0
    \]
    が元\(C(E) \in \Ext^n(X,Y)\)を定めることを示せ。
    この完全列を\textbf{\(X\)の\(Y\)による\(n\)-拡大}という。
    \item \label{1.39.2}
    任意の\(\Ext^n(X,Y)\)の元は\(C(E)\)の形で表すことができることを示せ。
    \item \label{1.39.3}
    \(E':0\to Y\to Z'_n\to \cdots \to Z'_1\to X\to 0\)を別の拡大とする。
    \(C(E) = C(E')\)であるための必要十分条件は、
    ある拡大\(E'': 0\to Y\to Z''_n\to \cdots \to Z''_1\to X\to 0\)
    と以下の可換図式が存在することであるということを示せ:
    \[
    \begin{CD}
      Y @>>> Z_n @>>> \cdots @>>> Z_1 @>>> X \\
      @| @AAA @. @AAA @| \\
      Y @>>> Z''_n @>>> \cdots @>>> Z''_1 @>>> X \\
      @| @VVV @. @VVV @| \\
      Y @>>> Z'_n @>>> \cdots @>>> Z'_1 @>>> X.
    \end{CD}
    \]
  \end{enumerate}
  \(\Ext^n(X,Y)\)をしばしば\textbf{Yoneda extension}という。
\end{prob}


\begin{proof}
  \ref{1.39.1}を示す。
  \(Z^i \dfn Z_{-i+1}\)と定義して、完全列\(E\)から\(X,Y\)を取り除いた複体を
  \[Z = (\cdots 0\to Z^{-n+1} \to \cdots \to Z^0\to 0 \to \cdots)\]
  と表す。
  このとき\(\tau^{\leq n-1}(Z) = Y[n-1]\)であり、
  \(\tau^{\geq 0}(Z) = X\)である。
  また、\(Z\)は\(-(n-2)\)次から\(-1\)次で完全なので、
  \(Y[n-1]\to Z\to X\xrightarrow{+1}\)は完全三角である。
  これに函手\(\Hom_{\sfD(\mcC)}(X,-)\)を適用することにより、
  射\(\Hom_{\sfD(\mcC)}(X,X)\to \Hom_{\sfD(\mcC)}(X,Y[n]) = \Ext^n(X,Y)\)を得る。
  \(\id_X\)の行き先を\(C(E)\)とすれば良い。
  以上で\ref{1.39.1}の証明を完了する。

  \ref{1.39.2}を示す。
  \(f\in \Ext^n(X,Y)\)を一つとる。
  定義より\(\Ext^n(X,Y) = \Hom_{\sfD(\mcC)}(X,Y[n])\)であるので、
  \(f\)は\(\sfD(\mcC)\)の射\(f:X\to Y[n]\)とみなせる。
  \(f\)を\(\sfD(\mcC)\)の完全三角
  \(X\xrightarrow{f} Y[n]\to Z\xrightarrow{+1}\)に伸ばす。
  このとき\(Y[n-1]\to Z[-1]\to X\xrightarrow{+1}\)も完全三角である。
  コホモロジーをとれば、
  \(H^i(Z[-1])\)は\(n-1\)次で\(H^{n-1}(Z[-1])\cong Y\)、
  \(0\)次で\(H^0(Z[-1])\cong X\)、他は\(0\)である。
  従って
  \[E : 0\to [Y\cong H^n(Z)]\to Z^{-n} \to \cdots \to Z^{-1} \to [X\cong H^{-1}(Z)]\to 0\]
  は完全である。
  完全三角\(Y[n-1]\to Z[-1]\to X\xrightarrow{+1}\)は
  \(X\xrightarrow{f} Y[n]\to Z\xrightarrow{+1}\)を
  \(-1\)方向に二つずらした完全三角なので、
  \(\Hom_{\sfD(\mcC)}(X,-)\)に入れると
  \(\id_X\)の行き先は\(f:X\to Y[n]\)に他ならない。
  このことは\(f = C(E)\)を意味している。
  以上で\ref{1.39.2}の証明を完了する。

  \ref{1.39.3}を示す。
  \(E,E'\)から\ref{1.39.1}のように定義した複体をそれぞれ\(Z,Z'\)と表す。
  十分性を示す。
  \(E''\)から\ref{1.39.1}のように定義した複体を\(Z''\)と表す。
  \ref{1.39.1}の証明より、\(\sfD(\mcC)\)の完全三角とその間の射
  \[
  \begin{CD}
    Y[n-1] @>>> Z @>>> X @> +1 >> \\
    @| @AAA @| @. \\
    Y[n-1] @>>> Z'' @>>> X @> +1 >> \\
    @| @VVV @| @. \\
    Y[n-1] @>>> Z' @>>> X @> +1 >>
  \end{CD}
  \]
  を得る。
  これを\(\Hom_{\sfD(\mcC)}(X,-)\)に入れると、
  アーベル群の可換図式
  \[
  \begin{CD}
    \Hom_{\sfD(\mcC)}(X,X) @= \Hom_{\sfD(\mcC)}(X,X) @= \Hom_{\sfD(\mcC)}(X,X) \\
    @V \delta VV @V \delta'' VV @VV \delta' V \\
    \Hom_{\sfD(\mcC)}(X,Y[n]) @= \Hom_{\sfD(\mcC)}(X,Y[n]) @= \Hom_{\sfD(\mcC)}(X,Y[n])
  \end{CD}
  \]
  を得る。
  ここで定義より\(C(E) = \delta(\id_X), C(E') = \delta'(\id_X)\)であるが、
  上の図式が可換であることは\(\delta = \delta'' = \delta'\)を意味するので、
  よって\(C(E) = \delta(\id_X) = \delta'(\id_X) = C(E')\)が成り立つ。
  以上で十分性の証明を完了する。

  必要性を示す。
  \(f = C(E) = C(E')\in \Ext^n(X,Y) = \Hom_{\sfD(\mcC)}(X,Y[n])\)とおく。
  \(f:X\to Y[n]\)を\(\sfD(\mcC)\)の完全三角
  \(X\xrightarrow{f} Y[n]\to Z'' \xrightarrow{+1}\)に伸ばす。
  \ref{1.39.2}の証明と同様に、このとき
  \[
  0\to Y\to {Z''}^{-n} \to \cdots \to {Z''}^{-1} \to X\to 0
  \]
  は完全である。
  \(Z''\)の\(-n-1\)次以下と\(0\)次以上を\(0\)で置き直した複体を再び\(Z''\)で表す。
  すると上の完全列により
  \(Y[n-1]\to Z'' [-1] \to X\xrightarrow{+1}\)が
  \(\sfK(\mcC)\)の完全三角であることが従う。
  \(f=C(E)\)であるから、三角圏の公理 (本文の命題1.4.4.の(TR4)) より
  \(\sfK(\mcC)\)の射\(Z''\to Z\)が存在して、
  \(\id_X,\id_{Y[n]}\)によって
  \(X\xrightarrow{f} Y[n]\to Z'' \xrightarrow{+1}\)から
  \(X\xrightarrow{f} Y[n]\to Z \xrightarrow{+1}\)への完全三角の射を形成する。
  同様に、\(f=C(E')\)であるから、完全三角の射を形成するような\(Z''\to Z'\)も存在する。
  よって\(\sfK(\mcC)\)の擬同型からなる図式
  \(Z \gets Z'' \to Z\)を得る。
  これらの射を代表する\(\Ch(\mcC)\)の擬同型からなる図式
  \(Z\gets Z'' \to Z\)を\(\mcC\)の図式として書き直すと、
  可換図式
  \[
  \begin{CD}
    H^{-n}(Z) @>>> Z^{-n} @>>> \cdots @>>> Z^{-1} @>>> H^{-1}(Z) \\
    @AAA @AAA @. @AAA @AAA \\
    H^{-n}(Z'') @>>> {Z''}^{-n} @>>> \cdots @>>> {Z''}^{-1} @>>> H^{-1}(Z'') \\
    @VVV @VVV @. @VVV @VVV \\
    H^{-n}(Z') @>>> {Z'}^{-n} @>>> \cdots @>>> {Z'}^{-1} @>>> H^{-1}(Z').
  \end{CD}
  \]
  を得る。
  \(H^{-n}(Z) \cong H^{-n}(Z'') \cong H^{-n}(Z') \cong Y\)と
  \(H^{-1}(Z) \cong H^{-1}(Z'') \cong H^{-1}(Z') \cong X\)に注意すれば
  所望の可換図式を得る。以上で必要性の証明を完了し、
  \ref{1.39.3}の証明を完了し、
  \autoref{1.39}の解答を完了する。
\end{proof}














\newpage
\section{Sheaves}

本文では、パラコンパクト空間であるという場合には、
ハウスドルフ性を常に仮定していることに注意しておく
(cf. 本文命題2.5.1直後の記述)。


\begin{prob}\label{2.1}
  \(\N\)を自然数の集合で、
  \(\{0,\cdots, n\}, n\geq -1\)たちが開となる最も粗い位相を入れる。
  このとき、\(\N\)上の前層\(F\)は
  各\(n\)に対するアーベル群\(F_n\dfn F(\{0,\cdots,n\})\)と
  \(n\geq m\)に対する開集合の包含
  \(\{0,\cdots, m\} \subset \{0,\cdots ,n\}\)
  により引き起こされる制限写像\(F_n\to F_m\)の族に
  唯一のアーベル群\(F_{\infty}\dfn F(\N)\)を添加したものと同一視される。
  \begin{enumerate}[start=0]
    \item \label{2.1.0}
    前層\(F\)が層であるための必要十分条件は
    \(\Gamma(\N,F) \cong \lim_n F_n\)であることを示せ。
    \item \label{2.1.1}
    各\(j\neq 0,1\)に対して\(H^j(\N,F) = 0\)であることを示せ。
    \item \label{2.1.2}
    \(H^1(\N,F) \cong (\prod_n F_n)/I\)であることを示せ。
    ただし\(I\)の定義は、
    \(f_{i,j}:F_i\to F_j\)を層\(F\)の制限写像とするとき、
    以下で定義される:
    \[
    I\dfn \left\{ (x_n)_{n\in \N}\in \prod_n F_n \middle|
    \forall n\in \N, \exists y_n\in F_n, x_n = y_n - f_{n+1,n}(y_{n+1})\right\}.
    \]
  \end{enumerate}
\end{prob}

\begin{proof}
  \ref{2.1.0}は自明。
  \ref{2.1.1}を示す。
  \(G_n \dfn \prod_{i\leq n}F_i\)と置く。
  射影\(G_n\to G_{n-1}\)らにより定まる\(\N\)上の層を\(G\)と置くと、
  構成からただちに\(G\)が脆弱層であることがわかる。
  各\(n\geq i\)に対して制限写像
  \(F_n \to F_i\)の族が単射
  \(F_n \to \prod_{i\leq n}F_i = G_n\)
  を引き起こす。
  これは制限写像\(F_n\to F_{n-1}\)と射影\(G_n\to G_{n-1}\)と可換し、
  よって層の単射\(F\to G\)を得る。

  層\(G/F\)の構造を決定する。
  層\(F\)の制限写像を\(f_{i,j}:F_i\to F_j\)と置く。
  \[
  \varphi_n ((x_i)_{i\leq n}) \dfn (x_i-f_{n,i}(x_n))_{i<n}
  \]
  で定まる射\(\varphi_n : G_n \to H_n \dfn \prod_{i<n}F_i\)
  は全射であり、核はちょうど\(\im(F_n\to G_n)\)である。
  また、\(m\leq n\)に対して\(h_{n,m}: H_n\to H_m\)を
  \[
  h_{n,m}((x_i)_{i<n}) \dfn (x_i - f_{m,i}(x_m))_{i<m}
  \]
  と定めれば、各\(i < m \leq n\)に対して
  \[
  (x_i - f_{n,i}(x_n)) - (f_{m,i}(x_m - f_{n,m}(x_n))
  = x_i - f_{m,i}(x_m)
  \]
  となるので、図式
  \[
  \begin{CD}
    G_n @> \varphi_n >> H_n \\
    @V \text{proj.} VV @VV h_{n,m} V \\
    G_m @> \varphi_m >> H_m
  \end{CD}
  \]
  は可換である。
  これらの\(H_n\)により定まる層\(H\)は\(G/F\)に他ならないが、
  各\(h_{n,m}\)は全射であるから、\(H\)は脆弱層である。
  以上より\(\N\)上の層の完全列
  \[
  \begin{CD}
    0 @>>> F @>>> G @>>> H @>>> 0
  \end{CD}
  \]
  で\(G,H\)が脆弱層となるものが構成できた。
  このことは\(H^j(\N,F) = 0 , j\geq 2\)を示している。
  以上で\ref{2.1.1}の証明を完了する。

  \ref{2.1.2}を示す。
  \ref{2.1.1}の証明中に得られた層\(H\)の大域切断を決定する。
  それは\(\lim_n H_n\)であるから、このアーベル群を計算する。
  \(\lim_nH_n\)は\(\prod_{n\in N}H_n\)の部分加群で、
  \[
  \left\{ (x^n \dfn (x_i^n)_{i<n})_{n\in N} \middle| x^n \in H_n,
  \forall (m \leq n), h_{n,m}(x^n) = x^m \right\}
  \]
  となるものと自然に同型である。
  従って各\(i < m\leq n\)に対して
  \(x^m_i = x^n_i - h_{m,i}(x^n_m)\)
  となる。
  従って、このような元の族\(x^n\in H_n\)は\(x^n_{n-1}\in F_{n-1}\)によって
  \(x^n_i = x^{n-1}_i + h_{n-1,i}(x^n_{n-1})\)
  の形で一意的に決定される。
  すなわち、
  射影\(H_n\to F_{n-1}\)を並べて得られる射影
  \(\prod_{n\in \N}H_n\to \prod_{n\in \N_{\geq 1}}F_{n-1} =
  \prod_{n\in \N}F_n\)
  を\(\lim_n H_n \subset \prod_{n\in \N}H_n\)へ制限すると同型射となる。
  従って\(\Gamma(\N,H) \cong \prod_{n\in \N}F_n\)となる。
  以上より、アーベル群の完全列
  \[
  \begin{CD}
    0 @>>> H^0(\N,F) @>>> \prod_{n\in \N} F_n @> \varphi >> \prod_{n\in \N} F_n \\
    @>>> H^1(\N,F) @>>> 0 @.
  \end{CD}
  \]
  を得る。
  問われていることは、\(\varphi\)の像を決定することである。
  各\(n\)について、\(\varphi_n:G_n\to H_n\)の像の
  \(F_{n-1}\)の成分を見ればそれは決定できる。
  \((x_n)_{n\in \N}\in \Gamma(\N,G) \cong \prod_{n\in \N}F_n\)を任意にとると、
  \(\varphi_n((x_i)_{i\leq n}) = (x_i-f_{n,i}(x_n))_{i<n}\)
  であるから、
  \(F_{n-1}\)の成分は\(x_{n-1}-f_{n,n-1}(x_n)\)である。
  従って、
  \[
  \varphi((x_n)_{n\in \N}) = (x_n - f_{n+1,n}(x_{n+1}))_{n\in \N}
  \]
  となる。
  従って\(\im(\varphi) = I\)がわかる。
  よって\(H^1(\N,F) \cong (\prod_{n\in \N}F_n)/I\)が示された。
  以上で解答を完了する。
\end{proof}












\begin{prob}\label{2.2}
  \(X\)を位相空間、\(A,B\subset X\)を閉集合とし、
  \(X=A\cup B\)であるとする。
  \(F\in \Ob(\mcD^+(X))\)に対して、
  自然に\((R\Gamma_B(F))_A \cong R\Gamma_B(F_A)\)となることを示せ。
\end{prob}

\begin{proof}
  函手\((-)_{X\setminus A}\)は完全なので、自然に
  \(R\Gamma_{X\setminus B}(-)_{X\setminus A}\cong
  R(\Gamma_{X\setminus B}(-)_{X\setminus A})\)
  が成り立つ。
  \((X\setminus A)\cap (X\setminus B) = \emptyset\)なので、
  \(\Gamma_{X\setminus B}(-)_{X\setminus A}=0\)が成り立ち、
  とくに\(R\Gamma_{X\setminus B}(-)_{X\setminus A}=0\)が成り立つ。
  \(F\)を\(X\)上の層の上に有界な複体とする。
  完全三角\(R\Gamma_B(F) \to F\to R\Gamma_{X\setminus B}(F)\xrightarrow{+1}\)
  に\((-)_{X\setminus A}\)を施すことにより、
  \(R\Gamma_B(F)_{X\setminus A}\xrightarrow{\sim}F_{X\setminus A}\)
  が従う。
  本文命題2.4.10の直前の記述にあるとおり、
  脆弱層のなす\(X\)の部分圏は
  \(\Gamma_{X\setminus B}(-)\)-injective
  である。
  また本文命題2.4.6 (i) より、
  脆弱層に対して\((-)|_{X\setminus B}\)を施したものも脆弱層である。
  よって\(i_B:X\setminus B\to X\)を包含射とすると、
  \(R\Gamma_{X\setminus B} \cong Ri_{B,*}\circ i_B^{-1}\)が成り立つ。
  \(i_B^{-1}((-)_{X\setminus A}) = 0\)であるから、
  任意の層に対して函手\((-)_{X\setminus A}\)を施したものは
  \(i_{B,*}\)に対してacyclicであり、
  よって自然に
  \(R\Gamma_{X\setminus B}((-)_{X\setminus A}) \cong
  R(\Gamma_{X\setminus B}((-)_{X\setminus A}))\)
  が成り立つ。
  \((X\setminus A)\cap (X\setminus B) = \emptyset\)なので、
  \(\Gamma_{X\setminus B}((-)_{X\setminus A}) = 0\)が成り立ち、
  とくに\(R\Gamma_{X\setminus B}((-)_{X\setminus A}) = 0\)が成り立つ。
  三角形
  \(R\Gamma_B(F_{X\setminus A})\to F_{X\setminus A}\to
  R\Gamma_{X\setminus B}(F_{X\setminus A})\xrightarrow{+1}\)
  が完全であることから、
  \(R\Gamma_B(F_{X\setminus A})\xrightarrow{\sim} F_{X\setminus A}\)
  は同型である。
  また、二つの図式
  \[
  \begin{CD}
    R\Gamma_B(F_{X\setminus A}) @>\sim>> F_{X\setminus A} @. \ \ \ \ \  @.
    R\Gamma_B(F)_{X\setminus A} @>\sim>> F_{X\setminus A} \\
    @VVV @VVV @. @VVV @VVV \\
    R\Gamma_B(F) @>>> F @. \ \ \ \ \ \ \ \ \ \ \ \ \ \ \ @. R\Gamma_B(F) @>>> F
  \end{CD}
  \]
  が可換であることから、
  \[
  \begin{CD}
    R\Gamma_B(F_{X\setminus A}) @>\sim>> R\Gamma_B(F)_{X\setminus A} \\
    @VVV @VVV  \\
    R\Gamma_B(F) @= R\Gamma_B(F)
  \end{CD}
  \]
  も可換である
  (二つの同型射を逆に辿って得られる二つの射
  \(F_{X\setminus A}\to R\Gamma_B(F)\)の差が\(0\)射である)。
  従って完全三角の間の同型射
  \[
  \begin{CD}
    R\Gamma_B(F_{X\setminus A}) @>>> R\Gamma_B(F) @>>> R\Gamma_B(F_A) @> +1 >> \\
    @V \cong VV @| @VV \cong V @. \\
    R\Gamma_B(F)_{X\setminus A} @>>> R\Gamma_B(F) @>>> R\Gamma_B(F)_A @> +1 >>
  \end{CD}
  \]
  を得る。
  以上で\autoref{2.2}の証明を完了する。
\end{proof}













\begin{prob}\label{2.3}
  \begin{enumerate}
    \item \label{2.3.1}
    \(U\subset X\)を開部分集合、
    \(x\in\bar{U}\setminus U\)として、
    層\(\Z_U\)について考えることによって
    \(\inHom(F,G)_x\cong \Hom(F_x,G_x)\)は一般には正しくないことを示せ。
    \item \label{2.3.2}
    次を満たす\(X\)上の層\(F\)と閉部分集合\(Z\subset X\)と開部分集合\(U\)の例を与えよ:
    \(Z\cap U=\emptyset\)であり、
    \(R\Gamma_Z(F_U)\neq 0\)である。
    \(\Gamma_Z(F_U)\)であることを確認し、
    このことから、一般に合成函手の導来函手が
    導来函手の合成とは異なることを帰結せよ。
  \end{enumerate}
\end{prob}

\begin{proof}
  \ref{2.3.1}を示す。
  \(F=\Z_U\)とおき、\(G\)は任意の層とする。
  \(x\not\in U\)なので\(F_x=0\)であり、
  従って、このとき、\(\Hom(F_x,G_x)=0\)が成り立つ。
  また、各開集合\(V\subset X\)に対して自然に
  \[
  \inHom(F,G)(V) = \inHom(\Z_U,G)(V) = \Hom(\Z_U|_V,G|_V)
  = \Hom(\Z_{U\cap V},G|_V) \cong G(U\cap V)
  \]
  が成り立つので、
  \(\inHom(F,G)\cong \Gamma_U(G)\)が成り立つ。
  従って、たとえば\(X=\R,U=\R\setminus \{0\},x=0,G=\Z\)とすると、
  \[
  \inHom(\Z_U,\Z)_x \cong \Gamma_U(G)_x
  \cong \Z\oplus\Z \neq 0 = \Hom(F_x,G_x)
  \]
  である。
  以上で\ref{2.3.1}の証明を完了する。

  \ref{2.3.2}を示す。
  まず一般に\(Z\cap U=\emptyset\)であれば、
  \(F_U\)の各切断は\(U\)の中に台を持つので
  \(\Gamma_Z(F_U)=0\)が成り立つ。
  \(X=\R_{\geq 0}, U=\R_{>0}\subset X, Z=\{0\}\)
  として\(F=\Z_X\)を定数層とする。
  このとき、\(F\)は定数層なので、
  各開集合\(V\subset X\)に対して
  \(s\in F(V)\)で\(Z\)の外で\(0\)となるものは\(0\)しかない
  (\(0\in V\)であり、\(s_0=n\neq 0\)であれば、
  \(0\)を含む\(V\)の連結成分の上で\(s=n\)である)。
  従って\(\Gamma_Z(F)=0\)である。
  完全列\(0\to F_U\to F\to F_Z\to 0\)に函手\(\Gamma_Z(-)\)を施すことにより、
  同型射
  \(\Z_x \cong \Gamma_Z(F_Z)\xrightarrow{\sim} H^1_Z(F_U)\)を得る。
  従ってこれが所望の例を与える。
  以上で\ref{2.3.2}の証明を完了し、
  \autoref{2.3}の解答を完了する。
\end{proof}














\begin{prob}\label{2.4}
  可縮な位相空間の上の局所定数層は定数層であることを示せ。
\end{prob}

\begin{proof}
  \(X\)を可縮な位相空間、\(F_1\)を\(X\)上の局所定数層とする。
  \(C(X)\)を\(X\)の錐とし、
  \(i:X\to C(X)\)を包含射とする。
  \(X\)は可縮なので、ある\(r:C(X)\to X\)が存在して、
  \(r\circ i = \id_X\)となる。
  従って\(i^{-1}(r^{-1}F_1) \cong F_1\)となる。
  よって\(r^{-1}F_1\)が定数層であることを証明すれば良い。
  \(v\in C(X)\)を錐の頂点とする。
  \(p:F\to C(X)\)を層\(r^{-1}F_1\)のエタール空間
  (cf. \cite[Exercise 1.13]{Ha}) とする。
  \(r^{-1}F_1\)は局所定数層であるから、\(p\)は\(C(X)\)の被覆空間である。
  \(r^{-1}F_1\)が定数層であることを証明するためには、
  \(p\)が自明な被覆空間であることを証明すれば良い。

  \(F\)の連結成分のなす集合を\(F_c\)とおき、
  \(F_c\)には離散位相を入れる。
  また、\(F_v\dfn p^{-1}(v)\)とおく。
  このとき、\(F_v\)の各点に対して、
  その点の属する連結成分を対応させることにより、
  写像\(F_v\to F_c\)を得る。
  各元\([Y]\in F_c\)に対して\(Y\subset F\)で対応する連結成分を表すとする。
  元\([Y]\in F_c\)と点\(y\in Y\)に対して、
  \(p(y)\)と\(v\)を結ぶ直線は\(F\)内のpathへと一意的にリフトする、
  すなわち、\(C(X)\)内の直線
  \(l:[0,1]\to C(X), l(0)=v,l(1)=p(y)\)に対して、
  あるpath \(l':[0,1]\to F\)が一意的に存在して、
  \(p\circ l' = l\)となる。
  \(l'\)の像は連結であり、\(y\in Y\)であり、\(Y\)は連結成分であるから、
  \(l'\)は\(Y\)を一意的に経由する
  (これは\([0,1]\)上の被覆空間が自明なものに限ることから従う)。
  従って、このことから、写像\(F_v\to F_c\)は全単射であり、
  さらに各元\([Y]\in F_c\)に対して
  合成射\(Y\subset F\to C(X)\)は全単射であることが従う。

  \(F\to F_c\)を各点に対してその点の属する連結成分を対応させる (連続) 写像とすると、
  この (連続) 写像と\(p:F\to C(X)\)によって、
  \(C(X)\)上の連続写像\(F\to F_c\times C(X)\)を得る。
  \(F_v\to F_c\)が全単射であることと、
  各元\([Y]\in F_c\)に対して
  合成射\(Y\subset F\to C(X)\)は全単射であることから、
  \(F\to F_c\times C(X)\)は\(C(X)\)上の被覆空間の間の全単射であることが従う。
  とくに同相写像である。
  よって\(p:F\to C(X)\)は自明な被覆空間である。
  以上で\autoref{2.4}の解答を完了する。
\end{proof}











\begin{prob}\label{2.5}
  \(X\)をパラコンパクトハウスドルフ空間とする。
  \(X\)上の層\(F\)が\textbf{soft}であるとは、
  任意の閉集合\(Z\subset X\)に対して
  \(F(X)\to F(Z)\)が全射であることを言う。
  \(F\)がsoftであるとき、
  任意の\(i > 0\)に対して\(H^i(X,F)=0\)であることを示せ。
\end{prob}

\begin{proof}
  \autoref{2.5}を示すには、
  softな層たちからなる\(\Ab(X)\)の充満部分圏が\(\Gamma(X,-)\)-injective
  であることを示すこと、
  従って、次の事柄を示すことが十分である
  (cf. 本文定義1.8.2から命題1.8.3までの記述):
  \begin{enumerate}
    \item \label{2.5.p1}
    脆弱層はsoftである
    (従って、とくに、任意の層\(F\)に対して、
    \(F\)を部分層として含むsoftな層\(G\)が存在する)。
    \item \label{2.5.p2}
    \(0\to F\to G\to H\to 0\)が層の完全列であるとき、
    \(F,G\)がsoftであるとすると、\(H\)もsoftである。
    \item \label{2.5.p3}
    \(0\to F\to G\to H\to 0\)が層の完全列であるとき、
    \(F\)がsoftであれば、
    次の列も完全である:
    \[
    0\to \Gamma(X,F) \to \Gamma(X,G) \to \Gamma(X,H)\to 0.
    \]
  \end{enumerate}
  \ref{2.5.p1}は本文命題2.5.1 (iii)よりただちに従う。
  \ref{2.5.p2}は、
  閉部分集合の上への制限をする函手が完全であること、
  softな層の閉部分集合への制限がsoftであること、
  \ref{2.5.p3}、
  へびの補題、より従う。
  残っているのは\ref{2.5.p3}を示すことである。

  \ref{2.5.p3}を示す。
  \(u\in \Gamma(X,H)\)を任意にとる。
  もとの層の列が完全であることから、\(u\)は局所的には\(G\)へと持ち上がる、
  すなわち、ある\(X\)の開被覆\(X=\bigcup_{i\in I}U_i\)と
  各\(U_i\)上の\(G\)の切断\(t_i^1\in \Gamma(U_i,G)\)が存在して、
  \(t_i\mapsto u|_{U_i}\)となる。
  \(U_i\)を局所有限開被覆による細分でおきかえて、
  \(t_i\)を制限することを考えれば、
  \ref{2.5.p3}を示すためには、
  \(U_i\)は局所有限であると仮定しても一般性を失わない。
  本文命題2.5.1の主張が終わるところからその証明が始まる前までの記述にあるとおり、
  \(U_i\)の開被覆による細分\((V_i)_{i\in I}\)であって、
  任意の\(i\in I\)に対して\(\bar{V}_i\subset U_i\)となるものが存在する。
  このとき、\((\bar{V}_i)_{i\in I}\)も局所有限である。
  \(i\in I\)に対して\(Z_i\dfn \bar{V}_i\)とおく。
  すると、\((Z_i)_{i\in I}\)が局所有限であることから、
  任意の\(J\subset I\)に対して
  \[Z_J\dfn \bigcup_{j\in J}Z_j = \overline{\bigcup_{j\in J}V_j} \subset X\]
  である (とくに\(Z_J\)は閉である)。
  \[
  \mcS \dfn \left\{ (J,t) \middle| J\subset I, t\in \Gamma(Z_J,G), \text{s.t.},
  t|_{Z_J}\mapsto u|_{Z_J}\right\}
  \]
  と定義して、
  \[
  (J_1,t_1)\leq (J_2,t_2) \ \ \deff \ \
  J_1\subset J_2 \text{かつ} t_1 = t_2|_{Z_{J_1}}
  \]
  と定義する。
  \(\mcT\subset \mcS\)を全順序部分集合とする。
  \(J_{\mcT} \dfn \bigcup_{(J,t_J)\in \mcT}J\)とおく。
  このとき、\(Z_{J_{\mcT}} = \bigcup_{(J,t_J)\in \mcT}Z_J\)であるから、
  各\((J,t_J)\in \mcT\)に対して層の全射
  \(G_{Z_{J_{\mcT}}} \to G_{Z_J}\)を得る。
  この全射は\(Z_J\)上の各点のstalkの間で同型射であるから、
  自然な射
  \(G_{Z_{J_{\mcT}}} \xrightarrow{\sim} \lim_{J\in \mcT}G_{Z_J}\)
  は同型射となる。
  大域切断をとることにより、
  同型射\(\Gamma(Z_{J_{\mcT}},G)\xrightarrow{\sim} \lim_{J\in \mcT}\Gamma(Z_J,G)\)
  を得る。
  従って、各切断\(t_J\in \Gamma(Z_J,G)\)は切断\(t\in \Gamma(Z_{J_{\mcT}},G)\)を定め、
  \(\mcT\)は上界\((J_{\mcT},t)\)を持つ。
  よって、Zornの補題により、\(\mcS\)には極大元\((J,t)\)が存在する。
  \(J=I\)であることを証明できれば、
  \(Z_J=X\)であるから、\ref{2.5.p3}の証明が完了する。
  よって、\ref{2.5.p3}が成り立つためには、\(J=I\)であることが十分である。
  元\(i\in I\setminus J\)が存在することを仮定する。
  \(t_i|_{Z_i\cap Z_J} - t|_{Z_i\cap Z_J}\mapsto 0\)であるから、
  \(t_i|_{Z_i\cap Z_J} - t|_{Z_i\cap Z_J}\in F(Z_i\cap Z_J)\)である。
  \(F\)はsoftであり、\(Z_i\cap Z_J\subset X\)は閉であるから、
  ある\(s\in \Gamma(X,F)\)が存在して、
  \(s|_{Z_i\cap Z_J} = t_i|_{Z_i\cap Z_J} - t|_{Z_i\cap Z_J}\)となる。
  \(t_i'\dfn t_i-s|_{U_i}\)と定義すると、
  \(s\)の定義より、
  \(t_i'|_{Z_i\cap Z_J} = t|_{Z_i\cap Z_J}\)が成り立つ。
  従って、本文命題2.3.6 (vi) より、
  \(J'\dfn J\cup\{i\}\)とおけば、
  ある\(t'\in G(Z_{J'})\)が存在して、\(t'|_{Z_i} = t_i'\)かつ
  \(t'|_{Z_J} = t\)となる。
  よって、再び本文命題2.3.6 (vi) より、
  \(t'\mapsto u|_{Z_{J'}}\)である。
  これは\((J,t) < (J',t')\)を意味し、\((J,t)\)の極大性に反する。
  以上で\(I=J\)が従い、
  \ref{2.5.p3}の証明を、
  従って、\autoref{2.5}の解答を、完了する。
\end{proof}




\begin{prob}\label{2.6}
  \(X\)を局所コンパクトハウスドルフ空間、
  \(F\)を\(X\)上の層とする。
  \begin{enumerate}
    \item \label{2.6.1}
    \(F\)が\(c\)-softであるための必要十分条件は
    任意の\(i>0\)と任意の開集合\(U\)に対して
    \(H^i_c(U,F) = 0\)となることである。
    ただしここで\(H^i_c(U,F)\dfn H^i_c(U,F|_U)\)である。
    \item \label{2.6.2}
    \(X\)が可算個のコンパクト部分集合の和集合として表すことができるとき、
    \(F\)が\(c\)-softであればsoftであることを示せ。
    \item \label{2.6.3}
    \(c\)-softであるという性質は局所的な性質であることを示せ。
  \end{enumerate}
\end{prob}

\begin{proof}
  \ref{2.6.1}を示す。
  必要性を示す。
  \(F\)を\(c\)-softであるとする。
  本文命題2.5.7 (i)より\(F|_U\)は\(c\)-softである。
  すると\(U\)上の\(c\)-softな層たちは
  \(\Gamma_c(U,-)\)-injectiveな圏であるので、
  とくに\(c\)-softな層は\(\Gamma_c(U,-)\)-acyclicである。
  よって、とくに、任意の\(i>0\)に対して\(H^i_c(U,F|_U)=0\)である。
  以上で必要性の証明を完了する。

  十分性を示す。
  任意の\(i>0\)と任意の開集合\(U\subset X\)に対して
  \(H^i_c(U,F)=0\)と仮定する。
  \(K\subset X\)をコンパクト部分集合とすれば、
  \(X\)はハウスドルフなので、\(K\subset X\)は閉である。
  従って\(U\dfn X\setminus K\)は開である。
  よって\(H^1_c(U,F)=0\)である。
  すると、本文 Remark 2.6.10 の最後の完全系列より、
  射\(F(X)\to F(K)\)は全射である。
  以上で\ref{2.6.1}の証明を完了する。

  \ref{2.6.2}を示す。
  \(F\)を\(X\)上の\(c\)-softな層であり、
  \(Z\subset X\)を閉集合とする。
  \(K_n\subset X, (n\in \N)\)をコンパクト部分集合の族で、
  \(K_n\subset \mathrm{Int}(K_{n+1})\)と
  \(X = \bigcup_{n\in \N}K_n\)を満たすものとする。
  \(Z_n\dfn Z\cap K_n\)とおくと、\(Z_n\)はコンパクトである。
  \(U_n\dfn K_n\setminus Z_n\)とおくと、\(U_n\subset X\)は局所閉集合である。
  \(F\)は\(c\)-softなので、本文命題2.5.7 (i)より
  \(F|_{K_n}\)は\(K_n\)上の\(c\)-softな層であり、
  \(F|_{U_n}\)は\(U_n\)上の\(c\)-softな層である。
  \(K_n,Z_{n+1}\subset K_{n+1}\)はいずれもコンパクト部分集合であり、
  \(F|_{K_{n+1}}\)は\(c\)-softなので、
  可換図式
  \[
  \begin{CD}
    F(K_{n+1}) @>>> F(Z_{n+1}) \\
    @VVV @VVV \\
    F(K_n) @>>> F(Z_n)
  \end{CD}
  \]
  において、いずれの射も全射である。
  \(K_n,Z_n\)はコンパクトであるから、
  \(F(K_n) = \Gamma_c(K_n,F), F(Z_n) = \Gamma_c(Z_n,F)\)であることに注意する。
  \(K_n\)上の層の完全列
  \[
  \begin{CD}
    0 @>>> (F|_{K_n})_{U_n} @>>> F|_{K_n} @>>> (F|_{K_n})_{Z_n} @>>> 0
  \end{CD}
  \]
  に函手\(\Gamma_c(K_n,-)\)を施すことにより、
  \[
  \begin{CD}
    0 @>>> \Gamma_c(U_n,F) @>>> F(K_n) @>>> F(Z_n) @>>> 0
  \end{CD}
  \]
  は完全であることが従うので、
  以上より、完全列の間の射
  \[
  \begin{CD}
    0 @>>> \Gamma_c(U_{n+1},F) @>>> F(K_{n+1}) @>>> F(Z_{n+1}) @>>> 0 \\
    @. @VVV @VVV @VVV @. \\
    0 @>>> \Gamma_c(U_n,F) @>>> F(K_n) @>>> F(Z_n) @>>> 0
  \end{CD}
  \]
  を得る。
  ただしここで、真ん中と右側の縦向きの射は全射である。
  極限\(\lim_{n\in \N}\)をとると、完全列
  \[
  \begin{CD}
    0 @>>> \lim_{n\in \N}\Gamma_c(U_n,F) @>>>
    \lim_{n\in \N}F(K_n) @>>> \lim_{n\in \N}F(Z_n)
  \end{CD}
  \]
  を得る。
  \(F(X) \xrightarrow{\sim} \lim_{n\in \N} F(K_n),
  F(Z) \xrightarrow{\sim} \lim_{n\in \N} F(Z_n)\)であるから、
  この完全列は完全列
  \[
  \begin{CD}
    0 @>>> \lim_{n\in \N}\Gamma_c(U_n,F) @>>> F(X) @>>> F(Z)
  \end{CD}
  \]
  と自然に同型である。
  従って、\(F(X)\to F(Z)\)が全射であるためには、
  \((\Gamma_c(U_n,F))_{n\in \N}\)が Mittag-Leffler 条件を満たすことが十分である。
  \(U_n\subset U_{n+1}\)は閉なので、
  \(U_{n+1}\)上の層の列
  \[
  \begin{CD}
    0 @>>> (F|_{U_{n+1}})_{U_{n+1}\setminus U_n}
    @>>> F|_{U_{n+1}} @>>> (F|_{U_{n+1}})_{U_n} @>>> 0
  \end{CD}
  \]
  は (本文命題2.6.6(v)より) 完全である。
  \(F|_{U_{n+1}}\)は\(c\)-softであるので、
  この完全列に函手\(\Gamma_c(U_{n+1},-)\)を施すことにより、
  \(\Gamma_c(U_{n+1},F)\to \Gamma_c(U_n,F)\)は全射であることが従う。
  よって、逆系\((\Gamma_c(U_n,F))_{n\in \N}\)は Mittag-Leffler 条件を満たす。
  従って、射\(F(X)\to F(Z)\)は全射であり、
  以上で\ref{2.6.2}の証明を完了する。

  \ref{2.6.3}を示す。
  \(X\)の開被覆\(X = \bigcup_{i\in I}U_i\)が存在して、
  各\(i\)に対して\(F|_{U_i}\)が\(U_i\)上の\(c\)-softな層であると仮定する。
  \(j:U_i\to X\)を包含射とする。
  本文命題2.5.4 (ii)より、
  \(F_{U_i} = j_!(F|_{U_i})\)であるから、
  \(F|_{U_i}\)が\(c\)-softであることから、
  \(F_{U_i}\)も\(c\)-softであることが従う。
  本文命題2.3.6 (vii) の完全列
  \[
  \begin{CD}
    0 @>>> F_{U_{i_1}\cap U_{i_2}} @>>> F_{U_{i_1}} \oplus F_{U_{i_2}}
    @>>> F_{U_{i_1}\cup U_{i_2}} @>>> 0
  \end{CD}
  \]
  において、左側と真ん中が\(c\)-softなので、
  本文系2.5.9より\(F_{U_{i_1}\cup U_{i_2}}\)も\(c\)-softである。
  従って、有限部分集合\(I_1\subset I\)に対して
  \(U_{I_1} \dfn \bigcup_{i\in I_1} U_i\)とおけば、
  \(F_{U_{I_1}}\)は\(c\)-softである。
  任意にコンパクト集合\(K\subset X\)をとれば、
  ある有限部分集合\(I_1\subset I\)が存在して
  \(K\subset U_{I_1}\)となる。
  \(F|_{U_{I_1}} = (F_{U_{I_1}})|_{U_{I_1}}\)は\(c\)-softであるから、
  \(\Gamma_c(U_{I_1},F|_{U_{I_1}}) = \Gamma_c(X,F_{U_{I_1}}) \to \Gamma_c(K,F)\)
  は全射である。
  また、\(F_{U_{I_1}}\)は\(F\)の部分層であるから、
  従って、\(\Gamma_c(X,F) \to \Gamma_c(K,F)\)も全射である。
  さらに\(\Gamma_c(X,F) \subset F(X)\)であり、
  \(\Gamma_c(K,F) = F(K)\)であるので、
  よって\(F(X) \to F(K)\)は全射である。
  以上で\ref{2.6.3}の証明を完了し、
  \autoref{2.6}の解答を完了する。
\end{proof}






\begin{prob}\label{2.7}
  \(X\)を局所コンパクトハウスドルフ空間として、
  \(R\)を\(X\)上の\(c\)-softな環の層とする。
  このとき、任意の\(R\)-加群は\(c\)-softであることを示せ。
\end{prob}

\begin{rem*}
  本文では\(X\)に関する仮定が何も書かれていないが、
  \(c\)-softな層に関する話は局所コンパクトハウスドルフ空間上で展開することが、
  本文では念頭に置かれているように思う。
  (もちろん、\(X\)が局所コンパクトでなくてもこの問題を解くことが可能かもしれないが...)
\end{rem*}

\begin{proof}
  \(M\)を\(R\)-加群とする。
  \(X\)は局所コンパクト空間であるから、
  閉包がコンパクトであるような開集合たちの和集合である。
  従って、\autoref{2.6} \ref{2.6.3}より、
  \(M\)が\(c\)-softであることを示すためには、
  \(X\)をコンパクトハウスドルフ空間であると仮定しても一般性を失わない。
  このとき、\(c\)-softであるという性質とsoftであるという性質は
  同等であることに注意しておく。
  とくに、\(R\)はsoftである。

  コンパクト部分集合\(K\subset X\)と
  切断\(m_K\in \Gamma(K,M)\)を任意にとる。
  本文命題2.5.1 (ii) より、
  ある開集合\(K\subset U\subset X\)と
  ある切断\(m_U\in \Gamma(U,M)\)が存在して、
  \(m_K=m_U|_K\)となる。
  \(K\subset V \subset \bar{V}\subset U\)となる開集合\(V\)を一つとる
  (\(X\)は局所コンパクトであり、
  \(K\)はコンパクトであるから、このような\(V\)が存在する)。
  \(R\)はsoftであり、
  \(K\cup (X\setminus V)\subset X\)は閉であるため、
  ある\(f\in \Gamma(X,R)\)が存在して
  \[
  f|_{K\cup (X\setminus V)} = (1|_K,0_{X\setminus V})
  \in \Gamma(K,R) \times \Gamma(X\setminus V,R)
  \cong \Gamma(K\cup (X\setminus V), R)
  \]
  となる。
  \(W\dfn X\setminus \bar{V}\)とおけば、
  \(\bar{V}\subset U\)なので\(W\cup U = X\)である。
  さらに\(U\cap W\subset X\setminus V\)であるから、
  \(f|_{U\cap W} = 0\)であり、
  とくに\(f|_{U\cap W} \times m_U|_{U\cap W} = 0\)である。
  従って、\(M\)は層であるから、
  \(W\)上での切断\(0\in \Gamma(W,M)\)を考えることにより、
  ある\(m\in \Gamma(X,M)\)が存在して\(f|_U\times m_U = m|_U, m|_W=0\)となる。
  \(f|_K=1\)なので、よって\(m|_K = f|_K\times m_U|_K = m_U|_K = m_K\)が従う。
  以上より\(\Gamma(X,M)\to \Gamma(K,M)\)は全射であり、
  \autoref{2.7}の証明を完了する。
\end{proof}







\begin{prob}\label{2.8}
  \(X\)を局所コンパクトハウスドルフ空間として、
  可算個のコンパクト部分集合の和集合であるとする。
  \(X\)上の層\(F\)が\textbf{しなやか} (supple) であるとは、
  任意の開集合\(U\subset X\)と
  \(U\)の二つの閉部分集合\(Z_1,Z_2\subset U\)に対して
  \(\Gamma_{Z_1}(U,F) \oplus \Gamma_{Z_2}(U,F) \to \Gamma_{Z_1\cup Z_2}(U,F)\)
  が全射であることを言う。
  \begin{enumerate}
    \item \label{2.8.1}
    脆弱層はしなやかであることを示せ。
    \item \label{2.8.2}
    層\(F\)がしなやかであれば、任意の閉部分集合\(Z\subset X\)に対して
    層\(\Gamma_Z(F)\)は\(c\)-softであることを示せ。
    \item \label{2.8.3}
    \(F\)を\(X\)上の層とする。
    \(X\)のある開被覆\(X=\bigcup_{i\in I}U_i\)が存在して、
    任意の\(i\)で\(F|_{U_i}\)が\(U_i\)上のしなやかな層であるとするとき、
    \(F\)もしなやかであることを示せ。
  \end{enumerate}
\end{prob}

\begin{proof}
  \ref{2.8.1}を示す。
  \(F\)を\(X\)上の脆弱層とする。
  開集合\(U\)と閉集合\(Z_1,Z_2\subset U\)を任意にとる。
  \(V_i\dfn U\setminus Z_i, (i=1,2)\)とすると、
  \(V_1,V_2\subset X\)は開である。
  \(F\)は脆弱であるから、
  可換図式
  \[
  \begin{CD}
    0 @>>> \Gamma_{Z_1\cap Z_2}(U,F) @>>>
    \Gamma_{Z_1}(U,F) \oplus \Gamma_{Z_2}(U,F) @>>> \Gamma_{Z_1\cup Z_2}(U,F) @. \\
    @. @VVV @VVV @VVV @. \\
    0 @>>> F(U) @>>> F(U) \oplus F(U) @>>> F(U) @>>> 0 \\
    @. @VVV @VVV @VVV @. \\
    0 @>>> F(V_1\cup V_2) @>>> F(V_1)\oplus F(V_2) @>>> F(V_1\cap V_2) @.
  \end{CD}
  \]
  において真ん中から下への縦向きの射はいずれも全射であり、
  さらに一番下の行は層であることの定義より完全である。
  蛇の補題を用いることで、
  \(\Gamma_{Z_1}(U,F) \oplus \Gamma_{Z_2}(U,F) \to \Gamma_{Z_1\cup Z_2}(U,F)\)
  が全射であることが従う。
  以上で\ref{2.8.1}の証明を完了する。

  \ref{2.8.2}を示す。
  \(F\)を\(X\)上のしなやかな層とする。
  まず、任意の閉部分集合\(Z\)に対して
  \(\Gamma_Z(F)\)が\(X\)上のしなやかな層であることを示す。
  \(U\subset X\)を開集合、
  \(Z'\subset U\)を\(U\)の閉部分集合とする。
  \(F\)の\(U\)上のsectionであって\(Z\)に台を持ち、
  さらに\(Z'\)にも台を持つものは\(Z\cap Z'\)に台を持つので、
  \[
  \Gamma_{Z'}(U,\Gamma_Z(F)) = \Gamma_{Z'\cap Z}(U,F)
  \]
  が成り立つ。
  二つの閉部分集合\(Z_1,Z_2\subset U\)に対して、
  \(Z'_1\dfn Z_1\cap Z, Z'_2\dfn Z_2\cap Z\)とおけば、
  \(F\)がしなやかであることから、
  \[
  \Gamma_{Z'_1}(U,F)\oplus \Gamma_{Z'_2}(U,F) \to \Gamma_{Z'_1\cup Z'_2}(U,F)
  \]
  は全射である。
  \(Z'_1\cup Z'_2 = (Z_1\cup Z_2)\cap Z\)であるので、従って
  \[
  \Gamma_{Z_1}(U,\Gamma_Z(F))\oplus \Gamma_{Z_2}(U,\Gamma_Z(F)) \to
  \Gamma_{Z_1\cup Z_2}(U,\Gamma_Z(F))
  \]
  も全射である。
  これは\(\Gamma_Z(F)\)がしなやかであることを意味する。
  以上より、\ref{2.8.2}を示すためには、
  任意のしなやかな層が\(c\)-softであることを示すことが十分である。

  \(F\)を\(X\)上のしなやかな層として、
  \(K\subset X\)をコンパクト部分集合とする。
  \(t\in F(K)\)を一つとる。
  本文命題2.5.1 (ii)より、
  ある開集合\(K\subset U\subset X\)と\(t_U\in F(U)\)が存在して、
  \(t_U|_K = t\)が成り立つ。
  \(K\subset V, \bar{V}\subset U\)となる開集合\(V\subset X\)を一つとる。
  \(Z\dfn \Supp(t_U)\setminus V\)とおくと、
  \(t_U\in \Gamma_{Z\cup \bar{V}}(U,F)\)である。
  \(F\)はしなやかであり、\(\bar{V},Z\subset U\)は閉集合であるから、
  ある\(u\in \Gamma_Z(U,F), v\in \Gamma_{\bar{V}}(U,F)\)が存在して
  \(t_U = u+v\)が成り立つ。
  \(Z\cap K \subset Z\cap V = \emptyset\)であるから、
  \(u|_K = 0\)である。
  従って\(t = t_U|_K = v|_K\)が成り立つ。
  \(\bar{V}\subset U\)であるから、
  \(\Gamma_{\bar{V}}(X,F) \xrightarrow{\sim} \Gamma_{\bar{V}}(U,F)\)
  は同型射である。
  従って\(v\in \Gamma_{\bar{V}}(X,F)\subset F(X)\)とみなすことができる。
  よって\(v|_K = t\)となる大域切断\(v\in F(X)\)が存在することが従い、
  \(F\)は\(c\)-softであることが従う。
  以上で\ref{2.8.2}の証明を完了する。

  \ref{2.8.3}を示す。
  まず以下の主張を証明する:
  \begin{enumerate}[label=(\fnsymbol*),start=2]
    \item \label{2.8.3.p1}
    \(X\)を位相空間、
    \(Z\subset X\)を閉部分集合、
    \(U\subset X\)を開集合、
    \(F\)を\(X\)上の層として、\(F|_U\)がしなやかであると仮定する。
    開集合\(V\)と閉集合\(C\)が\(V\subset C\subset U\)を満たしているとする。
    このとき、任意の切断\(t\in \Gamma_Z(X,F)\)に対し、
    ある\(u\in \Gamma_{Z\cap C}(X,F)\)と
    \(v\in \Gamma_{Z\setminus V}(X,F)\)が存在して
    \(t=u+v\)が成り立つ。
  \end{enumerate}
  \(t|_U\in \Gamma_{Z\cap U}(U,F)\)について考える。
  \(F|_U\)はしなやかであり、
  \(Z\cap U = (Z\cap U\cap C) \cup ((Z\cap U)\setminus V)
  = (Z\cap C) \cup ((Z\cap U)\setminus V)\)であるから、
  ある\(u_1\in \Gamma_{Z\cap C}(U,F)\)と
  \(v_1\in \Gamma_{(Z\cap U)\setminus V}(U,F)\)が存在して
  \(t|_U = u_1+v_1\)が成り立つ。
  \(Z\cap C\)は\(X\)の閉部分集合であるような\(U\)の部分集合であり、
  \(u_1|_{U\setminus (Z\cap C)} = 0\)であるから、
  \(X\setminus (Z\cap C)\)上の関数\(0\)と\(u_1\)が貼り合い、
  \(u|_U = u_1, \Supp(u) \subset Z\cap C\)
  となる大域切断\(u\in \Gamma_{Z\cap C}(X,F)\)が一意的に定義される。
  \(v \dfn t-u\)とおけば、
  \(u_1+v_1 = t|_U = u|_U + v|_U = u_1 + v|_U\)
  であるから\(v_1 = v|_U\)が成り立ち、
  従って\(\Supp(v|_U)\subset (Z\cap U)\setminus V\)である。
  一方、\(\Supp(u)\subset Z\cap C\)であるから、
  \(v|_{X\setminus(Z\cap C)} = t|_{X\setminus (Z\cap C)}\)であり、従って
  \(\Supp(v|_{X\setminus(Z\cap C)}) \subset Z\cap (X\setminus (Z\cap C)) = Z\setminus C\)
  が成り立つ。
  以上より
  \[
  \Supp(v)\subset (Z\setminus C)\cup ((Z\cap U)\setminus V) = Z\setminus V
  \]
  が成り立ち、\(v\in \Gamma_{Z\setminus V}(X,F)\)が成り立つ。
  以上で\ref{2.8.3.p1}の証明を完了する。

  次に以下の主張を証明する:
  \begin{enumerate}[label=(\fnsymbol*),start=3]
    \item \label{2.8.3.p2}
    \(X\)を位相空間、
    \(F\)を\(X\)上の層として、
    \(t_i\in \Gamma(X,F), (i\in I)\)を大域切断の族とする。
    閉集合族\((\Supp(t_i))_{i\in I}\)が局所有限であるとき、
    ある大域切断\(t\in \Gamma(X,F)\)が存在して、
    任意の\(x\in X\)で\(t_x = \sum_{i\in I}t_{i,x}\)が成り立つ
    (各\(x\in X\)に対して
    stalk \(t_{i,x}\)は有限個の\(i\in I\)を除き\(0\)なので、
    右辺が well-defined であることに注意)。
  \end{enumerate}
  各\(x\in X\)に対して
  \(U(x) \cap \Supp(t_i) \neq \emptyset\)となる\(i\in I\)が有限個
  となる開近傍\(x\in U(x)\)を一つずつ選び、
  \(I(x) \dfn \left\{i\in I\middle| U(x)\cap \Supp(t_i) \neq \emptyset\right\}\)
  定める。
  定義より\(I(x)\)は有限集合である。
  \(t(x)\dfn (\sum_{i\in I(x)}t_i)|_{U(x)}\)と定義する
  (\(I(x)\)が有限集合であることに注意)。
  このとき、\(x,y\in X\)に対して
  \(t(x)|_{U(x)\cap U(y)} = t(y)|_{U(x)\cap U(y)}\)
  が成り立つ
  (各stalkごとに、\(U(x)\cap U(y)\)上で\(0\)でない\(t_i\)たちの和と等しい)。
  \(F\)が層であることから、
  ある\(t\in \Gamma(X,F)\)が存在して
  任意の\(x\in X\)に対して\(t|_{U(x)} = t(x)\)が成り立つ。
  \(t(x)\)の定義より、この大域切断\(t\)が所望の大域切断である。
  以上で\ref{2.8.3.p2}の証明を完了する。

  本題に入る。
  \((U_i)_{i\in I}\)を\(X\)の開被覆として、\(F\)を\(X\)上の層とする。
  \(F|_{U_i}\)が\(U_i\)上のしなやかな層であるとする。
  各\(i\in I\)に対して、\(F|_{U_i\cap U}\)は\(U_i\cap U\)上のしなやかな層であるから、
  \ref{2.8.3}を示すためには、
  任意の閉集合\(Z_1,Z_2\subset X\)に対して
  \(\Gamma_{Z_1}(X,F) \oplus \Gamma_{Z_2}(X,F)\to \Gamma_{Z_1\cup Z_2}(X,F)\)
  が全射であることを証明することが十分である。
  \(X\)は局所コンパクトであり可算個のコンパクト部分集合の和であるから、
  パラコンパクトである (cf. 本文命題2.5.1の直後の段落)。
  従って、局所有限な開被覆による\((U_i)_{i\in I}\)の細分をとることによって、
  \ref{2.8.3}を示すためには、
  \((U_i)_{i\in I}\)が局所有限であると仮定しても一般性を失わない。
  開集合\(V_i\subset U_i\)を\(\bar{V}_i\subset U_i\)となるようにとると、
  閉被覆\((\bar{V}_i)_{i\in I}\)も局所有限である。
  \(I\)に整列順序を入れて順序数とみなしたものを\(\alpha\)とおき (整列可能定理)、
  各\(\beta<\alpha\)に対して対応するものを\(U_{\beta},V_{\beta}\)などと表す。
  各\(\beta<\alpha\)に対して
  \[
  U_{<\beta} \dfn \bigcup_{\gamma<\beta}U_{\gamma}, \ \
  U_{\leq\beta} \dfn U_{<\beta+1}, \ \
  V_{<\beta} \dfn \bigcup_{\gamma<\beta}V_{\gamma}, \ \
  V_{\leq\beta} \dfn V_{<\beta+1},
  \]
  とおく
  (ただし\(U_{<0}=V_{<0}=\emptyset\)と定義する)。
  このとき、任意の\(\beta<\alpha\)に対して
  \(\bar{V}_{<\beta} = \bigcup_{\gamma<\beta}\bar{V}_{\gamma}\)が成り立ち、
  また\(V_{<\alpha} = U_{<\alpha} = X\)が成り立つ。

  \(t\in \Gamma_{Z_1\cup Z_2}(X,F)\)を任意にとる。
  各\(\beta \leq \alpha\)に対して、
  大域切断
  \(t_{i,<\beta}\in \Gamma_{Z_i\cap \bar{V}_{<\beta}}(X,F),(i=1,2)\)
  であって
  \begin{equation}
    \label{eq 2.8.3}
    t|_{V_{<\beta}}=(t_{1,<\beta}+t_{2,<\beta})|_{V_{<\beta}}
    \tag{\(\bigstar\)}
  \end{equation}
  が成り立つものが存在することを示す。
  そのためには、超限帰納法により、\(\beta \leq \alpha\)を任意にとり、
  任意の\(\gamma<\beta\)に対して
  に対して等式\eqref{eq 2.8.3}を満たす
  \(Z_i\cap \bar{V}_{<\gamma}\)に台を持つ大域切断\(t_{i,<\gamma}\)が存在すると仮定して、
  \(\beta\)に対して等式\eqref{eq 2.8.3}を満たす
  \(Z_i\cap \bar{V}_{<\gamma}\)に台を持つ大域切断\(t_{i,<\gamma}\)が存在する
  ことを示すことが十分である。
  \(\beta = 0\)に対しては\(t_{1,<0} = t_{2,<0} = 0\)と定義すれば
  \eqref{eq 2.8.3}が成り立つ。
  (そうでなくても、\(V_{<0}=\emptyset\)なのでなんでも良い)。
  \(\beta \leq \alpha\)を任意にとる。
  任意の\(\gamma<\beta\)に対して
  に対して等式\eqref{eq 2.8.3}を満たす
  \(Z_i\cap \bar{V}_{<\gamma}\)に台を持つ大域切断\(t_{i,<\gamma}\)が存在すると仮定する。
  \(u_{\gamma} = t-(t_{1,<\gamma}+t_{2,<\gamma})\)とおく。
  \(u_{\gamma}|_{V_{<\gamma}} = 0\)であり、
  \(t\)は\(Z_1\cup Z_2\)に台を持ち、
  \(t_{i,<\gamma}\)は\(Z_i\cap \bar{V}_{<\gamma}\)に台を持つので、
  \(u_{\gamma}\)は\((Z_1\cup Z_2)\setminus V_{<\gamma}\)に台を持つ。
  \(\beta\)が極限順序数である場合は、
  各\(t_{i,<\gamma+1}-t_{i,<\gamma}\)は
  \(\bar{V}_{\gamma+1}\setminus V_{<\gamma}\)に台を持ち、
  閉集合族\((\bar{V}_{\gamma+1}\setminus V_{<\gamma})_{\gamma<\beta}\)は局所有限であるから、
  \ref{2.8.3.p2}を用いて
  \(t_{i,<\beta} \dfn \sum_{\gamma<\beta}(t_{i,<\gamma+1}-t_{i,<\gamma})\)
  と定義することで所望の大域切断を得る。
  \(\beta\)が\(\beta^-\)の後続順序数である場合に
  所望の大域切断\(t_{i,<\beta}\)の存在を示すことが残っている。
  \ref{2.8.3.p1}を
  \(U=U_{\beta^-}, Z=(Z_1\cup Z_2)\setminus V_{<\beta^-},
  C=\bar{V}_{\beta^-}, V=V_{\beta^-}\)
  に対して適用することにより、
  \((Z_1\cup Z_2)\setminus V_{<\beta}\)に台を持つ大域切断\(u'_{\beta}\)と
  \((Z_1\cup Z_2)\cap \bar{V}_{\beta^-}\)に台を持つ大域切断\(t_{\beta^-}\)が存在して、
  \(u_{\beta^-} = u'_{\beta} + t_{\beta^-}\)
  が成り立つ。
  \(F|_{U_{\beta^-}}\)はしなやかであるから、\(i=1,2\)に対して
  \(Z_i\cap \bar{V}_{\beta^-}\)に台を持つ\(U_{\beta^-}\)上の大域切断\(t_{i,\beta^-}\)が存在して
  \(t_{\beta}|_{U_{\beta^-}} = (t_{1,\beta^-}+t_{2,\beta^-})|_{U_{\beta^-}}\)が成り立つ。
  \(Z_i\cap \bar{V}_{\beta^-}, (i=1,2)\)は\(X\)の閉部分集合であるから、
  \(t_{i,\beta^-}\)は\(U_{\beta^-}\)の外で\(0\)とすることで大域切断に延長できる。
  よって、\(i=1,2\)に対して
  \(Z_i\cap \bar{V}_{\beta^-}\)に台を持つ大域切断\(t_{i,\beta^-}\)が存在して
  \(t_{\beta^-}=t_{1,\beta^-}+t_{2,\beta^-}\)が成り立つ。
  \(t_{i,<\beta}\dfn t_{i,<\beta^-} + t_{i,\beta^-}, (i=1,2)\)と定義すると、
  大域切断\(t_{i,<\beta}\)は\(Z_i\cap \bar{V}_{<\beta}\)に台を持ち、さらに
  \begin{align*}
    t|_{V_{<\beta}}
    &= (t_{1,<\beta^-}+t_{2,<\beta^-} + u_{\beta^-})|_{V_{<\beta}} \\
    &= (t_{1,<\beta^-}+t_{2,<\beta^-} + u'_{\beta} + t_{\beta^-})|_{V_{<\beta}} \\
    &= (t_{1,<\beta^-}+t_{2,<\beta^-} + u'_{\beta}
    + t_{1,\beta^-}+t_{2,\beta^-})|_{V_{<\beta}} \\
    &= (t_{1,<\beta}+t_{2,<\beta} + u'_{\beta})|_{V_{<\beta}} \\
    &= (t_{1,<\beta}+t_{2,<\beta})|_{V_{<\beta}} + u'_{\beta}|_{V_{<\beta}} \\
    &= (t_{1,<\beta}+t_{2,<\beta})|_{V_{<\beta}}
  \end{align*}
  が成り立つ。
  以上より、各\(\beta\leq \alpha\)に対して、
  \(Z_i\cap \bar{V}_{<\beta}\)に台を持つ大域切断
  \(t_{i,<\beta}\)であって
  \(t|_{V_{<\beta}} = (t_{1,<\beta}+t_{2,<\beta})|_{V_{<\beta}}\)
  を満たすものが存在することが示された。
  \(\beta=\alpha\)とすることで\ref{2.8.3}の証明が完了する。
  以上で\autoref{2.8}の解答を完了する。
\end{proof}


\begin{kansou*}
  \ref{2.8.3}を上手に示す方法を知っている (もしくは、上手に証明できた) 人は教えてください
  (上の証明はゴリ押し感が強いので)。
  本文で参照されている Bengal-Schapira はフランス語だったし
  どこにそれっぽい主張が書いてあるのかあんまりよくわからなかったので
  あんまり参考にしてません。
  \ref{2.8.3}より\ref{2.8.2}の方が難しくて苦労しました。
  慣れてなかっただけかもしれません。
\end{kansou*}


\begin{rem*}
  \(X\)がパラコンパクトであり\(F\)が\(X\)上のしなやかな層であるとすると、
  \ref{2.8.2}と同様の証明により、
  任意の閉部分集合\(Z\)に対して\(\Gamma_Z(F)\)はsoftであることを示すことができる。
\end{rem*}


\begin{rem*}
  \(X\)をハウスドルフとして、\(F\)を\(X\)上のしなやかな層であるとする。
  \(K\subset X\)を閉部分集合として、
  \(K\)がコンパクトであるか、または\(X\)がパラコンパクトであるとする。
  このとき、\(F|_K\)は\(K\)上のしなやかな層である。
  これを示す。
  明らかに\(F\)の開部分集合への制限はその開部分集合上しなやかな層となる。
  \(K\)の開部分集合は\(X\)の開集合\(W\subset X\)により
  \(K\cap W\)と表すことができる。
  \(Z_1,Z_2\subset K\cap W\)を閉集合とする。
  このとき\(Z_1,Z_2\)は\(W\)の閉集合である。
  \(K\cap W\subset W' \subset W\)となる開集合\(W'\subset X\)の族を
  包含関係の逆順に関して有向集合とみなしてそれを\(I\)とし、
  \(i\in I\)に対して対応する開集合を\(W_i\)と表す。
  各\(i\in I\)に対して
  \(F|_{W_i}\)は\(W_i\)上のしなやかな層であるから、
  \[
  \Gamma_{Z_1}(W_i,F|_{W_i}) \oplus \Gamma_{Z_2}(W_i,F|_{W_i})
  \to \Gamma_{Z_1\cup Z_2}(W_i,F|_{W_i})
  \]
  は全射である。
  \(i\in I\)に渡って余極限をとると、本文命題2.5.1より、
  \[
  \Gamma_{Z_1}(K\cap W,F|_{K\cap W}) \oplus \Gamma_{Z_2}(K\cap W,F|_{K\cap W})
  \to \Gamma_{Z_1\cup Z_2}(K\cap W,F|_{K\cap W})
  \]
  が全射であることが従う。
  これは\(F|_K\)がしなやかであることを意味する。
\end{rem*}




\begin{prob}\label{2.9}
  \(X\)を位相空間とする。
  \begin{enumerate}
    \item \label{2.9.1}
    \(F\)を\(X\)上の層として、
    \(n \geq 0\)を自然数とする。
    以下の条件が同値であることを示せ:
    \begin{enumerate}
      \item \label{2.9.1.1}
      完全列
      \(0\to F\to F^0\to \cdots \to F^n\to 0\)で
      各\(j=0,\cdots,n\)に対して\(F^j\)が脆弱であるものが存在する。
      \item \label{2.9.1.2}
      \(0\to F\to F^0\to \cdots \to F^n\to 0\)が完全であり、
      各\(j<n\)に対して\(F^j\)が脆弱であれば、\(F^n\)も脆弱である。
      \item \label{2.9.1.3}
      任意の閉部分集合\(Z\subset X\)と任意の\(k>n\)に対して
      \(H^k_Z(X,F) = 0\)が成り立つ。
      \item \label{2.9.1.4}
      任意の局所閉部分集合\(Z\subset X\)と任意の\(k>n\)に対して
      \(H^k_Z(X,F) = 0\)が成り立つ。
      \item \label{2.9.1.5}
      任意の閉部分集合\(Z\subset X\)と任意の\(k>n\)に対して
      \(H^k_Z(F) = 0\)が成り立つ。
      \item \label{2.9.1.6}
      任意の局所閉部分集合\(Z\subset X\)と任意の\(k>n\)に対して
      \(H^k_Z(F) = 0\)が成り立つ。
    \end{enumerate}
    これらの条件を満たす最小の\(n\geq 0\)を
    \(F\)の\textbf{脆弱次元} (flabby dimension) と言い、
    \(X\)上のすべての層\(F\)の脆弱次元のsupを\(X\)の\textbf{脆弱次元}と言う。
    \item \label{2.9.2}
    \(X\)を局所コンパクトハウスドルフであるとする。
    \(X\)上の層\(F\)の\textbf{\(c\)-soft dimension}を同様に定義して、
    この場合にも\ref{2.9.1}の条件
    \ref{2.9.1.1}から\ref{2.9.1.4}に対応するものが同値であることを確認せよ。
    \item \label{2.9.3}
    \(X\)を局所コンパクトハウスドルフであるとする。
    このとき、以下の不等式を証明せよ:
    \[
    \text{\(F\)の\(c\)-soft dimension} \leq \text{\(F\)の脆弱次元}
    \leq \text{\(F\)の\(c\)-soft dimension} + 1.
    \]
  \end{enumerate}
\end{prob}


\begin{proof}
  \ref{2.9.1}を示す。帰納法で証明する。
  \(n=0\)とする。
  条件\ref{2.9.1.1}と\ref{2.9.1.2}はどちらも
  「\(F\)は脆弱層である」と読むことができるので明らかに同値である。
  \ref{2.9.1.4} \(\Rightarrow\) \ref{2.9.1.3} と
  \ref{2.9.1.6} \(\Rightarrow\) \ref{2.9.1.5} が成り立つことは明らかである。
  また脆弱層は函手\(\Gamma_Z(X,-)\)や\(\Gamma_Z(-)\)に対してacyclicである
  (cf. 本文命題2.4.10の直前の記述) ので、
  \ref{2.9.1.1} \(\Rightarrow\) \ref{2.9.1.4}, \ref{2.9.1.5} が成り立つ。
  \(U\subset X\)を任意の開集合として、\(Z\dfn X\setminus U\)とおけば、
  \begin{align*}
    &0 \to H^0_Z(X,F) \to F(X) \to F(U) \to H^1_Z(X,F), \\
    &0 \to H^0_Z(F) \to F \to \Gamma_U(F) \to H^1_Z(F)
  \end{align*}
  は完全であるから、上の列が完全であることから
  \ref{2.9.1.3} \(\Rightarrow\) \ref{2.9.1.1}が成り立ち、
  下の列が完全であることと
  本文命題2.4.10の証明中で示されている主張 (2.4.1) より、
  \ref{2.9.1.5} \(\Rightarrow\) \ref{2.9.1.1}が成り立つ。
  以上で\(n=0\)の場合に
  条件\ref{2.9.1.1}から\ref{2.9.1.6}が全て同値であることが示された。
  ある\(n\)で所望の同値性が示されていると仮定して、
  \(n+1\)に対して所望の同値性を示す。
  \ref{2.9.1.4} \(\Rightarrow\) \ref{2.9.1.3} と
  \ref{2.9.1.6} \(\Rightarrow\) \ref{2.9.1.5} はいつでも成立する。
  また、\(n\)番目まで入射分解をとることによって、
  \ref{2.9.1.2} \(\Rightarrow\) \ref{2.9.1.1} が成り立つ。
  \(F\)が\(n+1\)に対して\ref{2.9.1.1}を満たすと仮定する。
  脆弱層への単射\(f:F\to F_0\)を任意にとる。
  \(\coker(f)\)は\(n\)に対して\ref{2.9.1.1}を満たすので、
  帰納法の仮定より、\(\coker(f)\)は\(n\)に対して\ref{2.9.1.2}を満たす。
  \(f\)の取り方は任意だったので、
  これは\(F\)が\(n+1\)に対して\ref{2.9.1.2}を満たすことを意味する。
  また、完全列\(0\to F\to F_0\to \coker(f) \to 0\)
  で局所コホモロジーをとると、\(F_0\)が脆弱層であることから、
  任意の局所閉集合\(Z\subset X\)と\(i\geq 1\)に対して同型射
  \(H^i_Z(X,\coker(f)) \xrightarrow{\sim} H^{i+1}_Z(X,F)\)と
  \(H^i_Z(\coker(f)) \xrightarrow{\sim} H^{i+1}_Z(F)\)を得る。
  \(\coker(f)\)は\(n\)に対して\ref{2.9.1.1}を満たすので、
  帰納法の仮定より、\(\coker(f)\)は\(n\)に対して\ref{2.9.1.4}と\ref{2.9.1.6}を満たす。
  よって、\(F\)が\(n+1\)に対して\ref{2.9.1.4}と\ref{2.9.1.6}を満たすことが従う。
  \(F\)が\(n+1\)に対して\ref{2.9.1.3}または\ref{2.9.1.5}を満たすと仮定する。
  脆弱層への単射\(f:F\to F_0\)を任意にとれば、先ほどと同様にして、
  \(\coker(f)\)が\(n\)に対して\ref{2.9.1.3}または\ref{2.9.1.5}を満たすことが従う。
  帰納法の仮定より、\(\coker(f)\)が\(n\)に対して\ref{2.9.1.1}を満たすことが従い、
  よって\(F\)が\(n\)に対して\ref{2.9.1.1}を満たすことが従う。
  以上で\ref{2.9.1}の証明を完了する。

  \ref{2.9.2}を示す。
  \(X\)を局所コンパクトハウスドルフ空間とする。
  \ref{2.9.1}の主張\ref{2.9.1.1}から\ref{2.9.1.4}に対応するのは
  以下の主張である (ほんまか??):
  \begin{enumerate}
    \item \label{2.9.2.1}
    完全列
    \(0\to F\to F^0\to \cdots \to F^n\to 0\)で
    各\(j=0,\cdots,n\)に対して\(F^j\)が \(c\)-soft であるものが存在する。
    \item \label{2.9.2.2}
    \(0\to F\to F^0\to \cdots \to F^n\to 0\)が完全であり、
    各\(j<n\)に対して\(F^j\)が \(c\)-soft であれば、\(F^n\)も \(c\)-soft である。
    \item \label{2.9.2.3}
    任意の開集合\(U\)と任意の\(k>n\)に対して
    \(H^k_c(U,F|_U)=0\)が成り立つ。
    \item \label{2.9.2.4}
    任意の局所閉集合\(U\)と任意の\(k>n\)に対して
    \(H^k_c(U,F|_U)=0\)が成り立つ。
  \end{enumerate}
  帰納法で証明する。まず\(n=0\)の場合にこれらの主張が同値であることを示す。
  \ref{2.9.2.1}と\ref{2.9.2.2}が同値であることは明らかである。
  また、\autoref{2.6} \ref{2.6.1}より、
  \ref{2.9.2.1}と\ref{2.9.2.3}も同値である。
  \ref{2.9.2.4}から\ref{2.9.2.3}が従うことは明らかである。
  さらに、\(c\)-soft な層の局所閉部分集合への制限はまた\(c\)-softであるから、
  \autoref{2.6} \ref{2.6.1}より、
  \ref{2.9.2.1}と\ref{2.9.2.2}と\ref{2.9.2.3}のいずれかを仮定すれば
  \ref{2.9.2.4}が導かれる。
  以上で\(n=0\)の場合の証明を完了する。
  ある\(n\)で所望の同値性が示されていると仮定して、
  \(n+1\)に対して所望の同値性を示す。
  \ref{2.9.2.4}から\ref{2.9.2.3}が従うことは明らかである。
  また、\(n\)番目までの入射分解をとれば、
  入射的な層は脆弱層であり、脆弱層は\(c\)-softであるから、
  これは\(n\)番目までの \(c\)-soft 分解を与えるので、
  その余核を考えることによって、
  \ref{2.9.2.2}から\ref{2.9.2.1}が導かれる。
  \(F\)が\(n+1\)に対して\ref{2.9.2.1}を満たすとする。
  \(c\)-soft な層への単射\(f:F\to F_0\)を任意にとる。
  このとき、\(\coker(f)\)は\(n\)に対して\ref{2.9.2.1}を満たす。
  従って、帰納法の仮定より、
  \(\coker(f)\)は\(n\)に対して\ref{2.9.2.2}を満たす。
  \(f\)の取り方は任意だったので、
  これは\(F\)が\(n+1\)に対して\ref{2.9.2.2}を満たすことを意味する。
  さらに、帰納法の仮定より、\(\coker(f)\)は\(n\)に対して\ref{2.9.2.4}を満たす。
  任意に局所閉集合\(U\)をとって、
  \(U\)に制限したあとでコンパクト台つきコホモロジーをとることにより、
  各\(i\geq 1\)に対して自然な同型射
  \(H^i_c(U,\coker(f)) \xrightarrow{\sim} H^{i+1}_c(U,F)\)
  を得る。
  \(\coker(f)\)は\(n\)に対して\ref{2.9.2.4}を満たすので、
  従って\(F\)は\(n+1\)に対して\ref{2.9.2.4}を満たす。
  \(F\)が\(n+1\)に対して\ref{2.9.2.3}を満たすと仮定する。
  \(c\)-soft な層への単射\(f:F\to F_0\)をとれば、
  各\(i\geq 1\)に対して自然な射
  \(H^i_c(U,\coker(f)) \xrightarrow{\sim} H^{i+1}_c(U,F)\)
  は同型射であるから、
  \(\coker(f)\)は\(n\)に対して\ref{2.9.2.3}を満たす。
  従って、帰納法の仮定より、\(\coker(f)\)は\(n\)に対して\ref{2.9.2.1}を満たす。
  \ref{2.9.2.1}によって存在が要請される
  \(\coker(f)\)の \(c\)-soft な層による長さ\(n\)の分解を
  \(f\)と繋げることにより、
  \(F\)の \(c\)-soft な層 による長さ\(n+1\)の分解を得るので、
  \(F\)は\(n+1\)に対して\ref{2.9.2.1}を満たす。
  以上で\ref{2.9.2}の証明を完了する。

  \ref{2.9.3}を示す。
  脆弱層が\(c\)-softであることから、不等式
  \(\text{\(F\)の\(c\)-soft dimension} \leq \text{\(F\)の脆弱次元}\)が従う。
  もう一つの不等式を証明する。
  \(F\)の \(c\)-soft dimension が\(n\)であるとする。
  完全列
  \(0\to F\to F^0\to \cdots \to F^n\)で
  各\(j\)に対して\(F^j\)が脆弱層であるものをとる。
  \(F\)の \(c\)-soft dimension が\(n\)であることから、
  \(\im(F^{n-1}\to F^n)\)は \(c\)-soft である。
  従って、\(F\)の脆弱次元が\(n+1\)以下であることを示すためには、
  次を示すことが十分である:
  \begin{enumerate}[label=(\fnsymbol*),start=2]
    \item \label{2.9.3.p}
    局所コンパクトハウスドルフな位相空間\(X\)上の層の完全列
    \(0\to F\to G\to H\to 0\)に対して、
    \(F\)が \(c\)-soft であり、
    \(G\)が脆弱層であるとき、
    \(H\)も脆弱層である。
  \end{enumerate}
  \(X\)は局所コンパクトであるので、
  各\(x\in X\)に対して開近傍\(x\in V\subset X\)であって
  \(\bar{V}\)がコンパクトとなるものが存在する。
  本文命題2.4.10の証明で示されている主張 (2.4.1) より、
  \ref{2.9.3.p}を示すためには、
  \(H|_V\)が脆弱であることを示すことが十分である。
  \(Z\subset V\)を閉集合とする。
  \(K\dfn \bar{Z}\cup(\bar{V}\setminus V)\)とおく。
  これはコンパクト空間\(\bar{V}\)の閉部分空間であるからコンパクトである。
  \(F\)は\(c\)-softであるから、
  \autoref{2.6} \ref{2.6.1}より、
  \(H^i_c(X,F)=H^i_c(X\setminus K,F) = 0, (\forall i>0)\)が成り立つ。
  各\(i\)に対して
  \(H^i_c(X\setminus K,F) \to H^{i+1}_{c,K}(X,F) \to H^{i+1}_c(X,F)\)
  は完全であるので、
  \(H^i_{c,K}(X,F)=0,(\forall i\geq 2)\)が成り立つ。
  \(K\)はコンパクトであるので、
  \(H^i_{c,K}(X,-)\cong H^i_K(X,-)\)が成り立つ。
  \(G\)は脆弱なので、\(H^i_K(X,G) = 0, (\forall i>0)\)が成り立ち、
  従って、完全列\(0\to F\to G\to H\to 0\)に函手\(\Gamma_K(X,-)\)を適用すると、
  \(H^i_K(X,H) = 0, (\forall i>0)\)が成り立つ。
  \(H(X)\to H(X\setminus K) \to H^1_K(X,H)\)は完全なので、
  従って、
  \(H(X)\to H(X\setminus K)\)は全射である。
  \(X\setminus K = (V\setminus Z)\cup (X\setminus \bar{V})\)
  なので、\(H(X\setminus K)\cong H(V\setminus Z)\times H(X\setminus \bar{V})\)
  が成り立つ。
  よって\(H(X)\to H(V\setminus Z)\)は全射であり、
  とくに\(H(V)\to H(V\setminus Z)\)も全射である。
  以上より\(H|_V\)は脆弱である。

  以上で\ref{2.9.3}の証明を完了し、
  \autoref{2.9}の解答を完了する。
\end{proof}


\begin{prob}\label{2.10}
  \(\mcR\)を\(X\)上の環の層として、\(M\)を\(\mcR\)加群とする。
  \begin{enumerate}
    \item \label{2.10.1}
    \(M\)が入射的であるための必要十分条件は、
    任意の部分\(\mcR\)-加群\(\mcI\subset \mcR\)
    (これを\(\mcR\)の\textbf{イデアル}という)
    に対して
    \[
    \Gamma(X,M) \cong \Hom_{\mcR}(\mcR,M)\to \Hom_{\mcR}(\mcI,M)
    \]
    が全射となることである。これを示せ。
    \item \label{2.10.2}
    \(A\)を体とする。
    \(A_X\)のイデアルはある開集合\(U\subset X\)を用いて\(A_U\)と表すことができる。
    このことから、\(A_X\)-加群\(M\)が入射的であるための必要十分条件は
    \(M\)が脆弱層であることであることを帰結せよ。
  \end{enumerate}
\end{prob}


\begin{proof}
  \ref{2.10.1}を示す。
  必要性は明らかであるので十分性が問題である。
  \(\mcR\)-加群\(F\)とその部分\(\mcR\)-加群\(G\subset F\)と
  射\(g:G\to M\)を任意にとる。
  集合
  \[
  S\dfn \left\{ (H,h)\middle| G\subset H\subset F, h|_G=g\right\}
  \]
  に
  \[(H_0,H_0) \leq (H_1,h_1) \ \iff \ H_0\subset H_1 \text{かつ} h_1|_{H_0}=h_0\]
  で順序を入れる。
  全順序部分集合\(S_0\subset S\)に対して、
  \(H_{S_0}\dfn \bigcup_{H\in S_0}H\)と定めて
  \(h_{S_0}:H_{S_0}\to M\)を余極限の普遍性により定まる自然な射とすると
  \((H_{S_0},h_{S_0})\)は\(S_0\)の上界である。
  よってZornの補題より\(S\)には極大限\((H,h)\)が存在する。
  \(H\neq F\)であるとする。
  このとき、開集合\(U\subset X\)と切断\(s\in F(U)\setminus H(U)\)が存在する。
  \(U\)上の切断\(s\)は\(\mcR\)-加群の射\(\mcR_U\to H\)と対応する。
  Fiber積をとって\(\mcI\dfn \mcR_U\times_FH\)とおけば、
  \(\mcI\)は\(\mcR_U\)の部分\(\mcR\)-加群である。
  ここで
  \[
  \Hom_{\mcR}(\mcR,M)\to \Hom_{\mcR}(\mcR_U,M)\to \Hom_{\mcR}(\mcI,M)
  \]
  の合成は全射であるから、
  \(\Hom_{\mcR}(\mcR_U,M)\to \Hom_{\mcR}(\mcI,M)\)も全射であり、
  従って、自然な射影と\(h\)の合成\(\mcI\to H\xrightarrow{h}M\)は
  射\(\mcR_U\to M\)へとリフトし、
  可換図式
  \[
  \begin{CD}
    \mcI @> \subset >> \mcR_U \\
    @VVV @VVV \\
    H @> h >> M
  \end{CD}
  \]
  を得る。
  Push-out をとることによって、
  射\(h':H'\dfn \mcR_U\coprod_{\mcI}H \to M\)を得る。
  一方、可換図式
  \[
  \begin{CD}
    \mcI @> \subset >> \mcR_U \\
    @VVV @VV s V \\
    H @> \subset >> F
  \end{CD}
  \]
  で push-out をとることにより、
  射\(H' \to F\)を得るが、
  \(\mcI=\mcR_U\times_F H\)であることと
  \autoref{1.6} \ref{1.6.3}より、
  \(H'\to F\)はモノ射である。
  従って\(H'\subset F\)とみなせる。
  \(s\not\in H(U)\)なので\(H\subsetneq H'\)である。
  これは\((H,h) < (H',h')\)を意味し、\((H,h)\)の極大性に反する。
  この矛盾は\(H\neq F\)と仮定したことにより引き起こされたので、
  \(H=F\)であることが帰結し、
  以上で、\(f|_G=g\)となる射\(f:F\to M\)の存在が示された。
  これは\(F\)が入射的層であることを示している。
  以上で\ref{2.10.1}の証明を完了する。

  \ref{2.10.2}を示す。
  \(A\)を体、\(\mcI\subset A_X\)をイデアルとする。
  各\(x\in X\)に対して
  \(\mcI_x\subset A_{X,x}\)はイデアルであるが、
  \(A_{X,x}\)は体なので、\(\mcI_x\)は\(0\)か\(A_{X,x}\)のいずれかである。
  \[S\dfn \left\{x\in X\middle| \mcI_x=A_{X,x}\right\}\]
  とおき、\(S\)が開であることを示す。
  \(x\in S\)を任意にとる。
  \(\mcI_x=A_{X,x}\)であるので、
  ある開近傍\(x\in U\)とある切断\(s\in \mcI(U)\)が存在して、
  任意の\(y\in U\)に対して\(s_y = 1\)が成り立つ。
  これから各\(y\in U\)で\(\mcI_y\neq 0\)であることが従い、
  \(\mcI_y\)は\(0\)か\(A_{X,y}\)のいずれかであったので、
  \(\mcI_y = A_{X,y}\)が従う。
  よって\(U\subset S\)が従い、これは\(S\)が開であることを示している。
  最後の主張を示す。
  入射的ならば脆弱層であるため、
  \(A_X\)-加群\(M\)が脆弱層である場合に\(M\)が入射的であることを示す。
  \(M\)が入射的であることを示すためには、\ref{2.10.1}より、
  任意のイデアル層\(\mcI\subset A_X\)と任意の\(A_X\)-加群の射\(\mcI\to M\)に対し、
  それが\(\mcI\subset A_X\)に沿ってリフトすることを示すことが十分である。
  既に証明したことにより、イデアル層\(\mcI\subset A_X\)に対して
  ある開集合\(U\subset X\)が存在して\(\mcI = A_U\)が成り立つ。
  \(A_X\)加群の射\(A_U\to M\)は\(M(U)\)の切断と対応し、
  \(M\)は脆弱層であるので、
  それは\(M(X)\)の元に延長することができる。
  このことは射\(A_U=\mcI\to M\)が\(A_U=\mcI\subset A_X\)に沿ってリフトすることを意味し、
  従って\(M\)は入射的である。
  以上で\ref{2.10.2}の証明を完了し、
  \autoref{2.10}の解答を完了する。
\end{proof}










\begin{prob}\label{2.11}
  \(f:Y\to X\)を局所コンパクトハウスドルフ空間の間の連続写像、
  \(G\)を\(Y\)上の層とする。
  以下の主張が同値であることを示せ:
  \begin{enumerate}
    \item \label{2.11.1}
    任意の\(x\in X\)に対して
    \(G|_{f^{-1}(x)}\)は\(c\)-softである。
    \item \label{2.11.2}
    任意の開集合\(V\subset Y\)と任意の\(j>0\)に対して
    \(R^jf_!G_V=0\)である。
  \end{enumerate}
\end{prob}

\begin{proof}
  \ref{2.11.1} \(\Rightarrow\) \ref{2.11.2}
  を示す。
  任意の\(x\in X\)に対して
  \(G|_{f^{-1}(x)}\)は \(c\)-soft であると仮定する。
  開集合\(V\subset Y\)と点\(x\in X\)を任意にとる。
  本文命題2.6.7より、
  各点\(x\in X\)に対して自然に
  \((R^jf_!G_V)_x \cong H^j_c(f^{-1}(x)\cap V,G|_{f^{-1}(x)})\)
  が成り立つ。
  ここで\(G|_{f^{-1}(x)}\)は \(c\)-soft であるので、
  \autoref{2.6} \ref{2.6.1}より、\(j>0\)に対して
  \(H^j_c(f^{-1}(x)\cap V,G|_{f^{-1}(x)})=0\)が成り立つ。
  よって層\(R^jf_!G_V\)の各点でのstalkは\(0\)であり、
  従って\(R^jf_!G_V=0\)である。

  \ref{2.11.2} \(\Rightarrow\) \ref{2.11.1}
  を示す。
  任意の開集合\(V\subset Y\)と任意の\(j>0\)に対して
  \(R^jf_!G_V=0\)であると仮定する。
  点\(x\in X\)と開集合\(V_x\subset f^{-1}(x)\)を任意にとる。
  このとき、ある開集合\(V\subset Y\)が存在して
  \(V_x = V\cap f^{-1}(x)\)が成り立つ。
  本文命題2.6.7より、各\(j>0\)に対して自然に
  \(H^j_c(V_x,G|_{f^{-1}(x)}) \cong (R^jf_!G_V)_x = 0\)
  が成り立つ。
  よって\autoref{2.6} \ref{2.6.1}より、
  \(G|_{f^{-1}(x)}\)は \(c\)-soft である。
  以上で\autoref{2.11}の解答を完了する。
\end{proof}










\begin{prob}\label{2.12}
  \(X\)を位相空間とする。
  \begin{enumerate}
    \item \label{2.12.1}
    \((F_{\lambda})_{\lambda\in \Lambda}\)
    を有向集合\(\Lambda\)で添字づけられた
    \(X\)上の層の順系とする。
    \(X\)がコンパクトハウスドルフであると仮定せよ。
    このとき、任意の\(k\in \N\)に対して
    \(\colim_{\lambda}H^k(X,F_{\lambda}) \cong
    H^k(X,\colim_{\lambda}F_{\lambda})\)
    が成り立つことを示せ。
    \item \label{2.12.2}
    \((F_n)_{n\in \N}\)を\(X\)上の層の逆系で、
    \textbf{各\(F_{n+1}\to F_n\)は全射である}とする。
    \(Z\subset X\)を局所閉部分集合とする。
    \(\{H^{k-1}_Z(X,F_n)\}_{n\in \N}\)が Mittag-Leffler 条件を満たすと仮定せよ。
    このとき、自然な同型射
    \(H^k_Z(X,\lim_nF_n)\xrightarrow{\sim} \lim_nH^k_Z(X,F_n)\)
    が存在することを示せ。
  \end{enumerate}
\end{prob}


\begin{rem*}
  \ref{2.12.2}は本文にはない仮定を置いている。
  本文を引用すると以下の通りである:

  Let \((F_n)_{n\in \N}\) be a projective system of sheaves on \(X\) and
  let \(Z\) be a locally closed subset of \(X\).
  Assuming that \(\{H^{k-1}_Z(X,F_n)\}_n\) satisfies the M-L condition,
  prove the isomorphism
  \(H^k_Z(X,\lim_n F_n) \xrightarrow{\sim} \lim_n H^k_Z(X,F_n)\).

  しかしこのままだと反例がある。
  \(X=Z=[0,1]\)とする。
  \(X=Z\)なので\(H^i_Z(X,-)\cong H^i(X,-)\)である。
  \(U_n=(1/2-1/(n+2),1/2)\cup (1/2,1/2+1/(n+2))\)とおき、
  \(F_n\dfn \Z_{U_n}\)と定める。
  \(U_{n+1}\subset U_n\)であるから\(F_{n+1}\subset F_n\)であり、
  これによって層の逆系\((F_n)_{n\in \N}\)ができる。
  \(k=1\)とする。
  定数層\(\Z_X\)の大域切断であって\(U_n\)に台を持つものは\(0\)しかないので
  \(H^0_Z(X,F_n)=H^0(X,F_n)=0\)が成り立ち、
  従って\(\{H^0_Z(X,F_n)\}_n\)は Mittag-Leffler 条件を満たす。
  \(\bigcap_{n=0}^{\infty}U_n=\emptyset\)であるので、
  各\(X\)の点で stalk をとることによって\(\lim F_n=0\)が成り立つ。
  従って\(H^1_Z(X,\lim F_n)=0\)である。
  さらに層の完全列
  \[0\to \Z_{U_n}\to \Z_X\to \Z_{X\setminus U_n}\to 0\]
  でコホモロジーをとる。
  \(X=[0,1]\)なので、命題2.7.3 (ii), (iii) より\(H^1(X,\Z_X)=0\)である。
  \(X\setminus U_n\)は連結成分が3つなので
  \(H^0(X,\Z_{X\setminus U_n})=\Z^3\)である。
  よって完全列
  \[
  0\to 0\to \Z \to \Z^3 \to H^1(X,F_n) \to 0
  \]
  を得る。
  従って\(H^1(X,F_n)\cong \Z^2\)が成り立つ。
  また、この同型射は\(H^1(X,F_{n+1})\to H^1(X,F_n)\)と可換するので、
  よって\(\lim H^1(X,F_n)\cong \Z^2\)が成り立つ。
  以上で\(\{H^0_Z(X,F_n)=0\}_n\)が Mittag-Leffler 条件を満たすのにもかかわらず
  \(0=H^1_Z(X,\lim F_n) \not\cong \Z^2\cong \lim H^1_Z(X,F_n)\)
  となる例が構成できた。
\end{rem*}



\begin{proof}
  \ref{2.12.1}を示す。
  \(X=\bigcup_{i=1}^r U_i\)を有限開被覆とする。
  Filtered colimitは有限極限と可換するので、
  \[
  \begin{CD}
    0 @>>> (\colim F_{\lambda})(X) @>>> \prod_{i=1}^r (\colim F_{\lambda})(U_i)
    @>>> \prod_{i,j=1}^r (\colim F_{\lambda})(U_i\cap U_j)
  \end{CD}
  \]
  は完全である。
  \(X=\bigcup_{i\in I}U_i\)を任意の開被覆とする。
  \(X\)はコンパクトであるから、
  \[S\dfn \left\{I_0\subset I\middle| X=\bigcup_{i\in I_0}U_i, |I_0|<\infty\right\}\]
  は空でない有向集合である。
  各\(I_0\subset I_1, I_0,I_1\in S\)に対して完全列の射
  \[
  \begin{CD}
    0 @>>> (\colim F_{\lambda})(X) @>>> \prod_{i\in I_1} (\colim F_{\lambda})(U_i)
    @>>> \prod_{i,j\in I_1} (\colim F_{\lambda})(U_i\cap U_j) \\
    @. @| @VVV @VVV \\
    0 @>>> (\colim F_{\lambda})(X) @>>> \prod_{i\in I_0} (\colim F_{\lambda})(U_i)
    @>>> \prod_{i,j\in I_0} (\colim F_{\lambda})(U_i\cap U_j)
  \end{CD}
  \]
  ができるので、\(I_0\in S\)に渡って逆極限をとることにより、
  \[
  \begin{CD}
    0 @>>> (\colim F_{\lambda})(X) @>>> \prod_{i\in I} (\colim F_{\lambda})(U_i)
    @>>> \prod_{i,j\in I} (\colim F_{\lambda})(U_i\cap U_j)
  \end{CD}
  \]
  が完全であることが従う。
  よって
  \(\colim_{\lambda}H^0(X,F_{\lambda}) \cong H^0(X,\colim_{\lambda}F_{\lambda})\)
  が成り立つ。

  \((I_{\lambda})_{\lambda\in \Lambda}\)を函手圏\([\Lambda,\Ab(X)]\)の入射的対象とする。
  任意の\(0\in \Lambda\)と任意の層の単射\(M\to N\)と任意の射\(f:M\to I_0\)に対し、
  \(M,N\)を\(0\)番目に配置して\(\Ab(X)\)の図式と考えると、
  \((I_{\lambda})_{\lambda\in \Lambda}\)が函手圏で入射的対象であるので、
  \(\lambda\)に関して函手的に\(f\)のリフト\(N_{\lambda}\to I_{\lambda}\)
  が得られるので、\(0\)番目をみることで、\(f\)のリフト\(N\to I_0\)を得る。
  従って、各\(\lambda\)に対して\(I_{\lambda}\)は入射的層である。
  とくに (\(c\)-)soft である。
  \(F\subset X\)を閉集合とすると、
  各\(I_{\lambda}(X)\to I_{\lambda}(F)\)は全射であるので、
  \(\colim(I_{\lambda}(X))\to \colim(I_{\lambda}(F))\)も全射であるが、
  ここで\(X,F\)はどちらもコンパクト (かつハウスドルフ) なので、
  すでに証明したことから、
  \(\colim(I_{\lambda}(X))\cong (\colim I_{\lambda})(X),
  \colim(I_{\lambda}(F))\cong (\colim I_{\lambda})(F)\)
  が成り立つ。
  従って\(\colim I_{\lambda}\)も (\(c\)-)soft であることが従う。
  \(X\)はコンパクトハウスドルフなので、
  従って\(\colim I_{\lambda}\)は大域切断函手に対して acyclic である。
  Filtered colimit をとる函手
  \(\colim:[\Lambda,\Ab(X)]\to \Ab(X)\)は完全函手であるから、
  以上より、函手
  \[
  [\Lambda,\Ab(X)]\to \Ab, \ \
  (F_{\lambda})_{\lambda\in \Lambda}\mapsto \Gamma(X,\colim F_{\lambda})
  \]
  の右導来函手は\(R\Gamma(X,-)\circ \colim\)と自然に同型である。
  同様に、\(\colim:[\Lambda,\Ab]\to \Ab\)は完全函手なので、函手
  \[
  [\Lambda,\Ab(X)]\to \Ab, \ \
  (F_{\lambda})_{\lambda\in \Lambda}\mapsto \colim \Gamma(X,F_{\lambda})
  \]
  の右導来函手は
  \(\colim \circ R\Gamma(X,(-)_{\lambda})\)
  と自然に同型である。
  ただし
  \(R\Gamma(X,(-)_{\lambda})\)は
  \(\sfD^{\geq 0}([\Lambda,\Ab(X)])\)から
  \(\sfD^{\geq 0}([\Lambda,\Ab])\)への函手である
  (\([\Lambda,\sfD^+(\Ab)]\)に値を持つのではない!)。
  すでに証明した\(0\)次の場合より、自然に
  \(\Gamma(X,\colim (-)_{\lambda}) \cong \colim \Gamma(X,(-)_{\lambda})\)
  が成り立つので、これらの右導来函手も自然に同型であり、
  \(R\Gamma(X,\colim (-)_{\lambda}) \cong \colim R\Gamma(X,(-)_{\lambda})\)
  が成り立つ
  (右辺の\(\colim\)は通常の余極限をとる函手の導来函手であり、
  \([\Lambda,\sfD^+(\Ab)]\)における余極限とは異なる)。
  \((F_{\lambda})_{\lambda\in \Lambda}\)を代入してコホモロジーをとると、
  余極限をとる函手が完全であることから
  \[
  H^i(X,\colim F_{\lambda})\cong
  H^i(\colim R\Gamma(X,F_{\lambda})) \cong
  \colim H^i(X,F_{\lambda})
  \]
  を得る。
  以上で\ref{2.12.1}の証明を完了する。

  \ref{2.12.2}を示す。
  \((I_n)_{n\in \N}\)を圏\([\N,\Ab(X)]\)の入射的対象とする。
  局所閉集合\(Z\subset X\)と\(n\)に対して切断\(s\in \Gamma_Z(X,I_n)\)を一つ選ぶと、
  \(s\)は層の射\(\Z_Z\to I_n\)を定める。
  \(n\)番目以前が\(\Z_Z\)でそれ以降\(0\)である逆系を\(\Z_Z(n)\)とおくと、
  \(s\)は逆系の射\(\Z_Z(n)\to (I_n)_{n\in \N}\)を定める。
  \((I_n)_{n\in \N}\)は入射的なので、
  逆系の単射\(\Z_Z(n)\subset \Z_Z(n+1)\)に沿って
  \(\Z_Z(n)\to (I_n)_{n\in \N}\)をリフトさせることにより、
  \(\Gamma_Z(X,I_{n+1})\to \Gamma_Z(X,I_n)\)
  が全射であることが従う。
  特に、各開集合\(U\subset X\)に対して\((\Gamma_Z(X,I_n))_{n\in \N}\)は
  Mittag-Leffler 条件を満たす、すなわち、\(\lim_n\)に対して acyclic である。
  よって\(R(\lim_n\circ\Gamma_Z(X,-)) \cong R\lim_n \circ R\Gamma_Z(X,-)\)
  が成り立ち、逆系\((F_n)_{n\in \N}\)に対してスペクトル系列
  \[
  E_2^{p,q}=R^p\lim_n H^q_Z(X,F_n) \ \Rightarrow \
  E^{p+q}=R^{p+q}(\lim_n\circ \Gamma_Z(X,-))(F_n)
  \]
  を得る。
  \(R^p\lim_n = 0, (p\neq 0,1)\)であるので、完全列
  \[
  0\to R^1\lim_n H^q_Z(X,F_n) \to E^{1+q} \to \lim_n H^{q+1}_Z(X,F_n)\to 0
  \]
  を得る。
  \(q=k-1\)とすれば、
  \((H^{k-1}_Z(X,F_n))_{n\in \N}\)が Mittag-Leffler 条件を満たすという仮定より、
  同型射\(E^k\xrightarrow{\sim} \lim_n H^{q+1}_Z(X,F_n)\)を得る。

  次に、\((I_n)_{n\in \N}\)を再び\([\N,\Ab(X)]\)の入射的対象とし、
  \(U\subset X\)を開集合として、切断\(s\in \lim_n I_n(U)\)を任意にとる。
  これは層の射\(\Z_U\to \lim_n I_n\)と対応するが、
  これは各番号に\(\Z_U\)が対応している自明な逆系\((\Z_U)_{n\in \N}\)からの逆系の射
  \((\Z_U)_{n\in \N}\to (I_n)_{n\in \N}\)と対応する。
  これを単射\((\Z_U)_{n\in \N}\to (\Z_X)_{n\in \N}\)に沿ってリフトさせることにより、
  \(\lim_nI_n(X)\to \lim_nI_n(U)\)が全射であることが従う。
  従って\(\lim_n I_n\)は脆弱層であり、
  とくに任意の局所閉集合\(Z\subset X\)に対する\(\Gamma_Z(X,-)\)に対して acyclic である。
  よって\(R(\Gamma_Z(X,-)\circ \lim_n) \cong R\Gamma_Z(X,-)\circ R\lim_n\)
  が成り立ち、逆系\((F_n)_{n\in \N}\)に対してスペクトル系列
  \[
  \bar{E}_2^{p,q}=H^p_Z(X,R^q\lim_n F_n) \ \Rightarrow \
  \bar{E}^{p+q}=R^{p+q}(\Gamma_Z(X,-)\circ \lim_n)(F_n)
  \]
  を得る。
  \(\Gamma_Z(X,-)\circ \lim_n\cong \lim_n\circ \Gamma_Z(X,-)\)
  であるので、自然に\(\bar{E}^{p+q}\cong E^{p+q}\)である。
  また、\(R^q\lim_n = 0, (q=0,1)\)であるので、
  完全列
  \[
  \cdots \to \bar{E}_2^{p-2,1} \to \bar{E}_2^{p,0} \to E^p
  \to \bar{E}_2^{p-1,1}\to \bar{E}_2^{p+1,0} \to E^{p+1} \to \cdots
  \]
  を得る。
  ここで各\(F_{n+1}\to F_n\)が全射であるという仮定より、
  \(R^1\lim_n F_n=0\)が成り立つので、
  \(\bar{E}_2^{\bullet,1}=0\)が成り立つ。
  従って各\(\bar{E}_2^{p,0} \xrightarrow{\sim} E^p\)は同型射である、
  すなわち、各\(p\)に対して
  \(H^p_Z(X,\lim_n F_n) \xrightarrow{\sim} E^p\)は同型射である。
  \(p=k\)とすることにより、同型射
  \[
  H^k_Z(X,\lim_n F_n) \xrightarrow{\sim}
  R^k(\lim_n\circ \Gamma_Z(X,-))(F_n) \xrightarrow{\sim}
  \lim_n H^k_Z(X,F_n)
  \]
  を得る。
  以上で\ref{2.12.2}の証明を完了し、
  \autoref{2.12}の解答を完了する。
\end{proof}









\begin{prob}\label{2.13}
  \(G\)を\(X\)上の層として、
  \(Z\subset X\)を局所閉集合とする。
  \(j<n\)に対して\(R^j\Gamma_Z(G)=0\)が成り立つと仮定せよ。
  このとき、前層\(U\mapsto H^n_Z(U,G)\)は層であり、
  さらにこれが\(R^n\Gamma_Z(G)\)と等しいことを示せ。
\end{prob}

\begin{proof}
  \(R^n\Gamma_Z(G)\)は層なので、
  \autoref{2.13}を示すためには、
  開集合\(U\subset X\)に対して自然に
  \(H^n_Z(U,G)\cong \Gamma(U,R^n\Gamma_Z(G))\)
  が成り立つことを証明することが十分である。
  \(F=\Gamma_Z(-), F'=\Gamma(U,-)\)として\autoref{1.22}を適用することにより、
  \(R^n(\Gamma(U,\Gamma_Z(-)))(G)\cong \Gamma(U,R^n\Gamma_Z(G))\)
  が成り立つ。
  ここで\(\Gamma(U,\Gamma_Z(-))\cong \Gamma_Z(U,-)\)であるので、
  よって\(H^n_Z(U,G)\cong \Gamma(U,R^n\Gamma_Z(G))\)が成り立つ。
  以上で\autoref{2.13}の解答を完了する。
\end{proof}













\begin{prob}\label{2.14}
  \(X = \bigcup_{i\in I}U_i\)を\(X\)の開被覆とする。
  各\(i\in I\)に対して\(F_i\)を\(U_i\)上の層として、
  各\((i,j)\in I^2\)に対して同型射
  \(\varphi_{ij}:F_j|_{U_i\cap U_j} \xrightarrow{\sim} F_i|_{U_i\cap U_j}\)
  が与えられているとする。
  \(\varphi_{ii}=\id_{F_i}\)であり、
  さらに任意の\((i,j,k)\in I^3\)に対して
  \(U_i\cap U_j\cap U_k\)上で
  \(\varphi_{ij}\circ \varphi_{jk}=\varphi_{ik}\)
  が成り立つと仮定せよ。
  このとき、\(X\)上の層\(F\)と
  各\(i\in I\)に対する同型射
  \(\varphi_i:F|_{U_i}\xrightarrow{\sim} F_i\)であって、
  任意の\((i,j)\in I^2\)に対して\(U_i\cap U_j\)上で
  \(\varphi_{ij} = \varphi_i\circ \varphi_j^{-1}\)
  が成り立つもの、が up to isomorphism で一意的に存在することを示せ。
\end{prob}


\begin{proof}
  圏\(\mcI\)を次で定義する:
  \begin{itemize}
    \item
    対象の集合は\(\mathrm{Ob}(\mcI) \dfn I^3\).
    \item
    \(\Hom_{\mcI}((i,j,k),(i',j',k'))\)は
    \(\{i,j,k\} \supset \{i',j',k'\}\)である場合は一点集合で、
    そうでない場合は\(\emptyset\)と定める。
  \end{itemize}
  \(\{i,j,k\} = \{i',j',k'\}\)
  である場合、またその場合に限り\((i,j,k)\to (i',j',k')\)は同型射である。

  \(U_{ij}\dfn U_i\cap U_j, U_{ijk}\dfn U_i\cap U_j\cap U_k\)とおき、
  \(f_i:U_i\to X, f_{ij}:U_{ij} \to X, f_{ijk}:U_{ijk}\to X\)
  をそれぞれ包含射とする。
  各\((i,j,k)\in \mcI\)に対して\(X\)上の層\(F(i,j,k)\)を
  \(P(i,j,k)\dfn f_{ijk,!}(F_i|_{U_{ijk}})\)と定義する
  (\(P(i,i,i) = f_{i!}F_i\)である)。
  各\(\mcI\)の射\(p:(i,j,k)\to (i',j',k')\)に対して
  \(U_{ijk}\subset U_{i'j'k'}\)であるので、
  自然な包含射
  \(\psi(p)(-):f_{i'j'k',!}((-))|_{U_{ijk}}) \subset
  f_{i'j'k',!}((-)|_{U_{i'j'k'}})\)
  がある。
  また、\(i'\in \{i,j,k\}\)であるので\(U_{ijk}\subset U_{ii'}\)である。
  \(P(p)\dfn \psi(p)(F_{i'})\circ f_{ijk,!}(\varphi_{i'i}|_{U_{ijk}})\)と定義する。
  この対応によって、\(P:\mcI\to \Ab(X)\)は函手になる。
  それを確かめるために、\(\mcI\)の射の列
  \((i,j,k)\xrightarrow{p}(i',j',k')\xrightarrow{q}(i'',j'',k'')\)
  を任意にとる。
  \(P\)が函手であるためには、
  \(P(q\circ p) = P(q)\circ P(p)\)が成り立つことが十分である。
  \(i''\in \{i',j',k'\}\subset \{i,j,k\}\)であるので、
  \(U_{ijk}\subset U_{ii'}\cap U_{i'i''}\)が成り立つ。
  従って
  \[
  f_{ijk,!}(\varphi_{i''i}|_{U_{ijk}})
  = f_{ijk,!}(\varphi_{i''i'}|_{U_{ijk}}\circ \varphi_{i'i}|_{U_{ijk}})
  = f_{ijk,!}(\varphi_{i''i'}|_{U_{ijk}})\circ f_{ijk,!}(\varphi_{i'i}|_{U_{ijk}})
  \]
  が成り立つ。
  また、定義より、函手の射として\(\psi(q\circ p)=\psi(q)\circ \psi(p)\)が成り立つ。
  また、\(\psi(p)\)が自然変換であることから、図式
  \[
  \begin{CD}
    f_{ijk,!}(F_{i'}|_{U_{ijk}})
    @> f_{ijk,!}(\varphi_{i''i'}|_{U_{ijk}})>>
    f_{ijk,!}(F_{i''}|_{U_{ijk}}) \\
    @V{\psi(p)(F_{i'})}VV
    @VV{\psi(p)(F_{i''})}V \\
    f_{i'j'k',!}(F_{i'}|_{U_{i'j'k'}})
    @> f_{i'j'k',!}(\varphi_{i''i'}|_{U_{i'j'k'}})>>
    f_{i'j'k',!}(F_{i''}|_{U_{i'j'k'}})
  \end{CD}
  \]
  は可換である。
  以上より、
  \begin{align*}
    P(q\circ p)
    &= \psi(q\circ p)(F_{i''}) \circ f_{ijk,!}(\varphi_{i''i'}|_{U_{ijk}}) \\
    &= \psi(q)(F_{i''}) \circ \psi(p)(F_{i''}) \circ
    f_{ijk,!}(\varphi_{i''i'}|_{U_{ijk}})\circ f_{ijk,!}(\varphi_{i'i}|_{U_{ijk}}) \\
    &= \psi(q)(F_{i''}) \circ f_{i'j'k',!}(\varphi_{i''i'}|_{U_{i'j'k'}})
    \circ \psi(p)(F_{i'}) \circ f_{ijk,!}(\varphi_{i'i}|_{U_{ijk}}) \\
    &= P(q)\circ P(p)
  \end{align*}
  が成り立つ。
  よって\(P:\mcI\to \Ab(X)\)は函手である。

  \(F\dfn \colim P\)とおく。
  各\(x\in X\)で stalk をとると図式\(P\)の射は\(0\)射と同型射の図式となる。
  従って、自然な射\(P_i:P(i,i,i)=f_{i,!}F_i\to F\)を\(U_i\)へと制限したものは
  層の同型射である。
  その逆射を\(\varphi_i\dfn P_i^{-1}:F|_{U_i}\to F_i\)とおく。
  図式
  \[
  \begin{CD}
    P(j,j,i) @>>> P(j,j,j) @> P_j >> F \\
    @V f_{ij,!}(\varphi_{ij}) VV @. @| \\
    P(i,j,j) @>>> P(i,i,i) @> P_i >> F
  \end{CD}
  \]
  は可換であり、
  \(P(j,j,i)\to P(j,j,j)\)と\(P(i,j,j)\to P(i,i,i)\)を\(U_{ij}\)へと制限すると
  \(\id_{F_j|_{U_{ij}}}\)と\(\id_{F_i|_{U_{ij}}}\)になるので、
  従って\(U_{ij}\)上で
  \(\varphi_{ij} = \varphi_i\circ \varphi_j^{-1}\)が成り立つ。

  別の\(F'\)がこの性質を満たせば、
  余極限の普遍性により射\(F\to F'\)が得られ、
  これは各点の stalk で同型射であるので、
  このような\(F\)は up to isom. で一意的に存在する。
  以上で\autoref{2.14}の証明を完了する。
\end{proof}










\begin{prob}\label{2.15}
  \begin{enumerate}
    \item \label{2.15.1}
    \(F^{\bullet}\)を下に有界な\(X\)上の層の複体とする。
    自然な射
    \(H^j(\Gamma(X,F^{\bullet})) \to H^j(R\Gamma(X,F^{\bullet}))\)
    を構成せよ。
    \item \label{2.15.2}
    \(\mcU = \{U_i\}_i\)を\(X\)の開被覆として、
    \(F\)を\(X\)上の層とする。
    自然な射\(H^j(C^{\bullet}(\mcU,F)) \to H^j(X,F)\)を構成せよ。
  \end{enumerate}
\end{prob}

\begin{proof}
  \ref{2.15.1}を示す。
  入射的層からなる複体へのモノな擬同型\(F^{\bullet}\xrightarrow{\text{qis}}I^{\bullet}\)
  をとれば複体の射
  \(\Gamma(X,F^{\bullet})\to \Gamma_(X,I^{\bullet})\cong R\Gamma(X,F^{\bullet})\)
  が得られるので、
  \(j\)次コホモロジーをとることによって射
  \(H^j(\Gamma(X,F^{\bullet})) \to H^j(R\Gamma(X,F^{\bullet}))\)
  を得る。
  以上で\ref{2.15.1}の証明を完了する。

  \ref{2.15.2}を示す。
  \(F\)を\(0\)次だけが\(F\)で他が\(0\)である自明な複体とみなすと、
  本文命題2.8.4より、augmentation map
  \(\delta:F\xrightarrow{\text{qis}} \mcC^{\bullet}(\mcU,F)\)
  は擬同型である。
  よって\(\sfD^+(\Ab(X))\)の同型射
  \(R\Gamma(X,\delta):R\Gamma(X,F)\xrightarrow{\sim}
  R\Gamma(X,\mcC^{\bullet}(\mcU,F))\)を得る。
  \ref{2.15.1}を\(F^{\bullet} = \mcC^{\bullet}(\mcU,F)\)に対して適用すると、
  \(\Gamma(X,\mcC^{\bullet}(\mcU,F)) \cong C^{\bullet}(\mcU,F)\)
  であるので、射
  \(H^j(C^{\bullet}(\mcU,F)) \to H^j(R\Gamma(X,\mcC^{\bullet}(\mcU,F))\)
  を得る。
  これに\(H^j(R\Gamma(X,\delta)^{-1})\)を合成することで
  射\(H^j(C^{\bullet}(\mcU,F))\to H^j(X,F)\)を得る。
  以上で\ref{2.15.2}の証明を完了し、
  \autoref{2.15}の解答を完了する。
\end{proof}










\begin{prob}\label{2.16}
  \(A\)を可換環、\(A^{\times}\)を単元のなす群とする。
  \(X\)を位相空間、\(\mcU=\{U_i\}_i\)を\(X\)の開被覆として、
  \(c\in C^2(\mcU,A_X^{\times})\)を\(\delta c = 0\)となる元とする。
  \(c'\)を\(c\)の\(H^2(C^{\bullet}(\mcU,A_X^{\times}))\)での剰余類として、
  \(c''\)を\(c'\)の\(H^2(X,A_X^{\times})\)での像とする
  (cf. \autoref{2.15} \ref{2.15.2})。
  圏\(\Sh(X,c)\)を次によって定義する:
  \begin{itemize}
    \item
    対象は\(A_{U_i}\)-加群\(F_i\)と
    同型射\(\rho_{ij}:F_j|_{U_i\cap U_j}\xrightarrow{\sim} F_i|_{U_i\cap U_j}\)
    の族\(\{F_i,\rho_{ij}\}\)で、
    任意の\(i,j,k\)に対して
    \[
    \rho_{ij}\rho_{jk}\rho_{ki} = c_{ijk}\id_{F_i|U_i\cap U_j\cap U_k}
    \]
    を満たすものとする。
    \item
    射\(f:\{F_i,\rho_{ij}\}\to \{F'_i,\rho_{ij}\}\)は
    \(U_i\)上の射の族\(f_i:F_i\to F'_i\)で
    \(U_i\cap U_j\)上で\(\rho'_{ij}\circ f_j = f_i\circ \rho_{ij}\)
    を満たすものとする。
  \end{itemize}
  以下を示せ:
  \begin{enumerate}
    \item \label{2.16.1}
    \(\Sh(X,c)\)はアーベル圏であることを証明せよ。
    \item \label{2.16.2}
    \(\tilde{c}\in C^2(\mcU,A_X^{\times})\)を別の元で
    \(\tilde{c}'' = c''\)を満たすものとする。
    \(\Sh(X,c)\)と\(\Sh(X,\tilde{c})\)の間の圏同値が存在することを示せ。
  \end{enumerate}
\end{prob}

\begin{proof}
  \ref{2.16.1}を示す。
  \(\Sh(X,c)\)は明らかな\(0\)-対象を持つ
  (各\(U_i\)上で\(0\)であるもの)。
  また、明らかに、二つの対象\(\{F_i,\rho_{ij}\},\{F_i',\rho_{ij}'\}\)に対して
  \(\{F_i\oplus F_i',\rho{ij}\oplus \rho_{ij}'\}\)は\(\Sh(X,c)\)の対象である。
  さらに、二つの対象の間の射\(f=(f_i):\{F_i,\rho_{ij}\}\to\{F_i',\rho_{ij}'\}\)に対し、
  各\(i\)ごとに\(\ker(f_i)\)をとり、
  \(\rho^{\ker(f)}_{ij}\dfn \rho_{ij}|_{\ker(f_i)|_{U_i\cap U_j}}\)
  と定めることにより、明らかに
  \(\{\ker(f_i),\rho^{\ker(f)}_{ij}\}\)は\(\Sh(X,c)\)の対象となる。
  余核についても同様である。
  核と余核が各\(i\)ごとに定義されるので、
  余像と像は一致し、これにより\(\Sh(X,c)\)がアーベル圏であることが従う。
  以上で\ref{2.16.1}の証明を完了する。

  \ref{2.16.2}を示す。
  \(\{F_i,\rho_{ij}\}\)を\(\Sh(X,c)\)の対象とする。
  \(\bar{c} \dfn c-\tilde{c}\)とおく。
  このとき\(\bar{c}''=0\)である。
  本文命題2.8.4より、augmentation map
  \(\delta:F\xrightarrow{\text{qis}} \mcC^{\bullet}(\mcU,F)\)
  は擬同型であるので、
  \(H^2(R\Gamma(X,\mcC^{\bullet}(\mcU,A_X^{\times})))\)での
  \(\bar{c}''\)の像は\(0\)である。
\end{proof}





\begin{prob}\label{2.17}
  \(X\)を局所コンパクト空間、
  \(\mcR\)を\(X\)上の (可換) 環の層で\(\wgld(\mcR)<\infty\)であるものとし、
  \(Z_1,Z_2\subset X\)を局所閉部分集合とする。
  \begin{enumerate}
    \item \label{2.17.1}
    \(F_1,F_2\in \sfD^+(\mcR)\)に対し、
    自然な射
    \[
    R\Gamma_{Z_1}(F_1)\otimes_{\mcR}^L R\Gamma_{Z_2}(F_2)
    \to R\Gamma_{Z_1\cap Z_2}(F_1\otimes_{\mcR}^L F_2)
    \]
    を構成せよ。
    \item \label{2.17.2}
    \(A\)を可換環として、\(\mcR=A_X\)であると仮定せよ。
    \(F_1,F_2\in \sfD^+(\mcR)\)に対し、
    自然な射
    \[
    R\Gamma_{Z_1}(X,F_1)\otimes_A^L R\Gamma_{Z_2}(X,F_2)
    \to R\Gamma_{Z_1\cap Z_2}(X,F_1\otimes_A^L F_2)
    \]
    を構成し、各\(p,q\in \Z\)に対して
    \[
    H^p_{Z_1}(X,F_1)\otimes_A H^q_{Z_2}(X,F_2)
    \to H^{p+q}_{Z_1\cap Z_2}(X,F_1\otimes_A^L F_2)
    \]
    を構成せよ。
    最後の射は\textbf{cup積}と呼ばれる。
  \end{enumerate}
\end{prob}

\begin{proof}
  \ref{2.17.1}を示す。
  本文 (2.6.9) より、自然に
  \(R\Gamma_{Z_i}(F_i) \cong R\inHom_{\mcR}(\mcR_{Z_i},F_i)\)が成り立つ。
  また、本文 (2.6.11) より、自然な射
  \begin{align*}
    R\inHom_{\mcR}(\mcR_{Z_1},F_1) \otimes_{\mcR}^L R\inHom_{\mcR}(\mcR_{Z_2},F_2)
    &\to R\inHom_{\mcR}(\mcR_{Z_1},F_1\otimes_{\mcR}^L R\inHom_{\mcR}(\mcR_{Z_2},F_2)) \\
    &\to R\inHom_{\mcR}(\mcR_{Z_1},R\inHom_{\mcR}(\mcR_{Z_2},F_1\otimes_{\mcR}^L F_2))
  \end{align*}
  を得る。
  さらに本文命題 2.6.3 (ii) より、
  自然な同型
  \[
  R\inHom_{\mcR}(\mcR_{Z_1},R\inHom_{\mcR}(\mcR_{Z_2},F_1\otimes_{\mcR}^L F_2))
  \cong R\inHom_{\mcR}(\mcR_{Z_1}\otimes_{\mcR}^L \mcR_{Z_2},F_1\otimes_{\mcR}^L F_2))
  \]
  を得る。
  ここで\(\mcR_{Z_i}\)は \(\mcR\)-flat であるので、自然に
  \(\mcR_{Z_1}\otimes_{\mcR}^L \mcR_{Z_2} \cong
  \mcR_{Z_1}\otimes_{\mcR} \mcR_{Z_2} \cong \mcR_{Z_1\cap Z_2}\)
  が成り立つ。
  再び本文 (2.6.9) を用いることで、自然に
  \(R\Gamma_{Z_1\cap Z_2}(F_1\otimes_{\mcR}^L F_2) \cong
  R\inHom_{\mcR}(\mcR_{Z_1\cap Z_2},F_1\otimes_{\mcR}^L F_2)\)
  が成り立つので、
  これらを組み合わせることによって所望の射を得る。
  以上で\ref{2.17.1}の証明を完了する。

  \ref{2.17.2}を示す。
  \(f:X\to \{\mathrm{pt}\}\)を自明な射とする。
  まず、\(Z_1=Z_2=X\)の場合に証明する。
  この場合、\(R\Gamma_{Z_i}(X,-)\cong Rf_*(-)\)が成り立つ。
  今、\(\mcR = A_X = f^{-1}A\)は定数層であるので、
  従って、本文命題 2.6.4 (ii) より、自然に
  \[
  \Hom_{\sfD^+(A)}(Rf_*F_1\otimes_A^L Rf_*F_2, Rf_*(F_1\otimes_{A_X}^L F_2))
  \cong \Hom_{\sfD^+(A_X)}(f^{-1}(Rf_*F_1\otimes_A^L Rf_*F_2),F_1\otimes_{A_X}^L F_2)
  \]
  が成り立つ。
  また、本文命題 2.6.5 より、自然に
  \(f^{-1}(Rf_*F_1\otimes_A^L Rf_*F_2)\cong
  (f^{-1}Rf_*F_1)\otimes_{A_X}^L (f^{-1}Rf_*F_2)\)
  が成り立つ。
  本文 (2.6.17) より、自然な射
  \((f^{-1}Rf_*F_1)\otimes_{A_X}^L (f^{-1}Rf_*F_2)\to F_1\otimes_{A_X}^LF_2\)
  があり、以上より射
  \begin{align*}
    \Hom_{\sfD^+(A_X)}(F_1\otimes_{A_X}^LF_2,F_1\otimes_{A_X}^L F_2)
    &\to \Hom_{\sfD^+(A_X)}((f^{-1}Rf_*F_1)\otimes_{A_X}^L (f^{-1}Rf_*F_2)
    ,F_1\otimes_{A_X}^L F_2) \\
    &\cong \Hom_{\sfD^+(A_X)}(f^{-1}(Rf_*F_1\otimes_A^L Rf_*F_2), F_1\otimes_{A_X}^L F_2) \\
    &\cong \Hom_{\sfD^+(A)}(Rf_*F_1\otimes_A^L Rf_*F_2, Rf_*(F_1\otimes_{A_X}^L F_2))
  \end{align*}
  を得る。
  \(\id\)の行き先が射
  \(Rf_*F_1\otimes_A^L Rf_*F_2\to Rf_*(F_1\otimes_{A_X}^L F_2)\)を与える。
  以上で\(Z_1=Z_2=X\)の場合の1つ目の射の構成を完了する。
  一般の場合、\ref{2.17.1}の自然な射に対して\(R\Gamma(X,-)\)を適用し、
  \(Z_1=Z_2=X\)の場合に得られた射と合成することによって、射
  \begin{align*}
    R\Gamma_{Z_1}(X,F_1)\otimes_A^L R\Gamma_{Z_2}(X,F_2)
    &\cong R\Gamma(X,R\Gamma_{Z_1}(F_1))\otimes_A^L R\Gamma(X,R\Gamma_{Z_2}(F_2)) \\
    &\to R\Gamma(X,R\Gamma_{Z_1}(F_1)\otimes_{A_X}^L R\Gamma_{Z_2}(F_2)) \\
    &\to R\Gamma(X,R\Gamma_{Z_1\cap Z_2}(F_1\otimes_{A_X}^L F_2)) \\
    &\cong R\Gamma_{Z_1\cap Z_2}(X,F_1\otimes_{A_X}^L F_2)
  \end{align*}
  を得る。
  以上で一般の場合の1つ目の射の構成を完了する。
  二つ目の射は、\autoref{1.24} \ref{1.24.1} で
  \(F=\otimes_A, X=R\Gamma_{Z_1}(X,F_1), Y=R\Gamma_{Z_2}(X,F_2)\)
  とすることにより、自然な射
  \begin{align*}
    H^p_{Z_1}(X,F_1)\otimes_A H^q_{Z_2}(X,F_2)
    &\to H^{p+q}(R\Gamma_{Z_1}(X,F_1)\otimes_A^L R\Gamma_{Z_2}(X,F_2)) \\
    &\to H^{p+q}(R\Gamma_{Z_1\cap Z_2}(X,F_1\otimes_{A_X}^L F_2))
  \end{align*}
  を得る。これが所望の射である。
  以上で\ref{2.17.2}の証明を完了し、
  \autoref{2.17}の解答を完了する。
\end{proof}








\begin{prob}\label{2.18}
  \(S\)を位相空間、
  \(X_1,X_2,Y_1,Y_2\)を\(S\)上の局所コンパクトハウスドルフ空間、
  \(f_i:Y_i\to X_i,(i=1,2)\)を\(S\)上の射とする。
  \(p_{Y_i}:Y_i\to S\)を構造射として、
  \(f=f_1\times_Sf_2:Y_1\times_S Y_2 \to X_1\times_S X_2\)とおく。
  \(\mcR\)を\(S\)上の可換環の層で、
  \(\wgld(\mcR) < \infty\)と仮定する。
  \(G_i\in \sfD^+(p_{Y_i}^{-1}\mcR)\)とする。
  \begin{enumerate}
    \item \label{2.18.1}
    以下の同型射の存在を示せ:
    \[
    Rf_{1!}G_1 \boxtimes_{S,\mcR}^L Rf_{2!}G_2
    \xrightarrow{\sim} Rf_!(G_1 \boxtimes_{S,\mcR}^L G_2).
    \]
    この同型射は\textbf{K\"{u}nnethの公式}として知られている。
    \item \label{2.18.2}
    \(S=X_1=X_2=\{\mathrm{pt}\}\)として、
    \(\mcR\)を体とする。
    以下を示せ:
    \[
    H^n_c(Y_1\times Y_2,G_1\boxtimes G_2) \cong
    \bigoplus_{p+q=n}(H^p_c(Y_1,G_1)\otimes H^q_c(Y_2,G_2)).
    \]
  \end{enumerate}
\end{prob}

\begin{proof}
  \ref{2.18.1}を示す。
  以下のように射に名前をつける
  (それぞれの四角形は Cartesian である):
  \[
  \begin{CD}
    Y_1\times_S Y_2 @>{f_1''}>> X_1\times_S Y_2 @>{q_2'}>> Y_2 \\
    @V{f_2''}VV @VV{f_2'}V @VV{f_2}V \\
    Y_1\times_S X_2 @>{f_1'}>> X_1\times_S X_2 @>{q_2'}>> X_2 \\
    @V{q_1'}VV @VV{q_1}V @VVV \\
    Y_1 @>{f_1}>> X_1 @>>> S.
  \end{CD}
  \]
  また、\(r_1\dfn q_1'\circ f_2'', r_2\dfn q_2'\circ f_1''\)とおき、
  \(X_1\times_S X_2 \to S\)を\(g\)とおいて、
  \(h\dfn g\circ f\)とおく。
  示すべきことは、自然な同型射
  \[
  q_1^{-1}Rf_{1!}G_1 \otimes_{g^{-1}\mcR}^L q_2^{-1}Rf_{2!}G_2
  \xrightarrow{\sim} Rf_!(r_1^{-1}G_1\otimes_{h^{-1}\mcR}^L r_2^{-1}G_2)
  \]
  の存在であるが、それは以下のように示される:
  \begin{align}
    q_1^{-1}Rf_{1!}G_1 \otimes_{g^{-1}\mcR}^L q_2^{-1}Rf_{2!}G_2
    &\xrightarrow{\sim}
    Rf_{1!}'{q_1'}^{-1}G_1 \otimes_{g^{-1}\mcR}^L Rf_{2!}'{q_2'}^{-1}G_2
    \label{eq: 2.18.1.1} \\
    &\xrightarrow{\sim}
    Rf_{1!}'({q_1'}^{-1}G_1 \otimes_{{f_1'}^{-1}g^{-1}\mcR}^L
    {f_1'}^{-1} Rf_{2!}'{q_2'}^{-1}G_2)
    \label{eq: 2.18.1.2} \\
    &\xrightarrow{\sim}
    Rf_{1!}'({q_1'}^{-1}G_1 \otimes_{{f_1'}^{-1}g^{-1}\mcR}^L
    Rf_{2!}''{f_1''}^{-1}{q_2'}^{-1}G_2)
    \label{eq: 2.18.1.3} \\
    &= Rf_{1!}'({q_1'}^{-1}G_1 \otimes_{{f_1'}^{-1}g^{-1}\mcR}^L
    Rf_{2!}''r_2^{-1}G_2)
    \label{eq: 2.18.1.4} \\
    &\xrightarrow{\sim} Rf_{1!}'Rf_{2!}''
    ({f_2''}^{-1}{q_1'}^{-1}G_1 \otimes_{{f_2''}^{-1}{f_1'}^{-1}g^{-1}\mcR}^L r_2^{-1}G_2)
    \label{eq: 2.18.1.5} \\
    &\xrightarrow{\sim} Rf_!(r_1^{-1}G_1 \otimes_{h^{-1}\mcR}^L r_2^{-1}G_2),
    \label{eq: 2.18.1.6}
  \end{align}
  ただしここで、
  \eqref{eq: 2.18.1.1}の部分に本文命題 2.6.7 を用い、
  \eqref{eq: 2.18.1.2}の部分に本文命題 2.6.6 を用い、
  \eqref{eq: 2.18.1.3}の部分に本文命題 2.6.7 を用い、
  \eqref{eq: 2.18.1.4}の部分に等式\(r_2=q_2'\circ f_1''\)を用い、
  \eqref{eq: 2.18.1.5}の部分に本文命題 2.6.6 を用い、
  \eqref{eq: 2.18.1.6}の部分に等式
  \(f = f_1'\circ f_2'', r_1 = q_1'\circ f_2'', h = g\circ f_1'\circ f_2''\)を用いた。
  以上で\ref{2.18.1}の証明を完了する。

  \ref{2.18.2}を示す。
  \(\mcR\)は体なので、
  任意の\(\mcR\)-加群 (\(f^{-1}\mcR\)-加群) は平坦であり、
  従って\(\otimes^L\cong \otimes, \boxtimes^L\cong \boxtimes\)が成り立つ。
  また、\ref{2.18.1}で\(S=X_1=X_2=\{\mathrm{pt}\}\)とすることで、同型射
  \[
  R\Gamma_c(X,G_1)\otimes R\Gamma_c(X,G_2) \xrightarrow{\sim}
  R\Gamma_c(X,G_1\boxtimes G_2)
  \]
  を得る。
  ここで \autoref{1.24} \ref{1.24.2} を
  \(F=\otimes, X=R\Gamma_c(X,G_1), Y=R\Gamma_c(X,G_2)\)として適用することにより、
  \[
  \bigoplus_{p+q=n}(H^p_c(X,G_1)\otimes H^q_c(X,G_2))
  \cong H^n(R\Gamma_c(X,G_1)\otimes R\Gamma_c(X,G_2))
  \xrightarrow{\sim} H^n_c(X,G_1\boxtimes G_2)
  \]
  を得る。
  以上で\ref{2.18.2}の証明を完了し、
  \autoref{2.18}の解答を完了する。
\end{proof}

\begin{kansou*}
  \ref{2.18.1}の本文のヒント、何あれ??
\end{kansou*}






\begin{prob}\label{2.19}
  \(X\)を局所コンパクト空間として、
  \(A\)を可換環で\(\wgld(A)<\infty\)であるものとする。
  \(F\in \sfD^+(A_X)\)として、
  \(\Omega,Z\subset X\)をそれぞれ開集合と閉集合とする。
  \(a_X\)を一意的な射\(X\to \{\text{pt}\}\)とする。
  以下を示せ:
  \begin{align}
    &R\Gamma(\Omega,F) \cong Ra_{X*}R\inHom(A_{\Omega},F),
    \label{2.19.1} \\
    &R\Gamma_c(\Omega,F) \cong Ra_{X!}(A_{\Omega}\otimes^L F),
    \label{2.19.2} \\
    &R\Gamma_Z(X,F) \cong Ra_{X*}R\inHom(A_Z,F),
    \label{2.19.3} \\
    &R\Gamma_c(Z,F) \cong Ra_{X!}(A_Z\otimes^L F).
    \label{2.19.4}
  \end{align}
\end{prob}

\begin{proof}
  \eqref{2.19.1}と\eqref{2.19.3}を示す。
  \(F\)を入射的とすると、本文命題2.4.6 (vii) より、
  \(\inHom(-,F)\)は脆弱層である。
  従って、\(a_{X*}=\Gamma(X,-)\)に関して acyclic である。
  よって
  \[
  R(a_{X*}\circ \inHom(A_{\Omega},-)) \cong Ra_{X*}\circ R\inHom(A_{\Omega},-), \ \
  R(a_{X*}\circ \inHom(A_Z,-)) \cong Ra_{X*}\circ R\inHom(A_Z,-),
  \]
  が成り立つ。
  ここで\(\Omega\subset X\)は開であるので、
  \(a_{X*}\circ \inHom(A_{\Omega},-)\cong \Gamma(\Omega,-)\)が成り立ち、
  \(Z\subset X\)は閉であるので、
  \(a_{X*}\circ \inHom(A_Z,-)\cong \Gamma_Z(X,-)\)が成り立つ。
  よって
  \[
  R\Gamma(\Omega,-) \cong Ra_{X*}\circ R\inHom(A_{\Omega},-), \ \
  R\Gamma_Z(X,-) \cong Ra_{X*}\circ R\inHom(A_Z,-),
  \]
  が成り立つ。
  以上で\eqref{2.19.1}と\eqref{2.19.3}の証明を完了する。

  \eqref{2.19.2}と\eqref{2.19.4}を示す。
  \(Z\)を局所閉集合とする。
  \(A_Z\)は \(A_X\)-flat なので、
  \(A_Z\otimes^L (-)\cong A_Z\otimes (-)\)
  が成り立つ。
  また、本文命題 2.3.10 より、
  \(A_Z\otimes (-)\cong (-)_Z\)が成り立つ。
  さらに、\((-)_Z\)は完全函手であるので、従って、
  \[Ra_{X!}\circ (-)_Z \cong R(a_{X!}\circ (-)_Z) \cong R\Gamma_c(Z,-)\]
  が成り立つ。
  \(Z\)を開または閉とすることにより、
  \eqref{2.19.2}と\eqref{2.19.4}が従う。
  以上で\autoref{2.19}の解答を完了する。
\end{proof}








\begin{prob}\label{2.20}
  \(A\)を可換環で、\(\wgld(A)<\infty\)であるものとする。
  \(E\)を有限次元実線形空間とする。
  \(s:E\times E \to E\)を足し算写像とし、
  \(F,G\in \sfD^+(A_E)\)に対して
  \(F*G \dfn Rs_!(F\boxtimes^LG)\)と定める。
  これを\(\sfD^+(A_E)\)上の
  \textbf{convolution作用素}という。
  \begin{enumerate}
    \item \label{2.20.1}
    \(F,G,H\in \sfD^+(A_E)\)に対し、
    \(F*G\cong G*F, F*(G*H)\cong (F*G)*H, A_{\{0\}}*F\cong F\)
    が成り立つことを示せ。
    \item \label{2.20.2}
    \(Z_1,Z_2\subset E\)をコンパクト凸集合とする。
    \(A_{Z_1}*A_{Z_2}\cong A_{Z_1+Z_2}\)であることを示せ。
    \item \label{2.20.3}
    \(\gamma\)を proper closed convex cone とするとき、
    \(A_{\gamma}*A_{\Int(\gamma)} = 0\)であることを示せ。
    \item \label{2.20.4}
    \(E = \R^n\)であると仮定せよ。
    \(Z_1 \dfn [-1,1]^n, Z_2\dfn (-1,1)^n\)とする。
    \(A_{Z_1}*A_{Z_2}\cong A_{\{0\}}[-n-1]\)であることを示せ。
  \end{enumerate}
\end{prob}

\begin{rem*}
  \ref{2.20.3}で「proper cone」の意味がよくわからなかった
  (本文に定義書いてましたっけ...) ので
  \href{https://ja.wikipedia.org/wiki/%E5%87%B8%E9%8C%90}{Wikipedia}
  を参考にして次の性質を満たす錐\(\gamma\)のことと解釈しました:
  \begin{itemize}
    \item \(\Int(\gamma) \neq \emptyset\)である。
    \item \(\{x,-x\}\in \gamma\)ならば\(x=0\)である
    (つまり\href{https://ja.wikipedia.org/wiki/%E5%87%B8%E9%8C%90}{Wikipedia}
    で\textbf{突錐}と呼ばれているものである)。
  \end{itemize}
  以下の解答ではこれら二つの条件はどちらも\ref{2.20.3}を解くのに用いられますが、
  別の解法で突錐であることを仮定せずとも\ref{2.20.3}が解けるのであれば、気になります。
\end{rem*}

\begin{rem*}
  \ref{2.20.4}は本文ではシフトが\(-n\)になっていたけど、
  \(-n-1\)な気がします。気のせいでしょうか。
\end{rem*}

\begin{proof}
  \ref{2.20.1}を示す。
  \(p_1,p_2:E\times E\to E\)を第一射影、第二射影として、
  \(p:E\times E\xrightarrow E\times E\)を
  成分を入れ替えることによって得られる同相写像とする。
  まず\(F*G\cong G*F\)を示す。
  \(p\)は同相写像であるので、\(Rp_! \cong Rp_*\cong p^{-1}\)が成り立つ。
  \(s = s\circ p, p_2 = p_1\circ p, p_1 = p_2\circ p\)であるので、従って、
  \begin{align*}
    Rs_!(p_1^{-1}F\otimes^L p_2^{-1}G)
    &\cong Rs_!Rp_!(p_1^{-1}F\otimes^L p_2^{-1}G) \\
    &\cong Rs_!p^{-1}(p_1^{-1}F\otimes^L p_2^{-1}G) \\
    &\cong Rs_!(p^{-1}p_1^{-1}F\otimes^L p^{-1}p_2^{-1}G) \\
    &\cong Rs_!(p_2^{-1}F\otimes^L p_1^{-1}G) \\
    &\cong Rs_!(p_1^{-1}G\otimes^L p_2^{-1}F)
  \end{align*}
  が成り立つ。
  以上で\(F*G\cong G*F\)が示された。

  次に\(F*(G*H)\cong (F*G)*H\)を示す。
  \(q_{ij}:E\times E\times E \to E\times E\)を第\(ij\)成分への射影とし、
  \(q_i:E\times E\times E \to E\)を第\(i\)成分への射影とする。
  \(\bar{s}:E\times E\times E \to E\)を足し算写像とする。
  このとき、図式
  \[
  \begin{CD}
    E\times E\times E @>{\id\times s}>> E\times E \\
    @V{q_{23}}VV @VV{p_2}V \\
    E\times E @>{s}>> E
  \end{CD}
  \]
  は Cartesian である。
  従って自然な同型射
  \(p_2^{-1}\circ Rs_!\xrightarrow{\sim} R(\id\times s)_!\circ q_{23}^{-1}\)
  が存在する。
  よって、
  \begin{align}
    F*(G*H)
    &= Rs_!(p_1^{-1}F\otimes^L p_2^{-1}Rs_!(p_1^{-1}G\otimes^L p_2^{-1}H)) \notag \\
    &\xrightarrow{\sim}
    Rs_!(p_1^{-1}F\otimes^L R(\id\times s)_!q_{23}^{-1}(p_1^{-1}G\otimes^L p_2^{-1}H))
    \notag \\
    &\xrightarrow{\sim}
    Rs_!(p_1^{-1}F\otimes^L R(\id\times s)_!
    (q_{23}^{-1}p_1^{-1}G\otimes^L q_{23}^{-1}p_2^{-1}H)) \notag \\
    &\cong Rs_!(p_1^{-1}F\otimes^L R(\id\times s)_!(q_2^{-1}G\otimes^L q_3^{-1}H))
    \label{eq: 2.20.1.1} \\
    &\xrightarrow{\sim}
    Rs_!R(\id\times s)_!((\id\times s)^{-1} p_1^{-1}F
    \otimes^L q_2^{-1}G\otimes^L q_3^{-1}H)
    \label{eq: 2.20.1.2} \\
    &\cong R\bar{s}_!(q_1^{-1}F \otimes^L q_2^{-1}G\otimes^L q_3^{-1}H)
    \label{eq: 2.20.1.3}
  \end{align}
  が成り立つ。
  ただしここで
  \eqref{eq: 2.20.1.1}の箇所に等式
  \(p_1\circ q_{23} = q_2, p_2\circ q_{23} = q_3\)を用い、
  \eqref{eq: 2.20.1.2}の箇所に本文命題 2.6.6 を用い、
  \eqref{eq: 2.20.1.3}の箇所に等式
  \(s\circ (\id\times s) = \bar{s}, p_1\circ (\id\times s) = q_1\)を用いた。
  同様に
  \((F*G)*H\cong R\bar{s}_!(q_1^{-1}F \otimes^L q_2^{-1}G\otimes^L q_3^{-1}H)\)
  が従う。
  以上より\(F*(G*H)\cong (F*G)*H\)が成り立つ。

  次に\(A_{\{0\}}*F\cong F\)を示す。
  \(i:E\cong \{0\}\times E \to E\times E\)を包含射とする。
  \(i\)は閉部分集合の上への同相写像なので、\(i_!\)は完全函手である
  (cf. 本文命題2.5.4 (i))。
  従って、
  \begin{align}
    A_{\{0\}}*F
    &= Rs_!(p_1^{-1}A_{\{0\}}\otimes^L p_2^{-1}F) \notag \\
    &= Rs_!(A_{\{0\}\times E}\otimes^L p_2^{-1}F) \notag \\
    &\cong Rs_!((p_2^{-1}F)_{\{0\}\times E}) \label{eq: 2.20.1.4} \\
    &\xrightarrow{\sim}
    Rs_!i_!(i^{-1}p_2^{-1}F) \label{eq: 2.20.1.5} \\
    &\xrightarrow{\sim} F \label{eq: 2.20.1.6}
  \end{align}
  が成り立つ。
  ただしここで
  \eqref{eq: 2.20.1.4}の箇所に本文命題2.3.10と
  \(A_{\{0\}\times E}\)が \(A_{E\times E}\)-flat であることを用い、
  \eqref{eq: 2.20.1.5}の箇所に本文命題2.5.4 (ii)を用い、
  \eqref{eq: 2.20.1.6}の箇所に等式
  \(s\circ i = \id_E, p_2\circ i = \id_E\)を用いた。
  以上で\ref{2.20.1}の証明を完了する。

  \ref{2.20.2}を示す。
  \(A_{Z_i}\)は \(A_E\)-flat なので、
  \(p_1^{-1}A_{Z_1} \cong A_{Z_1\times E}\)と
  \(p_2^{-1}A_{Z_2} \cong A_{E\times Z_2}\)も
  \(A_{E\times E}\)-flat である。
  従って
  \[
  p_1^{-1}A_{Z_1}\otimes^L p_2^{-1}A_{Z_2} \cong
  A_{Z_1\times E}\otimes A_{E\times Z_2}
  \cong A_{(Z_1\times E)\cap (E\times Z_2)}
  = A_{Z_1\times Z_2}
  \]
  が成り立つ。
  \(p:Z_1\times Z_2\to Z_1+Z_2\)を\(s\)の制限 (足し算写像) とする。
  \(A_{Z_1\times Z_2} \cong p^{-1}A_{Z_1+Z_2}\)が成り立つ。
  \(p\)はコンパクト空間からコンパクト空間への射なので固有である。
  従って\(p_!=p_*\)が成り立つ。
  \(Z_1,Z_2\)は凸であるので、
  \(Z_1\times Z_2\subset E\times E\)はコンパクト凸集合である。
  従って、任意の点\(z\in Z_1+Z_2\)に対して、
  \(p^{-1}(z) = (Z_1\times Z_2)\cap s^{-1}(z)\)
  はコンパクト凸集合と閉凸集合の共通部分であり、
  再びコンパクト凸集合、とくに可縮となる。
  すなわち、\(p\)の各 fiber は可縮である。
  よって、\(i:Z_1+Z_2\to E\)を包含射とすれば、本文系2.7.7.(iv) より、
  自然な射
  \(i^{-1}A_E\xrightarrow{\sim} Rp_*p^{-1}i^{-1}A_E \cong Rp_!p^{-1}i^{-1}A_E\)
  は同型射である。
  \(j:Z_1\times Z_2\to E\times E\)を包含射とすれば、
  \(i\circ p = s\circ j\)であるから、
  従って、とくに
  \[
  A_{Z_1+Z_2}\cong i_!i^{-1}A_E
  \cong i_!(Rp_!p^{-1}i^{-1}A_E)
  \cong R(s\circ j)_!(s\circ j)^{-1}A_E
  \cong Rs_!A_{Z_1\times Z_2}
  \cong A_{Z_1}*A_{Z_2}
  \]
  が成り立つ。
  以上で\ref{2.20.2}の証明を完了する。

  \ref{2.20.3}を示す。
  \(p_1,p_2:E\times E\to E\)を第一、第二射影とする。
  \(p_1^{-1}A_{\gamma}\otimes^L p_2^{-1}A_{\Int(\gamma)}
  \cong A_{\gamma\times \Int(\gamma)}\)である。
  \ref{2.20.3}を示すためには、
  \(Rs_!A_{\gamma\times \Int(\gamma)} = 0\)を示すことが十分である。
  各\(z\in E\)に対して
  \(i:E\cong s^{-1}(z) \to E\times E\)を包含射とする。
  このとき、本文命題2.6.7より、
  \((Rs_!A_{\gamma\times \Int(\gamma)})_z \cong
  R\Gamma_c(s^{-1}(z), A_{\gamma\times \Int(\gamma)}|_{s^{-1}(z)})\)
  が成り立つ。
  \autoref{2.19} \eqref{2.19.2} \eqref{2.19.4}の証明で行ったように、
  \(Z\)が局所閉集合である場合にも\autoref{2.19} \eqref{2.19.4}の等式が成立する。
  従って、本文 Remark 2.6.9 (iii) より、
  \begin{align*}
    R\Gamma_c(s^{-1}(z), A_{\gamma\times \Int(\gamma)}|_{s^{-1}(z)}) &\cong
    R\Gamma_c(s^{-1}(z)\cap (\gamma\times \Int(\gamma)), A_{s^{-1}(z)}) \\
    &\cong R\Gamma_c(s^{-1}(z)\cap (\gamma\times \Int(\gamma)),
    A_{s^{-1}(z)\cap (\gamma\times \Int(\gamma))})
  \end{align*}
  が成り立つ。
  従って、
  \(Rs_!A_{\gamma\times \Int(\gamma)} = 0\)を示すためには、
  各\(z\in E\)に対して
  \(R\Gamma_c(s^{-1}(z)\cap (\gamma\times \Int(\gamma)),
  A_{s^{-1}(z)\cap (\gamma\times \Int(\gamma))}) = 0\)
  であることを示すことが十分である。
  ここで\(s^{-1}(z)\cap (\gamma\times \Int(\gamma)) = \emptyset\)であれば
  明らかにこの等式が成り立つので、
  以下、\(s^{-1}(z)\cap (\gamma\times \Int(\gamma)) \neq \emptyset\)であると仮定して
  \(R\Gamma_c(s^{-1}(z)\cap (\gamma\times \Int(\gamma)),
  A_{s^{-1}(z)\cap (\gamma\times \Int(\gamma))}) = 0\)を示す。
  このとき、成分ごとに足すことによって\(z\in \Int(\gamma)\)であることが従う。
  簡単のため
  \(X \dfn s^{-1}(z)\cap (\gamma\times \Int(\gamma))\)とおく。
  示すべきことは\(R\Gamma_c(X,A_X)=0\)である。

  \(a\in \Int(\gamma)\)を一つとり、以下固定する。
  \(K_n \dfn (\gamma \times (a/n + \gamma)) \cap X \subset X\)とおく。
  このとき、\(X = \bigcup_{n\in \N}K_n\)が成り立つ。
  \(K_n\)に関して以下を主張を示す:
  \begin{enumerate}[label=(\fnsymbol*),start=2]
    \item \label{2.20.3.p1}
    \(K_n\)はコンパクトである。
    \item \label{2.20.3.p2}
    \(X\setminus K_n\)は可縮である。
  \end{enumerate}
  \ref{2.20.3.p1}を示す。
  もし\(K_n\)がコンパクトでなければ、点列コンパクトでないので、
  \(K_n\)内で収束しない点列\(v_i = (w_i,z-w_i)\in K_n\)が存在する。
  もし数列\(\| w_i\|\)が\(N\)で抑えられるとすれば、
  \(\gamma\)は閉であるから、\(\gamma\cap [-N,N]^{\dim E}\)はコンパクトであり、
  また、\(w_i\in \gamma\cap [-N,N]^{\dim E}\)であるので、
  従って\(w_i\)は\(\gamma\cap [-N,N]^{\dim E}\)内で収束する。
  これは\(v_i\)が\(K\)内で収束しないということに反する。
  従って\(\| w_i\|\)は非有界である。
  \(\gamma\)内の点列\(w_i/\|w_i\|\in \gamma\)と
  \((z-w_i)/\|w_i\|\)はノルムが有界なので\(\gamma\)内で収束する。
  \(w \dfn \lim w_i/\|w_i\|\in \gamma\)とおく。
  ここで\(\|w_i\| \to \infty\)であるから
  \(z/\|w_i\|\to 0\)であり、
  \((z-w_i)/\|w_i\| \to -w\in \gamma\)が成り立つ。
  一方、\(\gamma\)は突であるので、これは\(w = 0\)を意味する。
  しかしながら、\(\|w_i/\|w_i\|\| = 1\)であるため、\(\|w\|=1\)であり、
  これは矛盾している。
  以上より\(K_n\)は点列コンパクトである。
  今、\(K_n\)は有限次元実線形空間の部分空間なので、
  \(K_n\)はコンパクトである。

  \ref{2.20.3.p2}を示す。
  \(z\in \Int(\gamma)\)であるので、
  十分大きい\(N\gg n+1\)をとれば、
  \(z-a/N\in \gamma\)が成り立つ。
  \(a/N\not\in (a/n+\gamma)\)であるので、従って
  \((z-a/N,a/N)\in X\setminus K_n\)である。
  点\(v = (v_1,v_2)\in X\setminus K_n\)を任意にとる。
  このとき、\(v_2\not\in (a/n+\gamma)\)が成り立つ。
  ある\(t,u>0, t+u = 1\)が存在して、
  \(ta/N + uv_2 \in (a/n+\gamma)\)が成り立つと仮定する。
  このとき、ある\(v_3\in \gamma\)が存在して、
  \(ta/N + uv_2 = a/n + v_3\)が成り立つ。
  整理すれば、
  \begin{align*}
    uv_2 &= \frac{u}{n}a + \frac{1-u}{n}a + v_3 - \frac{t}{N}a \\
    v_2 &= \frac{1}{n}a + \frac{1}{u}\left( v_3 +
    \left(\frac{t}{n} - \frac{t}{N}\right)a \right)
  \end{align*}
  となるので、\(N \gg n+1\)であることから、
  \(v_2\in (a/n + \gamma)\)が従う。これは矛盾である。
  よって\(v_2\)と\(a/N\)を結ぶ線分は\(a/n+\gamma\)と交わらない。
  従って\(v=(v_1,v_2)\)と\((z-a/N,a/N)\)を結ぶ線分は
  \(K_n\)と交わらず、
  \(X\setminus K_n\)は星状であることが従う。
  よってとくに\(X\setminus K_n\)は可縮である。

  \(R\Gamma_c(X,A_X) = 0\)を示す。
  コンパクト部分集合\(K\subset X\)の集合は
  包含関係に関して有向集合であり、
  \(\{K_n | n\in \N\}\)はそのcofinalな部分集合をなす。
  従って、本文 Notations 2.6.8 の最後の記述より、
  任意の\(F\in \Ab(X)\)に対して
  \(H^j_c(X,F)\cong \colim_n H^j_{K_n}(X,F)\)
  が成り立つ。
  \(X\)は可縮であり、
  さらに十分大きな\(n\)に対して\(X\setminus K_n\)も可縮であるので、
  本文系 2.7.7 (iii)より、
  十分大きな\(n\)に対して
  \(A\cong R\Gamma(X,A_X)\cong R\Gamma(X\setminus K_n, A_{X\setminus K_n})\)
  が成り立つ。
  従って任意の\(i\)に対して
  \(H^i(X,A_X) \to H^i(X\setminus K_n, A_X)\)は同型射である
  (\(i=0\)の場合は\(\id_A\)で、他の次数ではどちらも\(0\))。
  よって任意の\(i\)に対して\(H^i_{K_n}(X,A_X)=0\)が従い、
  とくに\(H^i_c(X,A_X)=0\)が成り立つ。
  これは\(R\Gamma_c(X,A_X)=0\)を意味する。
  以上で\ref{2.20.3}の証明を完了する。

  \ref{2.20.4}を示す。
  \(p_1,p_2:E\times E\to E\)を第一、第二射影とする。
  \(p_1^{-1}A_{Z_1}\otimes^L p_2^{-1}A_{Z_2}\cong A_{Z_1\times Z_2}\)である。
  \(Rs_!A_{Z_1\times Z_2}\)を計算しなければならない。
  \(z\in E\)を任意にとる。
  \(S(z) = s^{-1}(z)\cap (Z_1\times Z_2)\)とおく。
  \(Rs_!A_{Z_1\times Z_2}\cong R\Gamma_c(S(z),A_{S(z)})\)
  である。
  従って、\ref{2.20.4}を示すためには、
  \((z,i)\neq (0,n)\)に対して\(H^i_c(S(z),A_{S(z)}) = 0\)であり、
  \(H^n_c(S(z),A_{S(z)})\cong A\)であることを示すことが十分である。
  \(S\subset E\)に対して
  \(z+S = \{z+v | v\in S\}\)とおく。
  \(v = (v_1,v_2)\in S(z)\)は
  \(v_1+v_2 = z, v_1\in Z_1,v_2\in Z_2\)を満たす。
  従って、\(v_1-z = -v_2 \in Z_2\)が成り立つ
  (\(Z_2\)は原点対称であることに注意)。
  すなわち、\(v_1\in Z_1\cap (z+Z_2)\)が成り立つ。
  よって\(Z_1\cap (z+Z_2) \to S(z), v_1\mapsto (v_1,z-v_1)\)
  は同相写像である。
  これにより\(S(z)\)を\(Z_1\cap (z+Z_2)\)と同一視する。
  \(S_k(z) \dfn Z_1\cap (z+[-1+1/k,1-1/k]^n)\)とおく。
  \(S(z) = \bigcup_{k\in \N}S_k(z)\)が成り立つ。
  また、\(S_k(z)\)はコンパクト空間二つの共通部分であるから、コンパクトである。

  \(z\neq 0\)に対して\(H^i_c(S(z),A_{S(z)})=0\)を示す。
  \(1/k_0 < \min\{|z_1|,\cdots,|z_n|\}\)となる\(k_0\)をとれば、
  任意の\(k \geq k_0\)に対して、
  \(S(z)\setminus S_k(z)\)は、
  \(z_i\neq 0\)となる座標を\(z_i/|z_i|\)側へと潰すホモトピーによって、
  可縮である。
  また、\(S(z) = \bigcup_{k\geq k_0}S_k(z)\)であるので、
  本文 Notations 2.6.8 の最後の記述より、
  \(H^i_c(S(z),A_{S(z)}) \cong \colim_{k\geq k_0} H^i_{S_k(z)}(S(z),A_{S(z)})\)
  が成り立つ。
  さらに、\(S(z)\)と\(S(z)\setminus S_k(z)\)はともに可縮であるから、
  本文系 2.7.7 (iii)より、
  \(R\Gamma(S(z),A_{S(z)})\cong
  R\Gamma(S(z)\setminus S_k(z),A_{S(z)\setminus S_k(z)})\cong A\)
  が成り立つ。
  従って、\(R\Gamma_{S_k(z)}(S(z),A_{S_k(z)})\cong 0\)であり、
  とくに\(H^i_{S_k(z)}(S(z),A_{S(z)}) = 0\)である。
  よって\(H^i_c(S(z),A_{S(z)}) = 0\)が従う。

  \(z=0\)とする。
  \(S(0)\setminus S_k(0)\)は\(n\)次元球面\(S^n\)とホモトピックであり、
  Mayer-Vietoris完全列 (cf. 本文 Remark 2.6.10) と
  本文系 2.7.7 (iii)を用いて、帰納法により、
  \(R\Gamma(S(0)\setminus S_k(0),A_{S(0)\setminus S_k(0)})\cong A\oplus A[-n]\)
  が従う。
  \(S(0)\)は可縮なので、系 2.7.7 (iii)より
  \(R\Gamma(S(0),A_{S(0)})\cong A\)である。
  以上より、\(R\Gamma_{S_k(0)}(S(0),A_{S(0)}) \cong A[-n-1]\)が成り立つ。
  従って、とくに
  \(i\neq n+1\)に対して
  \(H^i_{S_k(0)}(S(0),A_{S(0)})\cong 0\)であり、
  \(i=n+1\)に対しては\(H^{n+1}_{S_k(0)}(S(0),A_{S(0)})\cong A\)である。
  \(S(0)=\bigcup_{k\in \N}S_k(0)\)であるから、
  任意の\(i\)に対して
  \(H^i_c(S(0),A_{S(0)})\cong \colim_{k\in \N}H^i_{S_k(0)}(S(0),A_{S(0)})\)
  であり、
  よって\(H^i_c(S(0),A_{S(0)}) = 0, (i\neq n+1)\)と
  \(H^{n+1}_c(S(0),A_{S(0)}) \cong A\)が成り立つ。
  以上で\ref{2.20.4}の証明を完了し、
  \autoref{2.20}の解答を完了する。
\end{proof}









\begin{prob}\label{2.21}
  \(X\)を位相空間、\((X_n)_{n\in \N}\)を\(X\)の閉部分集合の減少列で
  \(X_n=X, (n \ll 0)\)と\(\bigcap_{n\in \N}X_n = \emptyset\)
  を満たすものとする。
  \(F\in \sfD^+(X)\)は\(k\neq n\)に対して
  \(H^k_{X_n\setminus X_{n+1}}(F)=0\)を満たすとする。
  完全三角
  \[
  R\Gamma_{X_{n+1}\setminus X_{n+2}}(F) \to
  R\Gamma_{X_n\setminus X_{n+2}}(F) \to
  R\Gamma_{X_n\setminus X_{n+1}}(F) \xrightarrow{+1}
  \]
  のコホモロジーをとって、連結準同型を
  \(d^n:H^n_{X_n\setminus X_{n+1}}(F) \to H^{n+1}_{X_{n+1}\setminus X_{n+2}}(F)\)
  と表す。
  \(K^n\dfn H^n_{X_n\setminus X_{n+1}}(F)\)と表す。
  \begin{enumerate}
    \item \label{2.21.1}
    \((K^{\bullet},d^{\bullet})\)は\(X\)上の層の複体であることを示せ。
    \item \label{2.21.2}
    \(k<n\)に対して\(H^k_{X_n}(F)=0\)であり、
    さらに\(H^n_{X_{n-1}}(F) \xrightarrow{\sim} H^n(F)\)は同型射であることを示せ。
    \item \label{2.21.3}
    \(G^n = \Gamma_{X_n}(F^n)\cap (d^n_F)^{-1}(\Gamma_{X_{n+1}}(F^{n+1}))\)
    とおく。
    射\(d_G^n:G^n\to G^{n+1}\)を構成して、
    \(G=(G^{\bullet},d^{\bullet})\)が複体であることを示せ。
    さらに\(G\to K\)と\(G\to F\)を構成して、
    各\(F^n\)が脆弱層である場合に擬同型となることを示せ。
    \(\sfD^+(X)\)において\(F\cong K\)であることを結論付けよ。
  \end{enumerate}
\end{prob}

\begin{proof}
  \ref{2.21.1}を示す。
  明らかに以下の図式が可換である:
  \[
  \begin{CD}
    0 @>>> \Gamma_{X_{n+2}\setminus X_{n+3}}(-)
    @>>> \Gamma_{X_n\setminus X_{n+3}}(-)
    @>>> \Gamma_{X_n\setminus X_{n+2}}(-) \\
    @. @VVV @| @VVV \\
    0 @>>> \Gamma_{X_{n+1}\setminus X_{n+3}}(-)
    @>>> \Gamma_{X_n\setminus X_{n+3}}(-)
    @>>> \Gamma_{X_n\setminus X_{n+1}}(-).
  \end{CD}
  \]
  従って、完全三角の間の射
  \[
  \begin{CD}
    R\Gamma_{X_n\setminus X_{n+3}}(-)
    @>>> R\Gamma_{X_n\setminus X_{n+2}}(-)
    @>>> R\Gamma_{X_{n+2}\setminus X_{n+3}}(-)[1]
    @>{+1}>> \\
    @| @VVV @VVV @. \\
    R\Gamma_{X_n\setminus X_{n+3}}(-)
    @>>> R\Gamma_{X_n\setminus X_{n+1}}(-)
    @>>> R\Gamma_{X_{n+1}\setminus X_{n+3}}(-)[1]
    @>{+1}>>
  \end{CD}
  \]
  を得る。
  縦に伸ばして横向きに書けば、完全三角の射
  \[
  \begin{CD}
    R\Gamma_{X_n\setminus X_{n+2}}(-)
    @>>> R\Gamma_{X_n\setminus X_{n+1}}(-)
    @>>> R\Gamma_{X_{n+1}\setminus X_{n+2}}(-)[1]
    @>{+1}>> \\
    @VVV @VVV @| @. \\
    R\Gamma_{X_{n+2}\setminus X_{n+3}}(-)[1]
    @>>> R\Gamma_{X_{n+1}\setminus X_{n+3}}(-)[1]
    @>>> R\Gamma_{X_{n+1}\setminus X_{n+2}}(-)[1]
    @>{+1}>>
  \end{CD}
  \]
  を得る。
  \(n\)次と\(n+1\)次の周辺でコホモロジーをとれば、可換図式
  \[
  \begin{CD}
    @>>> H^n_{X_n\setminus X_{n+1}}(-)
    @>{d^n}>> H^{n+1}_{X_{n+1}\setminus X_{n+2}}(-)
    @>>> H^{n+1}_{X_n\setminus X_{n+2}} (-)
    @>>> \\
    @. @VVV @| @VVV @. \\
    @>>> H^{n+1}_{X_{n+1}\setminus X_{n+3}}(-)
    @>>> H^{n+1}_{X_{n+1}\setminus X_{n+2}}(-)
    @>{d^{n+1}}>> H^{n+2}_{X_{n+2}\setminus X_{n+3}} (-)
    @>>>
  \end{CD}
  \]
  を得る。
  横向きは完全であるから、
  \(d^{n+1}\circ d^n = 0\)が従う。
  以上で\ref{2.21.1}の証明を完了する。

  \ref{2.21.2}を示す。
  完全三角
  \[
  R\Gamma_{X_{i+1}}(F) \to R\Gamma_{X_i}(F) \to
  R\Gamma_{X_i\setminus X_{i+1}}(F) \xrightarrow{+1}
  \]
  でコホモロジーをとる。
  各\(k < i\)に対して\(H^k_{X_i\setminus X_{i+1}}(F)\cong 0\)であるので、
  各\(k < i\)に対して同型射
  \(H^k_{X_{i+1}}(F) \xrightarrow{\sim} H^k_{X_i}(F)\)を得る。
  \(i\geq n\)としてこの同型射を繋ぐことによって、
  各\(k < n\)に対して同型射
  \(H^k_{X_i}(F) \xrightarrow{\sim} H^k_{X_n}(F), (i \gg 0)\)を得る。
  \(x\in X\)を任意にとれば、
  \(\bigcap_{i\in \N}X_i = \emptyset\)であるので、
  ある\(i\gg 0\)が存在して\(x\in X_i\)となる。
  点\(x\)で stalk をとることによって、
  \(0 = H^k_{X_i}(F)_x \xrightarrow{\sim} H^k_{X_n}(F)_x\)を得る
  (\(H^k_{X_i}(F)\)は\(X_i\)の上に台を持つ)。
  \(H^k_{X_n}(F)\)は任意の点の stalk が\(0\)であるので、
  \(H^k_{X_n}(F) = 0\)が従う。
  これが示すべきことの一つ目である。
  また、各\(k > i\)に対して\(H^k_{X_i\setminus X_{i+1}}(F)\cong 0\)であるので、
  各\(k > i+1\)に対して同型射
  \(H^k_{X_{i+1}}(F) \xrightarrow{\sim} H^k_{X_i}(F)\)を得る。
  \(k=n\)として\(i\leq n-2\)とすれば、この同型射を繋ぐことにより、
  同型射\(H^n_{X_{n-1}}(F) \xrightarrow{\sim} H^n_{X_i}(F), (i\ll 0)\)を得る。
  \(X_i = X, (i\ll 0)\)であるので、
  同型射\(H^n_{X_{n-1}}(F) \xrightarrow{\sim} H^n(F)\)を得る。
  これが示すべきことの二つ目である。
  以上で\ref{2.21.2}の証明を完了する。

  \ref{2.21.3}を示す。
  自然な包含射を\(i^n:\Gamma_{X_n}(F^n) \to F^n\)とおく。
  \(G^n\)の定義より、
  \[
  \begin{CD}
    G^n @>{q^n}>> \Gamma_{X_{n+1}}(F^{n+1}) \\
    @V{p^n}VV @VV{i^{n+1}}V \\
    \Gamma_{X_n}(F^n) @>{i^n\circ d^n}>> F^{n+1}
  \end{CD}
  \]
  は pull-back 図式である。
  また、\(p^n\)はモノ射である。
  さらに、
  \[
  d^{n+1}\circ i^{n+1}\circ q^n
  = d^{n+1}\circ d^n \circ i^n \circ p^n
  = 0
  \]
  であるので、
  \(i^{n+1}\circ q^n:G^n \to \Gamma_{X_{n+1}}(F^{n+1})\)と
  \(0\)-射\(G^n\to \Gamma_{X_{n+2}}(F^{n+2})\)は
  \(p^{n+1}\circ d_G^n = q^n, q^{n+1}\circ d_G^n=0\)となる
  射\(d_G^n:G^n\to G^{n+1}\)を一意的に定義する。
  このとき、
  \begin{align*}
    p^{n+2}\circ d_G^{n+1}\circ d_G^n
    &= q^{n+1}\circ d_G^n = 0, \\
    q^{n+2}\circ d_G^{n+1}\circ d_G^n
    &= 0\circ d_G^n = 0
  \end{align*}
  が成り立つ。
  従って\(d_G^{n+1}\circ d_G^n = 0\)であり、
  \((G^{\bullet},d_G^{\bullet})\)は層の複体である。
  また、\(p^n\circ i^n : G^n\to F^n\)は
  複体の射\(G\to F\)を与える。

  各\(F^n\)が脆弱層であるとする。
  このとき任意の局所閉集合\(?\subset X\)に対して
  \(R\Gamma_{?}(F)\cong \Gamma_{?}(F)\)が成り立つ。
  \(i^{n+1}\)はモノなので、
  \[
  \ker(d_G^n) = \ker(i^{n+1}\circ d_G^n) = \ker(d^n\circ p^n \circ i^n)
  = \ker(d^n)\cap \Gamma_{X_n}(F^n) = \Gamma_{X_n}(\ker(d^n))
  \]
  が成り立つ。
  定義より
  \(\im(d_G^{n-1}) = \Gamma_{X_n}(F^n)\cap \im(d^{n-1}) = \Gamma_{X_n}(\im(d^{n-1}))\)
  が成り立つ。
  従って、
  \(H^n(G) \cong \Gamma_{X_n}(\ker(d^n))/\Gamma_{X_n}(\im(d^{n-1}))\)
  であり、さらに
  \[
  \begin{CD}
    \im(d_G^{n-1}) @>>> \im(d^{n-1}) \\
    @VVV @VVV \\
    \ker(d_G^n) @>>> \ker(d^n)
  \end{CD}
  \]
  は Cartesian である。
  よって、\autoref{1.6} \ref{1.6.3}より、
  \(H^n(G) \to H^n(F)\)は単射である。
  また、複体の完全列
  \[
  0\to \Gamma_{X_n}(F) \to \Gamma_{X_{n-1}}(F) \to
  \Gamma_{X_n\setminus X_{n-1}}(F) \to 0
  \]
  でコホモロジーをとることにより、
  完全列
  \[
  H^n_{X_n}(F) \to H^n_{X_{n-1}}(F) \to H^n_{X_{n-1}\setminus X_n}(F)
  \]
  を得る。
  ここで仮定より、\(H^n_{X_{n-1}\setminus X_n}(F) = 0\)であり、
  さらに\ref{2.21.2}より、\(H^n_{X_{n-1}}(F)\cong H^n(F)\)であるので、
  \(H^n_{X_n}(F) \to H^n(F)\)は全射である。
  一方、\(\Im(d^{n-1})) \cong \Gamma_{X_n}(F^n)\times_{} \)
  層の複体の完全列
  \[
  0 \to \Gamma_{X_n}(F) \to F\to \Gamma_{X\setminus X_n}(F) \to 0
  \]

\end{proof}

\begin{kansou*}
  \ref{2.21.3}は、filtered complex のスペクトル系列の特別な場合。
\end{kansou*}

\newpage
\section{Poincar\'{e}-Verdier duality and Fourier-Sato transformation}

\begin{thebibliography}{9}
  \bibitem[Ha]{Ha}
  R.Hartshorne,
  \textit{Algebraic Geometry},
  Springer-Verlag,
  New York, 1977.
  Graduate Text in Mathematics, No. 52.
\end{thebibliography}

\end{document}
