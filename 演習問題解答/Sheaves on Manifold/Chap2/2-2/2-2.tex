\ifcsname Chap\endcsname\else
\documentclass[uplatex,dvipdfmx]{jsarticle}
\newcommand{\StylePath}{\ifcsname AllKS\endcsname KS-Style/KS-Style.sty\else
\ifcsname Chap\endcsname ../KS-Style/KS-Style.sty\else
../../KS-Style/KS-Style.sty\fi\fi}
\input{\StylePath}

\KSset{2}{2}

\setcounter{section}{\value{KSS}-1}

\begin{document}
\maketitle
\HeaderCommentA
\section{\KSsection{section}}
\setcounter{prob}{\value{KSP}-1}
\fi

\begin{prob}\label{2.2}
  \(X\)を位相空間、\(A,B\subset X\)を閉集合とし、
  \(X=A\cup B\)であるとする。
  \(F\in \Ob(\mcD^+(X))\)に対して、
  自然に\((R\Gamma_B(F))_A \cong R\Gamma_B(F_A)\)となることを示せ。
\end{prob}

\begin{proof}
  函手\((-)_{X\setminus A}\)は完全なので、自然に
  \(R\Gamma_{X\setminus B}(-)_{X\setminus A}\cong
  R(\Gamma_{X\setminus B}(-)_{X\setminus A})\)
  が成り立つ。
  \((X\setminus A)\cap (X\setminus B) = \emptyset\)なので、
  \(\Gamma_{X\setminus B}(-)_{X\setminus A}=0\)が成り立ち、
  とくに\(R\Gamma_{X\setminus B}(-)_{X\setminus A}=0\)が成り立つ。
  \(F\)を\(X\)上の層の上に有界な複体とする。
  完全三角\(R\Gamma_B(F) \to F\to R\Gamma_{X\setminus B}(F)\xrightarrow{+1}\)
  に\((-)_{X\setminus A}\)を施すことにより、
  \(R\Gamma_B(F)_{X\setminus A}\xrightarrow{\sim}F_{X\setminus A}\)
  が従う。
  本文\cite[Proposition 2.4.10]{kashiwara2002sheaves}の直前の記述にあるとおり、
  脆弱層のなす\(X\)の部分圏は
  \(\Gamma_{X\setminus B}(-)\)-injective
  である。
  また本文\cite[Proposition 2.4.6 (i)]{kashiwara2002sheaves}より、
  脆弱層に対して\((-)|_{X\setminus B}\)を施したものも脆弱層である。
  よって\(i_B:X\setminus B\to X\)を包含射とすると、
  \(R\Gamma_{X\setminus B} \cong Ri_{B,*}\circ i_B^{-1}\)が成り立つ。
  \(i_B^{-1}((-)_{X\setminus A}) = 0\)であるから、
  任意の層に対して函手\((-)_{X\setminus A}\)を施したものは
  \(i_{B,*}\)に対してacyclicであり、
  よって自然に
  \(R\Gamma_{X\setminus B}((-)_{X\setminus A}) \cong
  R(\Gamma_{X\setminus B}((-)_{X\setminus A}))\)
  が成り立つ。
  \((X\setminus A)\cap (X\setminus B) = \emptyset\)なので、
  \(\Gamma_{X\setminus B}((-)_{X\setminus A}) = 0\)が成り立ち、
  とくに\(R\Gamma_{X\setminus B}((-)_{X\setminus A}) = 0\)が成り立つ。
  三角形
  \(R\Gamma_B(F_{X\setminus A})\to F_{X\setminus A}\to
  R\Gamma_{X\setminus B}(F_{X\setminus A})\xrightarrow{+1}\)
  が完全であることから、
  \(R\Gamma_B(F_{X\setminus A})\xrightarrow{\sim} F_{X\setminus A}\)
  は同型である。
  また、二つの図式
  \[
  \begin{CD}
    R\Gamma_B(F_{X\setminus A}) @>\sim>> F_{X\setminus A} @. \ \ \ \ \  @.
    R\Gamma_B(F)_{X\setminus A} @>\sim>> F_{X\setminus A} \\
    @VVV @VVV @. @VVV @VVV \\
    R\Gamma_B(F) @>>> F @. \ \ \ \ \ \ \ \ \ \ \ \ \ \ \ @. R\Gamma_B(F) @>>> F
  \end{CD}
  \]
  が可換であることから、
  \[
  \begin{CD}
    R\Gamma_B(F_{X\setminus A}) @>\sim>> R\Gamma_B(F)_{X\setminus A} \\
    @VVV @VVV  \\
    R\Gamma_B(F) @= R\Gamma_B(F)
  \end{CD}
  \]
  も可換である
  (二つの同型射を逆に辿って得られる二つの射
  \(F_{X\setminus A}\to R\Gamma_B(F)\)の差が\(0\)射である)。
  従って完全三角の間の同型射
  \[
  \begin{CD}
    R\Gamma_B(F_{X\setminus A}) @>>> R\Gamma_B(F) @>>> R\Gamma_B(F_A) @> +1 >> \\
    @V \cong VV @| @VV \cong V @. \\
    R\Gamma_B(F)_{X\setminus A} @>>> R\Gamma_B(F) @>>> R\Gamma_B(F)_A @> +1 >>
  \end{CD}
  \]
  を得る。
  以上で\autoref{2.2}の証明を完了する。
\end{proof}



\ifcsname Chap\endcsname\else
\printbibliography
\end{document}
\fi
