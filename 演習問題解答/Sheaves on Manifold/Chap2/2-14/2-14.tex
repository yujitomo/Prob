\ifcsname Chap\endcsname\else
\documentclass[uplatex,dvipdfmx]{jsarticle}
\newcommand{\StylePath}{\ifcsname AllKS\endcsname KS-Style/KS-Style.sty\else
\ifcsname Chap\endcsname ../KS-Style/KS-Style.sty\else
../../KS-Style/KS-Style.sty\fi\fi}
\input{\StylePath}

\KSset{2}{14}
\setcounter{section}{\value{KSS}-1}
\begin{document}
\maketitle
\HeaderCommentA
\section{\KSsection{section}}
\setcounter{prob}{\value{KSP}-1}

\fi


\begin{prob}\label{2.14}
  \(X = \bigcup_{i\in I}U_i\)を\(X\)の開被覆とする。
  各\(i\in I\)に対して\(F_i\)を\(U_i\)上の層として、
  各\((i,j)\in I^2\)に対して同型射
  \(\varphi_{ij}:F_j|_{U_i\cap U_j} \xrightarrow{\sim} F_i|_{U_i\cap U_j}\)
  が与えられているとする。
  \(\varphi_{ii}=\id_{F_i}\)であり、
  さらに任意の\((i,j,k)\in I^3\)に対して
  \(U_i\cap U_j\cap U_k\)上で
  \(\varphi_{ij}\circ \varphi_{jk}=\varphi_{ik}\)
  が成り立つと仮定せよ。
  このとき、\(X\)上の層\(F\)と
  各\(i\in I\)に対する同型射
  \(\varphi_i:F|_{U_i}\xrightarrow{\sim} F_i\)であって、
  任意の\((i,j)\in I^2\)に対して\(U_i\cap U_j\)上で
  \(\varphi_{ij} = \varphi_i\circ \varphi_j^{-1}\)
  が成り立つもの、が up to isomorphism で一意的に存在することを示せ。
\end{prob}


\begin{proof}
  圏\(\mcI\)を次で定義する:
  \begin{itemize}
    \item
    対象の集合は\(\mathrm{Ob}(\mcI) \dfn I^3\).
    \item
    \(\Hom_{\mcI}((i,j,k),(i',j',k'))\)は
    \(\{i,j,k\} \supset \{i',j',k'\}\)である場合は一点集合で、
    そうでない場合は\(\emptyset\)と定める。
  \end{itemize}
  \(\{i,j,k\} = \{i',j',k'\}\)
  である場合、またその場合に限り\((i,j,k)\to (i',j',k')\)は同型射である。

  \(U_{ij}\dfn U_i\cap U_j, U_{ijk}\dfn U_i\cap U_j\cap U_k\)とおき、
  \(f_i:U_i\to X, f_{ij}:U_{ij} \to X, f_{ijk}:U_{ijk}\to X\)
  をそれぞれ包含射とする。
  各\((i,j,k)\in \mcI\)に対して\(X\)上の層\(F(i,j,k)\)を
  \(P(i,j,k)\dfn f_{ijk,!}(F_i|_{U_{ijk}})\)と定義する
  (\(P(i,i,i) = f_{i!}F_i\)である)。
  各\(\mcI\)の射\(p:(i,j,k)\to (i',j',k')\)に対して
  \(U_{ijk}\subset U_{i'j'k'}\)であるので、
  自然な包含射
  \(\psi(p)(-):f_{i'j'k',!}((-))|_{U_{ijk}}) \subset
  f_{i'j'k',!}((-)|_{U_{i'j'k'}})\)
  がある。
  また、\(i'\in \{i,j,k\}\)であるので\(U_{ijk}\subset U_{ii'}\)である。
  \(P(p)\dfn \psi(p)(F_{i'})\circ f_{ijk,!}(\varphi_{i'i}|_{U_{ijk}})\)と定義する。
  この対応によって、\(P:\mcI\to \Ab(X)\)は函手になる。
  それを確かめるために、\(\mcI\)の射の列
  \((i,j,k)\xrightarrow{p}(i',j',k')\xrightarrow{q}(i'',j'',k'')\)
  を任意にとる。
  \(P\)が函手であるためには、
  \(P(q\circ p) = P(q)\circ P(p)\)が成り立つことが十分である。
  \(i''\in \{i',j',k'\}\subset \{i,j,k\}\)であるので、
  \(U_{ijk}\subset U_{ii'}\cap U_{i'i''}\)が成り立つ。
  従って
  \[
  f_{ijk,!}(\varphi_{i''i}|_{U_{ijk}})
  = f_{ijk,!}(\varphi_{i''i'}|_{U_{ijk}}\circ \varphi_{i'i}|_{U_{ijk}})
  = f_{ijk,!}(\varphi_{i''i'}|_{U_{ijk}})\circ f_{ijk,!}(\varphi_{i'i}|_{U_{ijk}})
  \]
  が成り立つ。
  また、定義より、函手の射として\(\psi(q\circ p)=\psi(q)\circ \psi(p)\)が成り立つ。
  また、\(\psi(p)\)が自然変換であることから、図式
  \[
  \begin{CD}
    f_{ijk,!}(F_{i'}|_{U_{ijk}})
    @> f_{ijk,!}(\varphi_{i''i'}|_{U_{ijk}})>>
    f_{ijk,!}(F_{i''}|_{U_{ijk}}) \\
    @V{\psi(p)(F_{i'})}VV
    @VV{\psi(p)(F_{i''})}V \\
    f_{i'j'k',!}(F_{i'}|_{U_{i'j'k'}})
    @> f_{i'j'k',!}(\varphi_{i''i'}|_{U_{i'j'k'}})>>
    f_{i'j'k',!}(F_{i''}|_{U_{i'j'k'}})
  \end{CD}
  \]
  は可換である。
  以上より、
  \begin{align*}
    P(q\circ p)
    &= \psi(q\circ p)(F_{i''}) \circ f_{ijk,!}(\varphi_{i''i'}|_{U_{ijk}}) \\
    &= \psi(q)(F_{i''}) \circ \psi(p)(F_{i''}) \circ
    f_{ijk,!}(\varphi_{i''i'}|_{U_{ijk}})\circ f_{ijk,!}(\varphi_{i'i}|_{U_{ijk}}) \\
    &= \psi(q)(F_{i''}) \circ f_{i'j'k',!}(\varphi_{i''i'}|_{U_{i'j'k'}})
    \circ \psi(p)(F_{i'}) \circ f_{ijk,!}(\varphi_{i'i}|_{U_{ijk}}) \\
    &= P(q)\circ P(p)
  \end{align*}
  が成り立つ。
  よって\(P:\mcI\to \Ab(X)\)は函手である。

  \(F\dfn \colim P\)とおく。
  各\(x\in X\)で stalk をとると図式\(P\)の射は\(0\)射と同型射の図式となる。
  従って、自然な射\(P_i:P(i,i,i)=f_{i,!}F_i\to F\)を\(U_i\)へと制限したものは
  層の同型射である。
  その逆射を\(\varphi_i\dfn P_i^{-1}:F|_{U_i}\to F_i\)とおく。
  図式
  \[
  \begin{CD}
    P(j,j,i) @>>> P(j,j,j) @> P_j >> F \\
    @V f_{ij,!}(\varphi_{ij}) VV @. @| \\
    P(i,j,j) @>>> P(i,i,i) @> P_i >> F
  \end{CD}
  \]
  は可換であり、
  \(P(j,j,i)\to P(j,j,j)\)と\(P(i,j,j)\to P(i,i,i)\)を\(U_{ij}\)へと制限すると
  \(\id_{F_j|_{U_{ij}}}\)と\(\id_{F_i|_{U_{ij}}}\)になるので、
  従って\(U_{ij}\)上で
  \(\varphi_{ij} = \varphi_i\circ \varphi_j^{-1}\)が成り立つ。

  別の\(F'\)がこの性質を満たせば、
  余極限の普遍性により射\(F\to F'\)が得られ、
  これは各点の stalk で同型射であるので、
  このような\(F\)は up to isom. で一意的に存在する。
  以上で\autoref{2.14}の証明を完了する。
\end{proof}






\ifcsname Chap\endcsname\else
\printbibliography
\end{document}
\fi
