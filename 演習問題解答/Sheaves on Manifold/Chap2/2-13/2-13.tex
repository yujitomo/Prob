\ifcsname Chap\endcsname\else
\documentclass[uplatex,dvipdfmx]{jsarticle}
\newcommand{\StylePath}{\ifcsname AllKS\endcsname KS-Style/KS-Style.sty\else
\ifcsname Chap\endcsname ../KS-Style/KS-Style.sty\else
../../KS-Style/KS-Style.sty\fi\fi}
\input{\StylePath}

\KSset{2}{13}
\setcounter{section}{\value{KSS}-1}
\begin{document}
\maketitle
\HeaderCommentA
\section{\KSsection{section}}
\setcounter{prob}{\value{KSP}-1}

\fi


\begin{prob}\label{2.13}
  \(G\)を\(X\)上の層として、
  \(Z\subset X\)を局所閉集合とする。
  \(j<n\)に対して\(R^j\Gamma_Z(G)=0\)が成り立つと仮定せよ。
  このとき、前層\(U\mapsto H^n_Z(U,G)\)は層であり、
  さらにこれが\(R^n\Gamma_Z(G)\)と等しいことを示せ。
\end{prob}

\begin{proof}
  \(R^n\Gamma_Z(G)\)は層なので、
  \autoref{2.13}を示すためには、
  開集合\(U\subset X\)に対して自然に
  \(H^n_Z(U,G)\cong \Gamma(U,R^n\Gamma_Z(G))\)
  が成り立つことを証明することが十分である。
  \(F=\Gamma_Z(-), F'=\Gamma(U,-)\)として\autoref{1.22}を適用することにより、
  \(R^n(\Gamma(U,\Gamma_Z(-)))(G)\cong \Gamma(U,R^n\Gamma_Z(G))\)
  が成り立つ。
  ここで\(\Gamma(U,\Gamma_Z(-))\cong \Gamma_Z(U,-)\)であるので、
  よって\(H^n_Z(U,G)\cong \Gamma(U,R^n\Gamma_Z(G))\)が成り立つ。
  以上で\autoref{2.13}の解答を完了する。
\end{proof}





\ifcsname Chap\endcsname\else
\printbibliography
\end{document}
\fi
