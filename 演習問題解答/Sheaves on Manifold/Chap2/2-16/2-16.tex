\ifcsname Chap\endcsname\else
\documentclass[uplatex,dvipdfmx]{jsarticle}
\newcommand{\StylePath}{\ifcsname AllKS\endcsname KS-Style/KS-Style.sty\else
\ifcsname Chap\endcsname ../KS-Style/KS-Style.sty\else
../../KS-Style/KS-Style.sty\fi\fi}
\input{\StylePath}

\KSset{2}{16}
\setcounter{section}{\value{KSS}-1}
\begin{document}
\maketitle
\HeaderCommentA
\section{\KSsection{section}}
\setcounter{prob}{\value{KSP}-1}

\fi


\begin{prob}\label{2.16}
  \(A\)を可換環、\(A^{\times}\)を単元のなす群とする。
  \(X\)を位相空間、\(\mcU=\{U_i\}_i\)を\(X\)の開被覆として、
  \(c\in C^2(\mcU,A_X^{\times})\)を\(\delta c = 0\)となる元とする。
  \(c'\)を\(c\)の\(H^2(C^{\bullet}(\mcU,A_X^{\times}))\)での剰余類として、
  \(c''\)を\(c'\)の\(H^2(X,A_X^{\times})\)での像とする
  (cf. \autoref{2.15.2})。
  圏\(\Sh(X,c)\)を次によって定義する:
  \begin{itemize}
    \item
    対象は\(A_{U_i}\)-加群\(F_i\)と
    同型射\(\rho_{ij}:F_j|_{U_i\cap U_j}\xrightarrow{\sim} F_i|_{U_i\cap U_j}\)
    の族\(\{F_i,\rho_{ij}\}\)で、
    任意の\(i,j,k\)に対して
    \[
    \rho_{ij}\rho_{jk}\rho_{ki} = c_{ijk}\id_{F_i|U_i\cap U_j\cap U_k}
    \]
    を満たすものとする。
    \item
    射\(f:\{F_i,\rho_{ij}\}\to \{F'_i,\rho_{ij}\}\)は
    \(U_i\)上の射の族\(f_i:F_i\to F'_i\)で
    \(U_i\cap U_j\)上で\(\rho'_{ij}\circ f_j = f_i\circ \rho_{ij}\)
    を満たすものとする。
  \end{itemize}
  以下を示せ:
  \begin{enumerate}
    \item \label{2.16.1}
    \(\Sh(X,c)\)はアーベル圏であることを証明せよ。
    \item \label{2.16.2}
    \(\tilde{c}\in C^2(\mcU,A_X^{\times})\)を別の元で
    \(\tilde{c}'' = c''\)を満たすものとする。
    \(\Sh(X,c)\)と\(\Sh(X,\tilde{c})\)の間の圏同値が存在することを示せ。
  \end{enumerate}
\end{prob}

\begin{proof}
  \ref{2.16.1}を示す。
  \(\Sh(X,c)\)は明らかな\(0\)-対象を持つ
  (各\(U_i\)上で\(0\)であるもの)。
  また、明らかに、二つの対象\(\{F_i,\rho_{ij}\},\{F_i',\rho_{ij}'\}\)に対して
  \(\{F_i\oplus F_i',\rho{ij}\oplus \rho_{ij}'\}\)は\(\Sh(X,c)\)の対象である。
  さらに、二つの対象の間の射\(f=(f_i):\{F_i,\rho_{ij}\}\to\{F_i',\rho_{ij}'\}\)に対し、
  各\(i\)ごとに\(\ker(f_i)\)をとり、
  \(\rho^{\ker(f)}_{ij}\dfn \rho_{ij}|_{\ker(f_i)|_{U_i\cap U_j}}\)
  と定めることにより、明らかに
  \(\{\ker(f_i),\rho^{\ker(f)}_{ij}\}\)は\(\Sh(X,c)\)の対象となる。
  余核についても同様である。
  核と余核が各\(i\)ごとに定義されるので、
  余像と像は一致し、これにより\(\Sh(X,c)\)がアーベル圏であることが従う。
  以上で\ref{2.16.1}の証明を完了する。

  \ref{2.16.2}を示す。
  \(\{F_i,\rho_{ij}\}\)を\(\Sh(X,c)\)の対象とする。
  \(\bar{c} \dfn c-\tilde{c}\)とおく。
  このとき\(\bar{c}''=0\)である。
  本文\cite[Proposition 2.8.4]{kashiwara2002sheaves}より、augmentation map
  \(\delta:F\xrightarrow{\text{qis}} \mcC^{\bullet}(\mcU,F)\)
  は擬同型であるので、
  \(H^2(R\Gamma(X,\mcC^{\bullet}(\mcU,A_X^{\times})))\)での
  \(\bar{c}''\)の像は\(0\)である。
\end{proof}







\ifcsname Chap\endcsname\else
\printbibliography
\end{document}
\fi
