\ifcsname Chap\endcsname\else
\documentclass[uplatex,dvipdfmx]{jsarticle}
\newcommand{\StylePath}{\ifcsname AllKS\endcsname KS-Style/KS-Style.sty\else
\ifcsname Chap\endcsname ../KS-Style/KS-Style.sty\else
../../KS-Style/KS-Style.sty\fi\fi}
\input{\StylePath}

\KSset{2}{10}
\setcounter{section}{\value{KSS}-1}
\begin{document}
\maketitle
\HeaderCommentA
\section{\KSsection{section}}
\setcounter{prob}{\value{KSP}-1}
\fi

\begin{prob}\label{2.10}
  \(\mcR\)を\(X\)上の環の層として、\(M\)を\(\mcR\)加群とする。
  \begin{enumerate}
    \item \label{2.10.1}
    \(M\)が入射的であるための必要十分条件は、
    任意の部分\(\mcR\)-加群\(\mcI\subset \mcR\)
    (これを\(\mcR\)の\textbf{イデアル}という)
    に対して
    \[
    \Gamma(X,M) \cong \Hom_{\mcR}(\mcR,M)\to \Hom_{\mcR}(\mcI,M)
    \]
    が全射となることである。これを示せ。
    \item \label{2.10.2}
    \(A\)を体とする。
    \(A_X\)のイデアルはある開集合\(U\subset X\)を用いて\(A_U\)と表すことができる。
    このことから、\(A_X\)-加群\(M\)が入射的であるための必要十分条件は
    \(M\)が脆弱層であることであることを帰結せよ。
  \end{enumerate}
\end{prob}


\begin{proof}
  \ref{2.10.1}を示す。
  必要性は明らかであるので十分性が問題である。
  \(\mcR\)-加群\(F\)とその部分\(\mcR\)-加群\(G\subset F\)と
  射\(g:G\to M\)を任意にとる。
  集合
  \[
  S\dfn \left\{ (H,h)\middle| G\subset H\subset F, h|_G=g\right\}
  \]
  に
  \[(H_0,H_0) \leq (H_1,h_1) \ \iff \ H_0\subset H_1 \text{かつ} h_1|_{H_0}=h_0\]
  で順序を入れる。
  全順序部分集合\(S_0\subset S\)に対して、
  \(H_{S_0}\dfn \bigcup_{H\in S_0}H\)と定めて
  \(h_{S_0}:H_{S_0}\to M\)を余極限の普遍性により定まる自然な射とすると
  \((H_{S_0},h_{S_0})\)は\(S_0\)の上界である。
  よってZornの補題より\(S\)には極大限\((H,h)\)が存在する。
  \(H\neq F\)であるとする。
  このとき、開集合\(U\subset X\)と切断\(s\in F(U)\setminus H(U)\)が存在する。
  \(U\)上の切断\(s\)は\(\mcR\)-加群の射\(\mcR_U\to H\)と対応する。
  Fiber積をとって\(\mcI\dfn \mcR_U\times_FH\)とおけば、
  \(\mcI\)は\(\mcR_U\)の部分\(\mcR\)-加群である。
  ここで
  \[
  \Hom_{\mcR}(\mcR,M)\to \Hom_{\mcR}(\mcR_U,M)\to \Hom_{\mcR}(\mcI,M)
  \]
  の合成は全射であるから、
  \(\Hom_{\mcR}(\mcR_U,M)\to \Hom_{\mcR}(\mcI,M)\)も全射であり、
  従って、自然な射影と\(h\)の合成\(\mcI\to H\xrightarrow{h}M\)は
  射\(\mcR_U\to M\)へとリフトし、
  可換図式
  \[
  \begin{CD}
    \mcI @> \subset >> \mcR_U \\
    @VVV @VVV \\
    H @> h >> M
  \end{CD}
  \]
  を得る。
  Push-out をとることによって、
  射\(h':H'\dfn \mcR_U\coprod_{\mcI}H \to M\)を得る。
  一方、可換図式
  \[
  \begin{CD}
    \mcI @> \subset >> \mcR_U \\
    @VVV @VV s V \\
    H @> \subset >> F
  \end{CD}
  \]
  で push-out をとることにより、
  射\(H' \to F\)を得るが、
  \(\mcI=\mcR_U\times_F H\)であることと
  \autoref{1.6.3}より、
  \(H'\to F\)はモノ射である。
  従って\(H'\subset F\)とみなせる。
  \(s\not\in H(U)\)なので\(H\subsetneq H'\)である。
  これは\((H,h) < (H',h')\)を意味し、\((H,h)\)の極大性に反する。
  この矛盾は\(H\neq F\)と仮定したことにより引き起こされたので、
  \(H=F\)であることが帰結し、
  以上で、\(f|_G=g\)となる射\(f:F\to M\)の存在が示された。
  これは\(F\)が入射的層であることを示している。
  以上で\ref{2.10.1}の証明を完了する。

  \ref{2.10.2}を示す。
  \(A\)を体、\(\mcI\subset A_X\)をイデアルとする。
  各\(x\in X\)に対して
  \(\mcI_x\subset A_{X,x}\)はイデアルであるが、
  \(A_{X,x}\)は体なので、\(\mcI_x\)は\(0\)か\(A_{X,x}\)のいずれかである。
  \[S\dfn \left\{x\in X\middle| \mcI_x=A_{X,x}\right\}\]
  とおき、\(S\)が開であることを示す。
  \(x\in S\)を任意にとる。
  \(\mcI_x=A_{X,x}\)であるので、
  ある開近傍\(x\in U\)とある切断\(s\in \mcI(U)\)が存在して、
  任意の\(y\in U\)に対して\(s_y = 1\)が成り立つ。
  これから各\(y\in U\)で\(\mcI_y\neq 0\)であることが従い、
  \(\mcI_y\)は\(0\)か\(A_{X,y}\)のいずれかであったので、
  \(\mcI_y = A_{X,y}\)が従う。
  よって\(U\subset S\)が従い、これは\(S\)が開であることを示している。
  最後の主張を示す。
  入射的ならば脆弱層であるため、
  \(A_X\)-加群\(M\)が脆弱層である場合に\(M\)が入射的であることを示す。
  \(M\)が入射的であることを示すためには、\ref{2.10.1}より、
  任意のイデアル層\(\mcI\subset A_X\)と任意の\(A_X\)-加群の射\(\mcI\to M\)に対し、
  それが\(\mcI\subset A_X\)に沿ってリフトすることを示すことが十分である。
  既に証明したことにより、イデアル層\(\mcI\subset A_X\)に対して
  ある開集合\(U\subset X\)が存在して\(\mcI = A_U\)が成り立つ。
  \(A_X\)加群の射\(A_U\to M\)は\(M(U)\)の切断と対応し、
  \(M\)は脆弱層であるので、
  それは\(M(X)\)の元に延長することができる。
  このことは射\(A_U=\mcI\to M\)が\(A_U=\mcI\subset A_X\)に沿ってリフトすることを意味し、
  従って\(M\)は入射的である。
  以上で\ref{2.10.2}の証明を完了し、
  \autoref{2.10}の解答を完了する。
\end{proof}




\ifcsname Chap\endcsname\else
\printbibliography
\end{document}
\fi
