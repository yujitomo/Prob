\ifcsname Chap\endcsname\else
\documentclass[uplatex,dvipdfmx]{jsarticle}
\newcommand{\StylePath}{\ifcsname AllKS\endcsname KS-Style/KS-Style.sty\else
\ifcsname Chap\endcsname ../KS-Style/KS-Style.sty\else
../../KS-Style/KS-Style.sty\fi\fi}
\input{\StylePath}

\KSset{2}{1}

\setcounter{section}{\value{KSS}-1}

\begin{document}
\maketitle
\HeaderCommentA
\section{\KSsection{section}}
\setcounter{prob}{\value{KSP}-1}
\fi


\begin{prob}\label{2.1}
  \(\N\)を自然数の集合で、
  \(\{0,\cdots, n\}, n\geq -1\)たちが開となる最も粗い位相を入れる。
  このとき、\(\N\)上の前層\(F\)は
  各\(n\)に対するアーベル群\(F_n\dfn F(\{0,\cdots,n\})\)と
  \(n\geq m\)に対する開集合の包含
  \(\{0,\cdots, m\} \subset \{0,\cdots ,n\}\)
  により引き起こされる制限写像\(F_n\to F_m\)の族に
  唯一のアーベル群\(F_{\infty}\dfn F(\N)\)を添加したものと同一視される。
  \begin{enumerate}[start=0]
    \item \label{2.1.0}
    前層\(F\)が層であるための必要十分条件は
    \(\Gamma(\N,F) \cong \lim_n F_n\)であることを示せ。
    \item \label{2.1.1}
    各\(j\neq 0,1\)に対して\(H^j(\N,F) = 0\)であることを示せ。
    \item \label{2.1.2}
    \(H^1(\N,F) \cong (\prod_n F_n)/I\)であることを示せ。
    ただし\(I\)の定義は、
    \(f_{i,j}:F_i\to F_j\)を層\(F\)の制限写像とするとき、
    以下で定義される:
    \[
    I\dfn \left\{ (x_n)_{n\in \N}\in \prod_n F_n \middle|
    \forall n\in \N, \exists y_n\in F_n, x_n = y_n - f_{n+1,n}(y_{n+1})\right\}.
    \]
  \end{enumerate}
\end{prob}

\begin{proof}
  \ref{2.1.0}は自明。
  \ref{2.1.1}を示す。
  \(G_n \dfn \prod_{i\leq n}F_i\)と置く。
  射影\(G_n\to G_{n-1}\)らにより定まる\(\N\)上の層を\(G\)と置くと、
  構成からただちに\(G\)が脆弱層であることがわかる。
  各\(n\geq i\)に対して制限写像
  \(F_n \to F_i\)の族が単射
  \(F_n \to \prod_{i\leq n}F_i = G_n\)
  を引き起こす。
  これは制限写像\(F_n\to F_{n-1}\)と射影\(G_n\to G_{n-1}\)と可換し、
  よって層の単射\(F\to G\)を得る。

  層\(G/F\)の構造を決定する。
  層\(F\)の制限写像を\(f_{i,j}:F_i\to F_j\)と置く。
  \[
  \varphi_n ((x_i)_{i\leq n}) \dfn (x_i-f_{n,i}(x_n))_{i<n}
  \]
  で定まる射\(\varphi_n : G_n \to H_n \dfn \prod_{i<n}F_i\)
  は全射であり、核はちょうど\(\im(F_n\to G_n)\)である。
  また、\(m\leq n\)に対して\(h_{n,m}: H_n\to H_m\)を
  \[
  h_{n,m}((x_i)_{i<n}) \dfn (x_i - f_{m,i}(x_m))_{i<m}
  \]
  と定めれば、各\(i < m \leq n\)に対して
  \[
  (x_i - f_{n,i}(x_n)) - (f_{m,i}(x_m - f_{n,m}(x_n))
  = x_i - f_{m,i}(x_m)
  \]
  となるので、図式
  \[
  \begin{CD}
    G_n @> \varphi_n >> H_n \\
    @V \text{proj.} VV @VV h_{n,m} V \\
    G_m @> \varphi_m >> H_m
  \end{CD}
  \]
  は可換である。
  これらの\(H_n\)により定まる層\(H\)は\(G/F\)に他ならないが、
  各\(h_{n,m}\)は全射であるから、\(H\)は脆弱層である。
  以上より\(\N\)上の層の完全列
  \[
  \begin{CD}
    0 @>>> F @>>> G @>>> H @>>> 0
  \end{CD}
  \]
  で\(G,H\)が脆弱層となるものが構成できた。
  このことは\(H^j(\N,F) = 0 , j\geq 2\)を示している。
  以上で\ref{2.1.1}の証明を完了する。

  \ref{2.1.2}を示す。
  \ref{2.1.1}の証明中に得られた層\(H\)の大域切断を決定する。
  それは\(\lim_n H_n\)であるから、このアーベル群を計算する。
  \(\lim_nH_n\)は\(\prod_{n\in N}H_n\)の部分加群で、
  \[
  \left\{ (x^n \dfn (x_i^n)_{i<n})_{n\in N} \middle| x^n \in H_n,
  \forall (m \leq n), h_{n,m}(x^n) = x^m \right\}
  \]
  となるものと自然に同型である。
  従って各\(i < m\leq n\)に対して
  \(x^m_i = x^n_i - h_{m,i}(x^n_m)\)
  となる。
  従って、このような元の族\(x^n\in H_n\)は\(x^n_{n-1}\in F_{n-1}\)によって
  \(x^n_i = x^{n-1}_i + h_{n-1,i}(x^n_{n-1})\)
  の形で一意的に決定される。
  すなわち、
  射影\(H_n\to F_{n-1}\)を並べて得られる射影
  \(\prod_{n\in \N}H_n\to \prod_{n\in \N_{\geq 1}}F_{n-1} =
  \prod_{n\in \N}F_n\)
  を\(\lim_n H_n \subset \prod_{n\in \N}H_n\)へ制限すると同型射となる。
  従って\(\Gamma(\N,H) \cong \prod_{n\in \N}F_n\)となる。
  以上より、アーベル群の完全列
  \[
  \begin{CD}
    0 @>>> H^0(\N,F) @>>> \prod_{n\in \N} F_n @> \varphi >> \prod_{n\in \N} F_n \\
    @>>> H^1(\N,F) @>>> 0 @.
  \end{CD}
  \]
  を得る。
  問われていることは、\(\varphi\)の像を決定することである。
  各\(n\)について、\(\varphi_n:G_n\to H_n\)の像の
  \(F_{n-1}\)の成分を見ればそれは決定できる。
  \((x_n)_{n\in \N}\in \Gamma(\N,G) \cong \prod_{n\in \N}F_n\)を任意にとると、
  \(\varphi_n((x_i)_{i\leq n}) = (x_i-f_{n,i}(x_n))_{i<n}\)
  であるから、
  \(F_{n-1}\)の成分は\(x_{n-1}-f_{n,n-1}(x_n)\)である。
  従って、
  \[
  \varphi((x_n)_{n\in \N}) = (x_n - f_{n+1,n}(x_{n+1}))_{n\in \N}
  \]
  となる。
  従って\(\im(\varphi) = I\)がわかる。
  よって\(H^1(\N,F) \cong (\prod_{n\in \N}F_n)/I\)が示された。
  以上で解答を完了する。
\end{proof}



\ifcsname Chap\endcsname\else
\printbibliography
\end{document}
\fi
