\ifcsname Chap\endcsname\else
\documentclass[uplatex,dvipdfmx]{jsarticle}
\newcommand{\StylePath}{\ifcsname AllKS\endcsname KS-Style/KS-Style.sty\else
\ifcsname Chap\endcsname ../KS-Style/KS-Style.sty\else
../../KS-Style/KS-Style.sty\fi\fi}
\input{\StylePath}

\KSset{2}{9}
\setcounter{section}{\value{KSS}-1}
\begin{document}
\maketitle
\HeaderCommentA
\section{\KSsection{section}}
\setcounter{prob}{\value{KSP}-1}

本文では、局所コンパクト空間であるという場合には、
ハウスドルフ性を常に仮定していることに注意しておく
(cf. 本文\cite[Proposition 2.5.1]{kashiwara2002sheaves}直前の記述)。
\fi

\begin{prob}\label{2.9}
  \(X\)を位相空間とする。
  \begin{enumerate}
    \item \label{2.9.1}
    \(F\)を\(X\)上の層として、
    \(n \geq 0\)を自然数とする。
    以下の条件が同値であることを示せ:
    \begin{enumerate}
      \item \label{2.9.1.1}
      完全列
      \(0\to F\to F^0\to \cdots \to F^n\to 0\)で
      各\(j=0,\cdots,n\)に対して\(F^j\)が脆弱であるものが存在する。
      \item \label{2.9.1.2}
      \(0\to F\to F^0\to \cdots \to F^n\to 0\)が完全であり、
      各\(j<n\)に対して\(F^j\)が脆弱であれば、\(F^n\)も脆弱である。
      \item \label{2.9.1.3}
      任意の閉部分集合\(Z\subset X\)と任意の\(k>n\)に対して
      \(H^k_Z(X,F) = 0\)が成り立つ。
      \item \label{2.9.1.4}
      任意の局所閉部分集合\(Z\subset X\)と任意の\(k>n\)に対して
      \(H^k_Z(X,F) = 0\)が成り立つ。
      \item \label{2.9.1.5}
      任意の閉部分集合\(Z\subset X\)と任意の\(k>n\)に対して
      \(H^k_Z(F) = 0\)が成り立つ。
      \item \label{2.9.1.6}
      任意の局所閉部分集合\(Z\subset X\)と任意の\(k>n\)に対して
      \(H^k_Z(F) = 0\)が成り立つ。
    \end{enumerate}
    これらの条件を満たす最小の\(n\geq 0\)を
    \(F\)の\textbf{脆弱次元} (flabby dimension) と言い、
    \(X\)上のすべての層\(F\)の脆弱次元のsupを\(X\)の\textbf{脆弱次元}と言う。
    \item \label{2.9.2}
    \(X\)を局所コンパクトハウスドルフであるとする。
    \(X\)上の層\(F\)の\textbf{\(c\)-soft dimension}を同様に定義して、
    この場合にも\ref{2.9.1}の条件
    \ref{2.9.1.1}から\ref{2.9.1.4}に対応するものが同値であることを確認せよ。
    \item \label{2.9.3}
    \(X\)を局所コンパクトハウスドルフであるとする。
    このとき、以下の不等式を証明せよ:
    \[
    \text{\(F\)の\(c\)-soft dimension} \leq \text{\(F\)の脆弱次元}
    \leq \text{\(F\)の\(c\)-soft dimension} + 1.
    \]
  \end{enumerate}
\end{prob}


\begin{proof}
  \ref{2.9.1}を示す。帰納法で証明する。
  \(n=0\)とする。
  条件\ref{2.9.1.1}と\ref{2.9.1.2}はどちらも
  「\(F\)は脆弱層である」と読むことができるので明らかに同値である。
  \ref{2.9.1.4} \(\Rightarrow\) \ref{2.9.1.3} と
  \ref{2.9.1.6} \(\Rightarrow\) \ref{2.9.1.5} が成り立つことは明らかである。
  また脆弱層は函手\(\Gamma_Z(X,-)\)や\(\Gamma_Z(-)\)に対してacyclicである
  (cf. 本文\cite[Proposition 2.4.10]{kashiwara2002sheaves}の直前の記述) ので、
  \ref{2.9.1.1} \(\Rightarrow\) \ref{2.9.1.4}, \ref{2.9.1.5} が成り立つ。
  \(U\subset X\)を任意の開集合として、\(Z\dfn X\setminus U\)とおけば、
  \begin{align*}
    &0 \to H^0_Z(X,F) \to F(X) \to F(U) \to H^1_Z(X,F), \\
    &0 \to H^0_Z(F) \to F \to \Gamma_U(F) \to H^1_Z(F)
  \end{align*}
  は完全であるから、上の列が完全であることから
  \ref{2.9.1.3} \(\Rightarrow\) \ref{2.9.1.1}が成り立ち、
  下の列が完全であることと
  本文\cite[Proposition 2.4.10]{kashiwara2002sheaves}の証明中で示されている
  主張 (2.4.1) より、
  \ref{2.9.1.5} \(\Rightarrow\) \ref{2.9.1.1}が成り立つ。
  以上で\(n=0\)の場合に
  条件\ref{2.9.1.1}から\ref{2.9.1.6}が全て同値であることが示された。
  ある\(n\)で所望の同値性が示されていると仮定して、
  \(n+1\)に対して所望の同値性を示す。
  \ref{2.9.1.4} \(\Rightarrow\) \ref{2.9.1.3} と
  \ref{2.9.1.6} \(\Rightarrow\) \ref{2.9.1.5} はいつでも成立する。
  また、\(n\)番目まで入射分解をとることによって、
  \ref{2.9.1.2} \(\Rightarrow\) \ref{2.9.1.1} が成り立つ。
  \(F\)が\(n+1\)に対して\ref{2.9.1.1}を満たすと仮定する。
  脆弱層への単射\(f:F\to F_0\)を任意にとる。
  \(\coker(f)\)は\(n\)に対して\ref{2.9.1.1}を満たすので、
  帰納法の仮定より、\(\coker(f)\)は\(n\)に対して\ref{2.9.1.2}を満たす。
  \(f\)の取り方は任意だったので、
  これは\(F\)が\(n+1\)に対して\ref{2.9.1.2}を満たすことを意味する。
  また、完全列\(0\to F\to F_0\to \coker(f) \to 0\)
  で局所コホモロジーをとると、\(F_0\)が脆弱層であることから、
  任意の局所閉集合\(Z\subset X\)と\(i\geq 1\)に対して同型射
  \(H^i_Z(X,\coker(f)) \xrightarrow{\sim} H^{i+1}_Z(X,F)\)と
  \(H^i_Z(\coker(f)) \xrightarrow{\sim} H^{i+1}_Z(F)\)を得る。
  \(\coker(f)\)は\(n\)に対して\ref{2.9.1.1}を満たすので、
  帰納法の仮定より、\(\coker(f)\)は\(n\)に対して\ref{2.9.1.4}と\ref{2.9.1.6}を満たす。
  よって、\(F\)が\(n+1\)に対して\ref{2.9.1.4}と\ref{2.9.1.6}を満たすことが従う。
  \(F\)が\(n+1\)に対して\ref{2.9.1.3}または\ref{2.9.1.5}を満たすと仮定する。
  脆弱層への単射\(f:F\to F_0\)を任意にとれば、先ほどと同様にして、
  \(\coker(f)\)が\(n\)に対して\ref{2.9.1.3}または\ref{2.9.1.5}を満たすことが従う。
  帰納法の仮定より、\(\coker(f)\)が\(n\)に対して\ref{2.9.1.1}を満たすことが従い、
  よって\(F\)が\(n\)に対して\ref{2.9.1.1}を満たすことが従う。
  以上で\ref{2.9.1}の証明を完了する。

  \ref{2.9.2}を示す。
  \(X\)を局所コンパクトハウスドルフ空間とする。
  \ref{2.9.1}の主張\ref{2.9.1.1}から\ref{2.9.1.4}に対応するのは
  以下の主張である (ほんまか??):
  \begin{enumerate}
    \item \label{2.9.2.1}
    完全列
    \(0\to F\to F^0\to \cdots \to F^n\to 0\)で
    各\(j=0,\cdots,n\)に対して\(F^j\)が \(c\)-soft であるものが存在する。
    \item \label{2.9.2.2}
    \(0\to F\to F^0\to \cdots \to F^n\to 0\)が完全であり、
    各\(j<n\)に対して\(F^j\)が \(c\)-soft であれば、\(F^n\)も \(c\)-soft である。
    \item \label{2.9.2.3}
    任意の開集合\(U\)と任意の\(k>n\)に対して
    \(H^k_c(U,F|_U)=0\)が成り立つ。
    \item \label{2.9.2.4}
    任意の局所閉集合\(U\)と任意の\(k>n\)に対して
    \(H^k_c(U,F|_U)=0\)が成り立つ。
  \end{enumerate}
  帰納法で証明する。まず\(n=0\)の場合にこれらの主張が同値であることを示す。
  \ref{2.9.2.1}と\ref{2.9.2.2}が同値であることは明らかである。
  また、\autoref{2.6.1}より、
  \ref{2.9.2.1}と\ref{2.9.2.3}も同値である。
  \ref{2.9.2.4}から\ref{2.9.2.3}が従うことは明らかである。
  さらに、\(c\)-soft な層の局所閉部分集合への制限はまた\(c\)-softであるから、
  \autoref{2.6.1}より、
  \ref{2.9.2.1}と\ref{2.9.2.2}と\ref{2.9.2.3}のいずれかを仮定すれば
  \ref{2.9.2.4}が導かれる。
  以上で\(n=0\)の場合の証明を完了する。
  ある\(n\)で所望の同値性が示されていると仮定して、
  \(n+1\)に対して所望の同値性を示す。
  \ref{2.9.2.4}から\ref{2.9.2.3}が従うことは明らかである。
  また、\(n\)番目までの入射分解をとれば、
  入射的な層は脆弱層であり、脆弱層は\(c\)-softであるから、
  これは\(n\)番目までの \(c\)-soft 分解を与えるので、
  その余核を考えることによって、
  \ref{2.9.2.2}から\ref{2.9.2.1}が導かれる。
  \(F\)が\(n+1\)に対して\ref{2.9.2.1}を満たすとする。
  \(c\)-soft な層への単射\(f:F\to F_0\)を任意にとる。
  このとき、\(\coker(f)\)は\(n\)に対して\ref{2.9.2.1}を満たす。
  従って、帰納法の仮定より、
  \(\coker(f)\)は\(n\)に対して\ref{2.9.2.2}を満たす。
  \(f\)の取り方は任意だったので、
  これは\(F\)が\(n+1\)に対して\ref{2.9.2.2}を満たすことを意味する。
  さらに、帰納法の仮定より、\(\coker(f)\)は\(n\)に対して\ref{2.9.2.4}を満たす。
  任意に局所閉集合\(U\)をとって、
  \(U\)に制限したあとでコンパクト台つきコホモロジーをとることにより、
  各\(i\geq 1\)に対して自然な同型射
  \(H^i_c(U,\coker(f)) \xrightarrow{\sim} H^{i+1}_c(U,F)\)
  を得る。
  \(\coker(f)\)は\(n\)に対して\ref{2.9.2.4}を満たすので、
  従って\(F\)は\(n+1\)に対して\ref{2.9.2.4}を満たす。
  \(F\)が\(n+1\)に対して\ref{2.9.2.3}を満たすと仮定する。
  \(c\)-soft な層への単射\(f:F\to F_0\)をとれば、
  各\(i\geq 1\)に対して自然な射
  \(H^i_c(U,\coker(f)) \xrightarrow{\sim} H^{i+1}_c(U,F)\)
  は同型射であるから、
  \(\coker(f)\)は\(n\)に対して\ref{2.9.2.3}を満たす。
  従って、帰納法の仮定より、\(\coker(f)\)は\(n\)に対して\ref{2.9.2.1}を満たす。
  \ref{2.9.2.1}によって存在が要請される
  \(\coker(f)\)の \(c\)-soft な層による長さ\(n\)の分解を
  \(f\)と繋げることにより、
  \(F\)の \(c\)-soft な層 による長さ\(n+1\)の分解を得るので、
  \(F\)は\(n+1\)に対して\ref{2.9.2.1}を満たす。
  以上で\ref{2.9.2}の証明を完了する。

  \ref{2.9.3}を示す。
  脆弱層が\(c\)-softであることから、不等式
  \(\text{\(F\)の\(c\)-soft dimension} \leq \text{\(F\)の脆弱次元}\)が従う。
  もう一つの不等式を証明する。
  \(F\)の \(c\)-soft dimension が\(n\)であるとする。
  完全列
  \(0\to F\to F^0\to \cdots \to F^n\)で
  各\(j\)に対して\(F^j\)が脆弱層であるものをとる。
  \(F\)の \(c\)-soft dimension が\(n\)であることから、
  \(\im(F^{n-1}\to F^n)\)は \(c\)-soft である。
  従って、\(F\)の脆弱次元が\(n+1\)以下であることを示すためには、
  次を示すことが十分である:
  \begin{enumerate}[label=(\fnsymbol*),start=2]
    \item \label{2.9.3.p}
    局所コンパクトハウスドルフな位相空間\(X\)上の層の完全列
    \(0\to F\to G\to H\to 0\)に対して、
    \(F\)が \(c\)-soft であり、
    \(G\)が脆弱層であるとき、
    \(H\)も脆弱層である。
  \end{enumerate}
  \(X\)は局所コンパクトであるので、
  各\(x\in X\)に対して開近傍\(x\in V\subset X\)であって
  \(\bar{V}\)がコンパクトとなるものが存在する。
  本文\cite[Proposition 2.4.10]{kashiwara2002sheaves}の証明で示されている
  主張 (2.4.1) より、
  \ref{2.9.3.p}を示すためには、
  \(H|_V\)が脆弱であることを示すことが十分である。
  \(Z\subset V\)を閉集合とする。
  \(K\dfn \bar{Z}\cup(\bar{V}\setminus V)\)とおく。
  これはコンパクト空間\(\bar{V}\)の閉部分空間であるからコンパクトである。
  \(F\)は\(c\)-softであるから、
  \autoref{2.6.1}より、
  \(H^i_c(X,F)=H^i_c(X\setminus K,F) = 0, (\forall i>0)\)が成り立つ。
  各\(i\)に対して
  \(H^i_c(X\setminus K,F) \to H^{i+1}_{c,K}(X,F) \to H^{i+1}_c(X,F)\)
  は完全であるので、
  \(H^i_{c,K}(X,F)=0,(\forall i\geq 2)\)が成り立つ。
  \(K\)はコンパクトであるので、
  \(H^i_{c,K}(X,-)\cong H^i_K(X,-)\)が成り立つ。
  \(G\)は脆弱なので、\(H^i_K(X,G) = 0, (\forall i>0)\)が成り立ち、
  従って、完全列\(0\to F\to G\to H\to 0\)に函手\(\Gamma_K(X,-)\)を適用すると、
  \(H^i_K(X,H) = 0, (\forall i>0)\)が成り立つ。
  \(H(X)\to H(X\setminus K) \to H^1_K(X,H)\)は完全なので、
  従って、
  \(H(X)\to H(X\setminus K)\)は全射である。
  \(X\setminus K = (V\setminus Z)\cup (X\setminus \bar{V})\)
  なので、\(H(X\setminus K)\cong H(V\setminus Z)\times H(X\setminus \bar{V})\)
  が成り立つ。
  よって\(H(X)\to H(V\setminus Z)\)は全射であり、
  とくに\(H(V)\to H(V\setminus Z)\)も全射である。
  以上より\(H|_V\)は脆弱である。

  以上で\ref{2.9.3}の証明を完了し、
  \autoref{2.9}の解答を完了する。
\end{proof}




\ifcsname Chap\endcsname\else
\printbibliography
\end{document}
\fi
