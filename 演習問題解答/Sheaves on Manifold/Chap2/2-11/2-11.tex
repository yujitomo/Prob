\ifcsname Chap\endcsname\else
\documentclass[uplatex,dvipdfmx]{jsarticle}
\newcommand{\StylePath}{\ifcsname AllKS\endcsname KS-Style/KS-Style.sty\else
\ifcsname Chap\endcsname ../KS-Style/KS-Style.sty\else
../../KS-Style/KS-Style.sty\fi\fi}
\input{\StylePath}

\KSset{2}{11}
\setcounter{section}{\value{KSS}-1}
\begin{document}
\maketitle
\HeaderCommentA
\section{\KSsection{section}}
\setcounter{prob}{\value{KSP}-1}

本文では、局所コンパクト空間であるという場合には、
ハウスドルフ性を常に仮定していることに注意しておく
(cf. 本文\cite[Proposition 2.5.1]{kashiwara2002sheaves}直前の記述)。
\fi


\begin{prob}\label{2.11}
  \(f:Y\to X\)を局所コンパクトハウスドルフ空間の間の連続写像、
  \(G\)を\(Y\)上の層とする。
  以下の主張が同値であることを示せ:
  \begin{enumerate}
    \item \label{2.11.1}
    任意の\(x\in X\)に対して
    \(G|_{f^{-1}(x)}\)は\(c\)-softである。
    \item \label{2.11.2}
    任意の開集合\(V\subset Y\)と任意の\(j>0\)に対して
    \(R^jf_!G_V=0\)である。
  \end{enumerate}
\end{prob}

\begin{proof}
  \ref{2.11.1} \(\Rightarrow\) \ref{2.11.2}
  を示す。
  任意の\(x\in X\)に対して
  \(G|_{f^{-1}(x)}\)は \(c\)-soft であると仮定する。
  開集合\(V\subset Y\)と点\(x\in X\)を任意にとる。
  本文\cite[Proposition 2.6.7]{kashiwara2002sheaves}より、
  各点\(x\in X\)に対して自然に
  \((R^jf_!G_V)_x \cong H^j_c(f^{-1}(x)\cap V,G|_{f^{-1}(x)})\)
  が成り立つ。
  ここで\(G|_{f^{-1}(x)}\)は \(c\)-soft であるので、
  \autoref{2.6.1}より、\(j>0\)に対して
  \(H^j_c(f^{-1}(x)\cap V,G|_{f^{-1}(x)})=0\)が成り立つ。
  よって層\(R^jf_!G_V\)の各点でのstalkは\(0\)であり、
  従って\(R^jf_!G_V=0\)である。

  \ref{2.11.2} \(\Rightarrow\) \ref{2.11.1}
  を示す。
  任意の開集合\(V\subset Y\)と任意の\(j>0\)に対して
  \(R^jf_!G_V=0\)であると仮定する。
  点\(x\in X\)と開集合\(V_x\subset f^{-1}(x)\)を任意にとる。
  このとき、ある開集合\(V\subset Y\)が存在して
  \(V_x = V\cap f^{-1}(x)\)が成り立つ。
  本文\cite[Proposition 2.6.7]{kashiwara2002sheaves}より、各\(j>0\)に対して自然に
  \(H^j_c(V_x,G|_{f^{-1}(x)}) \cong (R^jf_!G_V)_x = 0\)
  が成り立つ。
  よって\autoref{2.6.1}より、
  \(G|_{f^{-1}(x)}\)は \(c\)-soft である。
  以上で\autoref{2.11}の解答を完了する。
\end{proof}





\ifcsname Chap\endcsname\else
\printbibliography
\end{document}
\fi
